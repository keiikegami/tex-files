\documentclass[dvipdfmx, 12pt]{beamer}
\usepackage{pxjahyper}
\usepackage{minijs}
\usepackage{otf}
\renewcommand{\kanjifamilydefault}{\gtdefault}
\usetheme{default}
\setbeamertemplate{navigation symbols}{}
\usepackage{url}
\usepackage{graphicx}


\title{銀行理論と金融危機\\4. 銀行システムと金融危機のマクロモデル}
\author{池上 慧 \\ 東京大学経済学部4年}

\begin{document}
\newcommand{\argmin}{\mathop{\rm arg~min}\limits}

\frame{\maketitle}

\begin{frame}\frametitle{問題意識}
	\begin{itemize}
		\item 銀行はアロー証券の取引が実現できない代わりにある程度のリスクをプールし負債を発行する主体
		\item 銀行が存在すること自体が、現実はアロー証券が完備された経済に比して不完備な市場であることの証左
		\item この事実はモデルの範疇を超えて成り立つ命題
		\item どう効率的でないのかを以下のモデルで見ていく
	\end{itemize}
\end{frame}

\begin{frame}\frametitle{モデルの概要}
	\begin{itemize}
		\item $t = 0,1,2$の3期間
		\item 経済主体は家計、銀行、企業家
		\item 無限の家計、無限の銀行、銀行と同数の企業家が存在
	\end{itemize}
\end{frame}

\begin{frame}\frametitle{$t = 0$}
	\begin{itemize}
		\item 家計 : 初期資産$1$を全額預金する。
		\item 銀行 : 預金された初期資産を全額企業家に貸し出す。また預金のグロスのリターン$D$を決定する。
		\item 企業家 : $1$単位の消費財をインプット
	\end{itemize}
\end{frame}

\begin{frame}\frametitle{$t = 1$}
	\begin{itemize}
		\item 家計 : 流動性選好$\theta$を知る。確定している生涯所得$m$に対して、金融危機発生の有無ごとの消費計画を立てる。この時同時に預金引き出し額(流動性需要関数)が決定される。
		\item 銀行 : 家計の流動性需要に応えるために、貸し出している企業家プロジェクトのうちで、次期にもたらす収益が現段階でプロジェクトを中断して利率$R$で再運用した際の収益を下回るものを中断させ、消費財として家計に払い出す(流動性供給関数)。
		\item 企業家 : 特に何もしない。銀行から中断と判断された企業家のみここで廃業する。
	\end{itemize}
\end{frame}

\begin{frame}\frametitle{$t = 2$}
	\begin{itemize}
		\item 家計 : $t=1$で決めた消費プロファイルに従って消費活動を行う。
		\item 銀行 : $t = 1$で潰さなかった企業家が生産した資本財のうち、自身の取り分として割合$\gamma$だけ所持する。企業家による資本財を用いた消費財生産が終了したら、所持している分の資本財を競争的な資本財市場で価格$q_2$で売払い、売却益を得る。
		\item 企業家 : $t = 1$で生き残った企業家のみが行動をとれる。資本財を$\omega$だけ生産する。生産した資本財を用いて消費財の生産を行い家計の消費分を賄う。消費財生産が終わった後、自身の所持割合である$1-\gamma$だけ資本財を競争的な資本財市場で価格$q_2$で売払い、売却益を得る。
	\end{itemize}
\end{frame}

\begin{frame}\frametitle{}
	\begin{itemize}
		\item k
	\end{itemize}
\end{frame}

\begin{frame}\frametitle{}
	\begin{itemize}
		\item k
	\end{itemize}
\end{frame}

\begin{frame}\frametitle{}
	\begin{itemize}
		\item k
	\end{itemize}
\end{frame}

\begin{frame}\frametitle{}
	\begin{itemize}
		\item k
	\end{itemize}
\end{frame}

\begin{frame}\frametitle{}
	\begin{itemize}
		\item k
	\end{itemize}
\end{frame}

\end{document}































