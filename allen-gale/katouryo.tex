\documentclass[dvipdfmx, 12pt]{beamer}
\usepackage{pxjahyper}
\usepackage{minijs}
\usepackage{otf}
\renewcommand{\kanjifamilydefault}{\gtdefault}
\usetheme{default}
\setbeamertemplate{navigation symbols}{}
\usepackage{url}
\usepackage{graphicx}


\title{銀行理論と金融危機\\4. 銀行システムと金融危機のマクロモデル}
\author{池上 慧 \\ 東京大学経済学部4年}

\begin{document}
\newcommand{\argmin}{\mathop{\rm arg~min}\limits}

\frame{\maketitle}

\begin{frame}\frametitle{問題意識}
	\begin{itemize}
		\item 銀行はアロー証券の取引が実現できない代わりにある程度のリスクをプールし負債を発行する主体
		\item 銀行が存在すること自体が、現実はアロー証券が完備された経済に比して不完備な市場であることの証左
		\item この事実はモデルの範疇を超えて成り立つ命題
		\item どう効率的でないのかを以下のモデルで見ていく
	\end{itemize}
\end{frame}

\begin{frame}\frametitle{モデルの概要}
	\begin{itemize}
		\item $t = 0,1,2$の3期間
		\item 経済主体は家計、銀行、企業家
		\item 無限の家計、無限の銀行、銀行と同数の企業家が存在
	\end{itemize}
\end{frame}

\begin{frame}\frametitle{$t = 0$}
	\begin{itemize}
		\item 家計 : 初期資産$1$を全額預金する。
		\item 銀行 : 預金された初期資産を全額企業家に貸し出す。また預金のグロスのリターン$D$を決定する。
		\item 企業家 : $1$単位の消費財をインプット
	\end{itemize}
\end{frame}

\begin{frame}\frametitle{$t = 1$}
	\begin{itemize}
		\item 家計 : 流動性選好$\theta$を知る。確定している生涯所得$m$に対して、金融危機発生の有無ごとの消費計画を立てる。この時同時に預金引き出し額(流動性需要関数)が決定される。
		\item 銀行 : 家計の流動性需要に応えるために、貸し出している企業家プロジェクトのうちで、次期にもたらす収益が現段階でプロジェクトを中断して利率$R$で再運用した際の収益を下回るものを中断させ、消費財として家計に払い出す(流動性供給関数)。
		\item 企業家 : 特に何もしない。銀行から中断と判断された企業家のみここで廃業する。
	\end{itemize}
\end{frame}

\begin{frame}\frametitle{$t = 2$}
	\begin{itemize}
		\item 家計 : $t=1$で決めた消費プロファイルに従って消費活動を行う。
		\item 銀行 : $t = 1$で潰さなかった企業家が生産した資本財のうち、自身の取り分として割合$\gamma$だけ所持する。企業家による資本財を用いた消費財生産が終了したら、所持している分の資本財を競争的な資本財市場で価格$q_2$で売払い、売却益を得る。
		\item 企業家 : $t = 1$で生き残った企業家のみが行動をとれる。資本財を$\omega$だけ生産する。生産した資本財を用いて消費財の生産を行い家計の消費分を賄う。消費財生産が終わった後、自身の所持割合である$1-\gamma$だけ資本財を競争的な資本財市場で価格$q_2$で売払い、売却益を得る。
	\end{itemize}
\end{frame}

\begin{frame}\frametitle{モデルを解く:家計の行動}
	\begin{itemize}
		\item $t = 1$における2期間の最適消費計画のみが決定事項
		\item 効用関数は$U = \theta {\rm log}(C_1) + (1 - \theta){\rm log}(C_2)$
		\item この時点での生涯所得は$m = w_1 + D + \frac{w_2}{R}$
		\item 金融危機の時は強制的に$(C_1, C_2) = (w_1+X, \underline{w})$
		\item 金融危機が起きない時の計画は、2期間モデルを解いて$(C_1, C_2) = (\theta m, (1-\theta)Rm)$であり、流動性需要は$g = \theta \left( \frac{w_2}{R} + D\right) - (1-\theta)w_1$
	\end{itemize}
\end{frame}

\begin{frame}\frametitle{モデルを解く:企業家の行動}
	\begin{itemize}
		\item こいつは何も決定しない
		\item 消費財から資本財を経てまた消費財を生み出す箱としてみる
	\end{itemize}
\end{frame}

\begin{frame}\frametitle{モデルを解く:銀行の行動$t = 1$}
	\begin{itemize}
		\item $t = 1$における企業家の選別と$t = 0$におけるリターン$D$の二つが決定事項
		\item モデルは$t=1$からバックワードに解く
		\item $RX = q_2\gamma \omega^*$を満たす資本財生産$\omega^*$を行う企業家に対して廃業と存続が無差別になる
		\item $\omega < \omega^*$ならば廃業、$\omega^* < \omega$ならば存続
		\item $\omega$が$[\omega_L, \omega_H]$上に一様に分布すると仮定しaggregateには不確実性がないとすると、流動性供給は$L(\frac{R}{q_2}) = \int_{\omega_L}^{\omega^*}X\mathrm{d}\omega$
	\end{itemize}
\end{frame}

\begin{frame}\frametitle{モデルを解く:銀行の行動$t = 0$}
	\begin{itemize}
		\item 2節Allen and Galeモデルの帰結より銀行は家計の期待効用を最大化するように動くのがナッシュ均衡なので、以下の問題を解けばいい。
	\end{itemize}
	\begin{align*}
	\scalebox{0.7}{$\displaystyle
	max_{D} \int_0^{\theta^*}\left\{ \theta{\rm log}(\tilde{C_1})+(1-\theta){\rm log}(\tilde{C_2}) \right\}\mathrm{d}F(\theta) + \int_{\theta^*}^1 \left\{ \theta{\rm log}(w_1+X)+(1-\theta){\rm log}(\underline{w}) \right\}\mathrm{d}F(\theta)
	$}
	\end{align*}
	\begin{align*}
	\scalebox{0.7}{$\displaystyle
		\text{where}\ \tilde{C_1} = \theta m,\ \tilde{C_2} = (1-\theta)Rm,\ \theta^* = \frac{L(\frac{R^*}{q_2^*}) + w_1}{w_1 + D + \frac{w_2}{R^*}}
		$}
	\end{align*}
\end{frame}

\begin{frame}\frametitle{モデルを解く:銀行の行動$t = 0$}
	\begin{itemize}
		\item $\theta^*$はデフォルト限界金利において流動性市場を均衡させる流動性の値
		\item デフォルト限界金利は銀行のソルベンシー制約を等式で満たす金利、すなわち$A(\cot)$を銀行の$t=1$における総資産の現在価値を示す関数として、$D = A\left( \frac{R^*}{q_2^*} \right)$を満たす$R^*$
		\item 流動性市場の均衡条件は$g = L\left(\frac{R}{q_2}\right)$
		\item $(\tilde{C_1}, \tilde{C_2})$は家計の最適行動の帰結(バックワードに解いてるのでこの結果を織り込んで最適な$D$を計算する)
	\end{itemize}
\end{frame}

\begin{frame}\frametitle{モデルを解く:銀行の行動$t = 0$}
	\begin{itemize}
		\item 上記の最適化問題を解くと式(39)をFOCとして得る
		\item FOCを満たす$D$は図5の示す通りただ一つ存在する
	\end{itemize}
\end{frame}

\begin{frame}\frametitle{モデルの帰結$1$}
	\begin{itemize}
		\item 以下で$LF$は資産財価格$q_2$を所与とした時の値を意味する
		\item $\frac{\mathrm{d}R^*}{\mathrm{d}D}|_{LF} < 0$
		\item $D$をあげるとより低い金利でもデフォルトするようになる
	\end{itemize}
\end{frame}

\begin{frame}\frametitle{モデルの帰結$2$}
	\begin{itemize}
		\item $\frac{\mathrm{d}\theta^*}{\mathrm{d}D}|_{LF} < 0$
		\item $D$をあげるとより低い流動性選好でもデフォルトするようになる
	\end{itemize}
\end{frame}

\begin{frame}\frametitle{モデルの帰結$3$}
	\begin{itemize}
		\item (39)
		\item 銀行はデフォルトのリスクと高いリターンを得ることのトレードオフを認識した上で最適な$D$を決定している
	\end{itemize}
\end{frame}

\begin{frame}\frametitle{PE}
	\begin{itemize}
		\item 以上のモデルでは$t=2$における資本財市場が競争的で$q_2$が所与であるとした
		\item しかし財市場の均衡も同時に考えることのできる社会型各社が銀行業を営むと$\frac{q_2^*}{D} \neq 0$である
		\item この時のマージナルソルベンシーを$\frac{\mathrm{d}R^*}{\mathrm{d}D}|_{SP}$で書く
		\item モデルの帰結は絶対値の意味で常に$\frac{\mathrm{d}R^*}{\mathrm{d}D}|_{SP} > \frac{\mathrm{d}R^*}{\mathrm{d}D}|_{LF}$となること
		\item すなわち銀行は上記のモデル内でマージナルソルベンシーを過小評価しており、これが$D$上昇の限界費用を過小評価することにつながっている
		\item 結果として$D$は過剰に大きくされデフォルトが頻発することになる
	\end{itemize}
\end{frame}

\begin{frame}\frametitle{結語}
	\begin{itemize}
		\item マクロ理論は金融危機のような頻度の低い現象を扱ってこなかったのは事実
		\item 主眼は景気変動や雇用、インフレだった
		\item 金融工学もシステミックリスクを扱ってこなかったのは事実
		\item 主眼は確率過程を外生的に与えた時のアセットプライシングだった
		\item 金融危機後、それ自体を研究対象とするムーブメントがあるからみんな乗り遅れないようにね!
	\end{itemize}
\end{frame}
\end{document}































