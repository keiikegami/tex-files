\documentclass[dvipdfmx, 12pt]{beamer}
\usepackage{pxjahyper}
\usepackage{minijs}
\usepackage{otf}
\renewcommand{\kanjifamilydefault}{\gtdefault}
\usetheme{Antibes}
\setbeamertemplate{navigation symbols}{}
\usepackage{url}
\usepackage{graphicx}
\usepackage{amsmath}
\usepackage{bm}
\usepackage{ascmac}


\title{Understanding Financial Crises\\Ch3 : Intermediation and Crises 前半}
\author{Kei Ikegami}

\begin{document}
\newcommand{\argmin}{\mathop{\rm arg~min}\limits}

\frame{\maketitle}

\section*{目次}
\begin{frame} \frametitle{発表の流れ}
\tableofcontents
\end{frame}

\section{introduction}
\begin{frame}\frametitle{なにするの?}
	\begin{itemize}
		\item 3章前半は2章でも用いたアイデアに基づいて銀行のモデル化をする。
		\item Bryant (1980), Diamond and Dybvig (1983)に依拠する。
	\end{itemize}
\end{frame}
\begin{frame}\frametitle{問題の所在:流動性問題}
	\begin{itemize}
		\item 覆水盆に返らず
	\end{itemize}
\end{frame}


\section{Agents}
\subsection{Market equilibirum}
\begin{frame}\frametitle{基本モデル}
	\begin{itemize}
		\item $t = 0,1,2$の3期間
		\item liquid asset (short)は今の財1単位を次期の財1単位に変換
		\item illiquid asset (long)は今の財1単位を2期後の財$R$単位に変換($R > 1$)
		\item agentは$t = 0$で所与の財1単位から上記二つのassetからポートフォリオ$(y, x)$を構成
		\item agentのタイプはearly($t = 1$で消費)とlate($t = 2$で消費)の二つで、$t = 1$に判明する
		\item earlyとlateは割合$(\lambda, 1 - \lambda)$で、これは個人が各タイプになる確率と等しい
	\end{itemize}
\end{frame}
\begin{frame}\frametitle{Autarky : 覆水盆に返らない}
	\begin{itemize}
		\item Ch2で扱った状況
		\item $(c_1, c_2) = (y, y + R(1 - y)), \text{where}\ 0 \leq y \leq 1$
	\end{itemize}
\end{frame}
\begin{frame}\frametitle{Market equilibrium : 覆水盆に返る}
	\begin{itemize}
		\item $t = 0$でt余分に買ってしまったlongやshortを売買できる市場が$t = 1$に開かれているケース
		\item 完全市場を想定(価格$p$でいくらでも売買できる)
		\item $c_1 = y + px$
		\item $c_2 = (x + \frac{y}{p})R$
		\item 均衡では$p = 1$なので$(c_1, c_2) = (1, R)$を得る。
	\end{itemize}
\end{frame}
\begin{frame}\frametitle{なんで$p = 1$?}
	\begin{description}
		\item[$p > 1$] $t = 0$でshort買うよりもlongを買って$t = 1$で売る方が儲かる。ゆえに誰もshortを買わず、$t = 1$の市場では誰もlongを欲しがらないため$p = 0$となり矛盾
		\item[$p < 1$] $t = 0$でshort買って$t = 1$で市場を通してlongを得た方が$t = 0$でlong買うよりもいい。なので$t = 0$で誰もlongを買わず、$t = 1$の市場でlongの供給が行われない。$p = R$ならばグロスのリターンが$1$となり売買が起こるが$R > 1$なので矛盾
	\end{description}
\end{frame}
\begin{frame}\frametitle{Autarky $\prec$ Market, しかし…}
	\begin{itemize}
		\item Fig 3.1.より、Marketが存在する時の消費計画はAutarkyの時よりも期待効用を大きくする(右上にあるので)
		\item しかしMarket equilibriumは"inefficient"
		\item ここで"inefficient"は、「各人のtypeも含め全ての情報を持つ社会計画者によって達成される期待効用の水準以下の水準しか達成できない」ことを意味
	\end{itemize}
\end{frame}

\subsection{Efficient solution}
\begin{frame}\frametitle{Social planner : \small 覆水しそうなやつは水少なめにしといたから}
	\begin{itemize}
		\item 今までとはモデルの解釈が違うので注意
		\item こいつにとってポートフォリオ$(x, y)$は社会全体で購入されるlongとshortの総量
		\item 1期の消費は$c_1$ずつ割合$\lambda$人が消費するので、消費総量は$\lambda c_1$
		\item 2期の消費は$c_2$ずつ割合$1-\lambda$人が消費するので、消費総量は$(1 - \lambda)c_2$
		\item 各期で消費量が財の供給量を上まらわないように$\lambda c_1 \leq y, (1 - \lambda)c_2 \leq Rx + (y - \lambda c_1)$
	\end{itemize}
\end{frame}
\begin{frame}\frametitle{Social planner : \small 覆水しそうなやつは水少なめにしといたから}
	目的関数は社会全体の期待効用の最大化(結果的に個人についての最適化と同じになる)。先の制約を合わせて社会計画者は以下の問題を解く。
	\begin{itembox}[l]{社会計画者の問題}
	\begin{align*}
		&\max_{c_1, c_2} \lambda U(c_1) + (1 - \lambda) U(c_2) \\
		&\text{s.t.}\
    		\begin{cases}
                		x + y = 1 \\
                		\lambda c_1 \leq y \\
                		\lambda c_1 + (1-\lambda)c_2 \leq Rx + y
    		\end{cases}
	\end{align*}
	\end{itembox}
\end{frame}
\begin{frame}\frametitle{Social planner : \small 覆水しそうなやつは水少なめにしといたから}
	\begin{itemize}
		\item 解は$(c_1, c_2) = (\frac{y}{\lambda}, \frac{R(1 - y)}{1 - \lambda})$
		\item お絵描きすると特殊な場合を除いてmarket equilibriumがこの問題の解とならないことがわかる(板書)
	\end{itemize}
\end{frame}
\begin{frame}\frametitle{Social planner : \small 覆水しそうなやつは水少なめにしといたから}
	\begin{itemize}
		\item $c_1 > 1$となる状況をliquidity insuranceと呼ぶ
		\item これは消費量の少ないearlyになるのが嫌すぎて、たとえearlyになってもちょっと多めに消費を行えるようにlateになった時の消費量を犠牲にすることを意味
		\item liquidity insuranceが発生する必要十分条件は$U^{'}(1) > R U^{'}(R)$
		\item 十分条件は$\eta(c) = -\frac{cU^{''}(c)}{U^{'}(c)} > 1\ \forall c$
	\end{itemize}
\end{frame}
\begin{frame}\frametitle{Contingent market : \small 器さえ持ってればあとで水注ぎ直すよ}
	\begin{itemize}
		\item 社会計画者が存在する時の効率的な資源配分は、$t = 0$で$t = 1$で判明したタイプごとに各期の消費量を保証してくれる権利を売買する市場が存在すれば市場に任せても実現できる
	\end{itemize}
\end{frame}
\begin{frame}\frametitle{Contingent market : \small 器さえ持ってればあとで水注ぎ直すよ}
	\begin{itemize}
		\item 確率$\lambda$で1期に$c_1$だけ欲しがる
		\item ということは1期に欲しがる期待消費は$\lambda c_1$
		\item 1期に1単位の消費を保証する権利の$t = 0$における市場での価格は$q_1$
		\item 以上より$c_1$の$t = 0$における現在価値は$q_1 \lambda c_1$
	\end{itemize}
\end{frame}
\begin{frame}\frametitle{Contingent market : \small 器さえ持ってればあとで水注ぎ直すよ}
	\begin{itemize}
		\item 確率$(1 -\lambda)$で2期に$c_2$を欲しがる
		\item ということは2期に欲しがる期待消費は$(1 - \lambda) c_2$
		\item $t = 1$の市場ではlongを価格$P$で買える。これが2期には$R$となるので、1期から2期にかけてのグロスのリターンは$\frac{R}{P}$
		\item よって、$t = 1$における価値は$\frac{(1 - \lambda) c_2}{\frac{R}{P}} = \frac{P}{R} (1 - \lambda) c_2 = s (1 - \lambda) c_2$
		\item 2期に1単位の消費を保証する権利の$t = 0$における市場での価格は$q_2$
		\item 以上より$c_2$の$t = 0$における現在価値は$q_2 s(1 - \lambda) c_2$
	\end{itemize}
\end{frame}
\begin{frame}\frametitle{Contingent market : \small 器さえ持ってればあとで水注ぎ直すよ}
	\begin{itemize}
		\item $t = 0$においては所与の財1単位しか持ってない
		\item 以上より権利市場での買い物は$q_1 \lambda c_1 + q_2 s(1 - \lambda) c_2 \leq 1$を満たす
	\end{itemize}
\end{frame}
\begin{frame}\frametitle{Contingent market : \small 器さえ持ってればあとで水注ぎ直すよ}
	目的関数は個人の期待効用。先の制約の中で効用最大化を行うので、agentの問題は以下のようにかける。
	\begin{itembox}[l]{Contingent marketにおけるagentの問題}
	\begin{align*}
		&\max_{c_1, c_2} \lambda U(c_1) + (1 - \lambda) U(c_2) \\
		&\text{s.t.}\
    		q_1 \lambda c_1 + q_2 s(1 - \lambda) c_2 \leq 1
	\end{align*}
	\end{itembox}
\end{frame}
\begin{frame}\frametitle{Contingent market : \small 器さえ持ってればあとで水注ぎ直すよ}
	\begin{itemize}
		\item FOCは$\frac{U^{'}(c_1)}{U^{'}(c_2)} = \frac{q_1}{q_2 s}$
		\item $t = 0$でassetに投資したほうが儲かる、またはcontingent claimをしたほうが儲かるという状況が起こらないような条件を考えると、$q_1 = 1, q_2 s = \frac{1}{R}$
		\item したがってFOCは$U^{'}(c_1) = R U^{'}(c_2)$と書き直せる
		\item これは社会計画者のいる時と同じリスクシェアリングが起こることを意味
	\end{itemize}
\end{frame}


\section{Banks}
\begin{frame}\frametitle{銀行の役目}
	\begin{itemize}
		\item
	\end{itemize}
\end{frame}
\begin{frame}\frametitle{Incentive compativility}
	\begin{itemize}
		\item
	\end{itemize}
\end{frame}
\begin{frame}\frametitle{Bank solution}
	\begin{itemize}
		\item
	\end{itemize}
\end{frame}
\begin{frame}\frametitle{}
	\begin{itemize}
		\item
	\end{itemize}
\end{frame}

\end{document}






















