\documentclass[dvipdfmx, 12pt]{beamer}
\usepackage{pxjahyper}
\usepackage{minijs}
\usepackage{otf}
\renewcommand{\kanjifamilydefault}{\gtdefault}
\usetheme{Antibes}
\setbeamertemplate{navigation symbols}{}
\usepackage{url}
\usepackage{graphicx}
\usepackage{amsmath}
\usepackage{bm}
\usepackage{ascmac}
\setbeamertemplate{footline}[frame number] 


\title{Understanding Financial Crises\\Ch10 : Contagion 後半}
\author{Kei Ikegami}

\begin{document}
\newcommand{\argmin}{\mathop{\rm arg~min}\limits}

\frame{\maketitle}

\section*{目次}
\begin{frame} \frametitle{発表の流れ}
\tableofcontents
\end{frame}

\begin{frame}\frametitle{何をするか}
	\begin{itemize}
	\item 銀行間のつながりがincomplete networkでもfirst bestを達成することはできた。しかし世界全体での流動性需要量が想定される量よりも多くなる想定外の事象の下では、incomplete networkだとbankruptが地域を超えて連鎖するContagionが発生する。ということの確認。
	\item Contagionが全地域に拡大すると世界に存在する資産の価値が低下するのでやだ!ということの確認。
	\item 銀行のデータを使ってContagionが発生する規模をシミュレーションする研究の紹介。
	\end{itemize}
\end{frame}

\begin{frame}\frametitle{設定}
	
\end{frame}

\begin{frame}\frametitle{}
	\begin{itemize}
	\item 
	\end{itemize}
\end{frame}

\begin{frame}\frametitle{}
	\begin{itemize}
	\item 
	\end{itemize}
\end{frame}

\begin{frame}\frametitle{}
	\begin{itemize}
	\item 
	\end{itemize}
\end{frame}

\begin{frame}\frametitle{}
	\begin{itemize}
	\item 
	\end{itemize}
\end{frame}

\begin{frame}\frametitle{}
	\begin{itemize}
	\item 
	\end{itemize}
\end{frame}

\begin{frame}\frametitle{}
	\begin{itemize}
	\item 
	\end{itemize}
\end{frame}

\end{document}
