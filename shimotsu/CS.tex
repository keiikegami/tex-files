\documentclass[dvipdfmx, 12pt]{beamer}
\usepackage{pxjahyper}
\usepackage{minijs}
\usepackage{otf}
\renewcommand{\kanjifamilydefault}{\gtdefault}
\usetheme{Antibes}
\setbeamertemplate{navigation symbols}{}
\usepackage{url}
\usepackage{graphicx}
\usepackage{amsmath}
\usepackage{bm}
\usepackage{ascmac}
\setbeamertemplate{footline}[frame number] 
\usepackage{bm}

\title{Tasting for Asymmetric Information in Insurance Markets \\Chiappori and Salanie (2000)}
\author{池上 慧}

\begin{document}
\newcommand{\argmin}{\mathop{\rm arg~min}\limits}

\frame{\maketitle}

\section*{目次}
\begin{frame} \frametitle{発表の流れ}
\tableofcontents
\end{frame}

\section{Introduction}
\subsection{目的}
\begin{frame}\frametitle{モチベーション}
	\begin{itemize}
	\item 契約理論において理論やモデルの発展は著しい
	\item 逆選択をはじめとして理論の示唆する現象の実証研究が少ない
	\item 本論文では情報の非対称性が存在するかをチェックするシンプルな手法を提案する
	\item その手法でフランスの自動車保険市場において情報の非対称性が存在しないことが否定できないことを示した
	\end{itemize}
\end{frame}

\begin{frame}\frametitle{理論}
	\begin{itemize}
	\item 逆選択に関してはRothschild and Stiglitz (1976)が基本となるモデルを提示した
	\item 本論文が検証する理論で示唆される現象は「補償の大きな契約は事故を起こしやすい人に選ばれる」という現象
	\item 観測された変数に置いて同質な個人について「事故を起こす」と「補償の大きい保険を選ぶ」が正の相関を持つかをチェックする
	\item 観測された変数が選択や事故に及ぼす影響については考慮しない
	\end{itemize}
\end{frame}

\begin{frame}\frametitle{\small 補償の大きな契約は事故を起こしやすい人に選ばれる?}
この現象は以下のように様々な状況に対して頑健な現象
	\begin{itemize}
	\item 保険業者の価格付けモデルに依存しない結果
	\item 消費者の効用関数への仮定に依存しない結果(モラルハザードや複数要素での逆選択があってもこの結果は観測できる)
	\item 一定の仮定の下では個人で事故の発生確率と重大度が異なっていても観測されるはず
	\item dynamic adverse selectionでも同じ現象が存在するはず
	\end{itemize}
\end{frame}

\begin{frame}\frametitle{モラルハザードとの関係}
	\begin{itemize}
	\item モラルハザードは「補償の大きな契約をする人は事故を起越さないようにするインセンティブが減る」現象
	\item データには事故発生と補償の大きな保険の契約とが正の相関を持つとして現れるはず
	\item 逆選択と区別できないがここでは区別せずに相関を分析する
	\end{itemize}
\end{frame}

\begin{frame}\frametitle{なぜ単なる回帰ではダメか}
$TR_i = \alpha + \beta Accident_i + X_i^{'}\gamma + \epsilon_i$で回帰して$\beta$が正となるかを見る。のでは以下が問題となる。
	\begin{itemize}
	\item $Accident_i$が事故発生のダミーであるので本来式に入れたい事故発生確率である$\pi_i$とは異なるためmeasurement errorが存在する。
	\item 補償を広げたために事故を起こしやすくなるという方向の因果関係(モラルハザード)の存在が想定されるため$\beta$に正のバイアスがかかる。
	\end{itemize}
\end{frame}

\subsection{先行研究}
\begin{frame}\frametitle{Puelz and Snow (1994)}
	\begin{itemize}
	\item ジョージア州の自動車保険市場を分析
	\item 逆選択の存在を示唆する結果
	\end{itemize}
\end{frame}

\begin{frame}\frametitle{Puelz and Snow (1994)の欠点}
	\begin{itemize}
	\item \textcolor{red}{Measurement error}
	
	deductibleを決定する要素である事故確率に実際事故が起きたかどうかのダミー変数を用いている。
	\item \textcolor{red}{Omitted variable bias}
	
	データとして用いている変数が20個しかない。
	\item \textcolor{red}{Linear function}
	
	deductibleの選択が線形関数で表現できるというのはかなりきつい制約。実際CARA関数を仮定すると非線形な関数でdeductibleが決定される。
	\end{itemize}
\end{frame}

\begin{frame}\frametitle{Puelz and Snow (1994)の欠点}
	\begin{itemize}
	\item \textcolor{red}{Heterogeneity}
	
	年齢と走行履歴がもたらす異質性について何の処理もしていない。これらの要素はdeductibleの決定に際して分散不均一をもたらす。特に走行履歴は保険業者が重視するデータであり、これをモデルに組み入れないのはOVBを招く決定的な要因となる。ただし、走行履歴は内生的な変数なのでその組み入れ方には注意がひつようである。
	\end{itemize}
\end{frame}

\begin{frame}\frametitle{対処方法}
	\begin{itemize}
	\item 理想的にはパネルデータを使う。(Chiappori and Heckman 1999)
	\item 本研究では初心者についてのデータに限定することで先の異質性を排除した。
	\item 用いる変数も55個と増やし、関数形によらない推定手法で検証を行った。
	\end{itemize}
\end{frame}


\section{Implementation}
\subsection{状況とデータ}
\begin{frame}\frametitle{自動車保険について}
\centering
        \includegraphics[width = 12cm]{ins.png}
\end{frame}

\begin{frame}\frametitle{フランスの自動車保険:二種類の保険}
以下の二つがdeductibleの差、premiumの差によって差別化されている。

	\begin{description}
	\item[RC] 日本でいう自賠責保険。法律により強制加入させられる最低限度の保険。
	\item[TR] 日本でいう任意保険。deductibleが増額するが、事故を起こした時により広い範囲で補償が受けられる。
	\end{description}
\end{frame}

\begin{frame}\frametitle{フランスの自動車保険:bonus/malus}
	\begin{itemize}
	\item premiumは「走行履歴によらない要素から決定される値」×「走行履歴に依存する係数」で決定される。
	\item 後者をbonus係数と呼ぶ。
	\item ある年に事故を起こすと25%増し、無事故なら5%減少、のように変化していく。
	\end{itemize}
\end{frame}

\begin{frame}\frametitle{データ:概要}
	\begin{itemize}
	\item フランス自動車保険市場のシェア7割を構成する21企業が加盟するFFSAが1990年に保険加入者に対して行った調査のデータを使う。
	\item 112万契約について41変数、12万件の事故について25変数が集められた。
	\item 先に述べたように初心者のデータに絞ると、1986年から1988年の間に免許を所得した20716人についてのデータとなる。
	\end{itemize}
\end{frame}

\begin{frame}\frametitle{データ:ex post moral hazard}
	\begin{itemize}
	\item 事故が発生した時、保険会社に連絡するかは個人の意思決定により決まる。
	\item 保険適用外の事故については報告しないはず。(これをex post moral hazardと呼ぶ)
	\item つまり得られたデータは単なる事故のデータではなく、内生的に決められた保険内容に従って報告すると判断されたもののみが現れているバイアスのかかったデータである。
	\item 本論文では2者が絡む事故のみに分析を絞ることで事故発生と保険会社への連絡がほぼ同義となっているデータを用いることでこの問題を回避した。
	\end{itemize}
\end{frame}

\begin{frame}\frametitle{(cf) ex post moral hazard}
	\begin{itemize}
	\item 逆選択も事前のモラルハザードも存在しない時、$TR_i$と$Accident_i$は条件付き独立である。
	\item しかし補償の範囲が大きい契約をしている運転者は、そうでない運転者よりも多様な事故について保険業者に報告する。
	\item すなわち$Claim_i$と$TR_i$は条件付き独立ではない。
	\item データ内では$Claim_i = Accident_i$なので、結果として逆選択も事前のモラルハザードも存在しなくても事故発生と補償範囲の広い保険の購入とが条件付き独立でなくなる。
	\end{itemize}
\end{frame}


\subsection{分析手法}
\begin{frame}\frametitle{3つの検定}
	\begin{itemize}
	\item Pair of Probits (parametric)
	\item Bivariate Probit (parametric)
	\item $\chi^2$ test (non parametric)
	\end{itemize}
\end{frame}

\begin{frame}\frametitle{Pair of Probits:ノーテーション}
	\begin{itemize}
	\item $y_i$はTRを購入したことのダミー変数
	\item $z_i$は契約者に非があると決定された事故を調査年に一度でも起こしたことのダミー変数
	\item $X_i$は外生変数
	\item $\epsilon_i, \eta_i$は独立な分散$1$平均$0$の正規分布に従う確率変数
	\end{itemize}
\end{frame}

\begin{frame}\frametitle{Pair of Probits:モデル}
個人ごとに走行日数$w_i$で重み付けをする。以下のプロビットで$\beta, \gamma$を推定する。
	\begin{align*}
	\begin{cases}
	y_i = {\bm 1}(X_i \beta + \epsilon_i > 0)\\[8pt]
	z_i = {\bm 1}(X_i \gamma + \eta_i > 0)
	\end{cases}
	\end{align*}
\end{frame}

\begin{frame}\frametitle{Pair of Probit:検定量}
推定した$\beta, \gamma$を用いて誤差項の推定値が以下のように得られる。
\begin{align*}
\begin{cases}
	\hat{\epsilon_i} = \frac{\phi(X_i \hat{\beta})}{\Phi(X_i \hat{\beta})} y_i - (1 - y_i) \frac{\phi(X_i \hat{\beta})}{\Phi(-X_i \hat{\beta})}\\[8pt]
	\hat{\eta_i} = \frac{\phi(X_i \hat{\gamma})}{\Phi(X_i \hat{\gamma})} z_i - (1 - z_i) \frac{\phi(X_i \hat{\gamma})}{\Phi(-X_i \hat{\gamma})}
\end{cases}
\end{align*}
これより帰無仮説$cov(\epsilon_i, \eta_i) = 0$の下で漸近的に$\chi^2(1)$に従う検定量$W = \frac{\left( \sum_{i} w_i \hat{\epsilon_i} \hat{\eta_i} \right)^2}{\sum_{i} w_i^2\hat{\epsilon_i^2}\hat{\eta_i^2}}$が得られる。
\end{frame}

\begin{frame}\frametitle{Bivariate Probit}
	\begin{itemize}
	\item プロビットを独立に行うことは帰無仮説の下ではefficientだが、そうでない時はinefficientである。
	\item $\epsilon_i, \eta_i$がそれぞれ$N(0,1)$に従いながら、相関係数$\rho$を持つとしてbivariate probitを行い、$\rho = 0$を帰無仮説として検定を行う。
	\end{itemize}
\end{frame}

\begin{frame}\frametitle{$\chi^2$ test}
	\begin{itemize}
	\item $m$個のbinaryな外生変数の実現値の全組み合わせは$2^m$組存在する。その一つ一つをcellと呼ぶ。
	\item cellに個人を割り当て、cellごとに$y_i, z_i$の実現値の組み合わせを用いて独立性の$\chi^2$検定量を計算する。
	\item これにより帰無仮説の下で漸近的に$\chi^2(1)$に従う統計量が$2^m$個得られる。
	\item 経験分布が$\chi^2(1)$であるかをKolmogorov-Smirnov検定を用いて検定する。
	\end{itemize}
\end{frame}

\begin{frame}\frametitle{$\chi^2$ test supplement}
	\begin{itemize}
	\item Kolmogorov-Smirnov検定はpowerが弱いことが知られている。
	\item これを補うために追加で二つの検定も行う。
	\item 一つ目は各cellで自由度1、有意水準5%の$\chi^2$検定を行う。帰無仮説の下では棄却された総数が$Bin(2^m, 0.05)$に従うので、これを用いて検定を行う。(Fischerのexact p-value?)
	\item 二つ目は$2^m$個のcellで得た検定量を足し合わせ、$\chi^2(2^m)$に従うとして仮説検定を行う。
	\end{itemize}
\end{frame}

\begin{frame}\frametitle{(cf) Kolmogorov-Smirnov test}
	\begin{itemize}
	\item 
	\end{itemize}
\end{frame}

\section{Results}
\begin{frame}\frametitle{Results on young drivers}
	\begin{itemize}
	\item いずれの検定でも独立であるという帰無仮説は棄却できなかった。
	\item しかし初心者は自身のタイプについてよくわからないまま契約を結ばざるを得ないことを考えるとこれは当然の結果とも言える。
	\end{itemize}
\end{frame}

\begin{frame}\frametitle{Results on senior drivers}
	\begin{itemize}
	\item 保険業者と運転者の間で情報の非対称性が拡大するのは、運転者がギリギリのところで事故を起こさないままずっと運転をし続けるケースである。
	\item この時運転者は自身が運転下手であることを知りながら保険業者はそのことについて何の情報も持たないという状況が生まれるからである。
	\item 1986年に免許を所得した人で1989年にbonus/malusが100を超えていない人に対象(該当者は6802人)を絞ることで、事故を起こしていない経験者という情報の非対称性が比較的大きいはずの対象に絞った分析ができる。
	\item ここでも情報の非対称性の存在がないことを否定する結果は得られなかった。
	\end{itemize}
\end{frame}

\begin{frame}\frametitle{Moral hazardに関して}
	\begin{itemize}
	\item 若い運転者は自動車を親からの借り物として登録することで、bonus/malusを最大50%オフの状態から始めることができる。
	\item 全体の$\frac{2}{3}$以上がこの恩恵を被っていた。
	\item premiumが安くなった時より広い補償の保険を買いやすくなることが予想され、実際データでもその傾向が観測される。
	\item この恩恵を受けていることと事故発生に正の相関が観測されればモラルハザードの存在が示唆されることになる。
	\end{itemize}
\end{frame}

\begin{frame}\frametitle{Moral hazardに関して}
	\begin{itemize}
	\item この恩恵を受けたか否かのダミー変数を追加してプロビットを再度行う。
	\item 結果は負の値で有意であった。
	\item よってモラルハザードについても存在は示唆されなかった。
	\end{itemize}
\end{frame}

\section{Conclusion}
\begin{frame}\frametitle{教訓}
	\begin{itemize}
	\item サンプルを同質と見るための努力をしよう。線形モデルが使えるのは同質なサンプルについてのみ。
	\item ネガティブな結果は頑健。
	\end{itemize}
\end{frame}

\begin{frame}\frametitle{結果について}
	\begin{itemize}
	\item 情報の非対称性がないというのは市場の意見と合致する。
	\item モラルハザードについても見つけられなかった。
	\item しかし情報の非対称性は別の形で存在するかもしれない(ex : cherry picking)
	\end{itemize}
\end{frame}

\begin{frame}\frametitle{Cherry Pickingで示唆されること}
	\begin{itemize}
	\item 
	\end{itemize}
\end{frame}

\begin{frame}\frametitle{Future work}
	\begin{itemize}
	\item Cherry Pickingの検証
	\item 学習を組み込む
	\item モラルハザードと逆選択の識別
	\item 他の市場における実証
	\end{itemize}
\end{frame}


\end{document}



















