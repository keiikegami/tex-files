\documentclass[dvipdfmx, 12pt]{beamer}
\usepackage{pxjahyper}
\usepackage{minijs}
\usepackage{otf}
\renewcommand{\kanjifamilydefault}{\gtdefault}
\usetheme{Antibes}
\setbeamertemplate{navigation symbols}{}
\usepackage{url}
\usepackage{graphicx}
\usepackage{amsmath}
\usepackage{bm}
\usepackage{ascmac}
\setbeamertemplate{footline}[frame number] 

\title{Tasting for Asymmetric Information in Insurance Markets \\Chiappori and Salanie (2000)}
\author{池上 慧}

\begin{document}
\newcommand{\argmin}{\mathop{\rm arg~min}\limits}

\frame{\maketitle}

\section*{目次}
\begin{frame} \frametitle{発表の流れ}
\tableofcontents
\end{frame}

\section{Introduction}
\subsection{目的}
\begin{frame}\frametitle{モチベーション}
	\begin{itemize}
	\item 契約理論において理論やモデルの発展は著しい
	\item 逆選択をはじめとして理論の示唆する現象の実証研究が少ない
	\item 本論文では情報の非対称性が存在するかをチェックするシンプルな手法を提案する
	\item その手法でフランスの車保険市場において情報の非対称性が存在しないことが否定できないことを示した
	\end{itemize}
\end{frame}

\begin{frame}\frametitle{理論}
	\begin{itemize}
	\item 逆選択に関してはRothschild and Stiglitz (1976)が基本となるモデルを提示した
	\item 本論文が検証する理論で示唆される現象は「補償の大きな契約は事故を起こしやすい人に選ばれる」という現象
	\item 観測された変数に置いて同質な個人について「事故を起こす」と「補償の大きい保険を選ぶ」が正の相関を持つかをチェックする
	\item 観測された変数が選択や事故に及ぼす影響については考慮しない
	\end{itemize}
\end{frame}

\begin{frame}\frametitle{\small 補償の大きな契約は事故を起こしやすい人に選ばれる?}
この現象は以下のように様々な状況に対して頑健な現象
	\begin{itemize}
	\item 保険業者の価格付けモデルに依存しない結果
	\item 消費者の効用関数への仮定に依存しない結果(モラルハザードや複数要素での逆選択があってもこの結果は観測できる)
	\item 一定の仮定の下では個人で事故の発生確率と重大度が異なっていても観測されるはず
	\item dynamic adverse selectionでも同じ現象が存在するはず
	\end{itemize}
\end{frame}

\begin{frame}\frametitle{モラルハザードとの関係}
	\begin{itemize}
	\item モラルハザードは「補償の大きな契約をする人は事故を起越さないようにするインセンティブが減る」現象
	\item データには事故発生と補償の大きな保険の契約とが正の相関を持つとして現れるはず
	\item 逆選択と区別できないがここでは区別せずに相関を分析する
	\end{itemize}
\end{frame}

\begin{frame}\frametitle{なぜ単なる回帰ではダメか}
	\begin{itemize}
	\item 
	\end{itemize}
\end{frame}

\subsection{先行研究}
\begin{frame}\frametitle{Puelz and Snow (1994)}
	\begin{itemize}
	\item ジョージア州の車保険市場を分析
	\item 逆選択の存在を示唆する結果
	\end{itemize}
\end{frame}

\begin{frame}\frametitle{Puelz and Snow (1994)の欠点}
	\begin{itemize}
	\item \textcolor{red}{Measurement error}
	
	deductibleを決定する要素である事故確率に実際事故が起きたかどうかのダミー変数を用いている。
	\item \textcolor{red}{Omitted variable bias}
	
	データとして用いている変数が20個しかない。
	\item \textcolor{red}{Linear function}
	
	deductibleの選択が線形関数で表現できるというのはかなりきつい制約。実際CARA関数を仮定すると非線形な関数でdeductibleが決定される。
	\end{itemize}
\end{frame}

\begin{frame}\frametitle{Puelz and Snow (1994)の欠点}
	\begin{itemize}
	\item \textcolor{red}{Heterogeneity}
	
	年齢と走行履歴がもたらす異質性について何の処理もしていない。これらの要素はdeductibleの決定に際して分散不均一をもたらす。特に走行履歴は保険業者が重視するデータであり、これをモデルに組み入れないのはOVBを招く決定的な要因となる。ただし、走行履歴は内生的な変数なのでその組み入れ方には注意がひつようである。
	\end{itemize}
\end{frame}

\begin{frame}\frametitle{対処方法}
	\begin{itemize}
	\item 理想的にはパネルデータを使う。(Chiappori and Heckman 1999)
	\item 本研究では初心者についてのデータに限定することで先の異質性を排除した。
	\item 用いる変数も55個と増やし、関数形によらない推定手法で検証を行った。
	\end{itemize}
\end{frame}


\section{Implementation}
\subsection{状況とデータ}
\begin{frame}\frametitle{}
	\begin{itemize}
	\item 
	\end{itemize}
\end{frame}

\begin{frame}\frametitle{}
	\begin{itemize}
	\item 
	\end{itemize}
\end{frame}

\begin{frame}\frametitle{}
	\begin{itemize}
	\item 
	\end{itemize}
\end{frame}

\subsection{分析手法}
\begin{frame}\frametitle{}
	\begin{itemize}
	\item 
	\end{itemize}
\end{frame}

\begin{frame}\frametitle{}
	\begin{itemize}
	\item 
	\end{itemize}
\end{frame}

\begin{frame}\frametitle{}
	\begin{itemize}
	\item 
	\end{itemize}
\end{frame}

\begin{frame}\frametitle{}
	\begin{itemize}
	\item 
	\end{itemize}
\end{frame}


\section{Results}
\subsection{Young drivers}
\begin{frame}\frametitle{}
	\begin{itemize}
	\item 
	\end{itemize}
\end{frame}

\subsection{Senior drivers}
\begin{frame}\frametitle{}
	\begin{itemize}
	\item 
	\end{itemize}
\end{frame}

\subsection{Moral hazard}
\begin{frame}\frametitle{}
	\begin{itemize}
	\item 
	\end{itemize}
\end{frame}


\section{Conclusion}
\begin{frame}\frametitle{}
	\begin{itemize}
	\item 
	\end{itemize}
\end{frame}

\begin{frame}\frametitle{}
	\begin{itemize}
	\item 
	\end{itemize}
\end{frame}

\begin{frame}\frametitle{}
	\begin{itemize}
	\item 
	\end{itemize}
\end{frame}


\end{document}



















