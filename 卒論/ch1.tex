
\section{Introduction}
データの拡充と推定手法の開発、そしてそれを可能にする計算環境の実現に伴い戦略的状況を扱った実証分析は近年大きく進展してきた。例えば\cite{Publishing2010}で行われた寡占市場の分析や\cite{Bresnahan1991},\cite{Tamer2003a},\cite{Seim2006}の参入の分析、\cite{Brock2001a}による社会的相互作用モデルを用いたpeer effectの分析などが静学ゲームの実証研究として行われており、また動学ゲームにおいても、\cite{Pakes1994}で開発された参入退出ゲームと\cite{Aguirregabiria2007}や\cite{Bajari2007}で開発された推定手法を用いて\cite{Exler2013}や\cite{Ryan2012}などで動学的な要素を考慮した戦略的意思決定の実証が行われている。

こういった戦略的状況の分析は、現実のデータがある種の均衡状態から生み出されたものであるという仮定を基盤とする。静学ゲームでは完備情報下ではナッシュ均衡、不完備情報下であればベイジアンナッシュ均衡がプレイされているとしてモデルのパラメータを推定する。動学ゲームにおいても同様にマルコフ完全均衡がプレイされているという仮定の下でモデルのパラメータを推定する。

しかし現実にはモデルの示唆する均衡が必ずしもプレイされるわけではないということが実験を通して明らかにされてきた。ナッシュ均衡の各種精緻化概念も現実をよりよく説明するために生み出されてきたものであり、また\cite{Economics2014}が提案した認知階層モデルなどの限定合理性モデルも既存の均衡では予測できない実験データを説明するために作られたモデルである。従来、こういったモデルの正当性は実験データをどれだけ説明できるかという基準で検証されるが、\cite{Haile2016}が指摘したようにデータへのフィットという観点でモデルの妥当性を図ることには正当性がない場合も存在する。

モデルや均衡概念の特定化の検証を現実のデータを用いて行った研究はわずかであり、それも極めて限定的な状況を利用するものが多い。\cite{Chiappori2002}はサッカーのペナルティキックにおける混合戦略ナッシュ均衡の実証を行い、\cite{Ostling2011}はスウェーデンで実際に行われたゲームを用いて実験的状況下にないデータから均衡概念のチェックを行っている。こういった研究において扱われる事象は極めて限定的かつ特殊なものであり、社会で広く観察される現象に対する均衡概念の検証はほとんど行われていない。この困難は研究者には観測できない変数の存在や異質性の問題、そしてそもそも利得が観測できずそれ自体が推定する対象であるパラメータであるという事実などによる。

本研究では広く観察される市場参入という現象に対して、純粋戦略ナッシュ均衡のプレイを仮説検定によって検証する手法を提案する。モデルは\cite{Bresnahan1991}で提案された完備情報参入ゲームを用いる。本論は以下のように進む。2章では関連する研究として\cite{Ostling2011}と\cite{Chiappori2002}をレビューし、均衡に対する検証がなぜ困難かを整理する。3章ではモデルを説明して均衡を計算する。4章では主要な結果であるパラメータの推定手法と検定方法を提案する。5章では結語としてより包括的な枠組みの可能性を述べる。
