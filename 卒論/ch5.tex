
\section{Monte Carlo Simulation}
サンプルデータを用いて上記の推定を行い、ハウスマン検定統計量を計算する。サンプルデータは説明変数として市場ごとの人口、企業ごとの利益を左右する変数として工場からの距離をそれぞれの企業について用意した。すなわち${\bf x}_{m, i}^{'} = (pop_m, dist_{m, i})$である。$3$つの変数は独立な標準正規分布からのサンプルである。単純化のために企業間で係数は等しいとして、$\beta = (\beta_1, \beta_2)^{'}$とする。

真のパラメータの値を${\theta^0}^{'} = ({\beta^0}^{'}, \delta) = (0.8, -0.5, -0.25)$とし、$w = 1$の時のサンプルデータを$30000$市場について$1000$期作成した。$M$市場のみのデータセットに関しては各市場の経験分布から実現した参入企業数を得た。このサンプルデータを用いて上記の推定を行い各データセットにおいてハウスマン検定統計量を計算した結果が表$3$である。

どちらのデータセットにおいてもRobust推定量は真の値に収束している一方で、PN推定量は誤った推定結果を出力していることが見て取れる。またハウスマン検定統計量を見てもどちらのデータセットにおいても純粋戦略ナッシュ均衡がプレイされるという帰無仮説は正しく棄却されている。

\begin{table}[t]
\begin{minipage}{\textwidth}
\centering
\def\sym#1{\ifmmode^{#1}\else\(^{#1}\)\fi}
\caption{推定結果とハウスマン検定統計量}
\begin{tabular}{l*{4}{c}}
\hline\hline
            &\multicolumn{1}{c}{(1)}&\multicolumn{1}{c}{(2)}&\multicolumn{1}{c}{(3)}&\multicolumn{1}{c}{(4)}\\
\hline
$\beta_1$&        0.755         &       0.828        &      0.737  &     0.840  \\
[1em]
$\beta_2$      &        -0.450          &       -0.509        &       -0.443        &       -0.519         \\
[1em]
$\delta$&        -0.160         &       -0.287        &       -0.092         &       -0.268        \\
\hline
H      &        \multicolumn{2}{c}{1.676e+12}       &         \multicolumn{2}{c}{1.345e+10}        \\
\hline\hline\\
\end{tabular}
\footnotetext[1]{$(1)$は$M$市場$T$期のデータを用いたPN推定量、$(2)$は$M$市場$T$期のデータを用いたRobust推定量、$(3)$は$M$市場のデータを用いたPN推定量、$(4)$は$M$市場のデータを用いたRobust推定量である。}
\footnotetext[2]{$H$はハウスマン検定統計量であり、帰無仮説の下で自由度$2$の$\chi^2$分布に従う。}
\footnotetext[3]{サンプルはRobust推定量の仮定を満たすものを用いた。}
\end{minipage}
\end{table}

