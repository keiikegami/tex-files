
\chapter{結論}
本論文では広く用いられる完備情報参入ゲームのモデルを用いて実験環境外のデータにおいて純粋戦略ナッシュ均衡がプレイされているかを統計的に検定できる簡便な手法を提案した。先行研究ではスポーツやゲームなどの限定的な状況でのみ均衡に関する実証研究が行われていたが、参入ゲームというより広く観察でき、複雑な利得構造を持つ現象についてもプレイされる均衡を統計的にテストできる枠組みを提示したことが本論文の主要な貢献である。これはまた、ゲーム理論の文脈では不安定な均衡として排除される混合戦略ナッシュ均衡の存在を実データを用いて検証することができるため、均衡の精緻化という観点でも有益な示唆を与える。さらに、混合戦略の存在を許しても部分識別の手法を用いることなく通常の推定法でモデルのパラメータを推定できる手法を提案したことも大きな貢献である。

\section{論点}
本論文では純粋戦略ナッシュ均衡の下で一致性を持つ推定量として\cite{Bresnahan1991}で提案された最尤推定量を用いた。しかし、本論でも述べた通り\cite{Tamer2003a}で提案されたセミパラメトリック最尤推定量の方が効率的な推定量である\footnote{\cite{Tamer2003a}で提案された推定量は$\largestar$における複数均衡を利用するものである。外生的な変数を$x$として、$H(x) = {\rm Pr}[(0,1)\mid x]$とする。この関数のノンパラメトリック推定値を$\hat{H(x)}$と書くと、$\sum_{m = 1}^{M} y_{m1}y_{m2} {\rm log}\ P_1(x_m, b) + (1-y_{m1})(1-y_{m2}) {\rm log}\ P_2(x_m, b) + (1-y_{m1})y_{m2} {\rm log}\ \hat{H(x_m, b)} + y_{m1}(1-y_{m2}) {\rm log}\ (1-P_1(x_m, b)-P_2(x_m, b) - \hat{H(x_m, b)}$が尤度である。この時$P_1(x_m, b)$はどちらも参入しない確率、$P_2(x_m, b)$はどちらも参入する確率を表し、$y_{m, i}$は市場$m$で企業$i$が参入していることのダミー変数である。これを最大化する$b$がセミパラメトリック最尤推定量である。ただし$\hat{H(\cdot)}$は規定の範囲に収まるように適切に処理する必要があることに注意。詳細は\cite{Tamer2003a}を参照。}。この推定量も純粋戦略ナッシュ均衡の仮定の下で一致性を持つ推定量であるため、上記のPN推定量としてこちらを用いることもできる。ここではサンプルデータとして$w = 1$のものを用いたため\cite{Bresnahan1991}の推定量でも十分に仮説を棄却できたためそちらのみを報告したが、$w$が不確実な現実にはより効率的な\cite{Tamer2003a}の推定量を用いた検定を行うことが望ましいであろう。この場合であっても上記に従って同様の検定量を作ることができる。

また、$M$市場について1回のみの参入結果を用いてRobust推定量を出す際には1段階目の参入社数の予測の精度が重要である。これは\cite{Seim2006}などで用いられる2段階推定に多く共通する事項である。このように予測結果のみが重要な場合においては近年発達した機械学習の手法を用いることができることが\cite{Mullainathan2017}や\cite{Igami2017}で述べられている。本研究もそれに従ってより柔軟な手法による予測の手法を用いることで$M$市場について1回しかデータが観測されていない場合についてもより正確な推定結果を得ることが可能になると想定される。

\section{さらなる研究}
さらなる研究が必要な部分としてまず、\cite{Ciliberto2009a}のように実際に完備情報ゲームとして寡占市場を分析した文献ではプレイヤーが$2$社以上のケースが扱われることに注意が必要である。\cite{Ciliberto2009a}ではアメリカにおける航空輸送産業が例として扱われているが、日本においてもコンビニエンスストアや低価格飲食チェーン店などの商業立地においては$2$社による競争よりも複数企業間での競争が多く観察され、またそれにより企業ごとに異なる競争効果を細かく分析することは当該産業における立地などの規制に関する政策評価を行う際に重要な要素となる。従ってプレイヤーが$2$社以上存在する市場に対しても適用できるように拡張する必要がある。

本論文においてはナッシュ均衡という構造を前提とし、さらに情報構造についても完備情報であるという仮定の下で、均衡の種類の特定化についての検定を提示した。混合戦略が現実にプレイされているかというゲーム理論の実証研究という観点でこれは重要なものであるが、実証産業組織論においてより有益な分析を行っていくためには、より基幹となる仮定であるナッシュ均衡という構造自体や情報構造についての検証を行っていくことも重要である。実際、\cite{Grieco2014a}や\cite{Magnolfi2015}などでは実際のスーパーマーケットの立地データを用いて情報構造についての検定が行われており、より柔軟な情報構造に頑健な均衡概念を用いた分析も行われている。また、ナッシュ均衡という構造自体に対して仮定を緩めることを試みる研究も存在する。例えば\cite{Aradillas-Lopez2008}では完備情報ゲームにおいて合理性の仮定を緩めた際の識別集合を様々なモデルについて報告している。さらに\cite{Kashaev2017},\cite{Kashaev2016}では様々な均衡概念を実データを用いて検定する手法が開発されている。これらの均衡概念の検証も未だ静学ゲームが主であり、動学参入退出のモデルや価格付けのモデルなど広いクラスの問題を扱うことのできる検定手法の開発が今後望まれる研究の方向性である。
