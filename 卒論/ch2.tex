
\chapter{先行研究}

\cite{Chiappori2002}はゲーム理論を実験環境外のデータを用いて検証した稀有な研究である。具体的にはサッカーのプロリーグにおけるペナルティキックでキッカーとキーパーのとった行動データを用いて、同時手番としてプレイされているかの検証と混合戦略がプレイされているかの検証を行った。手番の検証は自身の行動を相手の行動に回帰した時の係数の有意性により判断し、混合戦略に関しては実際の得失点を特定の行動に回帰した時の係数の有意性により判断している。結果としては同時手番という仮説は棄却されず、混合戦略をプレイしているという仮説も棄却されなかったため、実験環境外の戦略的状況下で混合戦略ナッシュ均衡がプレイされるということの証拠となった。しかしペナルティキックという限定的な事象に対する実証であることや、ナッシュ均衡が混合戦略でただ一つ存在する単純なゼロサムゲームについての分析であるということから、現実に観察される多様な経済現象を分析できる枠組みではないという弱みを持つ。本論文で扱う参入ゲームは幅広い産業分野で実証が重ねられてきた領域であり、利得構造もゼロサムではないより広範な利得構造を表現できるという点で\cite{Chiappori2002}から先に進んだものであると言える。

\cite{Ostling2011}はスウェーデンで実際に行われたゲームでの行動データからナッシュ均衡や限定合理性モデルの予測能力を検証したものである。実施されたのはLUPIと呼ばれるゲームであり、参加者は$1$から$99999$の間から好きな整数を選び、ただ1人にのみ選ばれた数字のうちで最も小さい数字を選んだプレイヤーが勝利者となり賞金を受け取るというゲームであった。筆者はこれを\cite{Myerson1998},\cite{Myerson2000}に従ってポワソンゲームとして定式化し、ポワソンゲームのナッシュ均衡では実際の行動データが説明できないこと、そして限定合理性のモデルの代表例である認知階層モデルでは実データの分散をよく説明できることを示した。しかし先にも述べたようにデータへのフィットの良さを裏付けとしてナッシュ均衡をはじめとしたモデルや構造を正当化することには懸念がある。ポワソンゲームにおける均衡は計算で求めることができるため、ナッシュ均衡とのフィットは一定の妥当性を持つことが想定されるが、認知階層モデルやlevel-kといったモデルはその予測がデータと良くフィットするように、パラメータの値や階層の分布、level-0の割合などが様々に特定化されている。これらを考慮した時、ある種のフィットの良さ以外にモデルや特定化のテストをする枠組みがゲームの構造に対しても必要であると考える。

\cite{Ostling2011}で引用されているRobert Aumannの言葉にある通り、戦略的な状況を記述し分析するためにはそのゲームが厳格なルールを持っている必要がある。特に均衡概念の特定化を検証するという目的のためにはこれが重要であり、コントロールされた利得を用いて均衡での挙動を計算できる\cite{Ostling2011}のようなゲーム的状況や\cite{Chiappori2002}で扱われたスポーツ、データの集積も盛んで厳密なルールが参加者にも徹底されるオークションなどの限定的な状況しか検証に用いることができなかった。

しかし一般に実証研究の対象となる経済事象はそのような理想的な状況ではなく、情報構造や利得の構造、さらに手番に関しても研究者にとっては不確実なものである。以下で扱う参入ゲームもその例外ではなく、実証に用いられるモデルは様々に特定化されている。その特定化の1つとして純粋戦略ナッシュ均衡がプレイされているという仮定をここでは取り上げる。これは実現した結果についてプレイヤーが両者ともに別の行動をとればよかったと後悔することがないということを意味するが、支配戦略が存在しない場合においては現実には強すぎる仮定である。通常のゲーム理論の文脈では参入ゲームにおける混合戦略ナッシュ均衡は安定でない均衡であるために予測としては用いられないが、実データにおいても不安定な均衡が排除されているかを検証するという意味においても、純粋戦略ナッシュ均衡がプレイされるという仮定を検証する枠組みを提示することには意義があると言える。

また、\cite{Ciliberto2009a}で用いられた部分識別による推定量は混合戦略がプレイされることに対しても頑健な推定を行うことができるが、混合戦略を許すことによりその推定幅が拡大する。この時、帰無仮説を純粋戦略ナッシュ均衡がプレイされるとした仮説検定を行うことができ、また帰無仮説が棄却されなければ混合戦略を許さない場合の推定によって識別集合の幅を小さくすることができる。この意味でも本研究で提案される検定は有用であると言える。

