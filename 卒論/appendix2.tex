\chapter{主張$3$の証明}
\begin{align*}
	&\theta^R = \argmin_{\theta} \sum_{m = 1}^M r(\hat{p^m} ; \theta)^2\\[10pt]
	&\text{where}\ r(\hat{p^m} ; \theta) = (\hat{p_2^m} - \hat{p_1^m}) - (q_1^1 q_2^1 - q_1^0 q_2^0)
\end{align*}
について\cite{Tamer2003a}と同様の除外制約の下で点識別可能であることを示す。

プレイヤー$2$の利得のみ動かす変数が存在する時、${\bf x}_{m, 2}^{'} \beta_2\ \to \infty$となる領域が存在する。この時$-{\bf x}_{m, 2}^{'} \beta_2\ \to -\infty$であることから以下が成立する。
\begin{align*}
\begin{cases}
	q_1^1 q_2^1 - q_1^0 q_2^0\ \to \Phi({\bf x}_{m, 1}^{'} \beta_1 + \delta_1)\ \text{where}\ {\bf x}_{m, 2}^{'} \beta_2\ \to \infty\\[10pt]
	q_1^1 q_2^1 - q_1^0 q_2^0\ \to -\Phi({\bf x}_{m, 1}^{'} \beta_1)\ \text{where}\ {\bf x}_{m, 2}^{'} \beta_2\ \to -\infty
\end{cases}
\end{align*}
これはプレイヤー$2$の行動がプレイヤー$1$の行動によらずに定るケースであり、プレイヤー$1$についても同様に考えることができるので、$E\left[ {\bf x}_{m, i} {\bf x}_{m, i}^{'} \right]$が$i = \left\{ 1,2 \right\}$についてどちらも逆行列が存在するならば$\theta$は識別可能である。