\documentclass[dvipdfmx, 12pt]{beamer}
\usepackage{pxjahyper}
\usepackage{minijs}
\usepackage{otf}
\renewcommand{\kanjifamilydefault}{\gtdefault}
\usetheme{Dresden}
\setbeamertemplate{navigation symbols}{}
\usepackage{url}
\usepackage{graphicx}
\usepackage{comment}
\usepackage{amsmath}
\newenvironment<>{varblock}[2][.9\textwidth]{%
  \setlength{\textwidth}{#1}
  \begin{actionenv}#3%
    \def\insertblocktitle{#2}%
    \par%
    \usebeamertemplate{block begin}}
  {\par%
    \usebeamertemplate{block end}%
  \end{actionenv}}


\title{\small Bounded Rationality and Industrial Organization}
\subtitle{Chapter 11 2nd part}
\author{Kei Ikegami}
\institute{Graduate School of Economics, The University of Tokyo}

\begin{document}
\newcommand{\argmin}{\mathop{\rm arg~min}\limits}

\frame{\maketitle}

\section*{Index}
	\begin{frame}\frametitle{Index}
	\tableofcontents
	\end{frame}
	
\section{Proposition $11.3$}
\begin{frame}\frametitle{Statement}
\begin{block}{Proposition $11.3$}
	Let $\sigma$ be a symmetric Nash equilibrium strategy. Then
	\begin{enumerate}
		\item Firms earn the max-min payoff $\frac{1}{2} - c_{x^{*}}$
		\item For every $M \in S(\sigma), |M| =2 \Rightarrow b(M) = x^{*}$
		\item $\beta_{\sigma}(x^{*}) = 1 -2 c_{r^{*}}$
	\end{enumerate}
\end{block}
\end{frame}

\begin{frame}\frametitle{About $1$}
\alert{Firms earn the max-min payoff $\frac{1}{2} - c_{x^{*}}$}

	\begin{itemize}
		\item This payoff coincides with the rational consumer benchmark.
		\item The main reason for this is that "$M$ beats $M^{'}$" needs not only the sensational temptation but also switching the default.
		\item In other words, $\left\{ x^{*} \right\}$ is never beaten in this sense.
	\end{itemize}
\end{frame}

\begin{frame}\frametitle{Proof sketch}
	\begin{itemize}
	\item From lemma 11.1, $x^{*}$ beats no menu in $S(\sigma)$. And it is not beaten by any menu in $S(\sigma)$ because $x^{*}$ is utility maximizer.
	\item So menu $\left\{ x^{*} \right\}$ always gives a market share $\frac{1}{2}$. And the cost is $c_{x^{*}}$. Then the payoff is $\frac{1}{2} - c_{x^{*}}$
	\item Then the expected payoff of this strategy is also $\frac{1}{2} - c_{x^{*}}$.
	\end{itemize}
\end{frame}

\begin{frame}\frametitle{About $2$}
\alert{For every $M \in S(\sigma), |M| =2 \Rightarrow b(M) = x^{*}$}

	\begin{itemize}
		\item This means that pure attention grabbers are included in a menu only when the menu has $x^{*}$ in equilibrium.
		\item If there is such a menu $M$ in $S(\sigma)$, $\left\{ x^{*} \right\}$, which is also included in $S(\sigma)$, has an incentive to include the same pure attention grabber of $M$.
		\item Then $\sigma$ is not an equilibrium.
	\end{itemize}
\end{frame}

\begin{frame}\frametitle{Proof sketch}
	\begin{itemize}
		\item Show its contraposition
		\item The condition for including some pure attention grabber in $M$ results in the profitable deviation from $\left\{ x^{*} \right\}$ to $\left\{ x^{*}, r(M)\right\}$, where $r(M)$ denotes the pure attention grabber in $M$.
	\end{itemize}
\end{frame}

\begin{frame}\frametitle{About $3$}
	\alert{$\beta_{\sigma}(x^{*}) = 1 -2 c_{r^{*}}$}
	
	\begin{itemize}
		\item This means that the probability utility maximizer is offered is entirely determined by the cost of the best attention grabber.
		\item As the sensations become costly, the less likely the utility maximizer is offered.
		\item This is directly derived from the fact $\left\{ x^{*} \right\}$ and $\left\{ x^{*} , r^{*}\right\}$ are indifferent. And both of them are included in $S(\sigma)$
	\end{itemize}
\end{frame}

\begin{frame}\frametitle{Proof sketch}
	\begin{itemize}
		\item Show there is no incentive to deviate from $\left\{ x^{*} \right\}$ to $\left\{ x^{*} , r^{*}\right\}$.
		\item To make it rational we confirm that the pure strategy $\left\{ r^{*} \right\}$ gives the better payoff than $\sigma$ if $\left\{ x^{*} , r^{*}\right\}$ is out of $\sigma$.
		\item At first glance $\left\{ x^{*} , r^{*}\right\}$ has wasteful costly alternative $r^{*}$, but it actually works for the higher market share.
	\end{itemize}
\end{frame}	
	
\section{Effective Marketing Property}	
\begin{frame}\frametitle{Relaxed R}
\begin{itemize}
	\item Consider when R need not be neither complete nor transitive.
	\item We interpret this relation as the similarity between the items rather than sensation.
	\item "Pure attention grabbers are offered only in conjunction with $x^{*}$ (Prop 11.3 (2))" does not hold in this more general relation.
\end{itemize}
\end{frame}

\begin{frame}\frametitle{Effective Marketing Property : Statement}
\begin{block}{Effective Marketing Property}
	Suppose that a symmetric Nash equilibrium strategy $\sigma$ induces the max-min payoff $\frac{1}{2} - c_{x^{*}}$. Let $M, M^{'} \in S(\sigma)$ that satisfy the below condition, 
	\begin{enumerate}
		\item $b(M^{'}) \neq x^{*}$
		\item $xRb(M^{'})$ for some $x \in M$
		\item $b(M) \lnot R b(M^{'})$
	\end{enumerate}
	Then $M$ beats $M^{'}$.
\end{block}
\end{frame}

\begin{frame}\frametitle{Effective Marketing Property : in Words}
\begin{itemize}
	\item When each firm obtains rational consumer benchmark payoff in symmetric Nash equilibrium,
	\item the attracted consumers by pure attention grabbers always switch their choice,
	\item unless the default menu does not have the utility maximizer.
\end{itemize}
\end{frame}

\begin{frame}\frametitle{Proof sketch}
\end{frame}

\end{document}























	
