\documentclass[dvipdfmx, 12pt]{beamer}
\usepackage{pxjahyper}
\usepackage{minijs}
\usepackage{otf}
\renewcommand{\kanjifamilydefault}{\gtdefault}
\usetheme{Dresden}
\setbeamertemplate{navigation symbols}{}
\usepackage{url}
\usepackage{graphicx}
\usepackage{comment}
\usepackage{amsmath}
\newenvironment<>{varblock}[2][.9\textwidth]{%
  \setlength{\textwidth}{#1}
  \begin{actionenv}#3%
    \def\insertblocktitle{#2}%
    \par%
    \usebeamertemplate{block begin}}
  {\par%
    \usebeamertemplate{block end}%
  \end{actionenv}}


\title{LEVEL-k AUCTIONS: CAN A NONEQUILIBRIUM MODEL OF STRATEGIC THINKING EXPLAIN THE WINNER'S CURSE AND OVERBIDDING IN PRIVATE-VALUE AUCTIONS?}
\author{Kei Ikegami}

\begin{document}
\newcommand{\argmin}{\mathop{\rm arg~min}\limits}

\frame{\maketitle}

\section*{Index}
	\begin{frame}\frametitle{Index}
	\tableofcontents
	\end{frame}

\section{Introduction}
	\begin{frame}\frametitle{Purpose}
	\end{frame}
	
	\begin{frame}\frametitle{Precedents}
	\end{frame}
	
	\begin{frame}\frametitle{What's new}
	\end{frame}
	
	\begin{frame}\frametitle{Result}
	\end{frame}
	
\section{Models}
	\subsection{Overview}
		\begin{frame}\frametitle{General Model}
			\begin{itemize}
			\item $N$ bidders bid for a single object.
			\item $X_i$ is bidder $i$'s private signal. $X = (X_1, \dots, X_N)$.
			\item $S_j$ is additional random variable which is informative about the value of the object. $S = (S_1, \dots, S_M)$.
			\item $V_i = u_i(S, X)$ is bidder $i$'s value of the object, where $u_i$ is symmetric across $i$.
			\item $V_i - p$ is the payoff for the bidder $i$ winning the auction by paying $p$.
			\item $Y$ is the highest signal among bidders other than $i$.
			\item $\upsilon(x, y) = E\left[ V_i | X_i = x, Y = y \right]$ is the expected value conditional on winning.
			\item $r(x) = E\left[ V_i | X_i = x \right]$ is the unconditional expected value.
			\end{itemize}
		\end{frame}
		
		\begin{frame}\frametitle{Classification of Auctions}
			\begin{itemize}
			\item First price auction vs Second price auction
			\item Independent private value auction(i.p.v) vs Common value auction(c.v)
			\item In i.p.v, the signals and values are independent among bidders.
			\item In c.v, the information of $i$ and $j$ is not independent and learning about the other bidders' information can cause the bidder to reassess his estimate of the value of the object. (e.g. Timber auction)
			\end{itemize}
		\end{frame}
		
	\subsection{Equilibrium}
		\begin{frame}\frametitle{First Price Auction}
			\begin{itemize}
			\item In c.v, the optimal bidding strategy is calculated as (1).
			\item In i.p.v, the optimal bidding strategy is calculated as (2), because $\upsilon(x,x) = x$ and $Y$ is independent of $X$.
			\end{itemize}
			
			\begin{align}
				a_*(x) &= \upsilon(x, x) - \int_{\underline{x}}^{x} \exp \left( -\int_{y}^{x} \frac{f_Y(t|t)}{F_Y(t|t)} \mathrm{d}t \right) \mathrm{d} (\upsilon(y,y)) \\[10pt]
				a_*(x) &= x - \int_{\underline{x}}^{x} \frac{F_Y(y)}{F_Y(x)} \mathrm{d}y
				= E\left[ Y | Y < X \right]
			\end{align}			
		\end{frame}
		
		\begin{frame}\frametitle{Second Price Auction}
			\begin{itemize}
			\item In c.v, the optimal bidding strategy is calculated as (3). This is not a weakly dominant strategy.
			\item In i.p.v, the optimal bidding strategy is calculated as (4), because $\upsilon(x,x) = x$. This is a weakly dominant strategy.
			\end{itemize}
			
			\begin{align}
				b_*(x) &= \upsilon(x,x) \\
				b_*(x) &= x
			\end{align}
			
		\end{frame}
		
	\subsection{Cursed Equilibrium}
		\begin{frame}[shrink]\frametitle{Points}
			\begin{itemize}
			\item $\chi$ is the parameter denoting the probability that the bidder think the other bidders bid independently of signals. 
			\item Cursed equilibrium for a given $\chi$ value is called $\chi$-cursed equilibrium.
			\item In ER's(2002, 2006), $\chi$-cursed equilibrium is the same as one in a hypothetical "$\chi$-virtual game", in which players believe that, with probability $\chi$, other's bid is independent of types.
			\item In i.p.v, the players bid independently of others' signals by definition, then the optimal bidding strategy in $\chi$-cursed equilibrium is the same as one in equilibrium ($\upsilon(x, x) = r(x) = x$).
			\item In c.v, since $\upsilon(x,x) \neq r(x)$, cursed equilibrium differs from equilibrium.
			\end{itemize}
		\end{frame}
		
		\begin{frame}[shrink]\frametitle{Common Value Auction}
			\begin{itemize}
			\item In first price auction, the optimal bidding strategy is calculated as (5).
			\item In second price auction, the optimal bidding strategy is calculated as (6).
			\item These calculations are exactly the same as ones in equilibrium.
			\end{itemize}
			
			\begin{align}
				a_{\chi}(x) &= \left\{ (1 - \chi)\upsilon(x,x) + \chi r(x) \right\} \nonumber \\
				&\quad - \int_{\underline{x}}^x \exp \left( - \int_y^x \frac{f_Y(t|t)}{F_Y(t|t)} \mathrm{d}t \right) \mathrm{d}\left\{ (1 - \chi)\upsilon(y,y) + \chi r(y) \right\} \\
				b_{\chi}(x) &= (1 - \chi)\upsilon(x,x) + \chi r(x)
			\end{align}
			
		\end{frame}
		
	\subsection{Nonequilibrium Level-k Models}
		\begin{frame}[shrink]\frametitle{Points}
			\begin{itemize}
			\item Level-k model allows behavior to be heterogeneous, but assumes that each player's behavior is drawn from the common distribution over the k types.
			\item In this paper there are 3 types (denoted by L k) which best response to the lower type. i.e. L1 best responses to L0, and L2 best responses to L1.
			\item The key assumption is the behavior of L0.  One is Random L0, in which the L0 bids uniformly randomly independent of its own signal. The other is Truthful L0, in which L0 bids the value suggested by its own signal.
			\item L1 and L2 are called Random L1 and L2 when L0 is set to Random L0.
			\item L1 and L2 are called Truthful L1 and L2 when L0 is set to Truthful L0. 
			\item In the latter slides I show the optimal bidding strategies of each player in each case one by one.
			\end{itemize}
		\end{frame}
		
		\begin{frame}\frametitle{Random L0}
			\begin{itemize}
			\item This player bids i.i.d. uniformly over the range $[\underline{z}, \bar{z}]$, which is determined by its private signal and the value $V_i = u_i (S, X)$.
			\end{itemize}
		\end{frame}
		
		\begin{frame}\frametitle{Random L1 in First Price Auction}
			\begin{itemize}
			\item Random L1 plays in the belief that all other players follow Random L0. 
			\item Let $Z$ be the highest bid among the others, the distribution function of $Z$ be $F_z (z)$, and the pdf be $f_z(z)$ (these two are from the ordered statistics).
			\item The optimal bidding strategy of L1 ($a_1^r(x)$) solves (7) and is characterized by (8).
			\end{itemize}
			
			\begin{align}
				\max_a \int_\underline{z}^a (r(x) - a) f_z(z) \mathrm{d}z \\
				(r(x) - a)f_z(a) - F_z(a) = 0
			\end{align}
		\end{frame}
		
		\begin{frame}\frametitle{Random L1 in First Price Auction}
			\begin{itemize}
			\item This optimal bidding strategy is common in i.p.v and c.v.
			\item There are two differences from the optimal bidding strategy in equilibrium.
			\item One is that $r(x)$ replaces $\upsilon(x,x)$, which reflects the fact that Random L1 think winning conveys no information about the value of the object even in c.v. (difference in value adjustment)
			\item Second is that the integral in (7) is over $Z$ rather than $Y$ (see (1) in paper). This is caused by L1 use nonequilibrium belief to evaluate the bidding trade-off.  (difference in bidding trade-off)
			\end{itemize}
		\end{frame}
		
		\begin{frame}\frametitle{Random L1 in Second Price Auction}
			\begin{itemize}
			\item The optimal bidding strategy ($b_1^r(x)$) solves (9)'s maximization problem. And get (10).
			\end{itemize}
			
			\begin{align}
				&\max_b \int_\underline{z}^b (r(x) - z) f_z(z)\mathrm{d}z\\
				&b_1^r(x) = r(x)
			\end{align}
		\end{frame}
		
		\begin{frame}\frametitle{Random L1 in Second Price Auction}
			\begin{itemize}
			\item One difference from equilibrium case is $r(x)$ replaces $\upsilon(x,x)$. (difference in value adjustment)
			\item Second difference from equilibrium case is that the player use nonequilibrium belief. However note that this does not result in the deviating from truthful bidding as in first price auction.
			\item In other words, we have no bidding trade-off term in the optimal bidding strategy just as in the equilibrium case.
			\item (10) coincides with (6) when $\chi = 1$, i.e. fully cursed equilibrium.
			\item (10) coincides with equilibrium in i.p.v, where $r(x) = x$.
			\end{itemize}
		\end{frame}
		
		\begin{frame}\frametitle{Random L2 , Truthful L1 and Truthful L2}
			\begin{itemize}
			\item Random L2 best responses to L1.
			\item (8) and (10) tell that L1's bidding strategy is an increasing function of its private signal both in first and second price auction.
			\item For L2 in this case, the high bid among the other players conveys the information about the value of the object, then L2 adjust its own estimated value according to the information. This means that we use $\upsilon(x, y)$ rather than $r(x)$.
			\item This structure is common in Truthful L1 and Truthful L2. Thus I provide the general bidding strategy which can be applied to the three cases.
			\end{itemize}
		\end{frame}
		
		\begin{frame}\frametitle{General Bidding in First Price Auction}
			\begin{itemize}
			\item Suppose that a level-k bidder expects others to bid according to the monotonically increasing bidding strategy $a_{k-1}(x)$.
			\item The bidder's optimal bidding strategy with $V_i$ and $X_i$ solves (11), and is characterized by the first order condition (12). 
			\end{itemize}
			
			\begin{align}
				&\max_a \int_\underline{x}^{a_{k-1}^{-1}(a)} (\upsilon(x, y) - a) f_Y(y | x) \mathrm{y} \\
				&(\upsilon(x, a_{k-1}^{-1}(a)) - a)f_Y(a_{k-1}^{-1}(a) | x) \frac{\partial a_{k-1}^{-1}(a)}{\partial a} - F_Y(a_{k-1}^{-1}(a) | x) = 0
			\end{align}
			
		\end{frame}
		
		\begin{frame}\frametitle{General Bidding in Second Price Auction}
			\begin{itemize}
			\item Suppose that a level-k bidder expects others to bid according to the monotonically increasing bidding strategy $b_{k-1}(x)$.
			\item The bidder's optimal bidding strategy with $V_i$ and $X_i$ solves (13), and is characterized by the first order condition (14). 
			\end{itemize}
			
			\begin{align}
				&\max_b \int_\underline{x}^{b_{k-1}^{-1}(b)} (\upsilon(x, y) - b_{k-1}(y)) f_Y(y | x) \mathrm{y} \\
				&b = \upsilon(x, b_{k-1}^{-1}(b))
			\end{align}
			
		\end{frame}
		
		\begin{frame}\frametitle{Random L2 in First Price Auction}
			\begin{itemize}
			\item 
			\end{itemize}
		\end{frame}
		
		\begin{frame}\frametitle{Random L2 in Second Price Auction}
			\begin{itemize}
			\item 
			\end{itemize}
		\end{frame}
		
		\begin{frame}\frametitle{Truthful L1 in First Price Auction}
			\begin{itemize}
			\item 
			\end{itemize}
		\end{frame}
		
		\begin{frame}\frametitle{Truthful L1 in Second Price Auction}
			\begin{itemize}
			\item 
			\end{itemize}
		\end{frame}
		
		\begin{frame}\frametitle{Truthful L2 in First Price Auction}
			\begin{itemize}
			\item 
			\end{itemize}
		\end{frame}
		
		\begin{frame}\frametitle{Truthful L2 in Second Price Auction}
			\begin{itemize}
			\item 
			\end{itemize}
		\end{frame}


\section{Comparing}
	\subsection{Optimal Bidding Strategy Comparing}
	
		\begin{frame}\frametitle{Value Adjustment and Bidding Trade-Off}
			\begin{itemize}
			\item 二つの言葉の定義を書く
			\end{itemize}
		\end{frame}
	
		\begin{frame}\frametitle{Summary Table}
			\begin{itemize}
			\item Table 1を挿入
			\end{itemize}
		\end{frame}
		
		\begin{frame}\frametitle{Equilibrium vs Cursed Equilibrium in First Price Auction}
			\begin{itemize}
			\item i.p.v
			\item c.v.
			\end{itemize}
		\end{frame}
		
		\begin{frame}\frametitle{Equilibrium vs Cursed Equilibrium in Second Price Auction}
			\begin{itemize}
			\item i.p.v
			\item c.v
			\end{itemize}
		\end{frame}
		
		\begin{frame}\frametitle{Equilibrium vs Random Level-k in First Price Auction}
			\begin{itemize}
			\item i.p.v
			\item c.v
			\end{itemize}
		\end{frame}
		
		\begin{frame}\frametitle{Equilibrium vs Random Level-k in Second Price Auction}
			\begin{itemize}
			\item i.p.v
			\item c.v
			\end{itemize}
		\end{frame}
		
		\begin{frame}\frametitle{Equilibrium vs Truthful Level-k in First Price Auction}
			\begin{itemize}
			\item i.p.v
			\item c.v
			\end{itemize}
		\end{frame}
		
		\begin{frame}\frametitle{Equilibrium vs Truthful Level-k in Second Price Auction}
			\begin{itemize}
			\item i.p.v
			\item c.v
			\end{itemize}
		\end{frame}
		
		\begin{frame}\frametitle{Cursed Equilibrium vs Random Level-k in First Price Auction}
			\begin{itemize}
			\item i.p.v
			\item c.v
			\end{itemize}
		\end{frame}
		
		\begin{frame}\frametitle{Cursed Equilibrium vs Random Level-k in Second Price Auction}
			\begin{itemize}
			\item i.p.v
			\item c.v
			\end{itemize}
		\end{frame}
		
		\begin{frame}\frametitle{Cursed Equilibrium vs Truthful Level-k in First Price Auction}
			\begin{itemize}
			\item i.p.v
			\item c.v
			\end{itemize}
		\end{frame}
		
		\begin{frame}\frametitle{Cursed Equilibrium vs Truthful Level-k in Second Price Auction}
			\begin{itemize}
			\item i.p.v
			\item c.v
			\end{itemize}
		\end{frame}
		
		\begin{frame}\frametitle{Summary: Where Level-k Model Can Improve?}
			\begin{itemize}
			\item 
			\end{itemize}
		\end{frame}
		
	\subsection{Econometrical Comparing}
		\begin{frame}\frametitle{Auction Examples: KL}
			\begin{itemize}
			\item 
			\end{itemize}
		\end{frame}
		
		\begin{frame}\frametitle{Auction Examples: AK}
			\begin{itemize}
			\item 
			\end{itemize}
		\end{frame}
		
		\begin{frame}\frametitle{Auction Examples: GHP}
			\begin{itemize}
			\item 
			\end{itemize}
		\end{frame}
		
		\begin{frame}\frametitle{Preparation for Comparing}
			\begin{itemize}
			\item 
			\end{itemize}
		\end{frame}
		
		\begin{frame}\frametitle{How to Compare}
			\begin{itemize}
			\item 
			\end{itemize}
		\end{frame}
		
		\begin{frame}\frametitle{Table3a}
			\begin{itemize}
			\item 
			\end{itemize}
		\end{frame}
		
		\begin{frame}\frametitle{Table3c}
			\begin{itemize}
			\item 
			\end{itemize}
		\end{frame}
		
		\begin{frame}\frametitle{Table3d}
			\begin{itemize}
			\item 
			\end{itemize}
		\end{frame}
		
		\begin{frame}\frametitle{Table3b}
			\begin{itemize}
			\item 他と比率が違う理由もかく
			\end{itemize}
		\end{frame}
		
		\begin{frame}\frametitle{Summary: Could Level-k Model really Improve?}
			\begin{itemize}
			\item 
			\end{itemize}
		\end{frame}
	
	
\section{Conclusion}
	\begin{frame}\frametitle{Summary}
		\begin{itemize}
		\item
		\end{itemize}
	\end{frame}
	
	\begin{frame}\frametitle{Implication}
		\begin{itemize}
		\item
		\end{itemize}
	\end{frame}
	

\end{document}
























