\documentclass[dvipdfmx, 12pt]{beamer}
\usepackage{pxjahyper}
\usepackage{minijs}
\usepackage{otf}
\renewcommand{\kanjifamilydefault}{\gtdefault}
\usetheme{Dresden}
\setbeamertemplate{navigation symbols}{}
\usepackage{url}
\usepackage{graphicx}
\usepackage{comment}
\newenvironment<>{varblock}[2][.9\textwidth]{%
  \setlength{\textwidth}{#1}
  \begin{actionenv}#3%
    \def\insertblocktitle{#2}%
    \par%
    \usebeamertemplate{block begin}}
  {\par%
    \usebeamertemplate{block end}%
  \end{actionenv}}


\title{LEVEL-k AUCTIONS: CAN A NONEQUILIBRIUM MODEL OF STRATEGIC THINKING EXPLAIN THE WINNER'S CURSE AND OVERBIDDING IN PRIVATE-VALUE AUCTIONS?}
\author{Kei Ikegami}

\begin{document}
\newcommand{\argmin}{\mathop{\rm arg~min}\limits}

\frame{\maketitle}

\section*{Index}
	\begin{frame}\frametitle{Index}
	\tableofcontents
	\end{frame}

\section{Introduction}
	\begin{frame}\frametitle{Purpose}
	\end{frame}
	
	\begin{frame}\frametitle{Precedents}
	\end{frame}
	
	\begin{frame}\frametitle{What's new}
	\end{frame}
	
	\begin{frame}\frametitle{Result}
	\end{frame}
	
\section{Models}
	\subsection{Overview}
		\begin{frame}\frametitle{General Model}
			\begin{itemize}
			\item $N$ bidders bid for a single object.
			\item $X_i$ is bidder $i$'s private signal. $X = (X_1, \dots, X_N)$.
			\item $S_j$ is additional random variable which is informative about the value of the object. $S = (S_1, \dots, S_M)$.
			\item $V_i = u_i(S, X)$ is bidder $i$'s value of the object, where $u_i$ is symmetric across $i$.
			\item $V_i - p$ is the payoff for the bidder $i$ winning the auction by paying $p$.
			\item $Y$ is the highest signal among bidders other than $i$.
			\item $\upsilon(x, y) = E\left[ V_i | X_i = x, Y = y \right]$ is the expected value conditional on winning.
			\item $r(x) = E\left[ V_i | X_i = x \right]$ is the unconditional expected value.
			\end{itemize}
		\end{frame}
		
		\begin{frame}\frametitle{Classification of Auctions}
			\begin{itemize}
			\item First price auction vs Second price auction
			\item Independent private value auction(i.p.v) vs Common value auction(c.v)
			\item In i.p.v, the signals and values are independent among bidders.
			\item In c.v, the information of $i$ and $j$ is not independent and learning about the other bidders' information can cause the bidder to reassess his estimate of the value of the object. (e.g. Timber auction)
			\end{itemize}
		\end{frame}
		
	\subsection{Equilibrium}
		\begin{frame}\frametitle{First Price Auction}
			\begin{itemize}
			\item In c.v, the optimal bidding strategy is calculated as follows 
			\end{itemize}
			
		\end{frame}
		
		\begin{frame}\frametitle{Second Price Auction}
			\begin{itemize}
			\item 
			\end{itemize}
		\end{frame}
		
	\subsection{Cursed Equilibrium}
		\begin{frame}\frametitle{Points}
			\begin{itemize}
			\item 
			\end{itemize}
		\end{frame}
		
		\begin{frame}\frametitle{First Price Auction}
			\begin{itemize}
			\item 
			\end{itemize}
		\end{frame}
		
		\begin{frame}\frametitle{Second Price Auction}
			\begin{itemize}
			\item 
			\end{itemize}
		\end{frame}
		
	\subsection{Nonequilibrium Level-k Models}
		\begin{frame}\frametitle{Points}
			\begin{itemize}
			\item 
			\end{itemize}
		\end{frame}
		
		\begin{frame}\frametitle{Random L1 in First Price Auction}
			\begin{itemize}
			\item 
			\end{itemize}
		\end{frame}
		
		\begin{frame}\frametitle{Random L1 in Second Price Auction}
			\begin{itemize}
			\item 
			\end{itemize}
		\end{frame}
		
		\begin{frame}\frametitle{Random L2 in First Price Auction}
			\begin{itemize}
			\item 
			\end{itemize}
		\end{frame}
		
		\begin{frame}\frametitle{Random L2 in Second Price Auction}
			\begin{itemize}
			\item 
			\end{itemize}
		\end{frame}
		
		\begin{frame}\frametitle{Truthful L1 in First Price Auction}
			\begin{itemize}
			\item 
			\end{itemize}
		\end{frame}
		
		\begin{frame}\frametitle{Truthful L1 in Second Price Auction}
			\begin{itemize}
			\item 
			\end{itemize}
		\end{frame}
		
		\begin{frame}\frametitle{Truthful L2 in First Price Auction}
			\begin{itemize}
			\item 
			\end{itemize}
		\end{frame}
		
		\begin{frame}\frametitle{Truthful L2 in Second Price Auction}
			\begin{itemize}
			\item 
			\end{itemize}
		\end{frame}


\section{Comparing}
	\subsection{Optimal Bidding Strategy Comparing}
		\begin{frame}\frametitle{Summary Table}
			\begin{itemize}
			\item Table 1を挿入
			\end{itemize}
		\end{frame}
		
		\begin{frame}\frametitle{Equilibrium vs Cursed Equilibrium in First Price Auction}
			\begin{itemize}
			\item i.p.v
			\item c.v.
			\end{itemize}
		\end{frame}
		
		\begin{frame}\frametitle{Equilibrium vs Cursed Equilibrium in Second Price Auction}
			\begin{itemize}
			\item i.p.v
			\item c.v
			\end{itemize}
		\end{frame}
		
		\begin{frame}\frametitle{Equilibrium vs Random Level-k in First Price Auction}
			\begin{itemize}
			\item i.p.v
			\item c.v
			\end{itemize}
		\end{frame}
		
		\begin{frame}\frametitle{Equilibrium vs Random Level-k in Second Price Auction}
			\begin{itemize}
			\item i.p.v
			\item c.v
			\end{itemize}
		\end{frame}
		
		\begin{frame}\frametitle{Equilibrium vs Truthful Level-k in First Price Auction}
			\begin{itemize}
			\item i.p.v
			\item c.v
			\end{itemize}
		\end{frame}
		
		\begin{frame}\frametitle{Equilibrium vs Truthful Level-k in Second Price Auction}
			\begin{itemize}
			\item i.p.v
			\item c.v
			\end{itemize}
		\end{frame}
		
		\begin{frame}\frametitle{Cursed Equilibrium vs Random Level-k in First Price Auction}
			\begin{itemize}
			\item i.p.v
			\item c.v
			\end{itemize}
		\end{frame}
		
		\begin{frame}\frametitle{Cursed Equilibrium vs Random Level-k in Second Price Auction}
			\begin{itemize}
			\item i.p.v
			\item c.v
			\end{itemize}
		\end{frame}
		
		\begin{frame}\frametitle{Cursed Equilibrium vs Truthful Level-k in First Price Auction}
			\begin{itemize}
			\item i.p.v
			\item c.v
			\end{itemize}
		\end{frame}
		
		\begin{frame}\frametitle{Cursed Equilibrium vs Truthful Level-k in Second Price Auction}
			\begin{itemize}
			\item i.p.v
			\item c.v
			\end{itemize}
		\end{frame}
		
		\begin{frame}\frametitle{Summary: Where Level-k Model Can Improve?}
			\begin{itemize}
			\item 
			\end{itemize}
		\end{frame}
		
	\subsection{Econometrical Comparing}
		\begin{frame}\frametitle{Auction Examples: KL}
			\begin{itemize}
			\item 
			\end{itemize}
		\end{frame}
		
		\begin{frame}\frametitle{Auction Examples: AK}
			\begin{itemize}
			\item 
			\end{itemize}
		\end{frame}
		
		\begin{frame}\frametitle{Auction Examples: GHP}
			\begin{itemize}
			\item 
			\end{itemize}
		\end{frame}
		
		\begin{frame}\frametitle{Preparation for Comparing}
			\begin{itemize}
			\item 
			\end{itemize}
		\end{frame}
		
		\begin{frame}\frametitle{How to Compare}
			\begin{itemize}
			\item 
			\end{itemize}
		\end{frame}
		
		\begin{frame}\frametitle{Table3a}
			\begin{itemize}
			\item 
			\end{itemize}
		\end{frame}
		
		\begin{frame}\frametitle{Table3c}
			\begin{itemize}
			\item 
			\end{itemize}
		\end{frame}
		
		\begin{frame}\frametitle{Table3d}
			\begin{itemize}
			\item 
			\end{itemize}
		\end{frame}
		
		\begin{frame}\frametitle{Table3b}
			\begin{itemize}
			\item 他と比率が違う理由もかく
			\end{itemize}
		\end{frame}
		
		\begin{frame}\frametitle{Summary: Could Level-k Model really Improve?}
			\begin{itemize}
			\item 
			\end{itemize}
		\end{frame}
	
	
\section{Conclusion}
	\begin{frame}\frametitle{Summary}
		\begin{itemize}
		\item
		\end{itemize}
	\end{frame}
	
	\begin{frame}\frametitle{Implication}
		\begin{itemize}
		\item
		\end{itemize}
	\end{frame}
	

\end{document}
























