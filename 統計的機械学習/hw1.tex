\documentclass{article}
\usepackage[margin = .7in]{geometry}
\usepackage[dvipdfmx]{graphicx}
\usepackage{listings}
\usepackage{amsmath}
\usepackage{amssymb}
\usepackage{amsfonts}
\usepackage{bm}
\usepackage{mathrsfs}
\lstset{%
  language={python},
  basicstyle={\small},%
  identifierstyle={\small},%
  commentstyle={\small\itshape},%
  keywordstyle={\small\bfseries},%
  ndkeywordstyle={\small},%
  stringstyle={\small\ttfamily},
  frame={tb},
  breaklines=true,
  columns=[l]{fullflexible},%
  numbers=left,%
  xrightmargin=0zw,%
  xleftmargin=3zw,%
  numberstyle={\scriptsize},%
  stepnumber=1,
  numbersep=1zw,%
  lineskip=-0.5ex%
}

\begin{document}
\title{統計的機械学習 \\ 
Homework 1}
\author{経済学研究科現代経済コース修士1年 / 池上 慧 (29186009) / sybaster.x@gmail.com}
\maketitle

\section{宿題1:非対称な損失}
 自動運転中に突然目の前に現れた物体の識別を例にとる。動く物体が人間か野生動物のどちらかに分類して、人間だと判断した時は急ブレーキをかけ、野生動物と判断した時はそのまま走行を続けるように決まっているとする。この時、野生動物を誤って人間と判断した時の損失は無駄な急ブレーキによって生じたむち打ちの治療費である30万円だが、人間を誤って野生動物と判断した時は死亡事故に対する賠償金や慰謝料は2000万から3000万円である。これは非対称な損失の例となっている。

\section{宿題2:計算機演習}
課題1
\begin{quotation}
    using Gadfly\\
    n = 1000\\
    x = randn(n)\\
    y1 = x + randn(n)\\
    y2 = -x + randn(n)\\
    y3 = randn(n)\\
    y4 = $x.^2$ + randn(n)\\
    plot(x = x, y = y1)\\
    plot(x = x, y = y2)\\
    plot(x = x, y = y3)\\
    plot(x = x, y = y4)\\
    println(cor(x, y1))\\
    println(cor(x, y2))\\
    println(cor(x, y3))\\
    println(cor(x, y4))
\end{quotation}
結果は上から順に、$0.7274246670322699$、$-0.7359980511037424$、$-0.019398550577429056$、$-0.043661238223850346$出会った。これより、相関係数が小さいことは確率変数の独立を意味しないことがわかる。
\\
\\
課題2

分散部分を大きくすると、より平坦な分布となる。共分散部分を正の方向に大きくすると$y = x$の方向にはなだらかに、$y = -x$の方向には急激な変化をするような形となり、今日分散部分を負に大きくするとその逆となる。

\section{Reference}
\begin{enumerate}
	\item 交通事故弁護士相談広場 https://www.jicobengo.com/damages/fatal-accident.html
	\item 誰でもわかる交通事故示談 https://xn--u8jvc1drby660ajea95wk08b0s0b5qybrha279c.com/9862
\end{enumerate}

\end{document}
