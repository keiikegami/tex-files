\documentclass{article}
\usepackage[margin = .7in]{geometry}
\usepackage{multirow,array}
\usepackage[dvipdfmx]{graphicx}
\usepackage{listings}
\usepackage{amsmath}
\usepackage{bm}

\begin{document}
\title{PET/MRI research}
\author{Kei Ikegami}
\maketitle

\section{Empirical Strategy}

$i \in \left\{ 1,2, \cdots, I\right\}$ denotes each hospital. And $y_i$ is the welfare variable, $PET_i$ is the number of PET at hospital $i$, $Total_i$ is the total number of owned PETs by hospitals existing in the same prefecture as $i$, $Other_i =Total_i - PET_i$, and $X_i$ is the control variables. Then consider the below model.

\begin{align*}
	y_i = \alpha + \beta\ PET_i + \gamma\ Total_i + \delta\ Other_i + X_i^{'} \eta + \epsilon_i
\end{align*}
$\beta$ is the parameter representing the welfare improvement by introducing PET, $\gamma$ is market expansion parameter, and $\delta$ is business stealing effect parameter. But by the variable construction, we cannot estimate all the parameter due to the perfect collinearlity. 

Then we make use of panel data structure. Now $p$ indicates the prefecture and $t$ refers to year. Let $\alpha_p, \alpha_t$ be the indicator of prefecture and year. Then we get the panel data version model.
\begin{align*}
	y_{ipt} = \beta\ PET_{ipt} + \gamma\ Total_{pt} + \delta\ Other_{ipt} + X_{ipt}^{'} \eta +W_{pt}^{'} \xi +  \alpha_t + \alpha_p + \epsilon_{ipt} + \nu_{pt}
\end{align*}
where $W_{pt}$ is the characteristics of each prefecture at each year, and $\nu_{pt}$ is the disturbance term common in individual hospitals.

In the first stage, we regress the below
\begin{align*}
	y_{ipt} = Z_{pt} + \beta\ PET_{ipt} + \delta\ Other_{ipt} + X_{ipt}^{'} \eta + \epsilon_{ipt}
\end{align*}
Then We get the estimates of $Z_{pt}$, denoted as $\hat{Z_{pt}}$.

If the size $P \times T$ is sufficiently large, we can use $\hat{Z_{pt}}$ as the dependent variable in the second stage regression.
\begin{align*}
	\hat{Z_{pt}} = \alpha_t + \alpha_p + \gamma\ Total_{pt} + W_{pt}^{'} \xi +  \nu_{pt}
\end{align*}
By this panel data regression, we also get the consistent estimate of $\gamma$. This argument is according to Imbens and Wooldridge's NBER summer lecture notes 10.

\end{document}