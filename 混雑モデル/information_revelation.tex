\documentclass{article}
\usepackage[margin = .7in]{geometry}
\usepackage[dvipdfmx]{graphicx}
\usepackage{listings}
\usepackage{amsmath}
\usepackage{amsfonts}
\usepackage{bm}
\lstset{%
  language={python},
  basicstyle={\small},%
  identifierstyle={\small},%
  commentstyle={\small\itshape},%
  keywordstyle={\small\bfseries},%
  ndkeywordstyle={\small},%
  stringstyle={\small\ttfamily},
  frame={tb},
  breaklines=true,
  columns=[l]{fullflexible},%
  numbers=left,%
  xrightmargin=0zw,%
  xleftmargin=3zw,%
  numberstyle={\scriptsize},%
  stepnumber=1,
  numbersep=1zw,%
  lineskip=-0.5ex%
}

\begin{document}
\title{Mitigating Congestion through Information Revelation}
\author{Kei Ikegami}
\maketitle

\section{Introduction}

\section{Model}
I consider a dynamic game, whose first stage is voting for trains and second one is boarding. In the first stage players are asked to vote for the most preferable train, and they can tell a lie in this phase. The result of voting is revealed after all players have voted, and then they choose which train they really board according to the population density estimated by the result, considering the tradeoff between disutility from congestion and the utility from getting on the appropriate train. This situation is modeled as a nonatomic infinite player game and solved by Perfect Bayesian Equilibrium concept.

\subsection{Second stage game}
At the station in suburb area, there is $K$ (even number) trains going to the center of the city in rush hour. All residents near by the station have to get on one of the $K$ trains in the same period like 7.a.m. ~ 9.a.m. Because there is a lot of commuters, they always suffer severe congestion everyday. They are the players in this model.

All the players have their first best train and second best train. The first best train means the player choose the train if there is no restriction or disutility effects like congestion. The second best train is the train they choose if they refrain from catching the first best train. I assume that all the player must board either the first best or the second best. In this case, we have ${}_K C _2$ types of players, but I additionally put assumptions for the sake of reality and computation.

To begin with I split the trains into two subgroups by their arriving time. Let $k = \left\{ 1, 2, \cdots, K \right\}$ represent earlier trains by earlier numbers, i.e. the earliest train is indexed by $1$, and the last train is by $K$. Then the former half train group is composed of $k = \left\{ 1,2,\cdots, \frac{K}{2} \right\}$ and the later half train group is $k = \left\{ \frac{K}{2}+1, \frac{K}{2} + 2, \cdots, K \right\}$. Furthermore I divide the players into two subgroups according to their type. Those who have the first best train in the former half train group is named as the former group, and the later half is defined analogously.

The first assumption is that the players in the former group have the one train before as their second best train and the player whose best train is $k = 1$ always get on the train. And the same restriction is put on the later half group. By this assumption we denote the type of the players by their first best train. For example, if $K = 6$, there are $6$ types players. Type $1$ is the player who always board the first train, in other words the players both of whose first best train and the second best train is train $1$. Type $2$ player is the player whose first best is train $2$ and second best is $1$. Type $5$ is the player whose first best is train $5$ and second best is train $6$.

The main reason why there exists super crowded trains in rush hour is that such trains arrive at the time when both groups of players are in the station. I model this situation by overlapping the two groups of trains. That is the last train in the former half train group is the first train in the later half train group. For example, when $K = 6$, there are $5$ trains. And the player in the first half player group prefer $1,2$, or $3$, while the player in the second half player group prefer $3,4$, or $5$. This is the second assumption.

The third assumption is the symmetric distribution of types in both of the groups. This absolutely simplify the computation. And I think this is not so unrealistic that it diminishes the importance of the model, because the commuters deriving the biased distribution should move their residence to the next station since they have the subsidy from their company. Furthermore the players are assumed to be divided into the first half and the second half with the same fraction, i.e. $\frac{1}{2}$. Then It is sufficient to put focus on the first half group. I denote the type distribution by $\left\{ q_k \right\}_{k = 1}^{\frac{K}{2}}$ where $\sum_{i = 1}^{\frac{K}{2}} q_t = \frac{1}{2}$.

Next we model the preference of players. All players have the common base utility $u_1, u_2$ specific to boarding the first best train and the second best train. And the key component is the disutility term resulted from the congestion. Let $\left\{ p_k \right\}$ be the realized distribution of passengers, the players boarding the train $k$ as his first best have the below form of utility.

\begin{align*}
	v_k = u_1 + \alpha p_k
\end{align*}
where $\alpha$ is the coefficient representing the marginal disutility from the congestion at the train. And of course those who board train $k$ as the second best have the below utility.
\begin{align*}
	v_k = u_2 + \alpha p_k
\end{align*} 
In the second stage of this game the players maximize their expected utility by choosing either his first best or second best train according to the above utility and the type distribution scheme. 

\subsection{Belief formation and the first stage game}
But the most important and confusing point of this model is the type distribution $\left\{ q_k \right\}_{k = 1}^{\frac{K}{2}}$ is not common knowledge but retrieved by the voting in the first stage.This setting is reasonable because we usually do not have a chance to get the information about the shape of the commuters distribution. All we know is how crowded the train which we really get on is. How much the other trains are congested is not always in our information. Certainly many station offer the knowledge to induce distributed boarding, but that is often ignored  since almost all passengers do not  know the existence and commuters in hurry do not rely on such not real time and scarce information.

Then I think there remains a opportunity to change their behavior by providing the real-time information about expected congestion. And the tool for that is voting in this model. In other words, in the second stage the players use the voting result to estimate the type distribution. So in the remaining I describe how the belief as to the distribution is made and the first stage game structure.

Belief formation is by Bayesian updating since  this case is modeled as Dirichlet - Multinomial. All players have the same Dirichlet prior and the voting result can be seen as the realization of multinomial distribution whose parameters is drawn from the Dirichlet prior. These two are conjugate, and so the posterior distribution is also Dirichlet distribution. The posterior mean or mode is used as an estimate of the type distribution.

In general,
\begin{align*}
	\begin{cases}
	x \sim Dir(\alpha_1, \cdots, \alpha_K)\\
	p \sim Multi(x_1,\cdots,x_K, N)
	\end{cases}
	\Rightarrow\ 
	x\mid p \sim Dir(\alpha_1 -1 + p_1, \cdots, \alpha_K -1 + p_K)
\end{align*}
and
\begin{align*}
	E[x \mid p] = \left( \frac{\alpha_1 - 1 + p_1}{\sum_{k = 1}^K \alpha_i -1 + p_i} , \cdots, \frac{\alpha_K - 1 + p_K}{\sum_{k = 1}^K \alpha_i -1 + p_i}\right)
\end{align*}
In current setting $N$ is the population using the station in rush hour. Note that the parameter of Dirichlet distribution is actually under the restriction for deriving the symmetric belief. This is described in detail later.



\section{Equilibrium}


\end{document}






























