\documentclass{jsarticle}
\usepackage[margin = .7in]{geometry}
\usepackage[dvipdfmx]{graphicx}
\usepackage{listings}
\usepackage{amsmath}
\usepackage{amsfonts}
\usepackage{bm}
\lstset{%
  language={python},
  basicstyle={\small},%
  identifierstyle={\small},%
  commentstyle={\small\itshape},%
  keywordstyle={\small\bfseries},%
  ndkeywordstyle={\small},%
  stringstyle={\small\ttfamily},
  frame={tb},
  breaklines=true,
  columns=[l]{fullflexible},%
  numbers=left,%
  xrightmargin=0zw,%
  xleftmargin=3zw,%
  numberstyle={\scriptsize},%
  stepnumber=1,
  numbersep=1zw,%
  lineskip=-0.5ex%
}

\begin{document}
\title{卒論テーマ候補 :混雑現象の構造推定}
\author{池上 慧}
\maketitle

\section{研究目的}
PMLは収束が保障されない。Kasahara and Shimotsuで言及された収束方法には別手法からのヤコビアンが必要。PMLによらない混雑パラメータの推定手法を提案。また、意思決定主体についての詳しいデータが存在しない状況(例えば満員電車での乗客について詳しいデータは知りようがない)でも用いることのできる推定手法であることを示す。

\section{モデル}

\section{シミュレーションデータでの推定}

\section{実験データでの推定}

\section{実証研究}
入試日程の設定と満員電車


\end{document}