\documentclass{jsarticle}
\usepackage[margin = .7in]{geometry}
\usepackage[dvipdfmx]{graphicx}
\usepackage{listings}
\usepackage{amsmath}
\usepackage{bm}
\usepackage{ascmac}
\lstset{%
  language={python},
  basicstyle={\small},%
  identifierstyle={\small},%
  commentstyle={\small\itshape},%
  keywordstyle={\small\bfseries},%
  ndkeywordstyle={\small},%
  stringstyle={\small\ttfamily},
  frame={tb},
  breaklines=true,
  columns=[l]{fullflexible},%
  numbers=left,%
  xrightmargin=0zw,%
  xleftmargin=3zw,%
  numberstyle={\scriptsize},%
  stepnumber=1,
  numbersep=1zw,%
  lineskip=-0.5ex%
}

\begin{document}
\title{現代西洋経済史A2まとめ}
\author{池上慧}
\maketitle

\section{第2次産業革命以前の世界}
東方貿易の開始や新大陸の発見などは商業の中心地をポルトガル、スペインへと変化させた商業革命を引き起こした。この時期、宗教改革の中で特に新教において教会を国家の統治機構の末端とみなす信教国家体制が確立、また内政不干渉の原則を掲げたウェストファリア体制の下で家父長的存在としての君主像と国家主権という概念が生み出されていた。こういった流れの中で明確になった国家という存在は互いに対立しあい、世界商業戦という様相を呈するようになる。各国が生産力の増大、国富の増大を目指して重商主義の立場をとったが、それによって発達した市場経済ではエンクロージャーによって村落共同体が崩壊し国家による管理が及ばない貧民が急増した。国家は救貧法によりこれらのうちで労働能力がない者、労働能力がありかつ労働意欲もある者についてのみ救済することで生産的な人口を増やし国富の増大という目標を追求し続けた。

しかし、市場経済と産業革命がもたらした都市部のスラム形成、コミュニティ崩壊による治安の悪化、労働者階級の誕生と彼らの組織化と政治化といった社会問題は産業革命期の負の側面として同時代から認識されており、またマルサス的な人口観、すなわち人口増大が富の増大を上回るために国富の増大という面で良いものでなくむしろ社会問題を悪化させうるものであるとの考え方、が支配的になったことも相まって、重商主義批判の土台が整備されてきた。そしてアダムスミスの国富論と経済学の誕生を受けて「国家と社会の分離」が重商主義に替わるものとしてこの批判の受け皿となった。この運動は産業革命で富を蓄えた中間層が担った。彼らは支配層と大衆との間に生まれたために団体に所属せず、公権力から排除された存在だったからである。彼らは自らの所属としてアソシエーションを組織した。サロンを中心に発達したこれらの団体は理性を基調とし、そこでの議論は自由な発言によるものであったために世論として重視され、それを伝えるためのメディアの発達ももたらした。彼らの関係性の基本は市民的公共性に基づき、家族を社会から隔絶した愛情によってのみ形成されるものとしたためにプライバシーという概念も生み出された。

だが、フランス革命を経て誕生した近代国民国家では信教国家は解体され中央集権的な官僚機構による統治が行われるようになる。これは国家と個人とを直接結びつけるもので、中間団体を否定する体制であった。中間層の思想はこの国家介入と対立するものであったため、コック内にそういった対立を内包したまま各国は第2次産業革命を迎えることになる。

\subsection{イギリス}
\begin{index}
	\item 
\end{index}

\subsection{フランス}

\subsection{ドイツ}


\section{第2次産業革命からWW1までの世界}

\subsection{イギリス}

\subsection{フランス}

\subsection{ドイツ}


\section{戦間期}

\subsection{イギリス}

\subsection{フランス}

\subsection{ドイツ}


\end{document}


























