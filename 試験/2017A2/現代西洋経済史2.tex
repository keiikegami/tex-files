\documentclass{jsarticle}
\usepackage[margin = .7in]{geometry}
\usepackage[dvipdfmx]{graphicx}
\usepackage{listings}
\usepackage{amsmath}
\usepackage{bm}
\usepackage{ascmac}
\lstset{%
  language={python},
  basicstyle={\small},%
  identifierstyle={\small},%
  commentstyle={\small\itshape},%
  keywordstyle={\small\bfseries},%
  ndkeywordstyle={\small},%
  stringstyle={\small\ttfamily},
  frame={tb},
  breaklines=true,
  columns=[l]{fullflexible},%
  numbers=left,%
  xrightmargin=0zw,%
  xleftmargin=3zw,%
  numberstyle={\scriptsize},%
  stepnumber=1,
  numbersep=1zw,%
  lineskip=-0.5ex%
}

\begin{document}
\title{現代西洋経済史A2まとめ}
\author{池上慧}
\maketitle

\section{第2次産業革命以前の世界}
東方貿易の開始や新大陸の発見などは商業の中心地をポルトガル、スペインへと変化させた商業革命を引き起こした。この時期、宗教改革の中で特に新教において教会を国家の統治機構の末端とみなす信教国家体制が確立、また内政不干渉の原則を掲げたウェストファリア体制の下で家父長的存在としての君主像と国家主権という概念が生み出されていた。こういった流れの中で明確になった国家という存在は互いに対立しあい、世界商業戦という様相を呈するようになる。各国が生産力の増大、国富の増大を目指して重商主義の立場をとったが、それによって発達した市場経済ではエンクロージャーによって村落共同体が崩壊し国家による管理が及ばない貧民が急増した。国家は救貧法によりこれらのうちで労働能力がない者、労働能力がありかつ労働意欲もある者についてのみ救済することで生産的な人口を増やし国富の増大という目標を追求し続けた。

しかし、市場経済と産業革命がもたらした都市部のスラム形成、コミュニティ崩壊による治安の悪化、労働者階級の誕生と彼らの組織化と政治化といった社会問題は産業革命期の負の側面として同時代から認識されており、またマルサス的な人口観、すなわち人口増大が富の増大を上回るために国富の増大という面で良いものでなくむしろ社会問題を悪化させうるものであるとの考え方、が支配的になったことも相まって、重商主義批判の土台が整備されてきた。そしてアダムスミスの国富論と経済学の誕生を受けて「国家と社会の分離」が重商主義に替わるものとしてこの批判の受け皿となった。この運動は産業革命で富を蓄えた中間層が担った。彼らは支配層と大衆との間に生まれたために団体に所属せず、公権力から排除された存在だったからである。彼らは自らの所属としてアソシエーションを組織した。サロンを中心に発達したこれらの団体は理性を基調とし、そこでの議論は自由な発言によるものであったために世論として重視され、それを伝えるためのメディアの発達ももたらした。彼らの関係性の基本は市民的公共性に基づき、家族を社会から隔絶した愛情によってのみ形成されるものとしたためにプライバシーという概念も生み出された。

だが、フランス革命を経て誕生した近代国民国家では信教国家は解体され中央集権的な官僚機構による統治が行われるようになる。これは国家と個人とを直接結びつけるもので、中間団体を否定する体制であった。中間層の思想はこの国家介入と対立するものであったため、コック内にそういった対立を内包したまま各国は第2次産業革命を迎えることになる。

\subsection{イギリス}
\begin{itemize}
	\item 慈善組織協会
	\item 共済団体
	\item 院内救済
	\item 劣等処遇
	\item 救貧法
	\item 工場法
	\item 地方自治体法
	\item 公衆衛生法
\end{itemize}

当時のイギリスにおける支配的な考えは経済的自由主義である。すなわち自由で平等な経済主体による市場を放任することが最も良いとされる考え方である。しかしこのような考えに基づいて進展した産業革命の負の側面として労使対立や住環境の悪化、貧困層の増大といった問題が噴出することになる。産業革命の担い手であった中間層は自助、共助を推し進めることで、より健全に市場メカニズムを機能させることができればこれらの問題を解決できると考えた。そのための枠組みを整備することが国家の役割であると考えられ、実際に、労働組合の合法化が労使における力関係の不均衡を是正するために、株式会社合法化が労働者の共済団体設立のために、工場法が女性と子供の保護のために施行された。これらの問題は市場経済という社会の中で自律的に解決されることが望ましいとみなされたものであった。このように自助、共助を助けるための仕組みや行いを慈善とよぶ。

貧困問題に対しても同様の発想で救貧法が施行された。これは市場の最低賃金よりも救済の水準を下げる劣等処遇を行うことで働く能力のあるものは皆働きに出るであろうという考えに基づいており、この下でも働きに出ないものは道徳的欠陥者であるとみなされ救貧院送りとなって市民的権利を剥奪されることとなっていた。このように市場経済内部には国家は介入するべきでなく、市場経済から追い出された存在に対しては社会ではなく国家が対応するべきだと明確に分けられたという意味で、イギリスは国家と社会の分離が明確になされた国であると言われる。

以上のように若干の修正を経ながらも経済的自由主義という原理に従って国家は運営されていたが、同時にフランス革命の影響を受けて中央集権的で官僚機構を備えた近代的統治機構の形成も進んでいた。1828年の地方自治体法によって信教国家が解体されたことを皮切りに19世紀半ばにかけて中央省庁が地方の行政を一括で行う流れが生ま、その中でさらなる国家介入が必要であるとするチャドウィクなども現れてきた。

\subsection{フランス}
\begin{itemize}
	\item ルシャ・ブリエ法
	\item シスモンディ(セー法則批判)
	\item ディリシズム
	\item ルプレー
	\item トクヴィル
	\item 1853年法
\end{itemize}

18世紀末のフランス革命では中間団体が徹底的に否定され、国家と直接つながる国民が誕生した。また同時に信教国家体制も否定され、国民の権利として救済権が認められるようになっていた。このような状況で、急速な人口増大と産業革命による下層民の誕生は「新しい貧困」と呼ばれる、野蛮で未開な危険な階級としても貧民像を確立した。シスモンディによるセー法則批判などを受けて、このような問題を市場メカニズムで解決しようというイギリス流の考えはフランスでは弱く、国家と社会は未分離のまま国家が問題に対処することが求められた。このような考えをディリシズムと呼ぶ。

しかし公的救貧制度は全国的に普及せず救済権が確立しないなど、国家による問題解決はうまくいかなかった。これを受けて19世紀に入るとルプレーの地域主義やトクヴィルによるフランスにおけるアソシエーションの欠如の指摘などを通して中間団体の見直しが始まった。だが、このような流れの下で生まれた共済組合も国家権力による認可登録制であり、英独に比してその発展が遅れることとなった。

\subsection{ドイツ}
\begin{itemize}
	\item 都市下層民
	\item 領邦絶対王政
	\item イヌング
	\item 良き行政
\end{itemize}

ドイツにおいても都市下層民は市民から排除された存在であるなど貧困問題は大きかった。3月革命以降の上からの近代化の中で、工業化の推進とそれによる負の側面に対応することの両方を行う主体としての行政が期待された。そのような「良き行政」の中では封建時代からの中間団体は国家の統治の下でその存在を推進されて行くことになる。これは、実際に国家のできることが小さかったために、古くからの中間団体における共助を利用し、支配体制に取り込むための策であった。そのため、イギリスにおけるアソシエーションなどと違い、中間層により形成されたこれらの団体には身分制が残り、労働者が取り込まれなかった。19世紀半ばにかけて国家主導で地域ごとに自助基金が設置され、労働者も一律の団体に所属できるように改善されて行く。


\section{第2次産業革命からWW1までの世界}
3つの要素がある。一つ目は階級対立を通じて国家と社会の分離が保てなくなること、二つ目は人口の質が重視されるようになること、三つ目は救貧制度が破綻して貧困観が転換することである。

第2次産業革命期には巨大資本の活用のために生産から供給までの市場の組織化が進んだ。それに伴う合理化は不熟練労働者を増加させ、慢性的な失業を発生させることとなった。労働組合の合法化も相まって労働者の政治参加が進み、今まで労組や共済から排除されていた不熟練労働者の間でも組織化が進んだ。労働者の政治参加は各国で労働者政党を発足させ、国家運営が階級間の利害関係の調整場としての機能を持つようになる。これは、社会から排除されてきたものたちが政治参加し国家による枠組み作成に関与するようになることを意味し、国家と社会の分離が保てなくなったことが明らかになった。一方で、そのような利害対立の場で議論の基礎となる社会における真実を発見するために社会科学が発展した。

帝国主義列強間での対立はその競争における人的資本の重要性を各国に認識させ、また第2次産業革命を経て人口成長を生産力が上回ったことによってマルサス的な人口観から各国が脱却したこともあり、人口の質をあげることが目指された。それは貧困問題の解決を目指すことを意味したが、この時期大不況による大量失業の発生によって旧来の救貧制度は破綻をきたしていた。そのような中で社会調査を通じて貧困道徳的な問題ではなく一般的な現象であることが科学的に発見され、以降は貧困を要因別に予防するための社会保障制度を充実させることが各国で目指されることとなった。

\subsection{イギリス}
\begin{itemize}
	\item レントナー化
	\item チェンバレンキャンペーン
	\item 帝国関税同盟
	\item 人民予算
	\item 反不労所得
	\item ウェッブ夫妻
	\item 最低賃金
	\item ナショナルミニマム
	\item 大蔵省
	\item 学校給食
	\item 無拠出制老齢年金
	\item 国民保険法
	\item 医療保険
	\item 失業保険
	\item 社会サービス協議会
\end{itemize}

第2次産業革命に立ち遅れたイギリスでは慢性的な失業問題と経済のレントナー化によるジェントリの資産転換で都市部、農村部のどちらでも貧困が深刻化しもはや院内救済の原則を守れないほどに救貧制度は限界を迎えていた。貧困を社会の責任とみなすようになったこと、列強との対立によって貧困問題の解決が喫緊の課題とみなされたことも相まって、今まで社会から排除されてきた存在、すなわち失業者や不熟練労働者、女性、ユダヤ人などの存在感が大きくなってきた。特に不熟練労働者が政治的に組織化する中でチェンバレンキャンペーンで帝国関税同盟の是非が問われた。1906年の総選挙で自由貿易派が勝利したことで新自由主義の社会改革が進展して行くことになる。

新自由主義は「資本主義的強制からの自由」を唄い、上向的自由競争を担保するものとしてのナショナルミニマムの実現を目指して改革を進めた。その根底には市場メカニズムへの懐疑が存在し、国家介入を強めて行く改革の方向性となった。救貧と公衆衛生に対しては大蔵省と商務省という中央省庁が今まで地方行政に任せていた救貧業務をまとめて担当することとし、同時に貧困の原因分析も進めた。これにより児童福祉や失業対策事業など、貧民のカテゴリー化と個別の対応が実現して行く。1908年には無拠出制の老齢年金を、1911年には医療と失業に対する国民保険法を施行した。ただし、医療保険においては既存の共済組合を運営主体とし、失業保険についても自助が困難な不熟練労働者にのみ対象を限定し給付水準も低く抑えることで、両者において自助、共助と矛盾しないナショナルミニマムの実現を目指した。さらにウェッブ夫妻の提言に乗っ取り苦汗産業については最低賃金、労働時間規制が実現したが、これについてもそれにより生産性の低い企業を排除することができるという市場機能の健全化という根本の思想が貫かれていることには注意が必要だ。

このような変化は従来型の共助団体の性質を変化させた。すなわち顔の知れたコミュニティにおける友愛組合から拠出と給付に基づく純粋な経済的関係のみによって成立する関係へと変化して行った。1910年には社会サービス協議会が設立され国家の支援の下で事前活動を行うようになる。これは当初国家への抵抗として成立したアソシエーションの経緯を考えると、大きな変化であり、同時期に発達した近代統治機構にこれらの団体が飲み込まれたことを意味している。

\subsection{フランス}
\begin{itemize}
	\item 第三共和制
	\item パストゥール革命
	\item 労働総同盟
	\item 連帯主義
	\item 労災保障法
	\item 老齢年金保険法
\end{itemize}

普仏戦争での敗北後第三共和制期に社会問題が噴出する。これは左右が極端で政情不安定なフランスにおいても1890年以降党派を超えて対処すべき問題として認識されるほどとなる。その一環としてフランス各目において徹底的に否定された中間団体の再生がこの時期目指されるが、共済組合については英仏に比して発展が遅れ、労組についても1884年にようやく合法化されて1895年に労働総同盟が組織されるが、組織率は低く、中間団体の組織はなかなか進まなかった。そのような中でパストゥール革命の影響から、中間団体の組織とそれらの有機的なつながりを妨げる要因が存在することが認識されるようになる。そこで注目されたのが、個人の自立を脅かす事故や病気、老齢などのリスクの存在であった。職業労働を探すこと、生活規律を正しく守り、衛生習慣を身につけ、家族を扶養するという個人としての責任を果たしている限りにおいて国家がこれらのリスクをケアすることが国家の役割として求められるようになる。この結果、1898年の労災保障法で事故の保障を、1910年の老齢年金保障法で自助が困難な労働者や農民を保護しようとした。しかし、当初強制保険を目指したこれらの社会保険は不徹底に終わったため効果も発揮しなかった。

\subsection{ドイツ}
\begin{itemize}
	\item ユンカー
	\item 労災保険
	\item 疾病保険
	\item 老齢年金
	\item ドイツ救貧慈善協会
	\item ドイツカトリックカリタス協会
\end{itemize}

第2帝政期は封建的な性格が強くユンカーを中心に家父長制的な雇用関係が結ばれ身分制秩序も残存していたが、大不況を期に困窮した旧中間層が中心となり社会主義運動が起こってくる。これに対応するために、旧中間層に対しては農業保護や手工業団体ツンフトを職業訓練制度の担い手として残存させるなどの政策を行った。また労働者に対しても保護を行ったが、実情としてはどれも国家による関与の小さい保険制度となっており、かつナショナルミニマムとしての性格よりも、身分保証的な性格の強いものとなっていた。具体的には労災保険においては職業危険の発想から経営者のみを拠出者とし、疾病保険においては使用者と労働者で分担して拠出させ、老齢年金においても地域別の団体に運営を任せていた。ドイツにおいても貧困の原因分析とその予防が重要視され、ドイツ救貧慈善協会において地域社会に社会問題解決のネットワークを創出しようとした。しかし宗教団体からの協力は得られなかった。


\section{戦間期}
WW1における各国の総力戦と総動員体制は、国内における兵隊の家族の保護、復員兵の保護などを必要とするものであった。このような保護対象の増大は、まず社会の自立性により対処できるものではなく、またすでに破綻していた公的救貧制度で対処できるものでは当然なかった。これに戦後の不安定な経済状況とそれによる失業問題が重なり、ロシアにおける社会主義革命の成功も相まって、従来社会から排除されていたグループの政治的組織とそれらを基盤とした国内における共産主義と資本主義という対立構造が生まれることになった。

資本主義体制を維持するためには、市場に任せても解決できないこれらの社会問題を国家として解決に導く必要があった。社会福祉政策の再編として無拠出の失業扶助が成立し、ここにおいて失業と貧困の区別が消失することとなる。また、この時期新たに顕在化した社会保障需要への対応としては社会衛生学に基礎を置く住宅政策と、少子化対策としての児童手当があった。大戦による人口減少を受けて人口の量も質も求められる時代の中で大衆を貧困に陥らせないための予防的社会政策が注目されるが、これは優生思想が社会に浸透する基礎となってしまった。

\subsection{イギリス}
\begin{itemize}
	\item 第4次選挙権改革
	\item 旧平化復帰
	\item 許可組合制度
	\item 無拠出制年金
	\item 失業手当
	\item ミーンズテスト
	\item 失業扶助
	\item 家族手当
	\item 社会サービス全国協議会
\end{itemize}

WW1以降、負傷兵や復員兵、兵隊の扶養家族など保護の対象が拡大し、また戦時下で抑圧されていた社会需要が爆発し従来社会から排除されてきたグループからの政治的要求が高まった。これは第4字選挙権改革に結びつき、男女普選と救貧受給者への選挙権付与が決定された。これは救貧受給者を市民として認めることを意味するが、この背景には綿工業や鉄鋼、造船などの旧来の中核産業の衰退と旧平値復帰によるデフレが不況を招き、結果として大量の失業者が発生していたことがある。これらの長期失業者に対しては無拠出の失業手当が準備され、労働党と保守党とのせめぎ合いの中で失業保険とともに対象の拡大と期間の延長が実現してきたが、大恐慌以降さらに深刻かした失業問題には対応できず、無拠出の失業扶助が新たに設けられた。このように失業問題に対しては長らく制度的な対応がなされてきたが、それでも対応しきれず救貧制度を適用する事例が多くなってきてしまう。1934年には失業扶助の給付水準が救貧制度のそれよりも低いという事態となり、劣等処遇の原則も守れなくなり、2次大戦後に救貧法の廃止へとつながることとなる。この背景にはすでに院内救済の原則が崩壊していたこと、そして選挙権の授与を通して市民的権利の剥奪も行われなくなっていたことがあったことに注意が必要である。

失業問題以外にも貧困の原因ごとに社会保障制度の改革が進められた。健康保険においては国家による認可組合制度が成立し、老齢年金に関しては受給者の増加とともに無拠出から拠出性へと移行がなされ、新たなニーズへの対応として施行された住宅政策では国家援助のもとで公営住宅の建設が行われ、家族政策についても少子化対策として一般家庭にまで保護の対象を広げ、1925年には家族手当によりどのような家庭においても一定の育児環境が保障されるようになった。これらは全て国家による社会問題の積極的な解決を意味し、自助、共助のための組織であった慈善団体、友愛組合は金銭的関係に基づく自助基金へと性質を転換させ、国家の庇護のもとで大規模化、官民協力の流れが進展した。


\subsection{フランス}
\begin{itemize}
	\item ポアンカレ内閣
	\item 社会保険法
	\item 単一金庫
	\item 出生率上級評議会
	\item 家族手当保障基金
\end{itemize}

WW1後、ドイツに対する後進性が強く認識され近代化が目指される。同時に返還されたアルザス・ロレーヌ地方を通してドイツの社会保障制度を知り、こちらにおいても後進性が認識される。しかし戦争被害や労使対立の激化などで改革は進まず、相対的安定期になってようやく社会保障制度改革が進められるようになる。1928年の社会保険法は医療年金出産の全てをカバーするもので、国家による援助と労使折半の拠出、年収制限などを特徴とした。国家介入への抵抗から単一金庫制にはならなかったが、30年に強制保険として施行される。家族政策については、少子高齢化の進展が早かったフランスにおいて出産と育児は国家主義と結びつき社会保健省の下に設置された出生率上級評議会を通して家族メダルなどの政策が実現した。家族手当についても複数企業が資金を融通し合う家族手当補償基金の設立などを経て、全ての子供を対象とする国家援助のもとの保健制度が成立した。

\subsection{ドイツ}
\begin{itemize}
	\item 社会国家
	\item ライヒスバンク
	\item 失業保険
	\item 強制同質化
	\item 頂上団体
	\item 社会衛生学
	\item 優生思想
	\item ライヒ扶助義務法
\end{itemize}

ワイマール期は政情不安と財政基盤の弱さゆえに当初社会保障制度改革はうまくいかない。失業保険は扶助や職業紹介所、職業訓練の整備にも関わらず失業者の増大に対処しきれず救貧受給者が増加し、高齢化で受給者が増えた年金も積立式から賦課方式へと転換させた。社会権を重視するワイマール体制の下ではこのように権利拡充と財政不安の間で右往左往することになる。

しかし相対的安定期に入り財政的な余裕が生まれると社会保障の拡充が進んだ。各種保健の条件緩和や給付額増加がおこなわれ、また予防的な処置が取られるようになる。このような流れの中で1923年に救貧法が廃止され、救済の権利化、標準化を目指して社会扶助が実現した。1927年には失業保険も整備されるが、大恐慌によって発生した大量の失業者への対処は行えず、社会扶助でも対応しきれなかった。

ナチス政権はこれらの社会権の権利性を否定し強制同質化政策で実質賃金の切り下げと労働力再配分を強行するとともに大規模な公共事業を行って失業問題を解決した。dこれによって社会保障財政が改善し、国民共同体としての意識の高まりとともに保護の対象が拡大、頂上団体を通して全国で平準化された保護を受けられるようになった。ただし、この背後には「国家に貢献できない者」「保護に値しない者」の排除が存在したことに注意が必要である。また、ナチスは優生思想や日常生活を脅かす性病やアルコール中毒への対処を目指す社会衛生学に則って人口政策を執り行った。民間慈善に関しては、WW1中から官民協力が進展したが、ワイマール期に社会科を恐れ独立性を保とうとしていた。こうして7頂上団体が結成されるが、相対的安定期には社会科の危機がさったために国家による民間の資源活用が進むようになった。1924年にはライヒ扶助義務法が成立し、社会サービスにおける民間団体の優先の原則が確立した。


\end{document}


























