\documentclass{jsarticle}
\usepackage[margin = .7in]{geometry}
\usepackage[dvipdfmx]{graphicx}
\usepackage{listings}
\usepackage{amsmath}
\usepackage{bm}
\usepackage{ascmac}
\lstset{%
  language={python},
  basicstyle={\small},%
  identifierstyle={\small},%
  commentstyle={\small\itshape},%
  keywordstyle={\small\bfseries},%
  ndkeywordstyle={\small},%
  stringstyle={\small\ttfamily},
  frame={tb},
  breaklines=true,
  columns=[l]{fullflexible},%
  numbers=left,%
  xrightmargin=0zw,%
  xleftmargin=3zw,%
  numberstyle={\scriptsize},%
  stepnumber=1,
  numbersep=1zw,%
  lineskip=-0.5ex%
}

\begin{document}
\title{都市経済A1:田淵}
\author{池上慧}
\maketitle

\section{語彙集}
\subsection{都市(1,2,3章)}
1章の内容
\begin{itemize}
\item 人口集中地区(DID)

人口密度4000人の地域が連続して人口5000人以上になるまで合わせた区域のこと
\item エッジシティ

都心の地価上昇に対応して、通勤費用や住宅費用の軽減のために、必ずしも都心になくてはならないわけではないオフィス機能を周辺に分散させた結果組織化が進んだと新周囲の都市群
\item パエリンクの都市発展プロセス
\item ランクサイズルール

都市の人口規模と、その人口規模順に並べたときのランクをかけると常に同じような値となる。(べき法則)
\end{itemize}

2章の内容
\begin{itemize}
\item 空間の不均一性(第1の自然/第2の自然)

都市への集中が進む要因の一つ。所与である天然資源や生産要素が空間上に不均一に存在するため、各地域で適した産業が発展することになる。河川や土地の肥沃度などの純粋な自然要因は第1の自然と呼ばれ、農作物やその加工製品の産業集積に影響する。また交通インフラや歴史的な観光資源などの人間が作った生活環境を第2の自然と呼び、これらも同様に産業の集積を起こす。港湾などの交通の要所を持つ都市は第1の自然に恵まれ人が集積し、それが第2の自然を整備した結果現在まで大都市であり続けているケースが多い。繁華街なども乗り換えの多い駅で発展するケースが多いが、これは大2の自然が産業を集積させた例である。
\item ロックイン効果(歴史的要因)

土地には所有権が残るので、過去に発展した都市は何らかの環境変化後もそのまま発展し続けることが多い。これをロックイン効果と呼ぶ。例えば戦後の復興においてかつて栄えた地域から順に再度都市化が進行した。
\item 規模の経済

1社についての話。ある企業が収穫逓増の技術を用いることで独占的にその財を生産することになり、合併や吸収を通して企業規模とともに都市規模も大きくなる。(しかし財の需要にも限度があるため、この力だけでは大都市には成長しない。)
\item 範囲の経済

1社についての話。規模の経済とは違い、複数財の生産に関して共通費用が存在する場合に一社が複数の財を生産することで一つの財に特化するよりも費用を節約でき、利潤を拡大できる現象を指す。例えば、鉄道会社が貨物輸送を担うのはレールや車両といった共通費用が存在するからである。
\item 集積の経済

現代の大都市は多様な企業の集積が起こっている。その要因として、以下の二つで構成される集積の経済という外部性の存在をあげる。
	\begin{itemize}
	\item 地域特化の経済
	
	特定地域に同一産業の企業が集中して立地することでバラバラに立地するよりも産業全体での生産額が増える現象のことである。小規模な地場産業の集積によくみられる。しかし1財への依存は都市の長期的発展を妨げるため、よくて中規模都市ぐらいで止まってしまう。
	\item 都市化の経済
	
	多様な産業が集積することで相互交流が起こり、都市全体の経済活動の水準が上昇する現象を指す。現代の大規模都市の成立にはこの効果が少なからず作用していることが想定される。
	\end{itemize}
\item 空間不可能性定理

輸送費が0でない時、収穫一定もしくは収穫逓減を仮定すると、競争均衡は存在せず、自給自足の経済が各地で営まれるだけである。すなわち交易は発生せず集積も起こらないとする定理
\item 土地集積の要因
	\begin{itemize}
        \item sharing
        
        同業者の集積は情報共有や原料輸送の同時発注などで便益をもたらす。また集積すれば取引費用や輸送費の削減で生産工程における分業がより進むことになり、結果として垂直的分業による効率的なネットワークが構成される。
        \item matching
        
        大都市では労働市場が巨大であるので、企業としては労働力の確保が容易に行えることになる。また消費者が集積していることは多様な財やサービスへの需要が存在することを意味し、企業としても自社とマッチする消費者を確保できる可能性が高くなるため生産を行いやすい。さらに、購買は複数材について同時に行われるため、小売業の集積も起こる。
        \item learning
        
        知識集約型の産業をはじめとして、企業活動におけるノウハウや技術知識は対面の関係性の中で拡散する。この時、ファッションやアート、ハイテク産業などの集積する地域ではそこでの知識拡散によって数多くのプロダクトが派生しまた企業が立ち上がるというプロセスが踏まれる。さらにそういった企業間で熟練労働者の転職などが盛んに行われることで一層知識のスピルオーバーガ加速する。このように密な関係性にある企業軍は熾烈な競争の中で収益性を急上昇させ、結果として産業全体の規模が大きくなる。それがまた人を呼び込み新たな知識を生む。こういった正のスパイラルは産業を集積させる。
        \end{itemize}
\end{itemize}

3章の内容
\begin{itemize}
\item 最適都市規模

一人あたり所得は都市規模に対して上に凸、一人当たり支出は都市規模に対して下に凸。前者は集積の経済が働くために都市規模の小さい時は生産性が急上昇し、ある程度大きくなるとその効果が逓減するため。後者は都市規模が小さい時は、産業の固定費用などを書く人で分担するので一人当たりコストが減少するが、ある程度大きくなると長時間通勤や大気汚染などによる外部不経済が増大するために一人当たりコストが上昇するからである。これにより一人当たりの効用は都市規模に対して上に凸な関数となるので、最大となる都市規模が唯一存在する。
\end{itemize}

\subsection{土地(7,8章)}
7章の内容
\begin{itemize}
\item 60年代前半の地価高騰

3度ある地価上昇期で最も上昇率が大きかった。高度経済成長期にあたり、コンビナートなどの大規模工業施設が建設されていく中で工業用地の地価が上昇、それを受けて住宅用地や商業用地の需要も膨れ上がりのちの地価神話を形作るほどの地価上昇が観察された。
\item 70年代前半の地価高騰

列島改造ブームで地方も巻き込んで大都市への人口移動が活発化した。所得の上昇も背景に膨れあがった住宅需要を満たすために都市周辺部の開発が行われ住宅用地の地価が上昇し、それを受けて全体としても地価の上昇が観察された。1975年ごろからのGDP成長率鈍化を反映して徐々に落ち着いて行った。
\item 80年代後半の地価高騰

金融自由化の中で世界金融の中心となった東京都心にオフィスビルが乱立。商業地の地価上昇をもたらす。これが周辺部の住宅用地へと波及し、全国的に地価上昇が観察された。バブル崩壊後地価は下落続きだが、2006年に底を打ち上昇傾向にある。しかし全国的な動きではなく、地価においても地域格差が広がっている。
\item 地価高騰の弊害
	\begin{itemize}
	\item 土地の供給不足
	
	値上がり期待による供給不足。これにより長時間出勤が必要な郊外に居住する労働者が増加し、また居住スペースの狭さも相まって労働者の勤労意欲を減退させた。
	\item 社会資本整備の妨げ
	
	道路や公共施設などの建設に際して地価が高すぎて最適な水準の投資が行われない。
	\item 所得格差の拡大
	
	地価変動が激しいと投機目的の土地購入が行われる。キャピタルゲインは運なので個人間の所得格差をもたらす。
	\end{itemize}
\item 土地の低度利用
	\begin{itemize}
	\item 資産目的の保有
	
	投機の対象とされる。また土地所有は税制上優遇されるため、わざと遊休地とされることもある。
	\item 建物の不可逆性
	
	一回建物を建ててしまうと、以降の建て替えには相当の費用がかかる。周囲の環境の変化に合わせて需要に最適化した利用目的の施設を作るためにはある程度様子見する期間が必要である。
	\item 借地法
	
	旧来の借地法、借家法の下では地主は一旦貸しに出すと権利が複雑化し容易に売却できなくなるので、現金化する必要が生じるまで歌詞もせず売りもしないでいることが多かった。借地借家法は従来の借地方では期限がなかった借地権に期限をつけることで賃貸市場の活発化を目的に作られた法律である。
	\end{itemize}
\item 土地の性質
	\begin{itemize}
	\item 超耐久財
	
	減少も減耗もしない耐久財。土地の供給量は既に決定されるているという意味で完全に弾力的なので、地価水準は需要サイドの要因だけで決まるという特性がある。
	\item 移動不可能性
	
	移動できないがゆえに周囲の環境や土地自体の性質には異質性が存在し、客観的な値付けは難しい。円滑な取引のためには不動産鑑定士による値付けが必須である。
	\item 生産要素
	
	最終消費財ではないということ。土地需要は生産活動に対する派生需要であると言える。
	\item 地価、地代、帰属地代
	
	土地自体の価格としての地価、一定期間提供されるサービスに対しての価格である地代の二つの価格を持つ。また所有者と利用者が同じである時は帰属地代と言われる。
	\end{itemize}
\item 土地市場の性質
株などと比べて土地市場の流動性は極めて低い。その要因として以下の3つがある。
	\begin{itemize}
	\item 土地取引費用
	
	登記費用などの額が大きな費用があり、これはサンクコストであるため、市場への参入が完全に自由ではない。
	\item 価格付けの困難性
	
	先にも述べた通り異質性により客観的に価格付けするのが難しい。
	\item 土地保有のインセンティブ
	
	弱者保護の目的で土地保有に関する税は優遇されている。その結果、土地という形での資産保有が行われ手放すインセンティブがないことになる。また投機目的の保有も多い。
	\end{itemize}
\item バブルの発生条件

耐久性、希少性、共通の確信(価格上昇への確信)が必要。バブル期の土地はこの条件を満たしていた。
\item 一物四価
以下は大きく乖離する。これは社会的弱者が土地を手放すことのないように相続税や固定資産税の算出に用いる表kがくを低く設定する行政側の配慮によるものである。しかしこれは様々な問題を起こしたため、今は相続税路線価を公示価格の8割、固定資産税評価がくを7割に近づけることとされた。
	\begin{itemize}
	\item 実勢価格 取引が成立する価格。しかし流動性が低いのでこれだけでは不十分。
	\item 公示価格 全国2万地点について不動産鑑定士による鑑定価格が公示される。
	\item 相続税路線価 相続税や贈与税の算出に用いられるもの。
	\item 固定資産税評価額 固定資産税の算出に用いられるもの。
	\end{itemize}
\item 土地保有税
\item 土地譲渡所得税
\item 相続税
\item 土地所得税
\end{itemize}

8章の内容
\begin{itemize}
\item フィルタリング現象
\item ジェントリフィケーしょん
\item 住宅の財としての特性
	\begin{itemize}
	\item 耐久財
	\item 移動の不可能性
	\item 不可分性
	\item 取引費用
	\item 情報の非対称性(2方向)
	\end{itemize}
\item 家賃統制
\end{itemize}

\end{document}

























