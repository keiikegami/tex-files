\documentclass{jsarticle}
\usepackage[margin = .7in]{geometry}
\usepackage[dvipdfmx]{graphicx}
\usepackage{listings}
\usepackage{amsmath}
\usepackage{bm}
\usepackage{ascmac}
\lstset{%
  language={python},
  basicstyle={\small},%
  identifierstyle={\small},%
  commentstyle={\small\itshape},%
  keywordstyle={\small\bfseries},%
  ndkeywordstyle={\small},%
  stringstyle={\small\ttfamily},
  frame={tb},
  breaklines=true,
  columns=[l]{fullflexible},%
  numbers=left,%
  xrightmargin=0zw,%
  xleftmargin=3zw,%
  numberstyle={\scriptsize},%
  stepnumber=1,
  numbersep=1zw,%
  lineskip=-0.5ex%
}

\begin{document}
\title{現代西洋経済史まとめ}
\author{池上慧}
\maketitle

\section{第2次産業革命}
1873年から1896年まで続いた大不況期に進展した第2次産業革命は、独占企業の登場に伴う市場の組織化とそれに対抗する形での労働組合の組織化を導いいた。マクロな視点ではイギリスの停滞と米独の台頭が顕著になり多極化の流れが鮮明となった。各国における第2次産業革命のあり方は第1次産業革命の起こり方が準備していた。
\subsection{第2次産業革命の具体的要素}
\begin{itemize}
\item 製鋼革命

鋼鉄の大量生産が可能となるベッセマー法が発明された。製鉄、製鋼、圧延を一社で行うことでコストを下げる範囲の経済がワークするようになった。
\item 新産業の誕生

電気、化学、石油、機械産業が勃興した。
\item 固定資本の巨大化と科学的管理法

規模の経済が存在するため大量生産を目指すようになる。そのために生産管理と労務管理が必要となり熟練労働者を用いての間接管理方式から科学的管理法に基づいた直接管理方式へと移行が進む。また同時に安定した供給を確保するために垂直的統合、水平的統合による市場の組織化が進展した。
\item 科学研究と技術の結びつき

研究開発への投資が行われるようになり、それに答えるための技術者要請が国の施策として行われるようになる。
\item 株式会社の誕生

社会的遊休資産集中の枠組みとして株式会社が合法化される。また有限責任が明記されたことでリスクを取りやすくなる。
\item 銀行資本と産業資本の有機的結びつき

銀行による株式発行、保有や役員派遣、交互計算業務などにより銀行と企業が有機的に結びつくようになる。これは銀行主導での企業グループが発生する基盤となった。
\end{itemize}
\subsection{第1次と第2次:イギリス}
\begin{itemize}
\item 
\end{itemize}
\subsection{第1次と第2次:フランス}
\begin{itemize}
\item 
\end{itemize}
\subsection{第1次と第2次:ドイツ}
\begin{itemize}
\item 
\end{itemize}
\subsection{第1次と第2次:アメリカ}
\begin{itemize}
\item 
\end{itemize}
\subsection{第1次と第2次:マクロな視点から}
\begin{itemize}
\item 
\end{itemize}

\section{戦間期}
\subsection{戦間期:マクロな視点から}
\subsection{戦間期:イギリス}
\subsection{戦間期:フランス}
\subsection{戦間期:ドイツ}
\subsection{戦間期:アメリカ}


\end{document}





















