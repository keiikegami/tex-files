\documentclass{jsarticle}
\usepackage[margin = .7in]{geometry}
\usepackage[dvipdfmx]{graphicx}
\usepackage{listings}
\usepackage{amsmath}
\usepackage{bm}
\usepackage{ascmac}
\lstset{%
  language={python},
  basicstyle={\small},%
  identifierstyle={\small},%
  commentstyle={\small\itshape},%
  keywordstyle={\small\bfseries},%
  ndkeywordstyle={\small},%
  stringstyle={\small\ttfamily},
  frame={tb},
  breaklines=true,
  columns=[l]{fullflexible},%
  numbers=left,%
  xrightmargin=0zw,%
  xleftmargin=3zw,%
  numberstyle={\scriptsize},%
  stepnumber=1,
  numbersep=1zw,%
  lineskip=-0.5ex%
}

\begin{document}
\title{現代西洋経済史まとめ}
\author{池上慧}
\maketitle

\section{第2次産業革命}
1873年から1896年まで続いた大不況期に進展した第2次産業革命は、独占企業の登場に伴う市場の組織化とそれに対抗する形での労働組合の組織化を導いいた。マクロな視点ではイギリスの停滞と米独の台頭が顕著になり多極化の流れが鮮明となった。各国における第2次産業革命のあり方は第1次産業革命の起こり方が準備していた。具体的にはイギリスにおいては産業革命が発生するような環境が整備された上での内発的な現象として産業革命が発生したが、諸外国では外発的に国主導で行われた。その際にイギリスを手本としつつ反面教師ともしつつ革命が進行したことに特徴がある。
\subsection{第2次産業革命の具体的要素}
\begin{itemize}
\item 製鋼革命

鋼鉄の大量生産が可能となるベッセマー法が発明された。製鉄、製鋼、圧延を一社で行うことでコストを下げる範囲の経済がワークするようになった。
\item 新産業の誕生

電気、化学、石油、機械産業が勃興した。
\item 固定資本の巨大化と科学的管理法

規模の経済が存在するため大量生産を目指すようになる。そのために生産管理と労務管理が必要となり熟練労働者を用いての間接管理方式から科学的管理法に基づいた直接管理方式へと移行が進む。また同時に安定した供給を確保するために垂直的統合、水平的統合による市場の組織化が進展した。
\item 科学研究と技術の結びつき

研究開発への投資が行われるようになり、それに答えるための技術者要請が国の施策として行われるようになる。
\item 株式会社の誕生

社会的遊休資産集中の枠組みとして株式会社が合法化される。また有限責任が明記されたことでリスクを取りやすくなる。
\item 銀行資本と産業資本の有機的結びつき

銀行による株式発行、保有や役員派遣、交互計算業務などにより銀行と企業が有機的に結びつくようになる。これは銀行主導での企業グループが発生する基盤となった。
\end{itemize}
\subsection{第1次と第2次:イギリス}
第1次キーワード
\begin{itemize}
\item 社会的分業
	\begin{itemize}
	\item 全国市場の成立
	\item 交通インフラの整備
	\item 銀行体系の確率
	\end{itemize}
\item パートナーシップ
\item 間接管理制
\item 経済的自由主義
\item クラフトユニオン
\end{itemize}
第2次キーワード
\begin{itemize}
\item 経験主義、現場主義
\item 植民地市場
\item 商業銀行
\item イギリス型独占
\item 労働組合法
\item 争議法
\end{itemize}

\subsection{第1次と第2次:フランス}
第1次キーワード
\begin{itemize}
\item フランス革命
\item ルシャ・プリエ法
\item 普仏戦争
\item 復古王政期
\item クレディモビリエ
\item サン・シモン会議
\end{itemize}
第2次キーワード
\begin{itemize}
\item 大企業と国内市場
\item 中小企業と輸出
\item 第3共和制
\item 4大銀行
\item 事業銀行
\item アンクタット/コートワール
\item サンディカリズム
\end{itemize}

\subsection{第1次と第2次:ドイツ}
第1次キーワード
\begin{itemize}
\item 2重革命
\item ユンカー経営
\item イヌング
\item ドイツ関税同盟
\item ルール工業地帯
\item プロイセン
\item 特殊ドイツ型銀行
\item 歴史学派経済学
\end{itemize}
第2次キーワード
\begin{itemize}
\item ドイツ帝国
\item ベルリン工科大学
\item ルール工業地帯
\item クルップ
\item ドイツ型銀行(主導性)
\item 生産材
\item カルテル
\item 家父長制的企業経営
\item 社会民主主義労働党
\end{itemize}

\subsection{第1次と第2次:アメリカ}
第1次キーワード
\begin{itemize}
\item 地域的分立
\item 州権主義
\item ハミルトン体制
\item 南北戦争(1865年)
\item アメリカ式工業生産体制
\item 独立自営農
\item 自己金融
\end{itemize}
第2次キーワード
\begin{itemize}
\item 全国的な金融市場(ニューヨーク金融市場)
\item ビッグビジネス
\item 消費財
\item カルテル/トラスト
\item シャーマン反トラスト法
\item 投資銀行
\item 機能別管理機構
\item ピラミッド型階層組織
\item ホワイトカラー
\item テイラー主義
\item AFL
\item 福祉資本主義
\end{itemize}

\subsection{第1次と第2次:マクロな視点から}
古典的世界市場(19c半ばまで)とその崩壊
\begin{itemize}
\item 世界の工場
\item 英仏通商条約
\item 自由貿易体制の確立
\item ロンドン金融市場
\item 鉄道債
\item 資本輸出(インフラ投資)
\item モノカルチャー
\end{itemize}
大不況(1873~1896)を経て古典的世界市場が崩壊。さらに第2次産業革命を経験した世界で多角的貿易決済機構が誕生
\begin{itemize}
\item 資金循環
\item 国際的金本位体制
\item 帝国関税同盟
\item チェンバレンキャンペーン
\item レントナー化
\end{itemize}


\section{戦間期}
第1次世界大戦での総力戦は財政規模の不可逆的な拡大をもたらし、世界はロシア革命を背景に国家が市場を組織化することを求めるようになる。一方で、戦時協力の報酬として存在を認められた労組が力を増し、さらにアメリカ的な生産様式の追及から各国で合理化が求められたことも相まって労働争議が戦後活発化した。
\subsection{戦間期:マクロな視点から}
キーワード
\begin{itemize}
\item 国際金融の役割分担
\item 国際資金循環
\item 賠償問題
\item ドーズ案(1924)
\item 相対的安定期
\item アメリカ株式ブーム
\item 近隣窮乏化政策
\end{itemize}

\subsection{戦間期:イギリス}
相対的安定期キーワード
\begin{itemize}
\item ポンド高
\item 合同運動
\item 経済構造の高度化
\item 合理化
\item 労働争議法、労働法改正(ムチ)
\item モンド・ターナー法(アメ)
\end{itemize}
恐慌後キーワード
\begin{itemize}
\item マクドナルド政権
\item スターリングブロック
\item 金本位制離脱
\end{itemize}

\subsection{戦間期:フランス}
相対的安定期キーワード
\begin{itemize}
\item フランス労働総同盟(CGT)
\item CGT分裂
\item 合理化運動が進まない
\end{itemize}
恐慌後キーワード
\begin{itemize}
\item 人民戦線内閣
\item 購買力政策
\item 小麦局
\item 金本位制離脱
\item アンタント
\item 自由主義経済と統制経済の中間
\end{itemize}

\subsection{戦間期:ドイツ}
相対的安定期キーワード
\begin{itemize}
\item 
\end{itemize}

\subsection{戦間期:アメリカ}
\begin{itemize}
\item 
\end{itemize}
\begin{itemize}
\item 
\end{itemize}


\end{document}





















