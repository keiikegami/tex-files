\documentclass{jsarticle}
\usepackage[margin = .7in]{geometry}
\usepackage[dvipdfmx]{graphicx}
\usepackage{listings}
\usepackage{amsmath}
\usepackage{bm}
\usepackage{ascmac}
\lstset{%
  language={python},
  basicstyle={\small},%
  identifierstyle={\small},%
  commentstyle={\small\itshape},%
  keywordstyle={\small\bfseries},%
  ndkeywordstyle={\small},%
  stringstyle={\small\ttfamily},
  frame={tb},
  breaklines=true,
  columns=[l]{fullflexible},%
  numbers=left,%
  xrightmargin=0zw,%
  xleftmargin=3zw,%
  numberstyle={\scriptsize},%
  stepnumber=1,
  numbersep=1zw,%
  lineskip=-0.5ex%
}

\begin{document}
\title{現代西洋経済史まとめ}
\author{池上慧}
\maketitle

\section{第2次産業革命}
1873年から1896年まで続いた大不況期に進展した第2次産業革命は、独占企業の登場に伴う市場の組織化とそれに対抗する形での労働組合の組織化を導いいた。マクロな視点ではイギリスの停滞と米独の台頭が顕著になり多極化の流れが鮮明となった。各国における第2次産業革命のあり方は第1次産業革命の起こり方が準備していた。具体的にはイギリスにおいては産業革命が発生するような環境が整備された上での内発的な現象として産業革命が発生したが、諸外国では外発的に国主導で行われた。その際にイギリスを手本としつつ反面教師ともしつつ革命が進行したことに特徴がある。
\subsection{第2次産業革命の具体的要素}
\begin{itemize}
\item 製鋼革命

鋼鉄の大量生産が可能となるベッセマー法が発明された。製鉄、製鋼、圧延を一社で行うことでコストを下げる範囲の経済がワークするようになった。
\item 新産業の誕生

電気、化学、石油、機械産業が勃興した。
\item 固定資本の巨大化と科学的管理法

規模の経済が存在するため大量生産を目指すようになる。そのために生産管理と労務管理が必要となり熟練労働者を用いての間接管理方式から科学的管理法に基づいた直接管理方式へと移行が進む。また同時に安定した供給を確保するために垂直的統合、水平的統合による市場の組織化が進展した。
\item 科学研究と技術の結びつき

研究開発への投資が行われるようになり、それに答えるための技術者要請が国の施策として行われるようになる。
\item 株式会社の誕生

社会的遊休資産集中の枠組みとして株式会社が合法化される。また有限責任が明記されたことでリスクを取りやすくなる。
\item 銀行資本と産業資本の有機的結びつき

銀行による株式発行、保有や役員派遣、交互計算業務などにより銀行と企業が有機的に結びつくようになる。これは銀行主導での企業グループが発生する基盤となった。
\end{itemize}
\subsection{第1次と第2次:イギリス}
第1次キーワード
\begin{itemize}
\item 社会的分業

市場の組織化
	\begin{itemize}
	\item 全国市場の成立
	\item 交通インフラの整備
	\item 銀行体系の確率
	\end{itemize}
\item パートナーシップ
\item 間接管理制
\item 経済的自由主義
\item クラフトユニオン
\end{itemize}
 市場の組織化によって利潤の再投資だけで資金調達が行えたイギリスでは株式会社を必要とせずに産業革命が内発的に発生した。この時の運営体制はパートナーシップと呼ばれ、無限責任であった。労務管理も熟練労働者に生産を委託する間接管理で、雇用すらも委託していた。しかし産業革命が進展する中で労働者と企業経営者の間で格差が拡大し、労働組合(craft union)結成による力関係の是正、株式会社承認による保険や協同組合の発達、有限責任の導入による参入の容易化などを通して市場の健全化が図られた。この、「独占容認、労組容認」の路線を経済的自由主義と呼ぶ。

第2次キーワード
\begin{itemize}
\item 経験主義、現場主義
\item 植民地市場
\item 商業銀行
\item イギリス型独占
\item 労働組合法(1871)
\item 争議法(1906)
\end{itemize}
 第1次産業革命を先導したイギリスは当時築いた世界帝国の市場を使う事で第2次産業革命を得ずとも経済を成立させることができてしまった。また経験主義、現場主義が根強く残ったため研究開発組織が未整備となったこと、クラフトユニオンを形成する熟練労働者が強く抵抗したことも相まって第2次産業革命に立ちおくれた。こういった事情は大企業の誕生を阻害し、市場の組織化が進まない要因ともなった。イギリスの銀行は本来商業銀行として発達したため産業との結びつきが弱く大企業化を主導できなかったこと、イギリスにおける独占の発展はのれんわけによる創業者利益狙いの合併によるところが大きかったため、旧来の経営方式から直接管理への移行が遅れたこともその要因である。一方で労使関係の面では、元からクラフトユニオンの力が強かったイギリスにおいては第1次大戦前に労働組合法と争議法が成立し、団結権と争議権獲得への一歩を踏み出すことができている。

\subsection{第1次と第2次:フランス}
第1次キーワード
\begin{itemize}
\item フランス革命
\item ルシャ・プリエ法
\item 復古王政期
\item クレディモビリエ
\item サン・シモン会議
\end{itemize}
 フランス革命での封建制廃止は徹底的に行われた。そこで行われた土地の再配分は零細農を大量に誕生させた。結果として国内市場は未発達のまま都市部への労働力供給も起こらず産業革命の発展を阻害することになった。封建制の廃止は中間団体の廃止に関しても徹底された。ルシャ・プリエ法でによってギルド解体と株式会社の禁止が決定する。これは産業革命の進行とともに見直されていくことになる。復古王政(1814~1830)期に産業革命は本格化し、株式会社の合法化後は株式投資銀行のクレディモビリエが活躍した。これらは産業発展を意識的に追及するサン・シモン主義の下で国家によって進められた。

第2次キーワード
\begin{itemize}
\item 普仏戦争(1870)
\item 大企業と国内市場
\item 中小企業と輸出
\item 第3共和制
\item 4大銀行
\item 事業銀行
\item アンクタット/コートワール
\item サンディカリズム
\end{itemize}
 普仏戦争での敗北で工業の中心地であったアルザス・ロレーヌを奪われたフランスは少子化などの人口動態の変化も受けて第2次産業革命に立ち後れる。石炭などの資源の枯渇もあり新産業の対外競争力は弱く大企業は国内市場においてしか存在感を発揮できなかった。一方で伝統的な手工業に基づく奢侈品を生産する中小企業は対外競争力が強く、輸出を行うようになった。故にクレディモビリエがありながら産業と金融とのつながりは弱いままとなる。この流れを受けて、第3共和制期に生まれた4大銀行も預金銀行としての業務が主であり、海外投資は行うものの国内市場とは遊離していた。フランスでの市場の組織化は中小企業を中心としたアンタットと呼ばれるカルテルが担い、これは大企業への対抗として発展した。したがって労働組合の組織率は低く、サンディカリズムなどの急進派と穏健派が混合したものとなっていた。

\subsection{第1次と第2次:ドイツ}
第1次キーワード
\begin{itemize}
\item 2重革命
\item ユンカー経営
\item イヌング
\item ドイツ関税同盟
\item ルール工業地帯
\item プロイセン
\item 特殊ドイツ型銀行
\item 歴史学派経済学
\end{itemize}
 フランス革命による近代化とイギリス産業革命による工業化の2重革命を受けて近代化が急速に進展する。その一つとして封建的な土地所有が廃止される。西部では有償廃業だったので零細農家が大量に発生、東部では分割方式だったため独立できず有力者への寄進が相次いだことでのちのプロイセン支配階級であるユンカーを生み出した。結果、東部においては裁判領主制と体僕領主制は維持される半封建的土地所有が存続した。ドイツにおいてはギルドに変わってイヌングによる職種の独占が存続しており、独占禁止や団結禁止の局面を経なかったことも特徴的である。国民市場もイギリスのように整備されていなかったので、ドイツ関税同盟という枠組みの中で成立が模索され、イギリスではすでに存在しその上で市場が成熟した鉄道もその市場形成のために敷設され、この結果ルール工業地帯が形成された。プロイセンによる軍事的な統一が行われる過程で軍需が増し、産業発展に寄与した。また、ドイツにおいては投資銀行が企業形成に大きく寄与し、株式会社も発達した。国内産業の弱いドイツは保護貿易主義を採るわけだが、その思想的な背景が歴史学派経済学であり、制度によってイギリスの経験した階層分化を回避しようとした。

第2次キーワード
\begin{itemize}
\item ドイツ帝国
\item ベルリン工科大学
\item ルール工業地帯
\item クルップ
\item ドイツ型銀行(主導性)
\item 生産材
\item カルテル
\item 家父長制的企業経営
\item 社会民主主義労働党
\end{itemize}
 ドイツ帝国は普仏戦争後の1871年に成立。ベルリン工科大学の設立をはじめとして技術と産業の結びつきを強く意識したこと、また第1次産業革命からの伝統で銀行と企業の結びつきが強く、交互計算業務、役員派遣、株式取引などを通して銀行主導によるカルテル形成(クルップ)が盛んに行われたことなどを背景に第2次産業革命をリードする立場となった。カルテルは生産材生産業において行われ、共倒れを防げることや国家と企業が結びついていたことなどを背景にカルテルには肯定的な態度が取られた。このようにして生まれた大企業においては家父長制的企業経営が行われ、福利厚生政策による労働者の怪獣が行われた。このため社会民主主義労働党の血糖などを経ても大企業まで労組の影響力が及ぶことはなく、中傷で飲み団結権が認められていた。

\subsection{第1次と第2次:アメリカ}
第1次キーワード
\begin{itemize}
\item 地域的分立
\item 州権主義
\item ハミルトン体制
\item 南北戦争(1865年)
\item アメリカ式工業生産体制
\item 独立自営農
\item 自己金融
\end{itemize}
 封建制が欠如し社会的分業が発達した保護貿易を求める東部、イギリス綿工業と結びついた奴隷制プランテーションに基づき自由貿易を求める南部、開拓地であり自営農による農業、牧畜が営まれる西部とに分立していた。州県主義が強く、これらを統合しようとしたハミルトン体制などはうまくいかない。19世紀に入り中西部で鉄機械工業が栄えた結果東部と西部が結びつき、南北戦争に勝利する。熟練労働者の不足、均質で広大な市場と高水準な賃金、人口が分散しているが故に簡単に修理できることが求められたことなどを背景に互換性部品を用いた均一な製品を大量生産するアメリカ式工業生産体制が確立する。労働環境としては、西部開拓が進行中である故に自営業化が頻発し、自分自身が労働者であるという意識が薄かったこと、また移民であるが故にアメリカ人としての意識も薄かったことから階級意識が薄く、労働運動の展開は他国に比べ遅れることになる。

第2次キーワード
\begin{itemize}
\item 全国的な金融市場(ニューヨーク金融市場)
\item ビッグビジネス
\item 消費財
\item カルテル/トラスト
\item シャーマン反トラスト法
\item 投資銀行
\item 機能別管理機構
\item ピラミッド型階層組織
\item ホワイトカラー
\item テイラー主義
\item AFL
\item 福祉資本主義
\end{itemize}
 南北戦争終了後に拡大市場や都市化の進行、大陸横断鉄道などの交通インフラの整備、ニューヨーク金融市場が統一された全国規模の金融市場として機能し始めたことなどを背景にビッグビジネスの準備が整う。標準化と機械化を成し遂げた消費財企業が統一された金融市場を背景に誕生した投資銀行の下で成長し19世紀の後半には多数の独占企業が誕生する。また恐慌以降その対策としてカルテルが横行し問題となったため、1890年にシャーマン反トラスト法が施行される。しかしこれは持ち株会社による独占の発生を招いただけであった。このような大企業は科学的管理法を重視しピラミッド型階層組織の中で直接管理が徹底されることになる。その中で新中間層として大衆文化を担うホワイトカラーが誕生する。こういった動きを受けてアメリカでも労使対立が発生する。1886年にAFLが組織され団結権を求めて争ったが、大企業の福利厚生を整える福祉資本主義にの中で労組は排除されていった。

\subsection{第1次と第2次:マクロな視点から}
古典的世界市場(19c半ばまで)とその崩壊
\begin{itemize}
\item 世界の工場
\item 英仏通商条約
\item 自由貿易体制の確立
\item ロンドン金融市場
\item 鉄道債
\item 資本輸出(インフラ投資)
\item モノカルチャー
\end{itemize}
 世界の工場として圧倒的な生産力を誇るイギリスを中心とした貿易体制は後発先進国の産業革命が進行する中で崩壊し、徐々に周辺地域に開国と自由貿易を求めうるようになる。1865年の英仏通商条約の締結はその流れを象徴するものである。これによってフランスを始めとする後発先進国は自発的に、周辺国は強制的に自由貿易を開始して自由貿易体制が確立した。この体制で流通したのはイギリスの発行する手形であり、その決済が行われるロンドンは世界の金融の中心になっていく。この中で各国の公債や鉄道債もロンドンで発行されるようになりイギリスからの資本輸出が大量に行われた。これは各国のインフラ投資に使われ長期的に見ればイギリスの優位性を損なうことになった。ただし、後発先進国では産業革命の準備となった鉄道が周辺国ではモノカルチャー経済の基盤としかならなkったことには注意が必要だ。

大不況(1873~1896)を経て古典的世界市場が崩壊。さらに第2次産業革命を経験した世界で多角的貿易決済機構が誕生
\begin{itemize}
\item 資金循環
\item 国際的金本位体制
\item 帝国関税同盟
\item チェンバレンキャンペーン
\item レントナー化
\end{itemize}
 銀銅山開発を受けて19世紀末には国際的な金本位体制が確立していた。これは国際取引を円滑にし、イギリスの金融センターとしての地位を確固たるものとした。逆にこれはイギリス国内の工業の地位を低下させ、20世紀初頭にはこの対立が総選挙で争われた。チェンバレンキャンペーンと呼ばれる保守的な関税導入と自由貿易の対立である。当時のイギリスは貿易収支が大幅な赤字、貿易外収支が大幅な黒字というレントナー化しており、結局自由貿易派が勝利する。これに対し、後発先進国は恐慌を受けての保護貿易主義を保っており、イギリスは悪質な製品を周辺国に、周辺国の1次産品が後発先進国に、後発先進国の良質な製品がイギリスへと輸出される資金循環が確立することになった。


\section{戦間期}
第1次世界大戦での総力戦は財政規模の不可逆的な拡大をもたらし、世界はロシア革命を背景に国家が市場を組織化することを求めるようになる。一方で、戦時協力の報酬として存在を認められた労組が力を増し、さらにアメリカ的な生産様式の追及から各国で合理化が求められたことも相まって労働争議が戦後活発化した。
\subsection{戦間期:マクロな視点から}
キーワード
\begin{itemize}
\item 国際金融の役割分担
\item 国際資金循環
\item 賠償問題
\item ドーズ案(1924)
\item 相対的安定期
\item アメリカ株式ブーム
\item 近隣窮乏化政策
\end{itemize}
 大戦を経て債権国となったアメリカとイギリスの間で、米国は長期金融を、英国は第3国間取引を、といったような国際金融取引の役割分担が行われるようになる。これは国際決済を非効率化させ、またアメリカの高自給率と圧倒的な工業生産力は国際貿易のバランスを崩し、資金循環がうまくいかなくなった。この背景にはまたドイツからの賠償が滞っていること、また英仏の米への借金返済がドイツからの賠償に頼っていることがあった。1924年に根本的な解決策としてドーズアンが採用、ドーズ債によってドイツへの貸付が行われた。これによって資金循環が回復し、相対的安定期が到来する。しかし1928年のアメリカ株式ブームで外国からの資産引き上げが始まり、それに対応して金融引き締めを行った各国からさらに資産が引き上げられるというスパイラルが起こる。1929年に株式が暴落すると各国は保護貿易に走り近隣窮乏化政策の中でブロック経済化していった。

\subsection{戦間期:イギリス}
相対的安定期キーワード
\begin{itemize}
\item ポンド高
\item 合同運動
\item 経済構造の高度化
\item 合理化
\item 労働争議法、労働法改正(ムチ)
\item モンド・ターナー法(アメ)
\end{itemize}
 工業力が衰退し海外資産も失ったイギリスでは不況に対抗するために合併が進行し独占化が進んだ。これを合同運動と呼ぶ。しかし体系化された研究開発機構が未発達であったイギリスでは合理化運動や産業構造の高度化にはこの動きが結びつかなかった。大戦中に力をつけた労組は賃下げと合理化に対抗し労使対立が激化した。先に述べたように企業の力があまり強くならなかったのでこの運動はある程度成功し、1920年代末から労働争議法や労働法の改正による労組運動の制限とともにモンド・ターナー法により対話が実現し労使協調へと動き出した。

恐慌後キーワード
\begin{itemize}
\item マクドナルド政権
\item スターリングブロック
\item 金本位制離脱
\end{itemize}
 他国に比して打撃は弱かった。マクドナルド政権は公共投資を行わず、輸入関税法の制定などを通して保護貿易かを進展させた。金本位制も離脱してスターリングブロックを形成。

\subsection{戦間期:フランス}
相対的安定期キーワード
\begin{itemize}
\item フランス労働総同盟(CGT)
\item CGT分裂
\item 合理化運動が進まない
\end{itemize}
 ドイツに対しての後進性を強く認識したフランスは合理化運動を進めるが、大戦で力をつけたCGTが戦後急進化し激しい労使対立となった。しかし1922年にCGTが社会主義と共産主義で分裂する。相対的安定期には変換されたアルザス・ロレーヌを基盤に新産業が勃興し、そこにおいての合理化が進むが、経済全体としては労働者の抵抗がありまた国内市場があればいいので今ひとつ進まなかった。

恐慌後キーワード
\begin{itemize}
\item 人民戦線内閣
\item 購買力政策
\item 小麦局
\item 金本位制離脱
\item アンタント
\item 自由主義経済と統制経済の中間
\end{itemize}
 反ファシズムで人民戦線内閣が結成される。この下で恐慌対策がとられ、有効需要喚起のために賃上げや団結権の強化を行った購買力政策、農産物について価格規制と生産調整を行った小麦局の設置、フランの切り下げを求めての金本位制離脱などを行った。イギリス同様大規模な公共投資を打てなかったという意味で均衡財政主義にとらわれていた。結果として構造改革無き再分配となり生産性拡大には結びつかなかった。また恐慌の中でアンタントによるカルテルを共倒れ阻止のために認めたが、中小企業は維持し、自由主義経済と統制経済の中間的な経済体制をとった。

\subsection{戦間期:ドイツ}
相対的安定期キーワード
\begin{itemize}
\item ワイマール体制
\item 政治的不安定
\item ルール占領
\item 産業合理化運動
\item 地位を確立した労組
\item 労組内対立
\item 失業問題と二極化
\item 新中間層
\end{itemize}
 1919年に成立したワイマール体制は社会権や普通選挙の実現といった点で極めて先進的であったが、その支持基盤は社会民主党をはじめとする左右両派から攻撃される立場であり政治的には不安定であった。そんな中ルール占領を受けハイパーインフレ陥ったがドーズアンで救済された。相対的安定期においてドイツ国内の投資が盛んとなり米式の大量生産様式の導入が進み産業合理化運動が進展する。その中で企業集中が加速した。ワイマール体制では労組も団結権と交渉権を与えられ地位を確立していたが、合理化が進展する中で使用者側は攻勢に出て失業問題が深刻化する。アメリカと違い生産財生産が工業の中心となったドイツでは合理化の恩恵として新中間層を形成できたのはごくわずかな労働者のみであり、彼らと失業の危機にさらされる労働者の間で対立が深丸。後者は共産主義の支持基盤となっていった。このような中で28年にはすでに貿易も滞っており、29年の大恐慌を迎える。

恐慌後キーワード
\begin{itemize}
\item ブリュニック内閣
\item 財政均衡主義
\item MEFO手形
\item 強制同質化
\item ドイツ労働戦線
\item 対外膨張
\end{itemize}
 ブリュニック内閣は財政均衡主義にとらわれ公共投資を打てなかった。これに代わったナチは均衡財政主義にとらわれないアウトバーンなどの公共支出、再軍備、軍備拡張を行う中で失業問題を解決した。また強制同質化の中で労組を解散させ他の職業団体と共にドイツ労働戦線にまとめ上げる。これによって実質賃金の切り下げを実現しアメリカが直面した企業利潤の追求との矛盾を解決した。これが公共投資だけでなく民間投資も呼び込み景気回復に成功する。しかし、MEFO手形の異常発行によるインフレも相まり経済は加熱しすぎる。これに対応する形で体外拡張を推し進めることになった。

\subsection{戦間期:アメリカ}
相対的安定期キーワード
\begin{itemize}
\item 債権国化
\item 戦時産業局
\item 全国戦時労働委員会
\item 農作物価格下落
\item 資本輸出国/多国籍企業
\item 新中間層
\item 大衆消費社会
\item AFL
\item WWI
\item 労使強調路線
\item 福祉資本主義
\end{itemize}
 労組のなかった米において戦時中に組織された労働市場安定のためのAFLが初めての労組となる。大戦後は債権国かと資本輸出による大企業の多国籍企業化も相まって経済的繁栄の下大衆消費社会が花ひらく。その中で耐久消費財の普及や多様な商品の浸透が起こり一人あたり所得も上昇する。大戦時に増産された農産供物が余ったために農作物価格が下落したことも受けて労働者の実質賃金が上昇したことになる。これは大企業におけるホワイトカラーを新中間層として大衆社会の担い手として準備した。大戦直後は急進化したAFLも好況を背景とした福祉資本主義の下で懐柔され、社会問題が小さいまま恐慌を迎える。

恐慌後キーワード
\begin{itemize}
\item フーバー政権
\item 金融恐慌
\item ニューディール政策
\item 農業調整法
\item 農業金融政策
\item 全国産業復興法(NIRA)
\item 財政均衡主義
\item CIO
\end{itemize}
 29年の株式暴落に対してフーバー政権が好況政策の拡充を打ったが、歯止めが効かなかった。さらに30年半の金融恐慌を受けて関税引き上げや銀行の救済、ペコラ委員会の組織などを行ったが、財政均衡主義によって決定打が打てない。33年にルーズベルトに敗北し、ニューディール政策が開始される。ルーズベルトの支持基盤はベルエポックで取り残された農民と非熟練労働者であった。この両者に向けて政策を打っていくことになる。まず農業政策として、価格と生産量の調整を行う農業調整法と、大戦時に進められた機械化による負債の解消を行う農業金融政策を打つ。しかしここで一つ目の矛盾として、農業価格の上昇と実質賃金の上昇の両立が出来ないという現実に直面する。また、労働者への施策として、全国産業復興法を制定する。この中で、価格維持のためのカルテル容認と労働者への生活可能な賃金の保障のために労組に団結権と交渉権を付与し、また最低賃金や労働時間規制も行った。しかしこの「適正利潤」と「生活可能な賃金」は矛盾をきたす。これが第2の矛盾となった。公共投資についても財政均衡主義が阻み、以上の矛盾から民間投資も伸び悩んだ。これによって1937年いは不況を迎えることになる。
 以前行われていた福祉資本主義は大恐慌の下で崩壊し、33年以降労働争議が頻発する。35年にNIRAが違憲判決で廃止され、そんな中でAFLから分派したCIOが勢力拡大するが、団結権や交渉権の獲得にはいたらないまま第2次大戦へと入っていく。

\end{document}





















