\documentclass[dvipdfmx, 12pt]{beamer}
\usepackage{pxjahyper}
\usepackage{minijs}
\usepackage{otf}
\renewcommand{\kanjifamilydefault}{\gtdefault}
\usetheme{Antibes}
\setbeamertemplate{navigation symbols}{}
\usepackage{url}
\usepackage{graphicx}
\usepackage{amsmath}
\usepackage{bm}
\usepackage{ascmac}
\setbeamertemplate{footline}[frame number] 

\title{卒論中間発表}
\author{池上 慧}

\begin{document}
\newcommand{\argmin}{\mathop{\rm arg~min}\limits}

\frame{\maketitle}

\section*{目次}
\begin{frame} \frametitle{発表の流れ}
\tableofcontents
\end{frame}

\section{目的}
\begin{frame}\frametitle{研究対象}
	\begin{itemize}
		\item 混雑現象の構造推定
		\item 混雑現象:プレイヤーは選択肢のうちどれかを必ず選ばなければならず、かつ選んだ選択肢を選んだ人数が多いほど効用が下がる状況を指す。
	\end{itemize}
\end{frame}

\begin{frame}\frametitle{具体例}
	\begin{itemize}
		\item 満員電車
		\item 私立大学の入試日決定
		\item テーマパークへの来場日決定
		\item 流行
	\end{itemize}
\end{frame}

\begin{frame}\frametitle{先行研究}
	\begin{enumerate}
		\item \textcolor{red}{Viauroux (2007)}
		
		バスと車の移動手段選択における混雑現象の影響を推定。個人が混雑回避的行動を取っていることを明らかにした。
		\item \textcolor{red}{柳沼・福田(2007,8)}
		
		Viauroux (2007)を定式化し直し、シミュレーションデータで推定が行えることを確認した。
		\item \textcolor{red}{松村他(2009)}
		
		乗車する電車の選択における混雑モデルを推定。混雑に対する不効用を個人が持つことを明らかにした。
	\end{enumerate}
\end{frame}

\begin{frame}\frametitle{先行研究の課題と問題意識}
	\begin{itemize}
		\item 何のデータも個人の意思決定に関するデータを用いている。
		\item 通勤時間などのデータについては他のデータソースから流用している。
		\item 推定手法のPMLは収束が保障されない。
	\end{itemize}
	
	\begin{itembox}[l]{着眼点}
	より簡単にアグリゲートされたデータから混雑現象の構造推定はできないか。
	\end{itembox}
\end{frame}


\section{モデル}
\begin{frame}\frametitle{}
	\begin{itemize}
		\item 
		\item 
	\end{itemize}
\end{frame}

\begin{frame}\frametitle{}
	\begin{itemize}
		\item 
		\item 
	\end{itemize}
\end{frame}

\section{実証}
\begin{frame}\frametitle{}
	\begin{itemize}
		\item 
		\item 
	\end{itemize}
\end{frame}

\begin{frame}\frametitle{}
	\begin{itemize}
		\item 
		\item 
	\end{itemize}
\end{frame}

\end{document}



























