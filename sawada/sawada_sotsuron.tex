\documentclass[dvipdfmx, 12pt]{beamer}
\usepackage{pxjahyper}
\usepackage{minijs}
\usepackage{otf}
\renewcommand{\kanjifamilydefault}{\gtdefault}
\usetheme{Antibes}
\setbeamertemplate{navigation symbols}{}
\usepackage{url}
\usepackage{graphicx}
\usepackage{amsmath}
\usepackage{bm}
\usepackage{ascmac}
\setbeamertemplate{footline}[frame number] 

\title{卒論中間発表}
\author{池上 慧}

\begin{document}
\newcommand{\argmin}{\mathop{\rm arg~min}\limits}

\frame{\maketitle}

\section*{目次}
\begin{frame} \frametitle{発表の流れ}
\tableofcontents
\end{frame}

\section{目的}
\begin{frame}\frametitle{研究対象}
	\begin{itemize}
		\item 混雑現象の構造推定
		\item 混雑現象:プレイヤーは選択肢のうちどれかを必ず選ばなければならず、かつ選んだ選択肢を選んだ人数が多いほど効用が下がる状況を指す。
	\end{itemize}
\end{frame}

\begin{frame}\frametitle{具体例}
	\begin{itemize}
		\item 満員電車
		\item 私立大学の入試日決定
		\item テーマパークへの来場日決定
		\item 流行
	\end{itemize}
\end{frame}

\begin{frame}\frametitle{先行研究}
	\begin{enumerate}
		\item \textcolor{red}{Viauroux (2007)}
		
		バスと車の移動手段選択における混雑現象の影響を推定。個人が混雑回避的行動を取っていることを明らかにした。
		\item \textcolor{red}{柳沼・福田(2007,8)}
		
		Viauroux (2007)を定式化し直し、シミュレーションデータで推定が行えることを確認した。
		\item \textcolor{red}{松村他(2009)}
		
		乗車する電車の選択における混雑モデルを推定。混雑に対する不効用を個人が持つことを明らかにした。
	\end{enumerate}
\end{frame}

\begin{frame}\frametitle{先行研究の課題と問題意識}
	\begin{itemize}
		\item 何のデータも個人の意思決定に関するデータを用いている。
		\item 通勤時間などのデータについては他のデータソースから流用している。
		\item 推定手法のPMLは収束が保障されない。
	\end{itemize}
	
	\begin{itembox}[l]{着眼点}
	より簡単にアグリゲートされたデータから混雑現象の構造推定はできないか。
	\end{itembox}
\end{frame}


\section{モデル}
\begin{frame}\frametitle{モデルのポイント}
	\begin{itemize}
		\item 「何割のプレイヤーがどの選択肢を選んだか」のデータと選択肢についてのデータのみを使用して推定できる。
		\item 個人についてのheteroeneityを推定に利用する。
		\item わかりやすいのでラッシュ時の電車選択を考える。
		\item まだ2選択肢についてしかできてない。
	\end{itemize}
\end{frame}

\begin{frame}\frametitle{設定}
	\begin{itemize}
		\item $M$駅の$T$日分の乗客数データ
		\item $m \in \left\{1\cdots M \right\}$で1つの駅を示すとして、そこで乗車する総人数は$N_m$
		\item $X_i$はプレイヤーの属性を含むベクトル
		\item $Z_j$は電車$j$の属性を含むベクトル
		\item $\eta_j^i$はプレイヤーごとに持つ電車$j$に対する観測不可能な選好(public information)
		\item $B_j^m$は駅$m$における電車$j$への実現した乗車人数
		\item $\epsilon_j^i$はプレイヤーが私的情報として持つ電車$j$への選好
	\end{itemize}
\end{frame}

\begin{frame}\frametitle{効用}
	\begin{itembox}[l]{駅$m$を利用する個人$i$の電車$j$に対しての効用}
		$X_i^{'}\beta + Z_j^{'}\gamma + \eta_j^i + \alpha B_j^m + \epsilon_j^i$
	\end{itembox}
\end{frame}

\begin{frame}\frametitle{ベイジアンナッシュ均衡}
	\begin{itemize}
		\item $\epsilon_j^i$は$i, j$それぞれに対して独立に同じ第1種極値分布に従う
		\item rational expectationの仮定の下では以下の方程式を満たすように毎日均衡としての電車1の期待乗車割合$(E[b_{t,m}^{N_m}])$が決定する
		\item $d_i^t = \eta_2^i - \eta_1^i$とする
	\end{itemize}
	\begin{itembox}[l]{駅$m$における$t$日の均衡}
		\begin{align*}
		\scalebox{1}{$
                	E[b_{t,m}^{N_m}] = \frac{1}{N_m} \sum_{i = 1}^{N_m} \frac{1}{1 + \exp\left( (z_2^m - z_1^m)^{'}\gamma + d_{i.m}^t + \alpha N \left(1 - 2E[b_{t,m}^{N_m}]\right) \right)} $}
                \end{align*}
	\end{itembox}
\end{frame}

\begin{frame}\frametitle{考え方}
	\begin{itemize}
		\item データとして得られた毎日の電車1の乗車割合は先の均衡である
		\item 日々の均衡として得られた電車1の乗車割合をたくさん集めたものの平均$(\widehat{E[b_m^{N_m}]})$は$t$によらないモデルの均衡に収束するはず(要証明)
		\item $t$によらないモデルの均衡を出すためにpublic informationである個人のheterogeneityが従う分布$F(\cdot)$を仮定する
		\item この操作は最尤推定量のアナロジー
	\end{itemize}
\end{frame}

\begin{frame}\frametitle{$t$によらないモデルの均衡}
	駅$m$を利用する個人$i$が電車$1$を利用する確率を離散選択の定式化で求めたものを$p_{1,m}^i$で記す。この時以下が成立する。
        \begin{align*}
        	p_{1,m}^i > \frac{1}{2} &\Leftrightarrow \frac{1}{1 + \exp\left( (z_2^m - z_1^m)^{'}\gamma + d_i + \alpha N \left(1 - 2E[b_m^{N_m}]\right) \right)} > \frac{1}{2}\\[8pt]
        	&\Leftrightarrow d_i < \alpha N_m \left(2E[b_m^{N_m}] - 1\right) - (z_2^m - z_1^m)^{'}\gamma
        \end{align*}
ただし、$E[b_m^{N_m}]$は今知りたいモデルの均衡として求まる乗車割合である。
\end{frame}

\begin{frame}\frametitle{$t$によらないモデルの均衡}
	ここから即座に以下を得る。
        \begin{align*}
        	E[b_m^{N_m}] = \frac{1}{2} + \frac{(z_2^m - z_1^m)^{'}\gamma + F^{-1} \left( E[b_m^{N_m}] \right)}{2\alpha N_m}
        \end{align*}
これより、分布$F(\cdot)$がサポートが有限で、$\left\{ z_j \right\}$が有限のサポートを持ちかつパラメータ$\gamma$も有限であるなら、$N \to \infty$で$E[b_m^{N_m}] \to \frac{1}{2}$であることがわかる。
\end{frame}

\begin{frame}\frametitle{帰結}
	以上の主張が下で、駅$m$について$T,N_m$が十分に大きい時に以下が成立する。
	
	\begin{itembox}[l]{モデルの帰結}
        \begin{align*}
        \scalebox{1}{$
        	\frac{1}{2} = \frac{1}{T}\sum_{t = 1}^T \frac{1}{N_m}\sum_{i = 1}^{N_m} \frac{1}{1 + \exp\left( (z_2^m - z_1^m)^{'}\gamma + d_{i,m}^t + \alpha N \left(1 - 2E[b_{t,m}^{N_m}]\right) \right)} $}
        \end{align*}
        \end{itembox}
\end{frame}

\section{推定}
\begin{frame}\frametitle{推定の手法}
	\begin{itemize}
		\item MSMのようなことをする
		\item heterogeneityの分布についても容易にパラメータ推定できるようにimportance samplingを用いる(Ackerberg 2009)
	\end{itemize}
\end{frame}

\begin{frame}\frametitle{推定の準備}
	\begin{itemize}
		\item $d_{i,m}^t$の分布をベータ分布$(Beta(a, b))$である
		\item $Beta(2,2)$をproposal distributionとするimportance samplingを用いる
		\item 乱数の発生回数を$R$とする
		\item $R$を増やすことは十分大きな$N_m$の下では結果的に$N_m$をより大きくするような働きをするので先の条件が成立するために必要な環境を整えることになる。(要証明)
	\end{itemize}
\end{frame}

\begin{frame}\frametitle{推定}
	駅$m$において満たすべき式は、$S_m = N_m R$として、以下のように書ける。
	
	\begin{align*}
	\scalebox{1}{$
	\iota_{s, m} \equiv \frac{1}{1 + \exp\left( (z_2^m - z_1^m)^{'}\gamma + x_{s,m}^t + \alpha N \left(1 - 2E[b_{t,m}^{N_m}]\right) \right)} \frac{\left(x_{s,m}^t + \frac{1}{2}\right)^{a - 2} \left(\frac{1}{2} - x_{s,m}^t \right)^{b-2}}{6B(a, b)} $}
	\end{align*}
	として、
	\begin{align*}
	\scalebox{1}{$
	\delta_m \equiv \frac{1}{2} - \frac{1}{T}\sum_{t = 1}^T \frac{1}{S_m}\sum_{s = 1}^{S_m} \iota_{s, m} = 0 $}
	\end{align*}
\end{frame}

\begin{frame}\frametitle{}
	\begin{itemize}
		\item 
		\item 
	\end{itemize}
\end{frame}

\end{document}



























