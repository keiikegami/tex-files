\documentclass[dvipdfmx, 12pt]{beamer}
\usepackage{pxjahyper}
\usepackage{minijs}
\usepackage{otf}
\renewcommand{\kanjifamilydefault}{\gtdefault}
\usetheme{Antibes}
\setbeamertemplate{navigation symbols}{}
\usepackage{url}
\usepackage{graphicx}
\usepackage{amsmath}
\usepackage{bm}
\usepackage{ascmac}
\setbeamertemplate{footline}[frame number] 

\title{卒論中間発表}
\author{池上 慧}

\begin{document}
\newcommand{\argmin}{\mathop{\rm arg~min}\limits}

\frame{\maketitle}

\section*{目次}
\begin{frame} \frametitle{発表の流れ}
\tableofcontents
\end{frame}

\section{目的}
\begin{frame}\frametitle{研究対象}
	\begin{itemize}
		\item 混雑現象の構造推定
		\item 混雑現象:プレイヤーは選択肢のうちどれかを必ず選ばなければならず、かつ選んだ選択肢を選んだ人数が多いほど効用が下がる状況を指す。
	\end{itemize}
\end{frame}

\begin{frame}\frametitle{具体例}
	\begin{itemize}
		\item 満員電車
		\item 私立大学の入試日決定
		\item テーマパークへの来場日決定
		\item 流行
	\end{itemize}
\end{frame}

\begin{frame}\frametitle{先行研究}
	\begin{enumerate}
		\item \textcolor{red}{Viauroux (2007)}
		
		バスと車の移動手段選択における混雑現象の影響を推定。個人が混雑回避的行動を取っていることを明らかにした。
		\item \textcolor{red}{柳沼・福田(2007,8)}
		
		Viauroux (2007)を定式化し直し、シミュレーションデータで推定が行えることを確認した。
		\item \textcolor{red}{松村他(2009)}
		
		乗車する電車の選択における混雑モデルを推定。混雑に対する不効用を個人が持つことを明らかにした。
	\end{enumerate}
\end{frame}

\begin{frame}\frametitle{先行研究の課題と問題意識}
	\begin{itemize}
		\item 何のデータも個人の意思決定に関するデータを用いている。
		\item 通勤時間などのデータについては他のデータソースから流用している。
		\item 推定手法のPMLは収束が保障されない。
	\end{itemize}
	
	\begin{itembox}[l]{着眼点}
	より簡単にアグリゲートされたデータから混雑現象の構造推定はできないか。
	\end{itembox}
\end{frame}


\section{モデル}
\begin{frame}\frametitle{モデルのポイント}
	\begin{itemize}
		\item 「何割のプレイヤーがどの選択肢を選んだか」のデータと選択肢についてのデータのみを使用して推定できる。
		\item 個人についてのheteroeneityを推定に利用する。
		\item わかりやすいのでラッシュ時の電車選択を考える。
		\item まだ2選択肢についてしかできてない。
	\end{itemize}
\end{frame}

\begin{frame}\frametitle{設定}
	\begin{itemize}
		\item $M$駅の$T$日分の乗客数データ
		\item $m \in \left\{1\cdots M \right\}$で1つの駅を示すとして、そこで乗車する総人数は$N_m$
		\item $X_i$はプレイヤーの属性を含むベクトル
		\item $Z_j$は電車$j$の属性を含むベクトル
		\item $\eta_j^i$はプレイヤーごとに持つ電車$j$に対する観測不可能な選好(public information)
		\item $B_j^m$は駅$m$における電車$j$への実現した乗車人数
		\item $\epsilon_j^i$はプレイヤーが私的情報として持つ電車$j$への選好
	\end{itemize}
\end{frame}

\begin{frame}\frametitle{効用}
	\begin{itembox}[l]{駅$m$を利用する個人$i$の電車$j$に対しての効用}
		$X_i^{'}\beta + Z_j^{'}\gamma + \eta_j^i + \alpha B_j^m + \epsilon_j^i$
	\end{itembox}
\end{frame}

\begin{frame}\frametitle{ベイジアンナッシュ均衡}
	\begin{itemize}
		\item $\epsilon_j^i$は$i, j$それぞれに対して独立に同じ第1種極値分布に従う
		\item rational expectationの仮定の下では以下の方程式を満たすように毎日均衡としての乗車人数の期待値$(E[b_{t,m}^{N_m}])$が決定する
		\item $d_i^t = \eta_2^i - \eta_1^i$とする
	\end{itemize}
	\begin{itembox}[l]{駅$m$における$t$日の均衡}
		\begin{align*}
		\scalebox{1}{$
                	E[b_{t,m}^{N_m}] = \frac{1}{N_m} \sum_{i = 1}^{N_m} \frac{1}{1 + \exp\left( (z_2^m - z_1^m)^{'}\gamma + d_{i.m}^t + \alpha N \left(1 - 2E[b_{t,m}^{N_m}]\right) \right)} $}
                \end{align*}
	\end{itembox}
\end{frame}

\section{実証}
\begin{frame}\frametitle{}
	\begin{itemize}
		\item 
		\item 
	\end{itemize}
\end{frame}

\begin{frame}\frametitle{}
	\begin{itemize}
		\item 
		\item 
	\end{itemize}
\end{frame}

\end{document}



























