\documentclass{jsarticle}
\usepackage[margin = .7in]{geometry}
\usepackage[dvipdfmx]{graphicx}
\usepackage{listings}
\usepackage{amsmath}
\usepackage{bm}
\usepackage{ascmac}
\lstset{%
  language={python},
  basicstyle={\small},%
  identifierstyle={\small},%
  commentstyle={\small\itshape},%
  keywordstyle={\small\bfseries},%
  ndkeywordstyle={\small},%
  stringstyle={\small\ttfamily},
  frame={tb},
  breaklines=true,
  columns=[l]{fullflexible},%
  numbers=left,%
  xrightmargin=0zw,%
  xleftmargin=3zw,%
  numberstyle={\scriptsize},%
  stepnumber=1,
  numbersep=1zw,%
  lineskip=-0.5ex%
}

\begin{document}
\title{院試まとめ4}
\author{池上慧}
\maketitle

\section{渡辺マクロまとめ}
\subsection{ホニャホニャデフレーター}
GDPデフレーターは以下のように定義される。

$GDP-deflator = \frac{GDP_n}{GDP} = P_C\frac{C}{GDP} + P_I\frac{I}{GDP} + P_G\frac{G}{GDP} + P_{EX}\frac{EX}{GDP} + P_{IM}\frac{IM}{GDP}$

ここで各項をそれぞれの要素についてのデフレーターという。例えば、輸入デフレーターは$\P_{IM}\frac{IM}{GDP}$である。
輸入品価格の上昇は上よりGDPデフレーターを低下させる。

\subsection{伸縮価格モデル}
古典派の二分法より、$Y$は供給サイドの利潤最大化より需要とは無関係に決定されるものである。この時、貯蓄である$S = Y-C$は実質利子率$r$に対して不変である。この状況では政府支出の増加に対する投資のクラウディングアウトがちょうど政府支出増加の直接的なGDPへの貢献を消すだけ働くので、政府支出では均衡GDPを増やすことはできない。

このモデルを改良し、$r,Y$が同時に決定するようにして、$G$の変更で$Y$が変化するようにしたのがIS-LMであり、ケインズの貢献である。

\subsection{フィリップス曲線の導出}
総供給曲線は「価格の事前契約モデル」、「予測誤差モデル」、「名目賃金の下方硬直性」などの要素、モデルからそれぞれ導ける。ここでは「価格の事前契約モデル」で導いた総需要曲線を用いてフィリップス曲線を導出する。

まず「価格の事前契約モデル」から導ける総供給曲線は、
\begin{align}
	{\rm log} P_t = E_{t-1} {\rm log}P_t + \frac{w}{1 -w} a ({\rm log}Y_t - {\rm log}Y^n)
\end{align}
である。ただし、$Y^n$はt期の潜在GDPを示している。

$k = \frac{w}{1 -w} a$として、両辺から${\rm log}P_{t-1}$を引くことで、インフレ率の定義より以下の関係式を得る。
\begin{align}
	\pi_t = E_{t-1}\pi_t + k ({\rm log}Y_t - {\rm log}Y^n)
\end{align}

ここで、オークンの法則、すなわち産出量ギャップと失業率との負の相関関係、${\rm log}Y_t - {\rm log}Y^n \sim -(u_t - u^n)$を代入して、
\begin{align}
	\pi_t = E_{t-1}\pi_t - \gamma (u_t - u^n)
\end{align}
を得る。

最後に、適合的期待として、$F_{t-1}\pi_t = \pi_{t-1}$、すなわち今期のインフレ率が来期も続くという期待を仮定すると、
\begin{align}
	\pi_t - \pi_{t-1}=  - \gamma (u_t - u^n)
\end{align}
でフィリップス曲線が得られた。

この含意として、$u_t \leq u^n$の時に、インフレ率の増加分が時間を経て増えていくことになることがわかる。つまり加速度的なインフレが発生するということである。この現象が起きない最低の失業率として$u^n$があり、これをインフレ非加速的失業率(NAIRU)と呼ぶ。

\subsection{地震の影響}
ここでは恒常所得仮説を採用する。この時震災が翌年の経常収支に及ぼす影響は以下のように考えられる。経常収支は、
\begin{align}
	X - M = (S - I) + (T - G)
\end{align}
の左辺を指す。すなわち、対外貿易の黒字分で、民間貯蓄のマイナス分と政府貯蓄のマイナス分を補っているということである。

ここで、$S = Y - C - T$であり、震災によって一時的に総生産の$Y$が減少すると考えられる。しかし、恒常所得仮説においては一時的な収入の増減は支出に反映されないので、$C$は変わらずに同じ水準を保つこととなる。従って民間貯蓄である$S$が減少する。他の要素を一定とすれば、これにより上の式から経常収支が減少することは明らかである。

\subsection{小話}
\begin{itemize}
	\item 完全失業率の低下はアベノミクスにより喚起された需要が生産の改善をもたらし実質GDPが上昇したから。
	\item 量的緩和の時期には貨幣の流通速度が低下する。これはケンブリッジ方程式より、$M = kPY$なので、$M$の上昇に対し$PY$が微小な変化であるうちは$k$が上昇する。これは$k = \frac{1}{V}$より貨幣の流通速度の低下を意味する。
	\item 完全に予測できるインフレ、例えば増税などは、フィッシャー方程式より実質利子率の$r$を引き下げる。これにより価格の高い耐久財への投資が起こりこれが駆け込み需要とみなされる。(2期間モデルでは名目利子率の上昇なので貯蓄する方向に動くはず?)
	\item 変動相場マンデルフレミングで財政政策が効かないのは純輸出のクラウディングアウトと呼ぶ。
	\item 90年以降のフィリップス曲線の傾きゆるまりについて。カルボモデルでは価格改定の確率が低下するとフィリップス曲線の傾きがゆるくなる。これが日本で起きたっぽいね。
	\item 明らかに自民が勝つ衆議院選挙などは公立市場仮説によると、すでに株価に織り込まれた情報なので、いざイベントが発生しても株価に変動は起こらない。
	\item インフレバイアス=「裁量型でのインフレ率」-「コミット型でのインフレ率」
	\item 貿易収支の悪化と為替レート:マンデルフレミングによれば貿易収支の悪化は$e$の上昇を招く。
\end{itemize}


\section{福田金融まとめ}
\subsection{2章 貯蓄行動}
\begin{itemize}
	\item ライフサイクル仮説の帰結:若い頃に貯蓄をし、年をとるとそれを切り崩す。すなわちある国について若年層の人口が多い時は資金市場への供給が増え、老年人口が多くなると供給が減る。日本でいうと団塊の世代の退職がこの契機。
	\item サンクトペテルブルクのパラドックス:払ってもいいのは4円(導出は余白)
\end{itemize}

\subsection{4章 資産価格}
\begin{itemize}
	\item 利回り=coupon rate(配当)+キャピタルゲイン。coupon rateが0の債券を割引債という。
	\item 流動性プレミアム仮説:長期の債券を持つことは、その保有期間における流動性を損なうことと、債券の不確実性が短期のものよりも大きいことを考慮して、純粋期待仮説で得られる利回りに少し割り増した利回りが長期の債券には与えられるとする仮説。この割り増し分を流動性プレミアムと呼ぶ。また、明らかに流動性プレミアムは債券の保有期間に対する増加関数である。
\end{itemize}

\subsection{9章 金融市場}
\begin{itemize}
	\item 短期金融市場の一つであるインターバンク市場、その中で中心的な役割を果たすのが、「金融機関相互の資金繰りを最終的に調整する場」であるコール市場である。
	\item コール市場で支払われる金利をコールレートと呼び、その中で無担保オーバーナイト物、すなわち無担保で貸し付け、翌日には決済を行うものにかかるコールレートが日銀の主要な操作目標である。これは、その他の市場などで金利が決定される際に参考される主なものがコール市場の無担保オーバーナイト物のコールレートだからである。
\end{itemize}

\subsection{10章 貨幣理論}
\begin{itemize}
	\item 準通貨とは流動性の低い定期性預金を指し、「M2=M1+準通貨」である。M1が交換手段としての貨幣価値に重きを置いた定義とすれば、M2は貨幣の保蔵手段としての価値に重きを置いたものである。
	\item 古典派の二分法の世界では、実質GDPである$Y$や貨幣の流通速度である$V$は貨幣量や価格とは別の世界で決定されるものであるので、フィッシャーの交換方程式である$MV = PY$の含意として、$M$の変化は同程度の$P$の変化にしかつながらない、ということがわかる。
	\item 古典派の貨幣数量説では貨幣需要に利子率は関係ない。しかし貨幣保有の機会費用としては名目利子率が重要となることは明らかである。これがボーモルトービンモデルであり、実質貨幣需要は$Y$についての増加関数、$i$についての減少関数である。
	\item 実質利子率の定義より、対数近似でフィッシャー方程式が得られる。
	\item マネタリストの主張:フリードマンの「最適貨幣量の理論」が述べる事は、緩やかなデフレにより名目利子率を0へ誘導するべきである、というもの。名目利子率は貨幣保有の機会費用なので、その機会費用を0にすることで経済主体の余剰を最大化できるという考えに基づくものである。フィッシャー方程式より名目利子率0の時は、実質利子率の非負制約よりインフレ率が負、すなわちデフレーションが発生するので、緩やかなデフレで名目利子率を0へと誘導できる。この議論は名目利子率に対して国民所得の$Y$が不変であることに依拠している。古典派の二分法のもとではこの仮定は正しいが一般には成立せず、またデフレは現実に起こると国民所得の停滞と不況を伴って発生するので、名目利子率を0にすることで最適な貨幣量とならないと考えられている。
\end{itemize}

\subsection{11章 日銀}
\begin{itemize}
	\item 日銀の最終目標は物価、景気の安定。その指標としてマネーストックと長期金利を中間目標とする。しかしこれらも直接に操作できるものではない。なぜならマネーストックには民間貯蓄という個人の意思決定による部分があるからで、長期金利は純粋期待仮説によればその期における民間の将来の短期金利に関する期待値に依存するからである。操作できるのはマネーストックに対してハイパワードマネー、長期金利に対して短期金利である。
	\item 伝統的にはハイパワードマネーを操作しての金融政策が行われてきた。これは貨幣乗数の安定性に依拠するものである。
	\item 短期金利の操作により目標とすべき名目利子率の水準をどこに設定すればいいかを教えてくれるのが「テイラールール」である。これは産出量ギャップとインフレ率と目標インフレ率との乖離の二つで名目利子率を決定するルールである。
\end{itemize}

\subsection{12章 伝統的経済政策}
\begin{itemize}
	\item 好況(好景気)とは、総需要が潜在GDPを上回ること。すなわちインフレギャップ($D(Y_f) - Y_f$)が正である状態で、この時物価上昇による需用量調整が必要となりインフレが発生する。
	\item 不況(不景気)とは、総需要が潜在GDPを下回ること。すなわちデフレギャップ($Y_f - D(Y_f)$)が正である状態で、この時は物価下落による需用量の調整ではなく、生産量の調整が大事になる。
	\item ケインズの枠組みはデフレギャップが正、すなわち不況の時の理論であり、この時需用量が増加すればその分生産量も増加する(上のように不況期は生産量で調整するから)ので、需要した分だけ作られるという意味で「有効需要の原理」と呼ばれる状況が成立している。
	\item 政府支出乗数はなぜ1より大きいのか:$Y = C + I + G$より、$G$の増加は同量の$Y$の増加に直接つながる。また、この$Y$の増加は国民の可処分所得を同程度に増加させ、消費である$C$も限界消費性向をかけた分だけ増加する。この二つの経路により1より大きい政府支出乗数が得られる。
	\item 均衡予算乗数は1
	\item 非ケインズ効果:財政赤字が過度に拡大した状況下で、ケインズ型消費関数でなくライフサイクル仮説のような将来時点での所得を考慮する消費関数を持つ家計は、赤字返済のための将来の増税を意識した結果消費を控える可能性がある。これは政府支出や減税による乗数効果が不況下であっても働かなくなる可能性を示唆するもので、非ケインズ効果と呼ばれる。
	\item 投資関数は実質利子率の関数、実質貨幣需要関数は名目利子率の関数。このことからIS-LMで名目利子率と実質利子率との違いを考慮すると、IS曲線がインフレ率上昇で右にシフトするという分析ができる。(これは実質利子率の低下による投資の拡大を反映)
	\item 流動性の罠:名目利子率が低すぎると、貨幣量の増加分は全て貨幣需要に吸収され、利子率に変化ゴキない。したがって投資の拡大も起きないので国民所得を増加させることができない。これを流動性の罠と呼ぶ。
\end{itemize}

\subsection{13章 インフレ、デフレ}
\begin{itemize}
	\item インフレ、デフレともに需要側が起こすもの(ディマンド・プル・インフレ/デフレ)と供給側が起こすもの(コスト・プッシュ・インフレ/デフレ)が存在する。
	\item ディマンド・プル・インフレ:ケンブリッジ方程式より、$\frac{\Delta P}{P} = \frac{\Delta M}{M} - \frac{\Delta Y}{Y}$なので、名目貨幣量の増加が実質国民所得の増加を上回った時にインフレが起こる。(古典派の二分法で$Y$が$M$と独立に決まれば貨幣の中立性が成立)
	\item コスト・プッシュ・インフレ:価格は、コストを$C$、マージンを$m$とした時に$P = ( 1 + m)C$として決定される。よって$C$の上昇は価格の上昇をもたらす。具体例としては、労働組合の交渉によって賃金が上昇する、石油危機などで原料輸入費が上昇する、などによってコストプッシュインフレが発生した。このタイプのインフレは物価上昇の一方で企業の収入を減少させ業績悪化につながる。よってインフレと不況が同時に発生する「スタグフレーション」が生まれがちである。
	\item ディマンド・プル・デフレ:総需要の減退は総供給曲線に沿って価格と総生産を同時に減少させる。これはすなわち物価下落と不況が同時発生する状況であり、この二つを持って一般にデフレと呼ぶ
	\item コスト・プッシュ・デフレ:原料価格の下落はインフレの時とは逆に価格の下落をもたらす。しかし、より大切なのは生産性向上による、価格の下落と生産量の拡大が同時に起きるタイプのものである。これは経済全体にとって良いことなので問題とはならない。
	\item デフレのコスト、メニューコストはデフレでもコストだが、靴のコストは、デフレ下において人々は手持ちの貨幣を少なくするメリットがないので発生しない。
	\item 「事後の実質利子率 = 名目利子率 - 実現したインフレ率」であり、「事前の実質利子率 = 名目利子率 - 期待インフレ率」である。これより、予測できないデフレは実質利子率を高止まりさせる。すなわち借り手から貸し手への予測できない所得移転をデフレが招く。またこれは設備投資の減少を招く。
	\item この所得移転は実質賃金についても起こる。名目賃金が事前に決定されている時、デフレが発生すると実質賃金は上昇する。デフレ下では労働生産性も低下するので、本来引き下げたいはずである。すなわちこの高止まりは企業側から労働者への予期せぬ所得移転となる。
	\item ただし、この実質賃金の高止まりは労働者全体に及ぶわけではない。インサイダーはこの恩恵を受ける一方で、アウトサイダーはインサイダーへの賃金が高止まりしたことに対する企業側の利潤現象を受けて実質賃金を低く提示されることになる。
	\item しかしデフレが長期化すると、企業側のロスが、リストラや賃金カットといった形でインサイダーへも波及してくる可能性が高くなる。(まとめると、デフレの初期は実質賃金の高止まりで予期せぬ所得移転が生じ労働分配率が上昇するが、長期のデフレの下では企業利潤の減少が賃金カットやリストラでインサイダーへも波及することで労働分配率が減少する。)
\end{itemize}

\end{document}
































