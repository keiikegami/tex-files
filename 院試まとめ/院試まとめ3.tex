\documentclass{jsarticle}
\usepackage[margin = .7in]{geometry}
\usepackage[dvipdfmx]{graphicx}
\usepackage{listings}
\usepackage{amsmath}
\usepackage{bm}
\usepackage{ascmac}
\lstset{%
  language={python},
  basicstyle={\small},%
  identifierstyle={\small},%
  commentstyle={\small\itshape},%
  keywordstyle={\small\bfseries},%
  ndkeywordstyle={\small},%
  stringstyle={\small\ttfamily},
  frame={tb},
  breaklines=true,
  columns=[l]{fullflexible},%
  numbers=left,%
  xrightmargin=0zw,%
  xleftmargin=3zw,%
  numberstyle={\scriptsize},%
  stepnumber=1,
  numbersep=1zw,%
  lineskip=-0.5ex%
}

\begin{document}
\title{院試まとめ3}
\author{池上慧}
\maketitle

\section{Definitions and Equations}
\subsection{Consumer theory definitions}
\begin{itembox}[l]{パレート効率的}
	$x^{*} \in \Pi_i^I x_i$ is Pareto efficient, if there is no feasible allocation $x$ which Pareto dominates $x^{*}$. \\
	where 
	\begin{itemize}
		\item $x$ is feasible allocation if $x \in \left\{ x \in \Pi_i^I x_i | \sum_i^I x_i \leq \bar{\omega} \right\}$
		\item $x$ Pareto Dominates $x^*$ if $\exists \ i\ \text{s.t.}\ x_i \succ_i x_i^*$ and $\forall \ i \ x_i \succeq_i x_i^*$
	\end{itemize}
\end{itembox}

\begin{itembox}[l]{ワルラス均衡}
	$(p^*, x^*) \in R^L\ \times \ \Pi_i^I x_i$ is Walrasian Equilibrium, if\\
	\begin{enumerate}
		\item $\forall_i \ \ x_i^* \succeq_i x_i\ \text{s.t.}\ x_i \in \left\{ x \in \Pi_i^I x_i | \sum_i^I x_i \leq \bar{\omega} \right\}$
		\item $\sum_i^I x_i = \bar{\omega}$
	\end{enumerate}
\end{itembox}

\begin{itembox}[l]{コア}
	$x$ is Core, if $x$ is Pareto efficient and $x$ is individually rational\\
	where $x$ is individually rational if $\forall  \ i \ x_i \succeq_i \omega_i$
\end{itembox}

\subsection{Equations}
$p,w,u$をそれぞれ価格、所得、効用水準とする。
\begin{itemize}
	\item 需要関数:$x(p, w)$
	\item 間接効用関数:$v(p, w)$
	\item 補償需要関数:$h(p, u)$
	\item 支出関数:$e(p, u)$
\end{itemize}
前の二つが効用最大化、後半二つが支出最小化。
\begin{itembox}[l]{スラツキー方程式(横)}
	$D_p x(p,w) = D_p h(p, u) - D_w x(p, w) x(p, w)^{T}$, where $u = V(p, w)$
\end{itembox}

\begin{itembox}[l]{ロワの恒等式(縦)}
	$x_l (p, w) = -\frac{\frac{\partial v(p, w)}{\partial p_l}}{\frac{\partial v(p, w)}{\partial w}}$
\end{itembox}

\begin{itembox}[l]{支出関数を微分したら補償需要関数(縦)}
	$\Delta_p e(p, u) = h(p, u)$
\end{itembox}

\begin{itembox}[l]{双対性(中)}
	\begin{itemize}
		\item $x(p, w) = h(p. v(p, w))$
		\item $h(p, u) = x(p, e(p, u))$
		\item $e(p, v(p, w)) = w$
		\item $v(p, e(p, u)) = u$
	\end{itemize}
\end{itembox}

\subsection{Game theory}
標準形ゲーム$G = (N, S, u)$を考える。ただし$S = \Pi_{i \in N} S_i$は戦略の組の集合であり、$u = (u_i)_{i \in N}$は効用の組、ただし$u_i : S \to R$で効用関数を表す。
\begin{itembox}[l]{ナッシュ均衡}
	$s^* \in S$ is Nash equilibrium, if \\
	$\quad \forall \ i \in N$ and $\forall \ s_i \in S_i$, $u_i(s^*) \geq u_i(s_i, s_{-i}^*)$\\
	where $s_{-i}$ is the strategies of the players other than $i$.
\end{itembox}

\begin{itembox}[l]{弱支配戦略}
	$s_i^* \in S_i$ is a weak dominant strategy, if \\
	$\forall \ s_{-i} \in S_{-i}$ and $\forall \ s_i \in S_i$, $u_i(s_i^*, s_{-i}) \geq u_i(s_i, s_{-i})$
\end{itembox}

\end{document}






















