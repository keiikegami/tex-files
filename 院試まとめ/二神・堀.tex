\documentclass{jsarticle}
\usepackage[margin = .7in]{geometry}
\usepackage[dvipdfmx]{graphicx}
\usepackage{listings}
\usepackage{amsmath}
\usepackage{bm}
\lstset{%
  language={python},
  basicstyle={\small},%
  identifierstyle={\small},%
  commentstyle={\small\itshape},%
  keywordstyle={\small\bfseries},%
  ndkeywordstyle={\small},%
  stringstyle={\small\ttfamily},
  frame={tb},
  breaklines=true,
  columns=[l]{fullflexible},%
  numbers=left,%
  xrightmargin=0zw,%
  xleftmargin=3zw,%
  numberstyle={\scriptsize},%
  stepnumber=1,
  numbersep=1zw,%
  lineskip=-0.5ex%
}

\begin{document}
\title{二神・堀まとめ}
\author{池上慧}
\maketitle

\section{第1章 SNA}
\subsection{国民経済計算と三面等価}
三面等価の関係とは、GDI(分配面)とGDP(生産面)と支出側GDP(支出面)の三つの値が一致することを指す。ここでGDIは国内総所得であり、「GDI = (間接税-補助金)+固定資本減耗+雇用者報酬+営業余剰+混合所得」である。またGDPは国内総生産であり、「GDP = 純付加価値+固定資本減耗」である。最後に支出側GDPは「C+I+G+EX - IM」でかける。ただしEX-IMは純輸出であり、NXで表記する。

\subsection{国際収支統計}
先では基本的に国内での経済活動に関する統計を見た。次に海外との取引を扱う統計、すなわち国際収支統計を見る。これは経常収支、金融収支、資本移転等収支の三つより構成されている。経常収支は「NX + 第1次所得収支 + 第2次所得収支」で定義される。ここで第1次所得収支は「海外からの所得受け取り-海外への所得支払い」で、第2次所得収支は「海外からの寄付等受け取り - 海外への寄付等支払い」で定義されるものである。
 資本移転等収支は、ODAなどの対価の受け取りをともなわない固定資本の移転などを指す。金融収支は「資本流出 - 資本流入 + 資本移転等収支」で定義される。ここで資本流出は海外資産の購入金額を指し、資本流入は国内資産が海外によって購入されることを指す。
 ここで「対外純資産の増加 = 金融収支 - 資本移転等収支」である。また経常収支は対外純資産の増加と等しいので、「経常収支 + 資本移転等収支= 金融収支」なる等式が成立する。

\subsection{物価水準の測定}
基準年を0で表すとし、t年度の物価測定を考える。経済にはN財あるとすると、
\begin{align*}
	GDPdeflator = \frac{Nominal GDP}{Real GDP} &= \frac{p_t^1 q_t^1 + \dots + p_t^N q_t^N}{p_0^1q_t^1 + \dots + p_0^Nq_t^N}\\
	&= \frac{p_0^1q_t^1}{p_0^1q_t^1 + \dots + p_0^Nq_t^N}\frac{p_t^1}{p_0^1} + \dots + \frac{p_0^Nq_t^N}{p_0^1q_t^1 + \dots + p_0^Nq_t^N}\frac{p_t^N}{p_0^N}
\end{align*}
で定義されるものがGDPデフレーターである。これは各財の物価上昇の加重平均である。加重平均を取る際のウェイトとなる部分に現在の数量が用いられる指数をパーシェ指数と呼び、ウェイトに基準年の数量を指数をラスパイレス指数と呼ぶ。ラスパイレス指数の例としてはCPI(消費者物価指数)などがある。

\section{第3章 投資}
\subsection{新古典派の投資理論}
t期粗投資を$I_t$、t期純投資を$K_{t+1} - K_t$と表記する。ここで$K_t$はt期の資本ストックである。両者の関係は
\begin{align}
	I_t = K_{t+1} - K_t + \delta K_t = K_{t+1} -(1 - \delta)K_t
\end{align}
である。

新古典派の投資理論とは、将来時点での最適な資本ストックが企業価値最大化の条件として導出され、それを達成するための手段として現在時点での投資量である$I_t$が式(1)にしたがって決定される、とするものである。

将来時点の最適な資本ストックは二期間モデルから導くこともできるが、解釈としては「資本の使用者費用」を考えたほうがいい。これは資本ストックを1単位使用する際のコストとして定義されるもので、「機会費用である利子(r)」と「資本減耗の費用$(\delta)$」の和として表される。っこで生産関数を$F(K)$で表記するとすると、利潤最大化の一階条件より
\begin{align}
	F^{'}(K_{t+1}^*) = r + \delta
\end{align}
が将来時点の最適な資本ストックの満たす式として導かれる。

生産関数がconcaveであるとすると、利子率の下落は将来時点の最適な資本ストック量を増やし、したがって投資量を増加させる。以上より、設備投資関数$(I(r))$は右下がりとなる。以上が新古典派の投資理論の帰結である。

\section{第4章 資産市場}
\subsection{資産の種類}
金融機関を介さず、株式や社債、国債で資金調達することを直接金融と呼び、銀行を介して家計から資金調達することを間接金融と呼ぶ。家計の目線から見て、収益が変動する資産のことを危険資産と呼び、現時点で将来受け取る収益が確定している資産を安全資産と呼ぶ。ここでリスクプレミアムとは危険資産の期待収益率が安全資産の収益率をどれだけ上回っているかの指標である。
\subsection{無裁定条件}
債券を保有する際に受け取る配当や利息をインカムゲインと呼び、債券自体の価値が上昇することによる収益をキャピタルゲインと呼ぶ。これを安全資産について考えると以下のように利子率が定義できる。
\begin{align}
	r_1 \equiv \frac{d_1 + p_1 -p_0}{p_0}
\end{align}
ここで、$d_1$がインカムゲインであり、$p_1 - p_0$がキャピタルゲインである。

一方で危険資産については同様の式で期待収益率が計算できる。
\begin{align}
	\frac{E[d_1 + p_1] -p_0}{p_0}
\end{align}
ここで期待値は将来時点での配当と債券価格が未確定であるので、それらについて期待値を撮ったものとなっている。

リスクプレミアムを$\rho$とすると、定義より無裁定条件とは以下の等式のことを指す。
\begin{align}
	\frac{E[d_1 + p_1] -p_0}{p_0} = r_1 + \rho
\end{align}

\subsection{資産価格}
資産価格の理論値が以上の議論より導ける。利子率が将来期間にわたって一定であるとすると、(3)式より
\begin{align}
	p_0 = \frac{d_1 + p_1}{1 + r}
\end{align}
を得る。ここで上の$p_1$についても同様の式が得られるので逐次代入していくと、
\begin{align*}
	p_0 = \sum_{t = 1}^{\infty} \frac{d_t}{(1+ r)^t} + \lim_{t \to \infty} \frac{p_t}{(1 + r)^t}
\end{align*}
を得る。この第2項が0に行くと仮定する(これは合理的バブルが存在しないことを仮定している)。すると、
\begin{align}
	p_0 = \sum_{t = 1}^{\infty} \frac{d_t}{(1+ r)^t}
\end{align}
が得られ、このように資産価値を求めることを配当の割引現在価値モデルと呼び、この理論価格を債券価格のファンダメンタルズと呼ぶ。

\subsection{短期金利と長期金利}
利子率は現実には複数存在する。例えば国債にしても短期国債と長期国債では利子率が異なる。短期と長期の金利に存在する関係性を金利の期間構造と呼ぶ。短期の資金取引を行う市場は短期金融市場と呼ばれ、インターバンク市場とオープン市場に分類される。前者は金融機関の身が参加可能な市場であり、この一部であるコール市場における無担保コールレートが日銀の政策目標をされるものである。

債券には保有期間中の利息、配当の支払いが発生する利付債と満期まで何も発生しない割引債の二つが存在する。ここでは割引債に注目し、その利回りは以下のように計算される。
\begin{align}
	p_{n, t} = \frac{f}{(1 + r_{n, t})^n}
\end{align}
ただし、$p_{n, t}$はt期における残存期間nの債券価格であり、$r_{n, t}$はその債券の利回り、$f$は満期の償還額である。ここでは安全資産にのみ注目すれば良いので$f$は確定していることに注意。

このように複数存在する利回りの間に存在する関係性については以下の無裁定条件が自然に考えられる。すなわち「残存期間nの長期債券を保有し続けることと、残存期間1年の短期債券をn年間毎年購入して資金運用することが同じ収益をもたらす」である。
将来の短期金利が未確定であることを考えると、これは以下のように表現できる。
\begin{align}
	(1 + r_{n, t})^n = E[(1 + r_{1,t})(1 + r_{1, t+1})\dots(1 + r_{1,t+n-1})]
\end{align}

これが成立すると仮定することを純粋期待仮説と呼ぶ。この下では長期金利が短期金利を上回っている場合に、将来時点で短期金利が上昇することが予想されているのであると言える。横軸に残存期間を、縦軸に利回りをとってプロットしたものをイールドカーブと呼ぶ。これよりイールドカーブが右上りであることは将来での金利上昇が投資家の間で期待されていることを示すことになる。

\subsection{フィッシャー方程式}
名目利子率を$i$とし、期待インフレ率を$E[\pi]$とし、実質利子率を$r$とした時に、以下のフィッシャー方程式が成立する。
\begin{align}
	i = E[\pi] + r
\end{align}
rは実質変数の世界で決定されるので、期待インフレ率が上がれば名目利子率は上昇する。期待が変化しても実質利子率に変化はない。






















\end{document}