\documentclass{jsarticle}
\usepackage[margin = .7in]{geometry}
\usepackage[dvipdfmx]{graphicx}
\usepackage{listings}
\usepackage{amsmath}
\usepackage{bm}
\lstset{%
  language={python},
  basicstyle={\small},%
  identifierstyle={\small},%
  commentstyle={\small\itshape},%
  keywordstyle={\small\bfseries},%
  ndkeywordstyle={\small},%
  stringstyle={\small\ttfamily},
  frame={tb},
  breaklines=true,
  columns=[l]{fullflexible},%
  numbers=left,%
  xrightmargin=0zw,%
  xleftmargin=3zw,%
  numberstyle={\scriptsize},%
  stepnumber=1,
  numbersep=1zw,%
  lineskip=-0.5ex%
}

\begin{document}
\title{二神・堀まとめ}
\author{池上慧}
\maketitle

\section{第1章 SNA}
\subsection{国民経済計算と三面等価}
三面等価の関係とは、GDI(分配面)とGDP(生産面)と支出側GDP(支出面)の三つの値が一致することを指す。ここでGDIは国内総所得であり、「GDI = (間接税-補助金)+固定資本減耗+雇用者報酬+営業余剰+混合所得」である。またGDPは国内総生産であり、「GDP = 純付加価値+固定資本減耗」である。最後に支出側GDPは「C+I+G+EX - IM」でかける。ただしEX-IMは純輸出であり、NXで表記する。

\subsection{国際収支統計}
先では基本的に国内での経済活動に関する統計を見た。次に海外との取引を扱う統計、すなわち国際収支統計を見る。これは経常収支、金融収支、資本移転等収支の三つより構成されている。経常収支は「NX + 第1次所得収支 + 第2次所得収支」で定義される。ここで第1次所得収支は「海外からの所得受け取り-海外への所得支払い」で、第2次所得収支は「海外からの寄付等受け取り - 海外への寄付等支払い」で定義されるものである。
 資本移転等収支は、ODAなどの対価の受け取りをともなわない固定資本の移転などを指す。金融収支は「資本流出 - 資本流入 + 資本移転等収支」で定義される。ここで資本流出は海外資産の購入金額を指し、資本流入は国内資産が海外によって購入されることを指す。
 ここで「対外純資産の増加 = 金融収支 - 資本移転等収支」である。また経常収支は対外純資産の増加と等しいので、「経常収支 + 資本移転等収支= 金融収支」なる等式が成立する。

\subsection{物価水準の測定}
基準年を0で表すとし、t年度の物価測定を考える。経済にはN財あるとすると、
\begin{align*}
	GDPdeflator = \frac{Nominal GDP}{Real GDP} &= \frac{p_t^1 q_t^1 + \dots + p_t^N q_t^N}{p_0^1q_t^1 + \dots + p_0^Nq_t^N}\\
	&= \frac{p_0^1q_t^1}{p_0^1q_t^1 + \dots + p_0^Nq_t^N}\frac{p_t^1}{p_0^1} + \dots + \frac{p_0^Nq_t^N}{p_0^1q_t^1 + \dots + p_0^Nq_t^N}\frac{p_t^N}{p_0^N}
\end{align*}
で定義されるものがGDPデフレーターである。これは各財の物価上昇の加重平均である。加重平均を取る際のウェイトとなる部分に現在の数量が用いられる指数をパーシェ指数と呼び、ウェイトに基準年の数量を指数をラスパイレス指数と呼ぶ。ラスパイレス指数の例としてはCPI(消費者物価指数)などがある。

\section{第3章 投資}
\subsection{新古典派の投資理論}
t期粗投資を$I_t$、t期純投資を$K_{t+1} - K_t$と表記する。ここで$K_t$はt期の資本ストックである。両者の関係は
\begin{align}
	I_t = K_{t+1} - K_t + \delta K_t = K_{t+1} -(1 - \delta)K_t
\end{align}
である。

新古典派の投資理論とは、将来時点での最適な資本ストックが企業価値最大化の条件として導出され、それを達成するための手段として現在時点での投資量である$I_t$が式(1)にしたがって決定される、とするものである。

将来時点の最適な資本ストックは二期間モデルから導くこともできるが、解釈としては「資本の使用者費用」を考えたほうがいい。これは資本ストックを1単位使用する際のコストとして定義されるもので、「機会費用である利子(r)」と「資本減耗の費用$(\delta)$」の和として表される。っこで生産関数を$F(K)$で表記するとすると、利潤最大化の一階条件より
\begin{align}
	F^{'}(K_{t+1}^*) = r + \delta
\end{align}
が将来時点の最適な資本ストックの満たす式として導かれる。

生産関数がconcaveであるとすると、利子率の下落は将来時点の最適な資本ストック量を増やし、したがって投資量を増加させる。以上より、設備投資関数$(I(r))$は右下がりとなる。以上が新古典派の投資理論の帰結である。

\section{第4章 資産市場}
\subsection{資産の種類}
金融機関を介さず、株式や社債、国債で資金調達することを直接金融と呼び、銀行を介して家計から資金調達することを間接金融と呼ぶ。家計の目線から見て、収益が変動する資産のことを危険資産と呼び、現時点で将来受け取る収益が確定している資産を安全資産と呼ぶ。ここでリスクプレミアムとは危険資産の期待収益率が安全資産の収益率をどれだけ上回っているかの指標である。
\subsection{無裁定条件}
債券を保有する際に受け取る配当や利息をインカムゲインと呼び、債券自体の価値が上昇することによる収益をキャピタルゲインと呼ぶ。これを安全資産について考えると以下のように利子率が定義できる。
\begin{align}
	r_1 \equiv \frac{d_1 + p_1 -p_0}{p_0}
\end{align}
ここで、$d_1$がインカムゲインであり、$p_1 - p_0$がキャピタルゲインである。

一方で危険資産については同様の式で期待収益率が計算できる。
\begin{align}
	\frac{E[d_1 + p_1] -p_0}{p_0}
\end{align}
ここで期待値は将来時点での配当と債券価格が未確定であるので、それらについて期待値を撮ったものとなっている。

リスクプレミアムを$\rho$とすると、定義より無裁定条件とは以下の等式のことを指す。
\begin{align}
	\frac{E[d_1 + p_1] -p_0}{p_0} = r_1 + \rho
\end{align}

\subsection{資産価格}
資産価格の理論値が以上の議論より導ける。利子率が将来期間にわたって一定であるとすると、(3)式より
\begin{align}
	p_0 = \frac{d_1 + p_1}{1 + r}
\end{align}
を得る。ここで上の$p_1$についても同様の式が得られるので逐次代入していくと、
\begin{align*}
	p_0 = \sum_{t = 1}^{\infty} \frac{d_t}{(1+ r)^t} + \lim_{t \to \infty} \frac{p_t}{(1 + r)^t}
\end{align*}
を得る。この第2項が0に行くと仮定する(これは合理的バブルが存在しないことを仮定している)。すると、
\begin{align}
	p_0 = \sum_{t = 1}^{\infty} \frac{d_t}{(1+ r)^t}
\end{align}
が得られ、このように資産価値を求めることを配当の割引現在価値モデルと呼び、この理論価格を債券価格のファンダメンタルズと呼ぶ。

\subsection{短期金利と長期金利}
利子率は現実には複数存在する。例えば国債にしても短期国債と長期国債では利子率が異なる。短期と長期の金利に存在する関係性を金利の期間構造と呼ぶ。短期の資金取引を行う市場は短期金融市場と呼ばれ、インターバンク市場とオープン市場に分類される。前者は金融機関の身が参加可能な市場であり、この一部であるコール市場における無担保コールレートが日銀の政策目標をされるものである。

債券には保有期間中の利息、配当の支払いが発生する利付債と満期まで何も発生しない割引債の二つが存在する。ここでは割引債に注目し、その利回りは以下のように計算される。
\begin{align}
	p_{n, t} = \frac{f}{(1 + r_{n, t})^n}
\end{align}
ただし、$p_{n, t}$はt期における残存期間nの債券価格であり、$r_{n, t}$はその債券の利回り、$f$は満期の償還額である。ここでは安全資産にのみ注目すれば良いので$f$は確定していることに注意。

このように複数存在する利回りの間に存在する関係性については以下の無裁定条件が自然に考えられる。すなわち「残存期間nの長期債券を保有し続けることと、残存期間1年の短期債券をn年間毎年購入して資金運用することが同じ収益をもたらす」である。
将来の短期金利が未確定であることを考えると、これは以下のように表現できる。
\begin{align}
	(1 + r_{n, t})^n = E[(1 + r_{1,t})(1 + r_{1, t+1})\dots(1 + r_{1,t+n-1})]
\end{align}

これが成立すると仮定することを純粋期待仮説と呼ぶ。この下では長期金利が短期金利を上回っている場合に、将来時点で短期金利が上昇することが予想されているのであると言える。横軸に残存期間を、縦軸に利回りをとってプロットしたものをイールドカーブと呼ぶ。これよりイールドカーブが右上りであることは将来での金利上昇が投資家の間で期待されていることを示すことになる。

\subsection{フィッシャー方程式}
名目利子率を$i$とし、期待インフレ率を$E[\pi]$とし、実質利子率を$r$とした時に、以下のフィッシャー方程式が成立する。
\begin{align}
	i = E[\pi] + r
\end{align}
rは実質変数の世界で決定されるので、期待インフレ率が上がれば名目利子率は上昇する。期待が変化しても実質利子率に変化はない。

\section{第5章 貨幣}
\subsection{貨幣乗数}
貨幣乗数は中央銀行がマネタリーベース(H)を1単位増やした時に増加する貨幣量(M)を示したものである。ここでマネタリーベースは「現金+準備預金」であり、貨幣量は「現金+預金通過」とする。この関係は以下の式で表される。
\begin{align*}
	mp = \frac{M}{H} = \frac{C + D}{C + R} = \frac{1 + c}{re + c}
\end{align*}
ただし、$c = \frac{C}{D}$であり、現金・預金比率と呼ばれ、$re = \frac{R}{D}$であり、預金準備率と呼ばれる。
ここで$re < 1$であるので、貨幣乗数は1より大きい値になる。すなわち中央銀行がマネタリーベースを1単位増やすと増やした量よりも貨幣量の増加分は必ず大きくなることがわかる。この効果を信用創造と呼び、以下のようなフローをたどる。

\begin{enumerate}
\item 中央銀行による民間金融機関が保持する国債の1単位購入
\item 中央銀行内の金融機関の口座で預金が1単位増える
\item 必要以上の準備預金が存在するので家計や企業への貸付が1単位増える
\item 1単位の借り入れを現金・預金比率に従って$\frac{1}{1 + c}$だけ預金が増え、$\frac{c}{1 + c}$だけ現金が増える
\item 新たに預けられた$\frac{1}{1 + c}$のうち$re$だけ準備預金が増え、$\frac{1 - re}{1 + c}$だけ新たに貸付が増える
\end{enumerate}
このループが繰り返された結果としてCの増加分は$\frac{c}{c + re}$で、Dの増加分は$\frac{1}{c + re}$で、Rの増加分は$\frac{re}{c + re}$である。これより確かにマネタリーベースの増加分で貨幣量の増加分をあると貨幣乗数に一致する。

このような貨幣量のコントロールを公開市場操作と呼び、これは貨幣乗数の安定性に依存するものである。他にも中央銀行から民間への貸付の際にかかる金利を変化させる補完貸付制度と預金準備率reを変化させる法定準備率操作が貨幣量コントロールの手段としてある。

\subsection{貨幣需要 古典派じゃない奴}
貨幣の特徴はその流動性にあり、流動性がもたらす便益、すなわち取引時の費用削減などと流動性の持つ費用、すなわち債券などに投資していれば得られていたであろう収益とを比較して貨幣保有を決定していると考える。
均衡では貨幣保有の限界便益と貨幣保有の限界費用が一致するように実質貨幣保有量が決定される。限界費用は常に名目利子率であり一定である。限界便益は、実質貨幣保有量が増えるに従って流動性の持つ紅葉は下がっていくと想定できるので右下がりである。

財やサービスの取引量増加、すなわち生産量拡大はより大きな貨幣保有の便益をより大きくするので限界便益曲線を上方向にシフトさせる。すなわち貨幣保有量を増加させる。一方で名目利子率の上昇は限界費用曲線を上方向にシフトさせるので貨幣保有量を低下させる。従って、実質貨幣需要関数は$L = L(Y, i)$とかけ、これは$Y$についての増加関数、$i$についての減少関数である。

\subsection{貨幣需要 古典派}
古典派の特徴は、貨幣需要が名目利子率に依存しない点である。
フィッシャーの交換方程式は、名目貨幣量を$M$、流通速度を$V$、物価水準を$P$、実質取引量を$T$とすると、
\begin{align}
	MV = PT
\end{align}
でかける。ただし流通速度は、PTの取引を行うのに1単位の貨幣が何回用いられるかを表す値である。

ケンブリッジ方程式は、実質GDPを$Y$と置いて、
\begin{align}
	M = kPY
\end{align}
である。ここで$k$はマーシャルの$k$と呼ばれる。フィッシャーの交換方程式と比較すれば、実質取引量を実質GDPとみなすとすれば$k = \frac{1}{V}$と解釈できることがわかる。(12)式を両辺$P$で割れば、$\frac{M}{P} = kY$となり、実質GDPのみに依存する(名目利子率には依存しない)実質貨幣需要関数が導ける。

\subsection{インフレ}
古典派じゃない方の理論を用いる。貨幣について需給が一致しているとすれば、先に導いた実質貨幣需要関数を用いて物価水準は以下のように欠書ける。
\begin{align*}
	\frac{M}{P} = L(Y, i)\quad \Leftrightarrow\quad P = \frac{M}{L(Y, i)}
\end{align*}
従って$M$の増加により物価水準が上がることが予想できる。これがインフレであり、そのもたらすコストとして、
\begin{itemize}
\item 予期しない所得移転
\item 靴のコスト
\item メニューコスト
\end{itemize}
などが挙げられるが、思ったほど自明なコストはない。

一方で貨幣発行が政府にもたらす便益があるので、これがインフレに伴う便益として存在する。これはシニョレージと呼ばれるもので、政府収入となるという意味で実質的には税金の役割を果たしているのでインフレ税とも呼ばれる。

\section{第11章 閉鎖短期}
短期ということは、資本ストックが一定であること、家計の消費が現在の可処分所得で決定されることを意味する。分析の枠組みは以下の三つ。
\begin{itemize}
\item 45度線分析:財市場のみを考えてGDPを均衡で決定する。(rもPも所与、というかPは関係ない)
\item IS-LM:財市場と貨幣市場での均衡からGDPとrを均衡で決定する。(Pは所与)
\item AD-AS:財市場、貨幣市場、労働市場での均衡からGDPとrとPを決定する。
\end{itemize}

\subsection{45度線分析}
以下の連立方程式の解として実質GDPを決定する。ただし$Y$は実質GDIであり、r,G,Tは外生、$C( )$は消費関数である。
\begin{align*}
	Y^D &= C(Y - T) + I(r) + G \\
	Y^S &= Y\\
	Y^D &= Y^S
\end{align*}
ここでケインズ型消費関数を用いれば、上の連立方程式は以下のように整理できる。
\begin{align}
	Y = A + c(Y - T) + I + G \quad \Leftrightarrow\quad Y^* = \frac{A - cT + I + G}{1 -c}
\end{align}
政府支出乗数は$\frac{d Y^*}{dG} = \frac{1}{1 -c}$であり、租税乗数雨は$\frac{dY^*}{dT} = \frac{-c}{1-c}$である。

\subsection{IS-LM}
(13)式でrを内生変数としてみることを考える。新古典派の投資理論より$I(r)$はrについての減少関数なので、均衡自室GDPはrについてのgんしょう関数である。このrと$Y^*$との組み合わせをIS曲線と呼ぶ。

次に貨幣市場での均衡を考えることで上と合わせてrとYを同時に決定させる。今、Pを所与としているのでインフレは存在しない。すなわち名目利子率と実質利子率は一致するので実質貨幣需要関数は$L(Y, r)$とかける。$\frac{M}{P}$は今の所しょとできるので、貨幣市場での均衡条件は以下の式出る。
\begin{align}
	\frac{M}{P} = L(Y, r)
\end{align}
左辺が定数であることと、実質貨幣需要関数が$Y$についての増加関数、$r$についての減少関数であることより、貨幣市場を均衡させるrとYとの組み合わせはr,Y平面上の右上がりの曲線で表せる。これをLM曲線と呼ぶ。以上で得られたIS曲線とLM曲線との交点でYとrが決定される。

すなわち解いている連立方程式は以下である。
\begin{align*}
	Y^D &= C(Y - T) + I(r) + G \\
	Y^S &= Y\\
	Y^D &= Y^S\\
	\frac{M}{P} &= L(Y, r)
\end{align*}

\subsection{クラウディングアウト}
政府支出の増加によりIS曲線は右方向にシフトする。そのシフト分は45度線分析から得られる分だけであるが、均衡実質GDPの上昇分は45度線分析の時よりも少なくなる。これはIS-LMのフレームワークにおいてはIS曲線の右シフトが均衡実質利子率の上昇も同時にもたらすので、これに伴って投資の減少が起こるからである。この、政府支出の増加に伴う設備投資の減少を、投資のクラウディングアウトと呼ぶ。

\subsection{AD-AS}
最後に労働市場を入れることでPも内生変数としたモデルを解く。

IS-LMでPを含むのはLM曲線のみであり、Pの上昇はLM曲線の上方シフトをもたらす。これより均衡で得られる均衡実質GDPは減少するので、P,Y平面上の右下がり曲線としてAD曲線を得る。

AS曲線は労働市場を考慮することで得られる。名目賃金の下方硬直性も予測誤差も存在しない場合、労働市場においてはどのような部下水準においても完全雇用が必ず必ず達成されるように名目賃金が決定される。これによって達成される雇用は常に一定であり、短期では資本ストックを変かっさせることもできないので、実質GDPが物価水準によらず決定されることとなる(これを完全雇用GDPと呼ぶ)。これよりP,Y平面上の垂直な線としてAS曲線が得られる。

ADとASとの交点でPとYが同時に決定され、IS-LMより決定されたYに応じたrが決定される。名目賃金の下方硬直性と予測誤差の存在はAS曲線をkinkさせる。が詳細は略。

\section{開放短期}
\subsection{金利平価式}
金利と為替についての無裁定条件は以下である。ただし$e_t$をt期の自国通貨建て為替レートとし、$i_{t+1}$をt期の自国名目金利、$i_{t+1}^{\$}$をt期の外国名目金利とする。
\begin{align}
	1 + i_{t+1} = \frac{e_{t+1}(1 + i_{t+1}^{\$})}{e_t}
\end{align}
左辺が1円を自国で1年運用した際の所得、右辺がt期に1円を外国通貨に両替し、外国で1年運用したのちに1年後のレートで自国通貨に両替し直した際の所得である。また、$x_t = {\rm ln} e_t$とおくと、金利平価式を$x_{t+1} - x_t = i_{t+1} - i_{t+1}^{\$}$と書き直せる。

ここで(15)式が将来時点の名目為替レートに不確実性がない場合の式であることに注意する。現実には名目為替レートの変動は激しく、不確実性が伴うものである。

1年後に取引する為替レートを現在時点で決定する契約を「フォワード契約」と呼び、その為替レートの対数値を$f_{t+1}$で書くと、金利平価式は
\begin{align*}
	f_{t + 1} - x_t = i_{t+1} - i_{t+1}^{\$}
\end{align*}
でかけ、この式を「カバー付き金利平価式」と呼ぶ。ここで$f_{t + 1} - x_t$をフォワードプレミアムと呼ぶ。

フォワード契約を考えない場合は
\begin{align*}
	E[x_{t + 1}] - x_t = i_{t+1} - i_{t+1}^{\$}
\end{align*}
であり、これを「カバーなし金利平価」と呼ぶ。

この二つの金利平価式より、$f_{t+1} = E[x_{t+1}]$であるはずだが、フォワードレートの上昇に対して将来の名目為替実現値が減少するなどの現象が観察されている。これをフォワードディスカウントパズルと呼ぶ。














\end{document}