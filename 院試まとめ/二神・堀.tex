\documentclass{jsarticle}
\usepackage[margin = .7in]{geometry}
\usepackage[dvipdfmx]{graphicx}
\usepackage{listings}
\usepackage{amsmath}
\usepackage{ascmac}
\usepackage{bm}
\lstset{%
  language={python},
  basicstyle={\small},%
  identifierstyle={\small},%
  commentstyle={\small\itshape},%
  keywordstyle={\small\bfseries},%
  ndkeywordstyle={\small},%
  stringstyle={\small\ttfamily},
  frame={tb},
  breaklines=true,
  columns=[l]{fullflexible},%
  numbers=left,%
  xrightmargin=0zw,%
  xleftmargin=3zw,%
  numberstyle={\scriptsize},%
  stepnumber=1,
  numbersep=1zw,%
  lineskip=-0.5ex%
}

\begin{document}
\title{院試まとめ1}
\author{池上慧}
\maketitle

\section{第1章 SNA}
\subsection{三面等価}
何はともあれ以下の4つを覚えればいい。生産側GDPを
\begin{screen}
	GDP(生産側)= ある一定期間である国の経済において生産されたサービス・財の付加価値の総和 
\end{screen}
として定義する。

このGDPは「分配面」と「支出面」という異なる二つの視点から見ることができ、分配面から見たGDPを「国内総所得(GDI)」と呼び、支出面から見たGDPを「支出側国内総生産」と呼ぶ。

国内総所得(GDI)は
\begin{screen}
	GDI = 雇用者報酬+営業余剰+混合所得+固定資本減耗+(間接税ー補助金)
\end{screen}
で定義できる。

国内支出側総生産は
\begin{screen}
	支出側GDP = C + I + G + EX - IM
\end{screen}
で定義できる。
この3つは全て等しい、というのが「三面等価の原則」であった。

以上の三つは「国内」について定義されたものである。日本国民の海外での所得等を考慮した「国民」の概念に基づく統計は「国民総生産(GNP)」と「国民総所得(GNI)」の二つがあるが、このうちGNPは政府の統計では正式に扱われていないので割愛する。国民総所
得(GNI)は
\begin{screen}
	GNI = GDI + 海外からの要素所得受け取り - 海外への要素所得支払い
\end{screen}
で定義できる。ただし「海外への要素所得支払い」は日本国内で外国人労働者が得る労働所得などを含むものである。

\subsection{ホニャホニャ収支}
当たり前だけど、国レベルの収支が発生するのは海外とのやりとりが発生する時なので、上記のSNAとは違うものとして以下の国際収支統計を理解すべき。

国際収支統計の構成要素は
\begin{screen}
\begin{enumerate}
	\item 経常収支 = 貿易・サービス収支 + 第1次所得収支 + 第2次所得収支
	\item 金融収支 = 資本流出 - 資本流入 + 資本移転等収支
	\item 資本移転等収支 = 対価をともなわない固定資本の移動(ODAなど)
\end{enumerate}
\end{screen}
の三つである。ただし「第1次所得収支」は「所得の純増」であり、「第2次所得収支」は「寄付の純増」である。また「資本流出」は「外国資産の購入」であり、「資本流入」は「自国資産の売却」である。

ここで「対外純資産の増加」という言葉を以下で定義する。
\begin{screen}
	対外純資産の増加 = 資産流出 - 資産流入
\end{screen}
以上の4つの式が定義である。

ここで以下の3つの恒等式が成立することに注意する。
\begin{screen}
	輸出代金の受け取り + 所得の受け取り = 外国への資本流出\\
	輸出代金の支払い + 所得の支払い = 外国からの資本流入\\
	経常収支 = (輸出代金の受け取り + 所得の受け取り) - (輸出代金の支払い + 所得の支払い)
\end{screen}
この恒等式と上の定義より
\begin{screen}
	経常収支 = 資本流出 - 資本流入  = 対外純資産の増加 = 金融収支 - 資本移転等収支
\end{screen}
が成立する。これすなわち「経常収支 + 資本移転等収支 = 金融収支」の恒等式である。

\subsection{物価水準の測定}
基準年を0で表すとし、t年度の物価測定を考える。経済にはN財あるとすると、
\begin{align*}
	GDPdeflator = \frac{Nominal GDP}{Real GDP} &= \frac{p_t^1 q_t^1 + \dots + p_t^N q_t^N}{p_0^1q_t^1 + \dots + p_0^Nq_t^N}\\
	&= \frac{p_0^1q_t^1}{p_0^1q_t^1 + \dots + p_0^Nq_t^N}\frac{p_t^1}{p_0^1} + \dots + \frac{p_0^Nq_t^N}{p_0^1q_t^1 + \dots + p_0^Nq_t^N}\frac{p_t^N}{p_0^N}
\end{align*}
で定義されるものがGDPデフレーターである。これは各財の物価上昇の加重平均である。加重平均を取る際のウェイトとなる部分に現在の数量が用いられる指数をパーシェ指数と呼び、ウェイトに基準年の数量を指数をラスパイレス指数と呼ぶ。ラスパイレス指数の例としてはCPI(消費者物価指数)などがある。

\section{第3章 投資}
\subsection{新古典派の投資理論}
t期粗投資を$I_t$、t期純投資を$K_{t+1} - K_t$と表記する。ここで$K_t$はt期の資本ストックである。両者の関係は
\begin{align}
	I_t = K_{t+1} - K_t + \delta K_t = K_{t+1} -(1 - \delta)K_t
\end{align}
である。

新古典派の投資理論とは、将来時点での最適な資本ストックが企業価値最大化の条件として導出され、それを達成するための手段として現在時点での投資量である$I_t$が式(1)にしたがって決定される、とするものである。

将来時点の最適な資本ストックは二期間モデルから導くこともできるが、解釈としては「資本の使用者費用」を考えたほうがいい。これは資本ストックを1単位使用する際のコストとして定義されるもので、「機会費用である利子(r)」と「資本減耗の費用$(\delta)$」の和として表される。っこで生産関数を$F(K)$で表記するとすると、利潤最大化の一階条件より
\begin{align}
	F^{'}(K_{t+1}^*) = r + \delta
\end{align}
が将来時点の最適な資本ストックの満たす式として導かれる。

生産関数がconcaveであるとすると、利子率の下落は将来時点の最適な資本ストック量を増やし、したがって投資量を増加させる。以上より、設備投資関数$(I(r))$は右下がりとなる。以上が新古典派の投資理論の帰結である。

\section{第4章 資産市場}
\subsection{資産の種類}
金融機関を介さず、株式や社債、国債で資金調達することを直接金融と呼び、銀行を介して家計から資金調達することを間接金融と呼ぶ。家計の目線から見て、収益が変動する資産のことを危険資産と呼び、現時点で将来受け取る収益が確定している資産を安全資産と呼ぶ。ここでリスクプレミアムとは危険資産の期待収益率が安全資産の収益率をどれだけ上回っているかの指標である。
\subsection{無裁定条件}
債券を保有する際に受け取る配当や利息をインカムゲインと呼び、債券自体の価値が上昇することによる収益をキャピタルゲインと呼ぶ。これを安全資産について考えると以下のように利子率が定義できる。
\begin{align}
	r_1 \equiv \frac{d_1 + p_1 -p_0}{p_0}
\end{align}
ここで、$d_1$がインカムゲインであり、$p_1 - p_0$がキャピタルゲインである。

一方で危険資産については同様の式で期待収益率が計算できる。
\begin{align}
	\frac{E[d_1 + p_1] -p_0}{p_0}
\end{align}
ここで期待値は将来時点での配当と債券価格が未確定であるので、それらについて期待値を撮ったものとなっている。

リスクプレミアムを$\rho$とすると、定義より無裁定条件とは以下の等式のことを指す。
\begin{align}
	\frac{E[d_1 + p_1] -p_0}{p_0} = r_1 + \rho
\end{align}

\subsection{資産価格}
資産価格の理論値が以上の議論より導ける。利子率が将来期間にわたって一定であるとすると、(3)式より
\begin{align}
	p_0 = \frac{d_1 + p_1}{1 + r}
\end{align}
を得る。ここで上の$p_1$についても同様の式が得られるので逐次代入していくと、
\begin{align*}
	p_0 = \sum_{t = 1}^{\infty} \frac{d_t}{(1+ r)^t} + \lim_{t \to \infty} \frac{p_t}{(1 + r)^t}
\end{align*}
を得る。この第2項が0に行くと仮定する(これは合理的バブルが存在しないことを仮定している)。すると、
\begin{align}
	p_0 = \sum_{t = 1}^{\infty} \frac{d_t}{(1+ r)^t}
\end{align}
が得られ、このように資産価値を求めることを配当の割引現在価値モデルと呼び、この理論価格を債券価格のファンダメンタルズと呼ぶ。

\subsection{短期金利と長期金利}
利子率は現実には複数存在する。例えば国債にしても短期国債と長期国債では利子率が異なる。短期と長期の金利に存在する関係性を金利の期間構造と呼ぶ。短期の資金取引を行う市場は短期金融市場と呼ばれ、インターバンク市場とオープン市場に分類される。前者は金融機関の身が参加可能な市場であり、この一部であるコール市場における無担保コールレートが日銀の政策目標をされるものである。

債券には保有期間中の利息、配当の支払いが発生する利付債と満期まで何も発生しない割引債の二つが存在する。ここでは割引債に注目し、その利回りは以下のように計算される。
\begin{align}
	p_{n, t} = \frac{f}{(1 + r_{n, t})^n}
\end{align}
ただし、$p_{n, t}$はt期における残存期間nの債券価格であり、$r_{n, t}$はその債券の利回り、$f$は満期の償還額である。ここでは安全資産にのみ注目すれば良いので$f$は確定していることに注意。

このように複数存在する利回りの間に存在する関係性については以下の無裁定条件が自然に考えられる。すなわち「残存期間nの長期債券を保有し続けることと、残存期間1年の短期債券をn年間毎年購入して資金運用することが同じ収益をもたらす」である。
将来の短期金利が未確定であることを考えると、これは以下のように表現できる。
\begin{align}
	(1 + r_{n, t})^n = E[(1 + r_{1,t})(1 + r_{1, t+1})\dots(1 + r_{1,t+n-1})]
\end{align}

これが成立すると仮定することを純粋期待仮説と呼ぶ。この下では長期金利が短期金利を上回っている場合に、将来時点で短期金利が上昇することが予想されているのであると言える。横軸に残存期間を、縦軸に利回りをとってプロットしたものをイールドカーブと呼ぶ。これよりイールドカーブが右上りであることは将来での金利上昇が投資家の間で期待されていることを示すことになる。

\subsection{フィッシャー方程式}
名目利子率を$i$とし、期待インフレ率を$E[\pi]$とし、実質利子率を$r$とした時に、以下のフィッシャー方程式が成立する。
\begin{align}
	i = E[\pi] + r
\end{align}
rは実質変数の世界で決定されるので、期待インフレ率が上がれば名目利子率は上昇する。期待が変化しても実質利子率に変化はない。

\section{第5章 貨幣}
\subsection{貨幣乗数}
貨幣乗数は中央銀行がマネタリーベース(H)を1単位増やした時に増加する貨幣量(M)を示したものである。ここでマネタリーベースは「現金+準備預金」であり、貨幣量は「現金+預金通過」とする。この関係は以下の式で表される。
\begin{align*}
	mp = \frac{M}{H} = \frac{C + D}{C + R} = \frac{1 + c}{re + c}
\end{align*}
ただし、$c = \frac{C}{D}$であり、現金・預金比率と呼ばれ、$re = \frac{R}{D}$であり、預金準備率と呼ばれる。
ここで$re < 1$であるので、貨幣乗数は1より大きい値になる。すなわち中央銀行がマネタリーベースを1単位増やすと増やした量よりも貨幣量の増加分は必ず大きくなることがわかる。この効果を信用創造と呼び、以下のようなフローをたどる。

\begin{enumerate}
\item 中央銀行による民間金融機関が保持する国債の1単位購入
\item 中央銀行内の金融機関の口座で預金が1単位増える
\item 必要以上の準備預金が存在するので家計や企業への貸付が1単位増える
\item 1単位の借り入れを現金・預金比率に従って$\frac{1}{1 + c}$だけ預金が増え、$\frac{c}{1 + c}$だけ現金が増える
\item 新たに預けられた$\frac{1}{1 + c}$のうち$re$だけ準備預金が増え、$\frac{1 - re}{1 + c}$だけ新たに貸付が増える
\end{enumerate}
このループが繰り返された結果としてCの増加分は$\frac{c}{c + re}$で、Dの増加分は$\frac{1}{c + re}$で、Rの増加分は$\frac{re}{c + re}$である。これより確かにマネタリーベースの増加分で貨幣量の増加分をあると貨幣乗数に一致する。

このような貨幣量のコントロールを公開市場操作と呼び、これは貨幣乗数の安定性に依存するものである。他にも中央銀行から民間への貸付の際にかかる金利を変化させる補完貸付制度と預金準備率reを変化させる法定準備率操作が貨幣量コントロールの手段としてある。

\subsection{貨幣需要 古典派じゃない奴}
貨幣の特徴はその流動性にあり、流動性がもたらす便益、すなわち取引時の費用削減などと流動性の持つ費用、すなわち債券などに投資していれば得られていたであろう収益とを比較して貨幣保有を決定していると考える。
均衡では貨幣保有の限界便益と貨幣保有の限界費用が一致するように実質貨幣保有量が決定される。限界費用は常に名目利子率であり一定である。限界便益は、実質貨幣保有量が増えるに従って流動性の持つ紅葉は下がっていくと想定できるので右下がりである。

財やサービスの取引量増加、すなわち生産量拡大はより大きな貨幣保有の便益をより大きくするので限界便益曲線を上方向にシフトさせる。すなわち貨幣保有量を増加させる。一方で名目利子率の上昇は限界費用曲線を上方向にシフトさせるので貨幣保有量を低下させる。従って、実質貨幣需要関数は$L = L(Y, i)$とかけ、これは$Y$についての増加関数、$i$についての減少関数である。

\subsection{貨幣需要 古典派}
古典派の特徴は、貨幣需要が名目利子率に依存しない点である。
フィッシャーの交換方程式は、名目貨幣量を$M$、流通速度を$V$、物価水準を$P$、実質取引量を$T$とすると、
\begin{align}
	MV = PT
\end{align}
でかける。ただし流通速度は、PTの取引を行うのに1単位の貨幣が何回用いられるかを表す値である。

ケンブリッジ方程式は、実質GDPを$Y$と置いて、
\begin{align}
	M = kPY
\end{align}
である。ここで$k$はマーシャルの$k$と呼ばれる。フィッシャーの交換方程式と比較すれば、実質取引量を実質GDPとみなすとすれば$k = \frac{1}{V}$と解釈できることがわかる。(12)式を両辺$P$で割れば、$\frac{M}{P} = kY$となり、実質GDPのみに依存する(名目利子率には依存しない)実質貨幣需要関数が導ける。

\subsection{インフレ}
古典派じゃない方の理論を用いる。貨幣について需給が一致しているとすれば、先に導いた実質貨幣需要関数を用いて物価水準は以下のように欠書ける。
\begin{align*}
	\frac{M}{P} = L(Y, i)\quad \Leftrightarrow\quad P = \frac{M}{L(Y, i)}
\end{align*}
従って$M$の増加により物価水準が上がることが予想できる。これがインフレであり、そのもたらすコストとして、
\begin{itemize}
\item 予期しない所得移転
\item 靴のコスト
\item メニューコスト
\end{itemize}
などが挙げられるが、思ったほど自明なコストはない。

一方で貨幣発行が政府にもたらす便益があるので、これがインフレに伴う便益として存在する。これはシニョレージと呼ばれるもので、政府収入となるという意味で実質的には税金の役割を果たしているのでインフレ税とも呼ばれる。

\section{第6,7章 長期の経済}
二期間モデルで話をするってだけ。$K_1$は所与とされる
\subsection{閉鎖経済 実質変数}
まずは政府部門を入れないで考える。また簡単のために資本減耗はないこととする。$K_1$が所与なので、$Y_1 = F(K_1)$で決定されている。よって知りたいのは、第2期の資本ストックを決定する投資額である。新古典派の投資理論より投資額は利子率の減少関数であることがわかっている。なので、まずは利子率を決定しなければならない。

利子率を決定するのは資金市場である。
\begin{align}
	Y_1 &= C_1 + I\\
	S &= Y_1 - C_1
\end{align}
より$I=S$である。これが資本市場の均衡条件である。$I$は先にも述べた通り$r$の減少関数であり、$S$は2期間モデルの示す通り$r$の増加関数である。

したがって$I = S$から均衡となる利子率$r^*$が決定され、同時に均衡での投資額$I(r^*)$が決定される。これより、$K_2^* = K_1 + I(r^*)$であり、$Y_2 = F(K_2^*)$で均衡GDPが決定される。

では政府部門を入れるとどうなるか。こちらも二期間モデルで考える。国債発行額を$B$で容器して、政府の予算制約式は第1期と第2期がそれぞれ
\begin{align}
	B &= G_1 - T_1\\
	T_2 &= (1+r)B + G_2
\end{align}
であり、ここから$B$を消去して、$G_1 + \frac{G_2}{1 + r} = T_1 + \frac{T_2}{1 + r}$を二期間を通じた予算制約式として得る。

資金市場については以下のような変更となる。
\begin{align}
	Y_1 &= C_1 + I + G_1\\
	S &= Y_1 - T_1 - C_1
\end{align}
したがって、$S = I + G_1 - T_1$が資金市場の均衡条件である。よって、$T_1 - G_1$の値によって資金の供給曲線がシフトして均衡が変化する。

\subsection{閉鎖経済 名目変数(物価)}
貨幣市場を考えて名目変数である物価の決定を行う。先に実質利子率は決定されているので、第2期の実質GDPは$Y_2 = Y_2(r^*)$ですでに決定されている。これより、第2期の貨幣市場での均衡は、3期がないので名目利子率は考慮されないことより、
\begin{align}
	\frac{M_2}{P_2} = L(Y_2(r^*))\quad \Leftrightarrow \quad P_2^* = \frac{M_2}{L(Y_2(r^*))}
\end{align}
となり、2期の物価水準が決定する。

今、長期の経済を考えているので期待インフレ率と現実のインフレ率は一致する。よってフィッシャー方程式より$i = r + \pi$であるので、第1期の名目利子率は$r^* + \pi$である。
これより、$\frac{M_1}{P_1} = L(Y_1^*, r^* + \pi)$が第1期の貨幣市場の均衡条件として得られる。ここでまだ決定されていないのはインフレ率と一気の物価水準である。

定義より、$\frac{P_2^*}{P_1} = 1 + \pi$であるので、以下を$\pi$についての均衡条件として得る。
\begin{align}
	\frac{M_1}{P_2^*}(1 + \pi) = L(Y_1^*, r^* + \pi)
\end{align}
これにより均衡のインフレ率が定まり、一期の物価水準も定まる。

\subsection{為替レート}
名目為替レート(自国通貨建て)は「外国通貨1単位と交換される時刻通貨の単位数」を示すものである。一方で実質為替レートは「外国の財1単位と交換される自国財の単位数」を表すものである。それぞれ、$e$と$\epsilon$で表記すると、両者の間には物価を通じて以下のような関係がある。
\begin{align}
	\epsilon = \frac{eP^{\$}}{P}
\end{align}
ただし、$P^{\$}$は外国の物価水準を、$P$は自国の物価水準を示している。

\subsection{購買力平価}
購買力平価とは、国際間の財取引によって為替レートが決定されるとする説であり、その帰結として、長期的には実質為替レートが1となること、すなわち$\epsilon^* = 1$が均衡であることを得る。これが一物一価の法則である。(21)式より均衡名目為替レートに関しては
\begin{align}
	e^* = \frac{P}{P^{\$}}
\end{align}
が得られる。ただしこれは財の移動にコストがかからない場合の結果であることに注意。

購買力平価の均衡の下では、小国の仮定を置くと、自国の物価水準が上昇すると自国通貨建ての名目為替レートが上昇、すなわち円安。ドル高となる。貨幣市場の均衡条件より、名目貨幣量の増加が円安ドル高を招くことが以上よりわかる。

\section{第11章 閉鎖短期}
短期ということは、資本ストックが一定であること、家計の消費が現在の可処分所得で決定されることを意味する。分析の枠組みは以下の三つ。
\begin{itemize}
\item 45度線分析:財市場のみを考えてGDPを均衡で決定する。(rもPも所与、というかPは関係ない)
\item IS-LM:財市場と貨幣市場での均衡からGDPとrを均衡で決定する。(Pは所与)
\item AD-AS:財市場、貨幣市場、労働市場での均衡からGDPとrとPを決定する。
\end{itemize}

\subsection{45度線分析}
以下の連立方程式の解として実質GDPを決定する。ただし$Y$は実質GDIであり、r,G,Tは外生、$C( )$は消費関数である。
\begin{align*}
	Y^D &= C(Y - T) + I(r) + G \\
	Y^S &= Y\\
	Y^D &= Y^S
\end{align*}
ここでケインズ型消費関数を用いれば、上の連立方程式は以下のように整理できる。
\begin{align}
	Y = A + c(Y - T) + I + G \quad \Leftrightarrow\quad Y^* = \frac{A - cT + I + G}{1 -c}
\end{align}
政府支出乗数は$\frac{d Y^*}{dG} = \frac{1}{1 -c}$であり、租税乗数雨は$\frac{dY^*}{dT} = \frac{-c}{1-c}$である。

\subsection{IS-LM}
(23)式でrを内生変数としてみることを考える。新古典派の投資理論より$I(r)$はrについての減少関数なので、均衡自室GDPはrについてのgんしょう関数である。このrと$Y^*$との組み合わせをIS曲線と呼ぶ。

次に貨幣市場での均衡を考えることで上と合わせてrとYを同時に決定させる。今、Pを所与としているのでインフレは存在しない。すなわち名目利子率と実質利子率は一致するので実質貨幣需要関数は$L(Y, r)$とかける。$\frac{M}{P}$は今の所しょとできるので、貨幣市場での均衡条件は以下の式出る。
\begin{align}
	\frac{M}{P} = L(Y, r)
\end{align}
左辺が定数であることと、実質貨幣需要関数が$Y$についての増加関数、$r$についての減少関数であることより、貨幣市場を均衡させるrとYとの組み合わせはr,Y平面上の右上がりの曲線で表せる。これをLM曲線と呼ぶ。以上で得られたIS曲線とLM曲線との交点でYとrが決定される。

すなわち解いている連立方程式は以下である。
\begin{align*}
	Y^D &= C(Y - T) + I(r) + G \\
	Y^S &= Y\\
	Y^D &= Y^S\\
	\frac{M}{P} &= L(Y, r)
\end{align*}

\subsection{クラウディングアウト}
政府支出の増加によりIS曲線は右方向にシフトする。そのシフト分は45度線分析から得られる分だけであるが、均衡実質GDPの上昇分は45度線分析の時よりも少なくなる。これはIS-LMのフレームワークにおいてはIS曲線の右シフトが均衡実質利子率の上昇も同時にもたらすので、これに伴って投資の減少が起こるからである。この、政府支出の増加に伴う設備投資の減少を、投資のクラウディングアウトと呼ぶ。

\subsection{AD-AS}
最後に労働市場を入れることでPも内生変数としたモデルを解く。

IS-LMでPを含むのはLM曲線のみであり、Pの上昇はLM曲線の上方シフトをもたらす。これより均衡で得られる均衡実質GDPは減少するので、P,Y平面上の右下がり曲線としてAD曲線を得る。

AS曲線は労働市場を考慮することで得られる。名目賃金の下方硬直性も予測誤差も存在しない場合、労働市場においてはどのような部下水準においても完全雇用が必ず必ず達成されるように名目賃金が決定される。これによって達成される雇用は常に一定であり、短期では資本ストックを変かっさせることもできないので、実質GDPが物価水準によらず決定されることとなる(これを完全雇用GDPと呼ぶ)。これよりP,Y平面上の垂直な線としてAS曲線が得られる。

ADとASとの交点でPとYが同時に決定され、IS-LMより決定されたYに応じたrが決定される。名目賃金の下方硬直性と予測誤差の存在はAS曲線をkinkさせる。が詳細は略。

\section{第12章 開放短期}
\subsection{金利平価式}
金利と為替についての無裁定条件は以下である。ただし$e_t$をt期の自国通貨建て為替レートとし、$i_{t+1}$をt期の自国名目金利、$i_{t+1}^{\$}$をt期の外国名目金利とする。
\begin{align}
	1 + i_{t+1} = \frac{e_{t+1}(1 + i_{t+1}^{\$})}{e_t}
\end{align}
左辺が1円を自国で1年運用した際の所得、右辺がt期に1円を外国通貨に両替し、外国で1年運用したのちに1年後のレートで自国通貨に両替し直した際の所得である。また、$x_t = {\rm ln} e_t$とおくと、金利平価式を$x_{t+1} - x_t = i_{t+1} - i_{t+1}^{\$}$と書き直せる。

ここで(15)式が将来時点の名目為替レートに不確実性がない場合の式であることに注意する。現実には名目為替レートの変動は激しく、不確実性が伴うものである。

1年後に取引する為替レートを現在時点で決定する契約を「フォワード契約」と呼び、その為替レートの対数値を$f_{t+1}$で書くと、金利平価式は
\begin{align*}
	f_{t + 1} - x_t = i_{t+1} - i_{t+1}^{\$}
\end{align*}
でかけ、この式を「カバー付き金利平価式」と呼ぶ。ここで$f_{t + 1} - x_t$をフォワードプレミアムと呼ぶ。

フォワード契約を考えない場合は
\begin{align*}
	E[x_{t + 1}] - x_t = i_{t+1} - i_{t+1}^{\$}
\end{align*}
であり、これを「カバーなし金利平価」と呼ぶ。

この二つの金利平価式より、$f_{t+1} = E[x_{t+1}]$であるはずだが、フォワードレートの上昇に対して将来の名目為替実現値が減少するなどの現象が観察されている。これをフォワードディスカウントパズルと呼ぶ。

\section{第13章 財政政策}
\subsection{プライマリー収支}
政府の予算制約式は、$B_t$をt期の国債発行残高、rを利子率として、
\begin{align*}
	G_t + rB_{t-1} - T_t = B_t - B_{t-1}
\end{align*}
でかける。ここで、$G_t - T_t$をプライマリー収支と呼び、これが正の値であることをプライマリー収支の赤字と呼ぶ。

\subsection{等価定理}
短期の分析では、政府支出の財源によって経済にもたらす効果に差があることがわかる。すなわち、公債によって資金調達した場合は政府支出乗数分だけGDPを増加させられる一方で、増税によって資金調達した場合は租税乗数によるマイナスの効果が働くので、政府支出によってGDPを増やすことはできない。

しかしこれは短期における話である。将来時点での増税によって公債の償還を行うであろうことが予想されている場合、それを加味して家計は行動するので現在時点での増税で政府支出を増やそうと、現在では公債によって資金調達をしようと家計の行動に影響は与えられない。これを等価定理と呼ぶ。

ただし、等価定理が成り立つのは流動性制約がないときのみである。すなわち現在時点での借り入れに制約が課されている場合には等価定理は成り立たない。これは二期間モデルで確認できる。

\section{金融政策}
\subsection{フィリップス曲線}
「失業率が高い時にはインフレ率が低くなり、失業率が低い時にはインフレ率が高くなる」という関係性を指摘したものがフィリップス曲線である。右下がりになる理由は、予測誤差モデルで説明できる。すなわち、物価上昇に伴う名目賃金上昇を実質賃金の上昇と勘違いした家計は労働供給を増やし、それに応じて労働需要が増加すれば失業率が低下するのである。

インフレ率0の時の失業率を自然失業率と呼び、$u_N$で表記する。この時短期フィリップス曲線は
\begin{align*}
	\pi_t = E_{t-1}[\pi_t] - \beta (u_t - u_N)
\end{align*}
でかける。

先の予測誤差モデルの説明で、勘違いした家計はやがて実質賃金が上昇していないことに気づく。その結果労働供給を減らし、自然失業率が達成される。しかし、その過程で生じたインフレは、すでに貨幣量を増やしてしまっているので低下させることはできない。よって、短期フィリップス曲線は徐々に情報へとシフトしていくことになる。また、自然失業率から垂直に伸びる直線を長期フィリップス曲線と呼ぶ。

\subsection{流動性効果}
貨幣市場における均衡条件より、名目貨幣量の予期せぬ増加が名目利子率の低下によって対応されることがわかる。これを流動性効果と呼ぶ。しかし、長期の目線では貨幣量の増加は物価水準の上昇を招くので、名目貨幣量の増加が必ずしも実質貨幣量の増加につながるとは限らない。なので、流動性効果はあくまでも短期において発現するものであることに注意。

\subsection{金融政策におけるルール}
$k\%$ルールとは、マネタリストの主張する金融政策のルールであり、「貨幣量の成長率を一定に保つよう金融政策を運営するべきである」というものである。これは経済政策のラグによる経済の混乱を最小限に抑えようと意図されたものであった。

インフレーションターゲットは、手段を定めずに、目標としたインフレ率を達成するように政策を変更していくもの。

テイラールールは、そもそも金融政策が目指すべき金利の水準についてのルールである。$i$を短期の政策金利、インフレ率を$\pi$、潜在GDPの対数値を$y^*$として、テイラールールは以下で与えられる。
\begin{align}
	i = \pi + 0.02 + 0.5(y - y^*) + 0.5(\pi -0.02)
\end{align}
これはすなわち潜在GDPにおける実質金利を2$\%$に保とうという目標設定がなされているということである。











\end{document}