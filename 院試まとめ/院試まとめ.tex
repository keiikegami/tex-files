\documentclass{jsarticle}
\usepackage[margin = .7in]{geometry}
\usepackage[dvipdfmx]{graphicx}
\usepackage{listings}
\usepackage{amsmath}
\usepackage{bm}
\usepackage{ascmac}
\lstset{%
  language={python},
  basicstyle={\small},%
  identifierstyle={\small},%
  commentstyle={\small\itshape},%
  keywordstyle={\small\bfseries},%
  ndkeywordstyle={\small},%
  stringstyle={\small\ttfamily},
  frame={tb},
  breaklines=true,
  columns=[l]{fullflexible},%
  numbers=left,%
  xrightmargin=0zw,%
  xleftmargin=3zw,%
  numberstyle={\scriptsize},%
  stepnumber=1,
  numbersep=1zw,%
  lineskip=-0.5ex%
}

\begin{document}
\title{院試まとめ2}
\author{池上慧}
\maketitle

\section{開放経済}
\subsection{長期開放経済の資金市場の均衡}
開放経済になって変化するのは資金市場のみであり、財市場、貨幣市場の均衡は長期閉鎖経済の時と変わらない。長期の開放経済における財市場の均衡は以下のとおり
\begin{align}
	Y_1 = C_1 + I + G_1 + NX_1
\end{align}
ここで$GNI = GDP + IB_1$(ただし$IB_1$は1期の所得収支を示す)なので、上の財市場の均衡式の両辺に$IB_1$を足して、
\begin{align*}
	Y_1 + IB_1 = C_1 + I + G_1 + NX_1 + IB_1
\end{align*}
である。ここで、経常収支$=NX_1 + IB_1$で、それを$CA$と書くとする。両辺から税金である$T_1$を引いて整理すると、
\begin{align}
	(GNI_1 - T_1 -C_1) = I + (G_1 - T_1) + CA
\end{align}
を得る。この左辺は国民民間貯蓄であり、経常収支はその定義より対外純資産の増加と等しいので、すなわちこれが資金市場の均衡式であり、「国民民間貯蓄が、設備投資と財政赤字と海外からの資金需要を補っている」ことを示す式と言える。

逆に、「経常収支=国民民間貯蓄 - 設備投資 - 財政赤字」であるので、経常収支の赤字とは国内の民間貯蓄で国内の資金需要が補えていない状態を示すことも言える。

\subsection{小国の仮定とGDP、経常収支の決定}
小国の仮定とは、$r = r^{\$}$が常に成立することを指す。二期間モデルで考えると、1期の資本ストックを所与とした時に1期の総生産は既に決定されているので、2期の総生産を決定すれば良い。しかし2期の資本ストックを決定する投資額は自国の実質利子率の関数で決定され、小国の仮定より実質利子率は外生的に決定していることより、2期の総生産も$Y_2 = Y_2(r^{\$})$で決定されることになる。これでGDPが決定された。

「経常収支はGDPと関係ない」ので注意。経常収支は小国の仮定の下では(2)式に従って資金市場で以下のように決定される。
\begin{align*}
	CA_1 = (GNI_1 - T_1 -C_1) - (G_1 - T_1) - I(r^{\$})
\end{align*}
しかしこれでは不完全。上式における国民民間貯蓄は、二期間モデルの帰結より、「実質利子率、1期の可処分所得、2期の可処分所得」の3つで決定される貯蓄関数と言える。これを$S(r^{\$}, Y_1 -T_1, Y_2(r^{\$}) - T_2)$と書くと、
\begin{align*}
	CA_1 = S(r^{\$}, Y_1 -T_1, Y_2(r^{\$}) - T_2) - (G_1 - T_1) - I(r^{\$})
\end{align*}
として経常収支が決定されることがわかる。

また、閉鎖経済であれば、経常収支の部分が0となるように$r$が資金市場の均衡で決定される。ここで$S$が$r$についての増加関数であること、$I$が$r$についての減少関数であることの二点より、「閉鎖経済での均衡実質利子率$<$外国の実質利子率」ならば経常収支が黒字となること、そして逆なら経常収支が赤字となることがモデルより判明する。

\subsection{開放短期 変動為替マンデルフレミング}
自国の物価、外国の物価を所与として、小国の仮定より$r = r^{\$}$が成立しているとする。ここで内生変数を「自国通貨建て名目為替レート$(e)$」と「実質GDP$(y)$」として両者が財市場と貨幣市場を均衡させるように同時に決定されるモデルをマンデルフレミングモデルという。

連立する二つの式はそれぞれ
\begin{align}
	Y &= C(Y-T) + I(r^{\$}) + G + NX(e\frac{P^{\$}}{P})\\
	\frac{M}{P} &= L(Y, r^{\$})
\end{align}
であり、(3)式をここでのIS曲線、(4)式をここでのLM曲線と呼ぶ。IS曲線について$e$を動かすと$Y$がどう変化するかを見る。これは全て純輸出$NX$を通しての変化であり、自国通貨建て名目為替レートの上昇は自国通貨安を意味し純輸出を増加させる。これがそのまま総生産を増加させるので、IS曲線はe,Y平面における右上がりの曲線となる。(ただし、このためにはマーシャル・ラーナー条件が必要である。)

LM曲線には名目為替レートが含まれていない。従ってこの一つの式より均衡実質GDPが決定される。これは全ての名目為替レートに対して成立するので、LM曲線はe,Y平面上の垂直な線として表現される。この二つの曲線の交点で$e,Y$が同時に決定される。

\subsection{開放短期 変動為替マンデルフレミングの政策効果}
財政政策は「名目為替レート下げるだけで総生産を増やせない」。一方で金融政策は「名目為替レートを上昇させ、総生産も増やす」。グラフで書けば両者当たり前だが、言葉で説明すると以下のよう。

財政政策は、
\begin{enumerate}
	\item 政府支出の増加は総需要を増加させる
	\item それによって実質貨幣需要が増加
	\item 貨幣市場での均衡を保つために実質利子率$r$が上昇する
	\item それに対応して日本の債券への需要が増える(経常収支の黒字)
	\item これにより円高、すなわち名目為替レートの低下が起こる
	\item それによって純輸出が低下して、当初に増加した分の総需要を減少させて$y$は変化しない
\end{enumerate}

金融政策は
\begin{enumerate}
	\item Mの増加は実質貨幣量を増加させる
	\item 均衡を保つために$r$が低下する
	\item 日本の債券への需要が減る(経常収支の赤字)
	\item 円安、すなわち名目為替レートの上昇が起こる
	\item これによって純輸出が増え、GDPも増える
\end{enumerate}

\subsection{開放短期 固定為替マンデルフレミング}
$e = \bar{e}$で固定されると、(3)式より均衡実質GDPが決定する。先のモデルで内生変数だった為替レートが外生となったので、代わりに内省となるのは、各国の中央銀行が固定為替レートを維持するために変動させる貨幣量$M$である。ここでは、(3)式で決定した$Y^{\$}$を所与として、(4)式から貨幣市場を均衡させる貨幣量$M$を均衡として得ることができる。これが固定為替でのマンデルフレミングの帰結である。

この時財政政策は「名目為替レートを上昇させ、実質GDPも増加させる」。一方で金融政策は「為替レートも変えられなければ実質GDPも変化させられない」。これはグラフより明らかである。

\section{ソロー}
\subsection{技術進歩なし}
「規模に関して収穫一定な生産関数、資本蓄積式、貯蓄関数、資金市場の均衡、人口成長率」の5つからソローモデルを描ける。それぞれ、
\begin{align*}
	y_t &= f(k_t)\\
	K_{t+1} &= K_t + I_t - \delta K_t\\
	S_t &= s Y_t\\
	S_t &= I_t\\
	\frac{L_{t+1} - L_t}{L_t} &= n
\end{align*}
である。ただし、$y_t = \frac{Y_t}{L_t}$、$k_t = \frac{K_t}{L_t}$、$\delta$は資本減耗率で、$s$は貯蓄率、$n$は人口成長率である。
全て整理すると、$k_{t+1} = \frac{sf(k_t) + (1 - \delta)k_t}{1+n}$を得る。これより一期間の一人当たり資本ストックの変化は、
\begin{align}
	k_{t+1} - k_t = \frac{sf(k_t) - (n+\delta)k_t}{1+n}
\end{align}
で表される。これより、$sf(k_t)$と$(n+\delta)k_t$との大小で一人当たり資本ストックの変化の方向を把握できることがわかる。

定常状態はすなわち$sf(k^*) = (n+\delta)k^*$となる$k^*$であり、ここにおいて一人当たり資本ストックは変化せず、当然一人当たりGDPも成長しない。しかし、全体としては資本ストック、GDPともに人口成長率の$n$で成長している。

\subsection{黄金律}
黄金律水準は一人当たり消費が最大となる一人当たり資本ストック$k_g$が満たすべき次の式を指す。
\begin{align*}
	f^{'}(k_g) = n+\delta
\end{align*}
この時に一人当たり消費が最大となることはグラフより明らかである。また黄金律水準を定常状態で達成することのできる貯蓄率$s_g$は、上の$k_g$に対応する$(n+\delta)k_g$と$s_g f(k_g)$が等しくなるように求めることができる。

\subsection{技術進歩あり}
技術進歩なしのモデルで一人当たりの変数として扱ったものを「効率労働当たりの」変数に変えるだけの違い。ここで効率労働とは、技術水準を表す変数として$A_t$を用いることで書ける$A_t L_t$を指すとする。

技術進歩率を$g^A$で書くと、一人当たりGDPも$g^A$で成長し、全体での資本ストックと実質GDPは$g^A + n$で成長することになる。

\section{恒常所得仮説}
所得を$Y_D$で書き、給料などの恒常的な所得を$Y_P$、ボーナスなどの突発的な所得を$Y_T$と書くとする。この時、所得は以下のように分解できる。
\begin{align*}
Y_D = Y_P + Y_T
\end{align*}
恒常所得仮説は、「消費が恒常所得$Y_P$の関数であり$Y_T$は消費の決定に影響しない」というものである。具体的には$C_t = \beta Y_P$であるとする。




\end{document}






















