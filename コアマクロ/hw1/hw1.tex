\documentclass{article}
\usepackage[margin = .7in]{geometry}
\usepackage[dvipdfmx]{graphicx}
\usepackage{listings}
\usepackage{amsmath}
\usepackage{bm}
\lstset{%
  language={python},
  basicstyle={\small},%
  identifierstyle={\small},%
  commentstyle={\small\itshape},%
  keywordstyle={\small\bfseries},%
  ndkeywordstyle={\small},%
  stringstyle={\small\ttfamily},
  frame={tb},
  breaklines=true,
  columns=[l]{fullflexible},%
  numbers=left,%
  xrightmargin=0zw,%
  xleftmargin=3zw,%
  numberstyle={\scriptsize},%
  stepnumber=1,
  numbersep=1zw,%
  lineskip=-0.5ex%
}

\begin{document}
\title{Macroeconomics 1 2018 S1S2 \\ 
Homework 1}
\author{Kei Ikegami (29186009)}
\maketitle

\section{Problem1}
 We need an additional assumption for stationarity.
 \begin{align}
 	&y_t = c + \phi y_{t-1} + \epsilon_t \nonumber \\
	\Leftrightarrow\ &(1-\phi L)y_t = c + \epsilon_t \nonumber \\
	\Leftrightarrow\ & y_t = (1-\phi L)^{-1} (c + \epsilon_t) = c + \sum_{u = 0}^{\infty} \phi^u L^u \epsilon_t = c + \sum_{u = 0}^{\infty} \phi^u \epsilon_{t-u}
 \end{align}
 For stationarity, it is necessary the above limit exists.  By Chaucy's convergence judgement, we know that it is necessary and sufficient for the convergence of series that the residual, i.e. $\sum_{u = n+1}^{\infty} \phi^u \epsilon_{t-u}$, converges to $0$. Usually, in time series analysis, the Hilbert space, whose product is defined by covariance, is used when the convergence is discussed. Thus if we want to know whether $\sum_{u = n+1}^{\infty} \phi^u \epsilon_{t-u} \to 0$ or not, we must check the convergence w.r.t the norm induced by the covariance product. And for the below calculation we need as assumption that states $Var(y_t) < \infty$. 
\begin{align*}
	\| \sum_{u = n+1}^{\infty} \phi^u \epsilon_{t-u} \| = Var(\sum_{u = n+1}^{\infty} \phi^u \epsilon_{t-u}) = \sum_{n+1}^{\infty} |\phi|^{2u}\sigma^2 = 0
\end{align*}
The second equality is followed by the additional assumption. The third equality is by the assumption $|\phi| < 1$. Now we have the result that $\sum_{u = 0}^{\infty} \phi^u \epsilon_{t-u}$ exists, in other words, the time series is stationary.

Next, I calculate the mean, variance, j-th autocovariance.
\begin{itemize}
	\item[Mean] By $(1)$, $E[y_t] = E\left[ c + \sum_{u = 0}^{\infty} \phi^u \epsilon_{t-u} \right] = c + 0 = c$
	\item[Variance] By $(1)$, $Var(y_t) = Var\left( \sum_{u = 0}^{\infty} \phi^u \epsilon_{t-u} \right) = \sum_{u=0}^{\infty} |\phi|^{2u} \sigma^2 = \frac{\sigma^2}{1-\phi^2}$
	\item[Autocovariance] $\gamma(t,t-j) = \phi Cov(Y_{t-1}, Y_{t-j}) = \cdots = \phi^j Var(Y_{t-j}) = \frac{\phi^u \sigma^2}{1-\phi^2}$
\end{itemize}

\section{Problem2}

\section{Problem3}

\section{Problem4}

\section{Problem5}

\end{document}
























