\documentclass{article}
\usepackage[margin = .7in]{geometry}
\usepackage[dvipdfmx]{graphicx}
\usepackage{listings}
\usepackage{amsmath}
\usepackage{amssymb}
\usepackage{amsfonts}
\usepackage{bm}
\usepackage{mathrsfs}
\lstset{%
  language={python},
  basicstyle={\small},%
  identifierstyle={\small},%
  commentstyle={\small\itshape},%
  keywordstyle={\small\bfseries},%
  ndkeywordstyle={\small},%
  stringstyle={\small\ttfamily},
  frame={tb},
  breaklines=true,
  columns=[l]{fullflexible},%
  numbers=left,%
  xrightmargin=0zw,%
  xleftmargin=3zw,%
  numberstyle={\scriptsize},%
  stepnumber=1,
  numbersep=1zw,%
  lineskip=-0.5ex%
}

\begin{document}
\title{測度論的確率論 2018 S1S2 \\ 
Homework 8}
\author{経済学研究科現代経済コース修士1年 / 池上 慧 (29186009) / sybaster.x@gmail.com}
\maketitle

\section{2.2.1}

\section{2.2.2}
\begin{align*}
	E\left[ \left(\frac{S_n}{n}\right)^2 \right] = \frac{1}{n^2} \sum_{1\geq i,j\geq n} E\left[ X_i X_j \right]
\end{align*}
である。コーシーシュワルツの不等式より、$E\left[ X_i X_j \right] \leq \left( E\left[ X_i^2 \right] E\left[ X_j^2 \right] \right)^{\frac{1}{2}} = r(0)$である。また仮定より、
\begin{align*}
	\forall \epsilon > 0, \exists K > 0\ \text{s.t.}\ k \geq K\ \Rightarrow\ r(k) < \epsilon
\end{align*}
である。これより、$|i-j| \leq K$に対しては$E[X_i X_j] \leq r(0)$とし、$|i-j| > K$に対しては$E[X_i X_j] \leq \epsilon$で抑えられる。すなわち、
\begin{align*}
	\frac{1}{n^2} \sum_{1\geq i,j\geq n} E\left[ X_i X_j \right] \leq \frac{1}{n^2} \left( n(2K-1)r(0) + n^2\epsilon \right) = \frac{2K+1}{n}r(0) + \epsilon
\end{align*}
である。$K$は$n$に依存しないので$\lim \frac{1}{n^2} E[S_n^2] \leq \epsilon$である。従って$0$に$L^2$収束するので$0$に確率収束する。

\section{2.2.3}
Th2.2.14の条件を満たすことを確認する。$U_i$が一様分布に従うことより以下が成立。
\begin{align*}
	E\left[ I_n \right] = E\left[ f(U_i) \right] = \int f(x) \mathrm{d}x
\end{align*}
さらに、$B$を$\mathbb{R}$上の可測集合とすると、$f$が可測関数であることより以下が成立する。
\begin{align*}
	P\left( \omega \in U_1^{-1}\left( f^{-1}(B) \right) \right) \times \cdots \times P\left( \omega \in U_n^{-1}\left( f^{-1}(B) \right) \right) = P\left( \omega \in \left(f\circ U_1\right)^{-1}(B) \right)\times \cdots \times  P\left( \omega \in \left(f\circ U_n\right)^{-1}(B) \right)
\end{align*}
また、
\begin{align*}
	P\left( \omega \in U_1^{-1}\left( f^{-1}(B) \right), \cdots, \omega \in U_n^{-1}\left( f^{-1}(B) \right) \right) = P\left( \omega \in \left(f\circ U_1\right)^{-1}(B), \cdots, \omega \in \left(f\circ U_n\right)^{-1}(B) \right)
\end{align*}
である。$\left\{ U_i \right\}$は独立であることより上二つの左辺は等しい。従って右辺も等しくなり、$\left\{ f(U_i) \right\}$も独立な確率変数である。以上でTh2.2.14の条件が成立することが確認されたので、$I_n \xrightarrow{p} \int_0^1 f(x) \mathrm{d}x$である。

次に$P\left( | I_n - I | > \frac{a}{\sqrt{n}} \right)$を推定するために一様分布の実現値$\left\{ u_i \right\}_{i=1}^n$を$M$セット発生させる。この時得られた$I_n$を$I_n^m$と記し、$\bar{I_n} = \frac{1}{M} \sum_{m=1}^M I_n^m$とする。
\begin{align*}
	P\left( | I_n - I | > \frac{a}{\sqrt{n}} \right) = E\left[ 1\left(  | I_n^m - I | > \frac{a}{\sqrt{n}} \right) \right]
\end{align*}
であるので、右辺のサンプル表記である$\frac{1}{M} \sum_{m=1}^M 1\left(  | I_n^m - \bar{I_n} | > \frac{a}{\sqrt{n}} \right)$が題意の推定をうまく行えることを以下で示す。
\begin{align*}
	E\left[ 1\left(  | I_n^m - \bar{I_n} | > \frac{a}{\sqrt{n}} \right) \right] &= P\left( | (I_n^m - I) + (I - \bar{I_n}) |  > \frac{a}{\sqrt{n}} \right) \\
	&\leq P\left( | I_n^m - I |+ |I - \bar{I_n}|  > \frac{a}{\sqrt{n}} \right)\\
	& \leq P\left( | I_n^m - I |> \frac{a}{\sqrt{n}} \right) + P\left( |I - \bar{I_n}|  > \frac{a}{\sqrt{n}} \right)
\end{align*}
$M$個のサンプルは独立に生成され、上より可積分なのでWLLNより、
\begin{align*}
\frac{1}{M} \sum_{m=1}^M 1\left(  | I_n^m - \bar{I_n} | > \frac{a}{\sqrt{n}} \right) \xrightarrow{p} E\left[ 1\left(  | I_n^m - \bar{I_n} | > \frac{a}{\sqrt{n}} \right) \right]
\end{align*}
である。さらに、チェビシェフの不等式より
\begin{align*}
	\| E\left[ 1\left(  | I_n^m - \bar{I_n} | > \frac{a}{\sqrt{n}} \right) \right] - P\left( | I_n^m - I |> \frac{a}{\sqrt{n}} \right)  \| &\leq P\left( |I - \bar{I_n}|  > \frac{a}{\sqrt{n}} \right)\\
	&\leq \frac{n}{a^2} E\left[ \left( \bar{I_n} - I \right)^2 \right]\\
	&= \frac{n}{Ma^2} Var\left(I_n^m \right)\\
	&= \frac{1}{a^2M} Var\left( f(U_i) \right)
\end{align*}
を得る。従って$M\to \infty$で$E\left[ 1\left(  | I_n^m - \bar{I_n} | > \frac{a}{\sqrt{n}} \right) \right] \to P\left( | I_n^m - I |> \frac{a}{\sqrt{n}} \right)$である。ここで「$a_n \xrightarrow{p} b_n \to b \Rightarrow a_n \xrightarrow{p} b$」であることより
\begin{align*}
	\frac{1}{M} \sum_{m=1}^M 1\left(  | I_n^m - \bar{I_n} | > \frac{a}{\sqrt{n}} \right) \xrightarrow{p} P\left( | I_n^m - I |> \frac{a}{\sqrt{n}} \right) = P\left( | I_n - I |> \frac{a}{\sqrt{n}} \right)
\end{align*}
であり、題意が示された。

\section{2.2.4}

\section{2.2.5}



\section{2.2.6}
フビニの定理より$E[X]$は以下のように変形できる。
\begin{align*}
	E\left[ X \right] = E\left[ \sum_{n = 1}^{\infty} 1(X \geq n) \right] = \sum_{n = 1}^{\infty} E[1(X \geq n)] = \sum_{n = 1}^{\infty} P(X \geq n)
\end{align*}
また同じくフビニの定理より、
\begin{align*}
	E[X^2] = E\left[ 2 \sum_{n=1}^X n - X \right] = E\left[ \sum_{n=1}^{\infty} (2n-1) 1(X \geq n) \right] = \sum_{n=1}^{\infty} (2n-1) P(X \geq n)
\end{align*}
を得る。

\end{document}





























