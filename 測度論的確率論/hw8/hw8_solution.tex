\documentclass{article}
\usepackage[margin = .7in]{geometry}
\usepackage[dvipdfmx]{graphicx}
\usepackage{listings}
\usepackage{amsmath}
\usepackage{amssymb}
\usepackage{amsfonts}
\usepackage{bm}
\usepackage{mathrsfs}
\lstset{%
  language={python},
  basicstyle={\small},%
  identifierstyle={\small},%
  commentstyle={\small\itshape},%
  keywordstyle={\small\bfseries},%
  ndkeywordstyle={\small},%
  stringstyle={\small\ttfamily},
  frame={tb},
  breaklines=true,
  columns=[l]{fullflexible},%
  numbers=left,%
  xrightmargin=0zw,%
  xleftmargin=3zw,%
  numberstyle={\scriptsize},%
  stepnumber=1,
  numbersep=1zw,%
  lineskip=-0.5ex%
}

\begin{document}
\title{測度論的確率論 2018 S1S2 \\ 
Homework 8}
\author{経済学研究科現代経済コース修士1年 / 池上 慧 (29186009) / sybaster.x@gmail.com}
\maketitle

\section{2.2.1}

\section{2.2.2}
\begin{align*}
	E\left[ \left(\frac{S_n}{n}\right)^2 \right] = \frac{1}{n^2} \sum_{1\geq i,j\geq n} E\left[ X_i X_j \right]
\end{align*}
である。コーシーシュワルツの不等式より、$E\left[ X_i X_j \right] \leq \left( E\left[ X_i^2 \right] E\left[ X_j^2 \right] \right)^{\frac{1}{2}} = r(0)$である。また仮定より、
\begin{align*}
	\forall \epsilon > 0, \exists K > 0\ \text{s.t.}\ k \geq K\ \Rightarrow\ r(k) < \epsilon
\end{align*}
である。これより、$|i-j| \leq K$に対しては$E[X_i X_j] \leq r(0)$とし、$|i-j| > K$に対しては$E[X_i X_j] \leq \epsilon$で抑えられる。すなわち、
\begin{align*}
	\frac{1}{n^2} \sum_{1\geq i,j\geq n} E\left[ X_i X_j \right] \leq \frac{1}{n^2} \left( n(2K-1)r(0) + n^2\epsilon \right) = \frac{2K+1}{n}r(0) + \epsilon
\end{align*}
である。$K$は$n$に依存しないので$\lim \frac{1}{n^2} E[S_n^2] \leq \epsilon$である。従って$0$に$L^2$収束するので$0$に確率収束する。

\section{2.2.3}

\section{2.2.4}

\section{2.2.5}



\section{2.2.6}
フビニの定理より$E[X]$は以下のように変形できる。
\begin{align*}
	E\left[ X \right] = E\left[ \sum_{n = 1}^{\infty} 1(X \geq n) \right] = \sum_{n = 1}^{\infty} E[1(X \geq n)] = \sum_{n = 1}^{\infty} P(X \geq n)
\end{align*}
また同じくフビニの定理より、
\begin{align*}
	E[X^2] = E\left[ 2 \sum_{n=1}^X n - X \right] = E\left[ \sum_{n=1}^{\infty} (2n-1) 1(X \geq n) \right] = \sum_{n=1}^{\infty} (2n-1) P(X \geq n)
\end{align*}
を得る。

\end{document}





























