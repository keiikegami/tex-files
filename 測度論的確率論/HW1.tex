\documentclass{article}
\usepackage[margin = .7in]{geometry}
\usepackage[dvipdfmx]{graphicx}
\usepackage{listings}
\usepackage{amsmath}
\usepackage{amssymb}
\usepackage{amsfonts}
\usepackage{bm}
\usepackage{mathrsfs}
\lstset{%
  language={python},
  basicstyle={\small},%
  identifierstyle={\small},%
  commentstyle={\small\itshape},%
  keywordstyle={\small\bfseries},%
  ndkeywordstyle={\small},%
  stringstyle={\small\ttfamily},
  frame={tb},
  breaklines=true,
  columns=[l]{fullflexible},%
  numbers=left,%
  xrightmargin=0zw,%
  xleftmargin=3zw,%
  numberstyle={\scriptsize},%
  stepnumber=1,
  numbersep=1zw,%
  lineskip=-0.5ex%
}

\begin{document}
\title{測度論的確率論 2018 S1S2 \\ 
Homework 1}
\author{経済学研究科現代経済コース修士1年 / 池上 慧 (2918009) / sybaster.x@gmail.com}
\maketitle

\section{問1}
 まず$\mathcal{A}$がfieldであることを示す。そのためには以下の三つの確認をすれば良い。

\begin{enumerate}
	\item $X^c = \phi$ であり、空集合はの要素数は$0$で有限なのでこれは有限集合。よって$X = \mathbb{N} \in \mathcal{A}$である。
	\item $A, B \in \mathcal{A}$を任意に取ってくる。この和集合が$\mathcal{A}$に入っていることを確かめる。$A, B$共に有限集合の時はその和集合も当然有限集合であり、$\mathcal{A}$に入る。片方がその補集合が有限集合である時は、一般性を失わずに$A^c$が有限集合であるとすると、$(A\cup B)^c = A^c \cap B^c$であり、これは有限集合である$A^c$の部分集合であるので有限集合である。従ってこの場合も和集合が$\mathcal{A}$に入る。最後にどちらも補集合が有限集合である場合を考える。この時は$(A\cup B)^c = A^c \cap B^c$であり、有限集合の部分集合となっているので、和集合は$\mathcal{A}$に入る。以上より、$\mathcal{A}$からどのような二つの要素を取ってきてもその和集合は$\mathcal{A}$に入っている。
	\item $A \in \mathcal{A}$を任意にとる。$A^c$が有限集合のケースは定義より$A^c \in \mathcal{A}$である。$A$が有限集合のケースは、$(A^c)^c = A$なので、$A^c$の補集合が有限集合であるので$A^c \in \mathcal{A}$である。従って$\mathcal{A}$の要素の補集合は必ず$\mathcal{A}$に入っている。
\end{enumerate}

次に$\mathcal{A}$が$\sigma$-fieldでないことを示す。このためには可算無限個の和集合について$\mathcal{A}$が閉じていないことを確認すれば良い。$i = 1, 2 \dots$について$A_i = \left\{2i -1\right\}$を$X$の部分集合として取ってくることを考える。これは全ての$i$について要素が一つの有限集合なので$\mathcal{A}$に入っている。

今、$\bigcup_{i = 1}^{\infty} A_i = \left\{ 2i-1 \mid i = 1,2,3, \dots \right\}$であり、これは奇数全体を表す。この集合は有限集合ではなく、またその補集合である偶数全体も有限集合ではない。従って$\bigcup_{i = 1}^{\infty} A_i \not\in \mathcal{A}$である。ゆえに可算無限個の和集合について閉じていないので$\mathcal{A}$は$\sigma$-fieldではない。

\section{問2}
 $m, n$を逆にすれば同じ議論が成り立つので、$\sum_{m=1}^{\infty} \sum_{n = 1}^{\infty} a_{n,m} = sup\left\{ \sum_{m=1}^{M} \sum_{n = 1}^{N} a_{n,m} \mid M, N \in \mathbb{N}\right\}$のみを示す。左を$p$、右を$q$と記す。まず、上限の定義より、任意の$M, N \in \mathbb{N}$で$\sum_{m=1}^{M} \sum_{n = 1}^{N} a_{n,m} \leq q$が成立している。大小関係は極限において保存されるので、左辺で$M, N$を順番に$+\infty$まで飛ばしても大小関係は保たれる。従って$p \leq q$が得られる。

さらに逆向きの不等号も成立することを示すことで題意を示す。$a_{m,n} \geq 0$であるので、任意の$M, N \in \mathbb{N}$について$\sum_{m=1}^{M} \sum_{n = 1}^{N} a_{n,m} \leq p$が成立する。この時$M, N$について左辺の上限を取った時に、$p$を上回ったとすると、上限の定義より$p$よりも大きく上限よりも小さい値をとる$M, N$の組みが存在することになる。しかし先の不等号は任意の$M, N$について成立しているのでこれはあり得ない。従って左辺の上限を取っても大小関係は保存される。よって$q \leq p$であり、逆むきの不等号も示された。以上より$p = q$である。

以上の議論は$p, q$が発散していても同じように成り立つので、その場合についても題意は示された。

\section{問3}
 $\mathcal{C}$を含む最小の$\sigma$-fieldは、以下の手順で構成できる。
\begin{enumerate}
	\item $\sigma(\mathcal{C}) = \phi, X$を用意する。
	\item $\sigma(\mathcal{C})$に$\mathcal{C}$の要素を全て入れる。
	\item 今$\sigma(\mathcal{C})$に入っている要素の補集合でまだ含まれていないものを$\sigma(\mathcal{C})$に加える。
	\item 今$\sigma(\mathcal{C})$に入っている要素の任意の可算個の組み合わせについて和集合をとり、それらでまだ含まれていないものを$\sigma(\mathcal{C})$に加える。
	\item 要素が増えなくなるまで3,4を繰り返す。
\end{enumerate}
このアルゴリズムの結果が一意に存在することはレクチャーノートのlemma1.1によって保証されている。またこのループを以下でサイクルと呼ぶ。

ここで、$\sigma(\mathcal{C})$から任意に一つの要素$B$を取ってきたとき、以下のケースが考えられる。
\begin{enumerate}
	\item $B$は一回前のサイクルで得られていた要素の補集合である
	\item $B$は一回前のサイクルで得られていた要素と、その補集合との可算回の和集合で表現できるもの
\end{enumerate}

2番目のケースで$B \in \sigma(\mathcal{C}_B)$となる$\mathcal{C}$の可算部分集合$\mathcal{C}_B$が存在することが示されたとする。この時、$\sigma$-fieldが要素の補集合について閉じていることを考えると、1番目のケースについては1サイクル前の終了時に得られている$\mathcal{C}$の要素を$B$としてひいた時に構成される$\mathcal{C}_B$に含まれていることから、自動的に示されることになる。従って、2番目のケースについてのみ考えれば良い。

2番目のケースについて題意を満たすような$\mathcal{C}_B$が存在することを帰納法で示す。先のアルゴリズムの2手順目終了時を0回目のサイクル終了時とする。この時、$B$が$\mathcal{C}$の要素であれば、$\mathcal{C}_B = B$とすれば$\mathcal{C}_B \subset \mathcal{C}$でかつ可算集合であり、さらに$B = \mathcal{C}_B \subset \sigma(\mathcal{C}_B)$であるので条件を満たす。また$B = \phi,X$の時についても任意の$\sigma$-fieldにこれらは含まれるので問題ない。

$n = 0,1,2,\dots$で$n$回目のサイクル終了時に存在する$\sigma(\mathcal{C})$の任意の要素$B$については題意を満たす可算集合$\mathcal{C}_B$が存在すると仮定する。この時、$n+1$回目の終了時に新たに得られた$\sigma(\mathcal{C})$の要素$B$について題意を満たす可算部分集合が存在することを示せば良い。

$\sigma(\mathcal{C})$の構成法より、$B = \bigcup_{i = 1}^{\infty} A_i$と書くことができる。ただし、$A_i$は$n$回目のサイクルの終了時に得られた$\sigma(\mathcal{C})$の要素と、その補集合とを合わせた集合の要素である。仮定と、補集合については先の1番目のケースについての議論を援用して、全ての$i$について$A_i \in \sigma(\mathcal{C}_{A_i})$なる可算集合で$\mathcal{C}_{A_i} \subset \mathcal{C}$となるものが存在する。これらに対して$\mathcal{C}_B = \left\{ \mathcal{C}_{A_i} \mid i = 1,2,3,\dots \right\}$とすると、$\mathcal{C}_B$は可算集合が可算個集まったものなので可算集合であり、$\mathcal{C}$の部分集合が集まったものであるため$\mathcal{C}$の部分集合である。また全ての$i$について$A_i \in \sigma(\mathcal{C}_{A_i})$であることより、$A_i \in \sigma(\mathcal{C}_B)$である。さらに$\sigma$-fieldは可算個の和集合について閉じているので、$B = \bigcup_{i = 1}^{\infty} A_i \in \sigma(\mathcal{C}_B)$である。以上より1番目のケースについて題意を満たす$\mathcal{C}_B$を構成することができた。



\section{問4}
 \subsection{(a)}
 問$1$と同様の三つの性質を確認すれば良い。
\begin{enumerate}
	\item 全事象を$X$で表すと、全ての$n$について$X \in \mathcal{A}_n$であるので$X \in \bigcup_n \mathcal{A}_n$である。
	\item $A, B \in \bigcup_n \mathcal{A}_n$を任意にとる。この時、$A\cup B$はある大きな$N$が存在して$\forall n \geq N$に対して$A, B \in \mathcal{A}_n$である。$\sigma$-fieldでは和集合について閉じているため、このような$n$については$A\cup B \in \mathcal{A}_n$である。これより$A\cup B \in \bigcup_n \mathcal{A}_n$である。
	\item $A \in \bigcup_n \mathcal{A}_n$を任意にとる。ここで$\sigma$-fieldの性質より$A^c$は$A$を含む$\mathcal{A}_n$に含まれているため、$A^c \in \bigcup_n \mathcal{A}_n$である。
\end{enumerate}
以上より$\bigcup_n \mathcal{A}_n$がfieldであることが示された。

\subsection{(b)}
 $X = [0,1]$として、$\mathcal{A}_n$を以下のように定める。
\begin{itemize}
	\item $\mathcal{A}_1$ = $\left\{ \phi, X \right\}$
	\item $n \geq2$について$\mathcal{A}_n = \sigma\left(\left\{ \left[0, \frac{1}{2^{n-1}}\right], \left(\frac{1}{2^{n-1}}, \frac{2}{2^{n-1}}\right), \dots, \left(\frac{2^{n-1}-1}{2^{n-1}}, 1\right] \right\}\right)$
\end{itemize}
この時、明らかに$\mathcal{A}_1 \subset \mathcal{A}_2 \subset \mathcal{A}_3 \dots$となる$\sigma$-fieldである。ここで、$B_n = \left( \frac{1}{2^{n-1}}, 1\right]$とすると、ある大きな$N$が存在して$n \geq N$に対しては常に$B_n \in \mathcal{A}_n$である。すなわち$B_n \in \bigcup_n \mathcal{A}_n$であるが、$\bigcup_n B_n = \left(0, 1\right]$であって、これはいかなる大きな$n$に対しても$\mathcal{A}_n$とすることはできない。これは$\bigcup_n \mathcal{A}_n$が可算無限回の和集合について閉じていないことを意味しているため、上のように作成した$\sigma$-fieldの列ではその和集合が$\sigma$-fieldにならないことが確認できた。 

\section{問5}
 $\left( X, \mathcal{A}, \mu \right)$が測度空間であるためには以下の三つを示せば良い。
\begin{enumerate}
	\item $\mathcal{A} \subset 2^X$が$\sigma$-fieldである
	\item $\mu(\phi) = 0$である。
	\item $A_n \in \mathcal{A},\ n = 1,2,3,\dots$が排反の時、$\mu\left( \bigcup_{n = 1}^{\infty} A_n \right) = \sum_{n = 1}^{\infty} \mu\left( A_n \right)$である。
\end{enumerate}

 まず1番目を示すためには、問1と同じように三つの条件を確認する。$\phi = X^c$であり、$\phi$は要素数$0$の可算集合なので$X \in \mathcal{A}$である。$A \in \mathcal{A}$を任意に取ると、$A = (A^c)^c$であるので、$A$が可算集合の時は$A^c$の補集合である$A$が可算集合なので$A^c \in \mathcal{A}$であり、$A^c$が可算集合の時は定義より$A^c \in \mathcal{A}$であるのでいずれにせよ$A$の補集合は$\mathcal{A}$の要素である。$\left\{ A_i \right\}_{i = 1}^{\infty} \in \mathcal{A}$を取ってくることを考える。$\mathcal{A}$の定義より、全ての$i$について$A_i$か$A_i^c$が可算集合であることがわかっている。ここで、全ての$i$について$A_i$が可算集合の時は$\bigcup_i A_i$が可算集合の可算個の和集合になっているのでこれ自体が可算集合となり$\mathcal{A}$に入る。$1$つ以上の$i$について$A_i^c$が可算集合である場合は、$\left( \bigcup_i A_i\right)^c = \bigcap_i A_i^c$であり、仮定よりこれは可算集合の部分集合となるため可算集合である。従ってこの場合も$ \bigcup_i A_i \in \mathcal{A}$である。これより可算有限個の和集合についても閉じているので$\mathcal{A}$は$\sigma$-fieldである。

 2番目の性質は$\phi$が可算集合であることから測度の定義より成立する。

 最後に3番目の性質を確認する。測度の定義域である$\mathcal{A}$から排反に$\left\{ A_i \right\}_{i = 1}^{\infty}$を取った時に、非加算集合が複数この集合に含まれているとする。そこから$2$個の非可算集合をとることを考え、それを$A_m, A_n$と記す。すなわち$A_m \cap A_n = \phi$であるが、どちらも$\mathcal{A}$に含まれるため、$A_m^c, A_n^c$がどちらも可算集合である必要がある。$\left( A_m \cap A_n \right)^c = \phi^c = \mathbb{R}$でありこれは非可算集合であるが、$\left( A_m \cap A_n \right)^c = A_m^c \cup A_n^c$であって、先に述べたようにこれは可算集合の和集合であり可算集合である。これより$A_m^c \cup A_n^c = \mathbb{R}$は成立しない。従って$\mathcal{A}$から排反に$\left\{ A_i \right\}_{i = 1}^{\infty}$を取った時に、非加算集合が複数この集合に含まれていることはありえず、排反に取ったとkには高々一つしか非可算集合が含まれないことがわかる。

 まず$\left\{ A_i \right\}_{i = 1}^{\infty}$の要素が全て可算集合である場合を考えると、この時は$\bigcup_i A_i$が可算集合の可算個の和集合であるため可算集合であり、$\mu \left( \bigcup_i A_i \right) = 0 = \sum_i \mu \left( A_i \right)$が成立する。

 次に$\left\{ A_i \right\}_{i = 1}^{\infty}$の中に一つ、$A_{i^*}^c$が可算集合となるものが含まれているケースを考えると、$\left( \bigcup_i A_i \right)^c = \bigcap_i A_i^c$は可算集合の部分集合であるため可算集合であり、$\mu \left(  \bigcup_i A_i  \right) = 1$である。この時、$\sum_i \mu \left( A_i \right) = \mu \left( A_{i^*} \right) = 1$であるため、こちらでも$\mu \left(  \bigcup_i A_i  \right) = \sum_i \mu \left( A_i \right)$が成立し、3番目の性質についても確認できた。

\end{document}




















