\documentclass{article}
\usepackage[margin = .7in]{geometry}
\usepackage[dvipdfmx]{graphicx}
\usepackage{listings}
\usepackage{amsmath}
\usepackage{amssymb}
\usepackage{amsfonts}
\usepackage{bm}
\usepackage{mathrsfs}
\lstset{%
  language={python},
  basicstyle={\small},%
  identifierstyle={\small},%
  commentstyle={\small\itshape},%
  keywordstyle={\small\bfseries},%
  ndkeywordstyle={\small},%
  stringstyle={\small\ttfamily},
  frame={tb},
  breaklines=true,
  columns=[l]{fullflexible},%
  numbers=left,%
  xrightmargin=0zw,%
  xleftmargin=3zw,%
  numberstyle={\scriptsize},%
  stepnumber=1,
  numbersep=1zw,%
  lineskip=-0.5ex%
}

\begin{document}
\title{測度論的確率論 2018 S1S2 \\ 
Homework 5}
\author{経済学研究科現代経済コース修士1年 / 池上 慧 (29186009) / sybaster.x@gmail.com}
\maketitle

\section{Ex 4.12}
\subsection{(a)}
$\left\{ A_n \right\} \in \mathcal{A}, n = 1,2, \cdots$を排反にとってくる。この時、
\begin{align*}
	\nu \left( \bigcup_{n = 1}^{\infty} A_n \right) &= \int_{\bigcup_{n = 1}^{\infty} A_n} f \mathrm{d}\mu = \int f \left( 1_{A_1} + 1_{A_2} + \cdots \right) \mathrm{d}\mu = \int \lim_{N \to \infty} f \sum_{n = 1}^N 1_{A_n} \mathrm{d}\mu 
\end{align*}
$f$が可測関数であり、指示関数も可測関数であることからCorollary 4.1より
\begin{align*}
	 \int \lim_{N \to \infty} f \sum_{n = 1}^N 1_{A_n} \mathrm{d}\mu = \lim_{N \to \infty} \sum_{n = 1}^N \int f 1_{A_n}\mathrm{d}\mu = \sum_{n = 1}^{\infty} \nu(A_n)
\end{align*}
である。また、$\nu(\phi) = \int_{\phi} f \mathrm{d}\mu$であって、$\mu (\phi) = 0$であるので、$\nu(\phi) = 0$を得る。従って確かに$\nu$は測度となる。

\subsection{(b)}
簡単な関数から順番に示す。まず、$g = 1_B$のような指示関数の場合は、
\begin{align*}
	\int g \mathrm{d}\nu = \int 1_B \mathrm{d}\nu = \nu (B)
\end{align*}
は必ず定義できる。この時、$\nu$の定義より以下が成立し、確かに値は同じになる。
\begin{align*}
	\int gf \mathrm{d}\mu = \int 1_B f \mathrm{d}\mu = \int_B f \mathrm{d}\mu = \nu (B)
\end{align*}
次に$g = \sum_{i = 1}^m g_i 1_{B_i}, g_i \geq 0, \left\{B_1, \cdots, B_m\right\} \in \mathcal{A}$で排反とかけるsimple functionの場合を考える。この時、
\begin{align*}
	\int g \mathrm{d}\mu =\sum_{i = 1}^m g_i \int 1_{B_i} \mathrm{d}\nu = \sum_{i = 1}^m g_i \int 1_{B_i} f \mathrm{d}\mu = \int \left( \sum_{i = 1}^m g_i 1_{B_i} \right) f \mathrm{d}\mu = \int gf \mathrm{d}\mu
\end{align*}
が成立する。ただし二つ目の等号は指示関数のケースの結果より得られる。さらに非負関数$g$について考えると、lemma 4.2より$g_n \uparrow g$となる非負単関数の列が必ず存在するので、
\begin{align*}
	\int g \mathrm{d}\nu = \lim_{n \to \infty} \int g_n \mathrm{d}\nu =\lim_{n \to \infty} \int g_n f \mathrm{d}\mu = \int \lim_{n \to \infty}  g_n f \mathrm{d}\mu = \int gf \mathrm{d}\mu
\end{align*}
が成立する。ここで一つ目と三つ目の等号はMCTより成立し、二つ目の等号は非負単関数のケースの結果より得られる。最後に一般の可測関数$g : X \to \bar{\mathbb{R}}$について題意を示す。非負関数のケースより、以下の等号が成立する。
\begin{align*}
	\int g \mathrm{d}\nu \equiv \int g^{+} \mathrm{d}\nu - \int g^{-} \mathrm{d}\nu = \int g^{+} f \mathrm{d}\mu - \int g^{-} f \mathrm{d}\mu \equiv \int gf \mathrm{d}\mu
\end{align*}
どちらかの項が有限の時、上の積分は定義できる。確かにどちらかの積分が定義できればもう片方もそれに対応する項が有限となるのでもう片方の積分も定義できていることが確認された。

\section{Ex 4.13}
$x \in [0, n]$で$e ^{-x} \geq \left( 1 -\frac{x}{n} \right)^n$を示す。どちらも正なので対数をとって、$-x \geq n {\rm log} \left(1-\frac{x}{n} \right)$を示す。
\begin{align*}
	n {\rm log} \left(1-\frac{x}{n} \right) = n \left\{ -\frac{x}{n} - \frac{1}{2} \frac{x^2}{n^2} - \frac{1}{3} \frac{x^3}{n^3} - \cdots \right\} = -x -\left( \frac{1}{2} \frac{x^2}{n^2} + \frac{1}{3} \frac{x^3}{n^3} + \cdots \right)
\end{align*}
であり、最終項は$x\in [0,n]$で正なので確かに$e ^{-x} \geq \left( 1 -\frac{x}{n} \right)^n$である。これより、$\left| \left( 1 -\frac{x}{n} \right)^n 1_{[0, n]} \right| \leq  e^{-x}$である。また、指数関数の定義より$\lim_{n \to \infty} \left( 1 -\frac{x}{n} \right)^n 1_{[0, n]} = e^{-x}$であるので、優収束定理より以下が成立する。
\begin{align*}
	\lim_{n \to \infty} \int _0^n \left( 1 -\frac{x}{n} \right)^n \mathrm{d}x = \lim_{n \to \infty} \int_0^{\infty} \left( 1 -\frac{x}{n} \right)^n 1_{[0, n]} \mathrm{d}x= \int_0^{\infty} \lim_{n \to \infty} \left( 1 -\frac{x}{n} \right)^n 1_{[0, n]} \mathrm{d}x = \int_0^{\infty} e^{-x} \mathrm{d}x = 1
\end{align*}

\section{Ex 4.14}
以下のような関数列を考える。
\begin{align*}
	f_n (x)= \begin{cases}
	\frac{1}{x} 1_{[1, \infty)} \ \text{$n = 1$}\\[8pt]
	\frac{1}{nx} 1_{[1, n)} \ \text{otherwise}
	\end{cases}
\end{align*}
この時、
\begin{align*}
	\lim_{n \to \infty} \int_{\mathbb{R}} f_n (x) \mathrm{d}x = \lim_{n \to \infty} \int_1^{\infty} \frac{1}{nx} 1_{[1, n]} \mathrm{d}x = \lim_{n \to \infty} \frac{1}{n} \int_1^n \frac{1}{x} \mathrm{d}x = \lim_{n\to \infty} \frac{{\rm log} n}{n} = 0
\end{align*}
であり、また任意の$x\in \mathbb{R}$に対して十分大きな$N$が存在して、$n \geq N$であれば任意の$\epsilon > 0$よりも$\frac{1}{nx}$を小さくできるので$0$に各点収束することもわかる。しかし、全ての$n$について上の関数列の絶対値を抑えるためには$[1, \infty)]$の範囲で$\frac{1}{x}$よりも大きくないといけない。ここで$\int_1^{\infty} \frac{1}{x} \mathrm{d}x = \lim_{n \to \infty} {\rm log}n = \infty$であるため、そのような関数の積分は必ず無限に発散するため$L^1$に入らないことが確認できた。

\section{Ex 4.15}
任意の$n$について、$\left| f_n \right| \leq g_n\ \Rightarrow -g_n \leq f_n \leq g_n\ \Rightarrow f_n + g_n \geq 0,\ g_n -f_n \geq 0$が成立する。この時Fatouの補題より、
\begin{align}
	\int \liminf_{n\to \infty} (f_n + g_n) \mathrm{d}\mu \leq \liminf_{n\to \infty}\left( \int f_n + g_n \mathrm{d}\mu\right)\\
	\int \liminf_{n\to \infty} (g_n - f_n) \mathrm{d}\mu \leq \liminf_{n\to \infty}\left(  \int g_n - f_n \mathrm{d}\mu\right)
\end{align}
が成立する。上の左辺について、
\begin{align*}
	\int \liminf_{n\to \infty}( f_n + g_n) \mathrm{d}\mu = \int (f + g) \mathrm{d}\mu = \int f \mathrm{d}\mu + \int g \mathrm{d}\mu
\end{align*}
である。ただし二つ目の等号は、$|f| \leq g$であることから$f$が可積分であることより得られる。右辺については、全ての$n$について$f_n, g_n$が可積分であることと、「どちらも有界な数列で、$b_n$が$b$に収束する数列の時、$\liminf (a_n + b_n) = \liminf a_n + b$(主張$1$)」を使うと、
\begin{align*}
	\liminf_{n\to \infty} \left(  \int f_n + g_n \mathrm{d}\mu\right) = \liminf_{n \to \infty} \left( \int f_n \mathrm{d}\mu + \int g_n \mathrm{d}\mu \right) = \liminf_{n \to \infty} \int f_n \mathrm{d}\mu + \int g \mathrm{d}\mu
\end{align*}
以上より、
\begin{align*}
	\int f \mathrm{d}\mu + \int g \mathrm{d}\mu \leq  \liminf_{n \to \infty} \int f_n \mathrm{d}\mu + \int g \mathrm{d}\mu\ \Rightarrow \int f \mathrm{d}\mu \leq \liminf_{n \to \infty} \int f_n \mathrm{d}\mu
\end{align*}
を得る。同様にして$(1)$の下の不等式から、
\begin{align*}
	\limsup_{n \to \infty} \int f_n \mathrm{d}\mu \leq \int f \mathrm{d}\mu
\end{align*}
が得られる。以上より、$\liminf_{n \to \infty} \int f_n \mathrm{d}\mu \leq \limsup_{n \to \infty} \int f_n \mathrm{d}\mu$であることから、$\lim_{n \to \infty} \int f_n \mathrm{d}\mu = \int f \mathrm{d}\mu$であることが示された。

\subsection{主張$1$の証明}
$\left\{ a_n \right\}$と$\left\{ b_n \right\}$をどちらも有界な数列とする。この時、まず以下が成立する。
\begin{align*}
	\liminf(a_n) + \liminf(b_n) \leq \liminf(a_n + b_n)
\end{align*}
なぜなら、任意の$k \geq 1$について、
\begin{align*}
	\inf_{n \geq k} a_n + \inf_{n \geq k} b_n \leq a_m + b_m\ \forall\ m \geq k
\end{align*}
であるので、
\begin{align*}
	\inf_{n \geq k} a_n + \inf_{n \geq k} b_n \leq \inf_{n \geq k} (a_n + b_n)
\end{align*}
が成立する。ここで$k \to \infty$をとると、題意を得る。

ここでさらに、$b_n \to n$であるとする。上の結果を使うと、
\begin{align*}
	\liminf (a_n + b_n) \geq \liminf(a_n) + \liminf(b_n) = \liminf(a_n) + b
\end{align*}
である。$\left\{ -b_n \right\}$も有界な数列であるので、$a_n = (a_n+b_n) + (-b_n)$に対して上の結果を使うと、
\begin{align*}
	\liminf(a_n) \geq \liminf(a_n+b_n) + \liminf(-b_n) = \liminf(a_n+b_n) - b\ \Rightarrow \ \liminf(a_n) + b \geq \liminf(a_n+b_n)
\end{align*}
以上より両方の向きについて不等号が成立したので、
\begin{align*}
	\liminf(a_n+b_n) =  \liminf(a_n) + b
\end{align*}
であることが確認できた。

\section{Ex 4.21}
\subsection{(a)}
$f$が非負関数の場合についてまず示す。単関数$s$で$f$を近似し、階段関数$q$で$s$を近似できることを以下で示す。$\left\{ s_i \right\}_{i = 1}^n \in \mathbb{R}$と排反な$\left\{ A_i \right\} \in \mathcal{A}$を使って$s$を以下で定義する。
\begin{align*}
	s(x) = \sum_{i = 1}^n s_i 1_{A_i}
\end{align*}
この時、
\begin{align*}
	\int_{\mathbb{R}} |f-s| \mathrm{d}\mu &= \int_{\mathbb{R}}\left| \sum_{i = 1}^n (f-s_i)1_{A_i} + \left(f- \sum_{i = 1}^n f1_{A_i} \right) \right| \mathrm{d}\mu\\
	&\leq \sum_{i = 1}^n \left( \int_{A_i} f\mathrm{d}\mu - s_i \mu(A_i) \right) + \int_{\mathbb{R}} f\left( 1 - \sum_{i=1}^n 1_{A_i} \right) \mathrm{d}\mu
\end{align*}
である。ここで第2項を$\epsilon$で抑えられることを確認する。Exercise 4.16より、任意の$\epsilon > 0$に対してある$\delta > 0$が存在し、$A\in \mathcal{A}$で、$\mu(A) < \delta \Rightarrow \int_A f \left( 1 - \sum_{i=1}^n 1_{A_i} \right) \mathrm{d}\mu < \frac{\epsilon}{2}$である。ここで、$\left\{ A_i \right\} \in \mathcal{A}$が排反であることより、
\begin{align*}
	\int_A f \left( 1 - \sum_{i=1}^n 1_{A_i} \right) \mathrm{d}\mu = \int_A f1_{\mathbb{R}\setminus \bigcup_i A_i} \mathrm{d}\mu = \int f1_{A \cap \left( \mathbb{R}\setminus \bigcup_i A_i \right)} \mathrm{d}\mu
\end{align*}
とかける。ここで、$A \in \mathcal{A}$であることより以下は$\mathcal{A}$に入ることがわかる。
\begin{align*}
	A \cap \left( \mathbb{R}\setminus \bigcup_i A_i \right) = A \cap \left( \bigcap_i (\mathbb{R} \setminus A_i) \right) = \bigcap_i \left( (\mathbb{R} \setminus A_i) \cap A \right) = \mathbb{R} \setminus \bigcup_i \left( \mathbb{R}\setminus \left( (\mathbb{R} \setminus A_i) \cap A \right) \right)
\end{align*}
$A_i^{*} = \mathbb{R}\setminus \left( (\mathbb{R} \setminus A_i) \cap A \right)$と置き直すと、
\begin{align*}
	\int_{\mathbb{R}} \left( f - \sum_i^n 1_{A_i^{*}} \right) \mathrm{d}\mu < \frac{\epsilon}{2}
\end{align*}
とすることができる。ここで$f$が可積分であることから、$s_i^{*} = \frac{\int_{A_i^{*}} f\mathrm{d}\mu}{\mu(A_i^{*})}$とおくことで第1項は$0$とできるので、単関数を使って$f$を近似できることが確認できた。

次に階段関数$q$で$s$を近似することを示す。$q$を以下で定義する。
\begin{align*}
	q = \sum_{j = 1}^k c_j 1_{(a_{j-1}, a_j)}
\end{align*}
lemma 2.1より単関数を定義している$A_i$それぞれについて任意の$\epsilon > 0$に対して$\frac{\epsilon}{2n\sum_j c_j} > 0$をとることができ、$\left\{ D_1^i, D_2^i, \cdots, D_M^i \right\}$なる開区間が存在して、
\begin{align*}
	\lambda\left( A_i \Delta \left( \bigcup_{m = 1}^M D_m^i \right) \right) < \frac{\epsilon}{2n\sum_j c_j} 
\end{align*}
とできる。ここで$q$を以下のように定義する。
\begin{align*}
	q = \begin{cases}
	s_i \ &\text{when}\ x \in D_m^i\ \forall\ i,\forall\ m\\
	0\ &\text{otherwise}
	\end{cases}
\end{align*}
この時、
\begin{align*}
	\int_{\mathbb{R}} \left| s - q \right| \mathrm{d}\mu < \frac{\epsilon}{2n\sum_j c_j} \times n \times \sum_i s_i = \frac{\epsilon}{2}
\end{align*}
である。以上より、
\begin{align*}
	\int |f-q|\mathrm{d}\mu \leq \int |f-s| \mathrm{d}\mu + \int |s-q | \mathrm{d}\mu < \frac{\epsilon}{2} + \frac{\epsilon}{2} = \epsilon
\end{align*}
とすることができる。

一般の可測関数$f$について、積分を定義したときと同様に$f^{+}, f^{-}$をそれぞれ非負関数として定義すると、
\begin{align*}
	\int |f-q|\mathrm{d}\mu = \int |f^{+} - q^{+}| \mathrm{d}\mu + \int |f^{-} - q^{-}| \mathrm{d}\mu
\end{align*}
とできるような階段関数がそれぞれに対して存在することが以上よりわかっている。ここで、階段関数$q = q^{+} + q^{-}$を用いれば、任意の$\epsilon > 0$に対してそれぞれの項を$\frac{\epsilon}{2}$で抑えるように$q$を作ることが必ずできる。従って、
\begin{align*}
	\int |f-q|\mathrm{d}\mu < \epsilon
\end{align*}
を得る。

\subsection{(b)}
$q$を$g$で近似できることを確認すれば、$(a)$の結果と合わせて題意を得る。階段関数の各不連続点$a_j$について二点$(a_j-\delta, c_j)$と$(a_j + \delta, c_{j+1})$とを直線で結ぶことで連続関数にする。この時、
\begin{align*}
	\int |g -q|\mathrm{d}\mu = \sum_j \left( \frac{|c_{j+1} - c_j|}{2} \times 2\delta \times \frac{1}{2} \right) = \frac{\delta}{2} \sum_j | c_{j+1} - c_j |
\end{align*}
であるので、$\delta \to 0$で任意の$\frac{\epsilon}{2} > 0$で抑えることができる。従って、
\begin{align*}
	\int |f-g|\mathrm{d}\mu \leq \int |f-q| \mathrm{d}\mu + \int |q-g | \mathrm{d}\mu < \frac{\epsilon}{2} + \frac{\epsilon}{2} = \epsilon
\end{align*}
を得る。この関数は明らかに有界区間以外では$0$をとるので題意は示された。

\section{Ex 4.22}
$f$が階段関数の時にこの補題が成立することをまず確認する。$a = x_0 < x_1 < \cdots < x_{J-1} < x_J = b$として、$[a,b]$上の階段関数$s$を以下のように定義する。
\begin{align*}
	s(x) = c_j\ \text{when $x_{j-1} \leq x < x_j $}
\end{align*}
この時、
\begin{align*}
	\left| \int s(x) \cos(nx) \mathrm{d}x \right|= \left| \sum_{j = 1}^J \int_{x_{j-1}}^{x_j} c_j \cos(nx) \mathrm{d}x \right|= \left| \sum_{j=1}^J c_j \frac{\sin(nx_{j}) - \sin(nx_{j-1})}{n}\right| \leq \left| \sum_{j = 1}^J c_j \frac{2}{n} \right|
\end{align*}
であり、$n\to \infty$で$0$に収束する。今、コンパクトな台$[a, b]$を持つ関数$f$についてのみ考えると、階段関数$s(x)$を使って、三角不等式より以下が得られる。
\begin{align*}
	\left| \int_a^b f(x) \cos(nx) \mathrm{d}x \right|= \left| \int_a^b (f(x)-s(x)) \cos(nx) + s(x)\cos(nx) \mathrm{d}x \right| \leq \left| \int_a^b (f(x)-s(x)) \cos(nx) \mathrm{d}x \right| + \left| \int_a^b s(x)\cos(nx) \mathrm{d}x \right|
\end{align*}
Ex 4.21より任意の$\epsilon > 0$について、最後の第1項を$\frac{\epsilon}{2}$で抑えるような階段関数$s(x)$が必ず存在する。また、階段関数$s(x)$については先の議論よりリーマンルベーグの補題がすでに示されているので、ある大きな$N$で、$n \geq N$であれば第2項を$\frac{\epsilon}{2}$で抑えることができるようなものが存在する。この時、同じ$N$について、
\begin{align*}
	n \geq N \Rightarrow \left| \int_a^b f(x) \cos(nx) \mathrm{d}x \right| < \frac{\epsilon}{2} + \frac{\epsilon}{2} = \epsilon
\end{align*}
とできるので、コンパクトな台を持つ関数$f$については題意を示すことができた。ここで、$|f(x) \cos(nx)| \leq |f|$であり、仮定より$\int |f| \mathrm{d}\mu < \infty$であるので、DCTが使えて以下が成立する。
\begin{align*}
	\lim_{n \to \infty} \int_{\mathbb{R}} f(x) \cos(nx) \mathrm{d}x &= \int \lim_{n \to \infty} f(x) \cos(nx) \mathrm{d}x\\
	&= \int_{\mathbb{R}} \lim_{m \to \infty} \left( \lim_{n \to \infty} f(x) \cos(nx) \right) 1_{[-m, m]} \mathrm{d}x
\end{align*}
さらに、$\left|\lim_{n \to \infty} f(x) \cos(nx) 1_{[-m, m]}\right| < |f|$であるので、同様にDCTを適用でき、
\begin{align*}
	\int_{\mathbb{R}} \lim_{m \to \infty} \left( \lim_{n \to \infty} f(x) \cos(nx) \right) 1_{[-m, m]} \mathrm{d}x &= \lim_{m\to \infty} \int_{\mathbb{R}} \lim_{n \to \infty} f(x) \cos(nx) 1_{[-m, m]} \mathrm{d}x\\
	&= \lim_{m \to \infty} \lim_{n \to \infty} \int_{-m}^m f(x) \cos(nx) \mathrm{d}x
\end{align*}
である。ここで任意の$m$について$[-m, m]$はコンパクトな台で、そのような関数に対してはリーマンルベーグの補題が示されているので、上は$0$となり、一般の可測関数$f$について題意が示された。

\section{Ex 5.3}
対称性より$\rm(\hspace{.18em}i\hspace{.18em}) \Leftrightarrow \rm(\hspace{.08em}ii\hspace{.08em})$のみ示せば十分である。$\rm(\hspace{.08em}ii\hspace{.08em})$の否定は以下のように書ける。
\begin{align*}
	\mu\left( \left\{ x \mid \nu\left( \left\{ y \mid f(x, y) \neq 0 \right\} \right) > 0 \right\} \right) > 0
\end{align*}
これは以下と同値である。
\begin{align*}
	\exists A \in \mathcal{A}\ \text{s.t.}\ \mu(A) > 0,\ \nu\left( \left\{ y \mid f(x, y) \neq 0 , x\in A \right\} \right) > 0
\end{align*}
さらにこれは以下を意味する。
\begin{align*}
	\exists A \in \mathcal{A}, B \in \mathcal{C}\ \text{s.t.}\ \begin{cases} \mu(A) > 0, \nu(B) > 0 \\[8pt] f(x, y) \neq 0\ \forall (x, y) \in A \times B \end{cases}
\end{align*}
この時$\rho(\left\{ (x, y)\mid f(x, y) \neq 0 \right\}) \geq \rho(A \times B) = \mu(A) \nu(B) > 0$なので、対偶から$\rm(\hspace{.18em}i\hspace{.18em}) \Rightarrow \rm(\hspace{.08em}ii\hspace{.08em})$が示された。

次に$\rm(\hspace{.08em}ii\hspace{.08em}) \Rightarrow \rm(\hspace{.18em}i\hspace{.18em})$を示す。こちらについても対偶を考える。
\begin{align*}
	\rho\left( (x, y) \mid f(x, y) \neq 0\right) > 0 \ \Rightarrow \ \mu\left( \left\{ x \mid \nu\left( \left\{ y \mid f(x, y) \neq 0 \right\} \right) > 0 \right\} \right) > 0
\end{align*}
を示せばよい。左辺は以下を意味する。
\begin{align*}
	\exists A\in \mathcal{A}, B \in \mathcal{C}\ \text{s.t.}\ \rho(A \times B) = \mu (A) \nu(B) > 0
\end{align*}
ここで、$x\in A$を任意にとると、そのような$x$について$y \in B$は必ず$f(x, y) \neq 0$となる。すなわち、
\begin{align*}
	A \subset \left\{ x \mid \nu\left( \left\{ y \mid f(x, y) \neq 0 \right\} \right) > 0 \right\} 
\end{align*}
である。測度の単調性より、
\begin{align*}
	\mu\left( \left\{ x \mid \nu\left( \left\{ y \mid f(x, y) \neq 0 \right\} \right) > 0 \right\} \right) \geq \mu(A) > 0
\end{align*}
を得る。これより対偶が示された。

\section{Ex 5.4}
\subsection{(a)}
以下の集合を考える。ただし$\left\{ (z_i ,z_{i+1}] \right\}_{i = 0}^{n-1}$は$(a, b]$の分割で、$a = z_0 < z_1 < \cdots < z_{n-1} < z_n = b$である実数から作られ、これは$[a,b]$を$n$等分する分割であるとする。
\begin{align*}
	B_n = \bigcup_{i = 1}^n (a, z_i] \times (z_{i-1}, z_i]
\end{align*}
ここで、
\begin{align}
	\left\{ (x, y) \mid a < x \leq y \leq b \right\} = \lim_{n\to \infty} B_n
\end{align}
であることをまず示す。$B_n$の構成より、任意の$n$について$\left\{ (x, y) \mid a < x \leq y \leq b \right\} \subset B_n$であるので、$\left\{ (x, y) \mid a < x \leq y \leq b \right\} \subset \lim_{n\to \infty} B_n$である。また、任意の$n$について$(x, y) \in B_n$であるような点をとると、$B_n$の構成より明らかに$a < x \leq b, a < y \leq b$である。ここでもし$x > y$であるとすると、$\frac{x-y}{\sqrt{2}} > 0$である。$B_n$が$\left\{ (x, y) \mid a < x \leq y \leq b \right\}$からはみ出ている部分は、最大でも$y = x$の直線との距離が$\frac{b-a}{\sqrt{2}n}$より短い部分までであるので、$n$を十分大きく取った時、必ず$x > y$であるような点を含まない$B_n$を作ることができる。すなわちそのような$(x, y)$は$\lim_{n\to \infty} B_n$に含まれず、またその構成から明らかに$x\leq y$となる点は任意の$n$について含まれている。以上より$(3)$が示された。

$B_n$は長方形集合の可算個の和集合でかけているので、$(\mu \times \nu)$を定義に従って計算することができる。測度の連続性より以下を得る。
\begin{align*}
	(\mu \times \nu)\left( \left\{ (x, y) \mid a < x \leq y \leq b \right\} \right) = (\mu \times \nu) \left( \lim_{n\to \infty} B_n \right) = \lim_{n \to \infty} (\mu \times \nu)\left( B_n \right)
\end{align*}
ここで、$B_n$を構成する長方形集合は互いに排反なので以下のように計算できる。
\begin{align*}
	(\mu \times \nu)\left( B_n \right) = \sum_{i = 1}^n \mu \left( (a, z_i] \right) \nu \left( (z_{i-1},z_i] \right) = \sum_{i = 1}^n \nu \left( (z_{i-1},z_i] \right) \left( F(z_i) - F(a) \right)
\end{align*}
ここで、$H_n = \sum_{i = 1}^n \left( F(z_i) - F(a) \right) i_{(z_{i-1}, z_i]}$と置くと、単関数についてのルベーグ積分の定義より、
\begin{align*}
	\sum_{i = 1}^n \nu \left( (z_{i-1},z_i] \right) \left( F(z_i) - F(a) \right) = \int_{(a, b]} H_n \mathrm{d}\nu
\end{align*}
が成立する。ここで、$H_n$は可測関数であり、$F$が非現象な関数であることより$F(x) - F(a)$に下から近づいて行く関数であり、さらに$F$が右連続であるからこれは$F$に収束する関数列となる。($F$が右連続でない時、不連続点において$H_n$の値が右極限よりも小さな左極限に固定されてしまい、これでは$F - F(a)$を近似することはできない。)従って単調収束定理を適用でき、以下を得る。
\begin{align*}
	(\mu \times \nu)\left( \left\{ (x, y) \mid a < x \leq y \leq b \right\} \right) &= \lim_{n \to \infty} (\mu \times \nu)\left( B_n \right)\\
	&=  \lim_{n \to \infty} \int_{(a, b]} H_n \mathrm{d}\nu\\
	&= \int_{(a,b]}(F(x) - F(a)) \mathrm{d}\nu\\
	&= \int_{(a, b]} (F(y) - F(a)) \mathrm{d}G(y)
\end{align*}

\subsection{(b)}
まず以下の積分を考える。
\begin{align*}
	\int_{(a,b]} \left( G(b) - G(x) \right) \mathrm{d}F(x)
\end{align*}
問題$(a)$と同様の議論によりこれは以下の表現と同じである。
\begin{align*}
	\lim_{n \to \infty} \sum_{i = 1}^n \left( G(b) - G(z_i) \right) \mu\left( (z_{i-1}, z_i] \right) = \lim_{n \to \infty} \sum_{i = 1}^n \nu((z_i,b]) \mu\left( (z_{i-1}, z_i] \right)
\end{align*}
ここで、以下の集合を考える。
\begin{align*}
	C_n = \bigcup_{i = 1}^n (z_{i-1}, z_i] \times (z_i , b]
\end{align*}
これを用いると、積測度の定義より以下が成立する。
\begin{align*}
	\lim_{n \to \infty} \sum_{i = 1}^n \nu((z_i,b]) \mu\left( (z_{i-1}, z_i] \right) = (\mu \times \nu) \left( \bigcup_{n=1}^{\infty} C_n \right)
\end{align*}
この$C_n$の和集合については以下の関係が成立する。
\begin{align}
	\bigcup_{n=1}^{\infty} C_n = \left\{ (x, y) \mid a < x \leq y \leq b \right\} \setminus \left\{ (x, y) \mid a < x = y \leq b \right\}
\end{align}
まず$C_n$の構成より、任意の$n$について$C_n \subset  \left\{ (x, y) \mid a < x \leq y \leq b \right\} \setminus \left\{ (x, y) \mid a < x = y \leq b \right\}$である。従って$\bigcup_{n=1}^{\infty} C_n$が右辺に含まれることが確認できる。また、右辺の集合から任意に要素$(x, y)$をとると、
\begin{align*}
\begin{cases}
	a < x \leq b\\
	a < y \leq b\\
	x < y
\end{cases}
\end{align*}
を満たす。$C_n$の構成より、$(4)$の右辺の集合において、$y = x$の直線から$\frac{b-a}{\sqrt{2}n}$より離れた部分は全て$C_n$に含まれる。従って、取ってきた点と直線$y = x$との距離である$\frac{y-x}{\sqrt{2}}$は、十分大きな$n$の時に必ず$\frac{b-a}{\sqrt{2}n}$よりも大きくなってしまう。これより、どこかしらの$C_n$に含まれるため、$\bigcup_{n=1}^{\infty} C_n$に含まれることになる。こうして逆向きの包含関係も示されたため$(4)$が示された。

以上より、$(a)$と合わせて以下の関係が成立する。
\begin{align*}
	\begin{cases}
		\int_{(a, b]} F(y) \mathrm{d}G(y) = \int_{(a, b]} F(a) \mathrm{d}G(y) + (\mu \times \nu)\left( \left\{ (x, y) \mid a < x \leq y \leq b \right\} \right)\\[8pt]
		\int_{(a, b]} G(x) \mathrm{d}F(y) = \int_{(a, b]} G(b) \mathrm{d}F(x) - (\mu \times \nu)\left( \left\{ (x, y) \mid a < x \leq y \leq b \right\} \right) + (\mu \times \nu) \left( \left\{ (x, y) \mid a < x = y \leq b \right\} \right)
	\end{cases}
\end{align*}
この辺々を足し合わせて、
\begin{align*}
	\int_{(a, b]} F(y) \mathrm{d}G(y) + \int_{(a, b]} G(x) \mathrm{d}F(y) &= \int_{(a, b]} F(a) \mathrm{d}G(y) + \int_{(a, b]} G(b) \mathrm{d}F(x) + (\mu \times \nu) \left( \left\{ (x, y) \mid a < x = y \leq b \right\} \right)\\[8pt]
	&= F(b)G(b) - F(a)G(a) + \sum_{a < x \leq b} \mu\left( \left\{ x \right\} \right) \nu\left( \left\{ x \right\} \right)
\end{align*}
であり題意は示された。

\section{Ex 5.5}
\begin{align*}
	\int_{\mathbb{R}} \left\{ F(x + c) - F(x) \right\}\mathrm{d}x = \int_{\mathbb{R}} \left\{  \mu((-\infty, x+c]) - \mu((-\infty, x]) \right\}\mathrm{d}x = \int_{\mathbb{R}} \int_{(x, x+c]} \mathrm{d}\mu \mathrm{d}x
\end{align*}
「Lebesgue測度は$\sigma$ finiteである(主張$2$)」を所与とする。この時$\mu$が有限測度であることから、どちらかの逐次積分が発散しなければフビニの定理を用いることができる。
\begin{align*}
	 \int_{\mathbb{R}} \int_{\mathbb{R}} 1_{(x, x+c]} \mathrm{d}x \mathrm{d}\mu = \int_{\mathbb{R}} c \mathrm{d}\mu = c \mu(\mathbb{R}) < \infty
\end{align*}
であるのでフビニの定理より上の値が求めたい積分の値と一致する。

\subsection{主張$2$の証明}
可測空間$(\mathbb{R}, \mathcal{B}, \lambda)$を考えたき、$\mathbb{R} = \bigcup_{k = 1}^{\infty} [-k,k]$とかける。ここで任意の$k$について$\lambda\left( [-k, k] \right) = 2k < \infty$であるので、定義より確かにルベーグ測度は$\sigma$ finiteである。

\section{Ex 5.6}
ヒントに従い下の積分を得る。
\begin{align*}
	\int_0^a e^{-xy} \sin(x) \mathrm{d}y = -\frac{\sin(x)}{x} e^{-ax} + \frac{\sin(x)}{x}
\end{align*}
これより、
\begin{align*}
	\int_0^b \frac{\sin(x)}{x} e^{-ax} \mathrm{d}x = \int_0^b \left\{ \frac{\sin(x)}{x} - \int_0^a e^{-xy}\sin(x) \mathrm{d}y \right\} \mathrm{d}x = \int_0^b \frac{\sin(x)}{x} \mathrm{d}x - \int_0^b \int_0^a e^{-xy}\sin(x) \mathrm{d}y \mathrm{d}x
\end{align*}
が成立する。「この第2項にフビニの定理を用いることができる(主張$3$)」とすると、さらに以下のように変形できる。
\begin{align*}
	\int_0^b \frac{\sin(x)}{x} \mathrm{d}x - \int_0^a \int_0^b e^{-xy}\sin(x) \mathrm{d}x \mathrm{d}y = \int_0^b \frac{\sin(x)}{x} \mathrm{d}x + \int_0^a \frac{1}{1 + y^2} \left( ye^{-by} \sin(b) + e^{-by} \cos(b) -1 \right) \mathrm{d}y
\end{align*}
ここで第1項に注目する。同様の議論により、
\begin{align*}
	\int_0^b \frac{\sin(x)}{x} \mathrm{d}x = \int_0^b \left( \int_0^{\infty} e^{-xy} \mathrm{d}y \right) \sin(x) \mathrm{d}x
\end{align*}
であり、「この積分にはフビニの定理が使える(主張$4$)」ので、
\begin{align*}
	\int_0^{\infty} \int_0^b e^{-xy} \sin(x) \mathrm{d}x \mathrm{d}y =\frac{\pi}{2} - \int_0^{\infty} \frac{1}{1 + y^2} \left( ye^{-by} \sin(b) + e^{-by} \cos(b)\right) \mathrm{d}y
\end{align*}
とできる。ただし、$\int0^{\infty} \frac{1}{1 + y^2} \mathrm{d}y$が$y = \tan(\theta)$の変数変換で$\frac{\pi}{2}$となることを用いている。以上を合わせると、与式の左辺は以下のように変形できる。
\begin{align*}
	\left| \int_0^b \frac{\sin(x)}{x} e^{-ax} \mathrm{d}x - \frac{\pi}{2} \right| &= \left| \int_0^{\infty} \frac{1}{1 + y^2} \left( ye^{-by} \sin(b) + e^{-by} \cos(b)\right) \mathrm{d}y - \int_0^a \frac{1}{1 + y^2} \left( ye^{-by} \sin(b) + e^{-by} \cos(b) \right) \mathrm{d}y \right|\\
	&= \left| \int_0^a \frac{1}{1 + y^2} \mathrm{d}y +  \int_a^{\infty}  \frac{1}{1 + y^2} \left( ye^{-by} \sin(b) + e^{-by} \cos(b)\right) \mathrm{d}y \right|\\[8pt]
	&\leq \left|  \int_0^a \frac{1}{1 + y^2} \mathrm{d}y \right| + \left|  \int_a^{\infty}  \frac{1}{1 + y^2} \left( ye^{-by} \sin(b) + e^{-by} \cos(b)\right) \mathrm{d}y \right|
\end{align*}
ただし最終行には三角不等式を用いた。ここで$y = \tan(\theta)$で変数変換を行うと、$a > 0$より第1項は$\arctan(a)$。第2項は$b$を極限に飛ばした時の挙動さえ分かれば良いので、優収束定理を適用できることを確認する。
\begin{align*}
	 \int_a^{\infty} \left|  \frac{1}{1 + y^2} \left( ye^{-by} \sin(b) + e^{-by} \cos(b)\right) \right| \mathrm{d}y = \int_0^{\frac{\pi}{2}} \left|  \frac{\tan(\theta) \sin(b) + \cos(b)}{e^{b\tan(\theta)}} \right| \mathrm{d}\theta
\end{align*}
であり、これに対して、
\begin{align*}
	\left|  \frac{\tan(\theta) \sin(b) + \cos(b)}{e^{b\tan(\theta)}} \right| &\leq \left| \frac{\tan(\theta) \sin(b) + \cos(b)}{1 + b\tan(\theta)} \right| \leq \left| \frac{\sin(b)\tan(\theta)}{1 + b\tan(\theta)}\right| +  \left|\frac{\cos(b)}{1 + b\tan(\theta)} \right| \\[8pt]
	&\leq \left| \frac{\sin(b)}{b} \right| + \left| \cos(b) \right| \leq 2
\end{align*}
であることから有界範囲$[0, \frac{\pi}{2}]$についてこれが可積分であるので優収束定理を用いることができる。ただし、一つ目の不等号には$e^x \geq 1 + x$を、二つ目の不等号には三角不等式を用い、三つ目の不等式は$b > 0, \tan(\theta) > 0$であることから得られ、最後の不等式は「$\frac{\sin(x)}{x} < 1$(主張$5$)」から得られる。

 以上から、$\delta(b) = \int_a^{\infty}  \frac{1}{1 + y^2} \left( ye^{-by} \sin(b) + e^{-by} \cos(b)\right) \mathrm{d}y $とおくと、
\begin{align*}
	\lim_{b \to \infty} \delta(b) = \int_a^{\infty} \lim_{b \to \infty} \frac{1}{1 + y^2} \left( ye^{-by} \sin(b) + e^{-by} \cos(b)\right) \mathrm{d}y = \int_a^{\infty} 0 \mathrm{d}y = 0
\end{align*}
となるので確かに題意である、
\begin{align*}
	\left| \int_0^b \frac{\sin(x)}{x} e^{-ax} \mathrm{d}x - \frac{\pi}{2} \right| = \arctan(a) + \delta (b)
\end{align*}
なる表記を得られた。

\subsection{主張$3,4$の証明}
主張$3$は主張$4$が成立すれば明らかに成立するので、主張$4$のみ示す。ルベーグ測度が$\sigma$ finiteであることに注意すると、以下の積分が発散しないことを示せばフビニの定理が適用できる。
\begin{align*}
	\int_0^b \int_0^{\infty} \left| e^{-xy}  \sin(x) \right| \mathrm{d}y \mathrm{d}x
\end{align*}
$\left| \sin(x) \right| \leq 1$であることより、
\begin{align*}
	\int_0^b \int_0^{\infty} \left| e^{-xy}  \sin(x) \right| \mathrm{d}y \mathrm{d}x \leq \int_0^b \int_0^{\infty} e^{-xy} \mathrm{d}y \mathrm{d}x = \int_0^b \frac{1}{x} \mathrm{d}x = {\rm log}b < \infty
\end{align*}
が成立するので確かにフビニの定理が適用できる。

\subsection{主張$5$の証明}
$x \in (0,1]$の範囲でのみ示せば、それ以外では明らかである。中間値の定理より、
\begin{align*}
	\frac{\mathrm{d} \sin(x_0)}{\mathrm{d}x} = \frac{\sin(x)}{x}
\end{align*}
を満たす$x_0$が$(0,x)$の範囲で必ず存在する。すなわち、
\begin{align*}
	1 > \cos(x_0) = \frac{\sin(x)}{x}
\end{align*}
であり主張が示された。

\end{document}































