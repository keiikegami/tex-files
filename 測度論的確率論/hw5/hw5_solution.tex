\documentclass{article}
\usepackage[margin = .7in]{geometry}
\usepackage[dvipdfmx]{graphicx}
\usepackage{listings}
\usepackage{amsmath}
\usepackage{amssymb}
\usepackage{amsfonts}
\usepackage{bm}
\usepackage{mathrsfs}
\lstset{%
  language={python},
  basicstyle={\small},%
  identifierstyle={\small},%
  commentstyle={\small\itshape},%
  keywordstyle={\small\bfseries},%
  ndkeywordstyle={\small},%
  stringstyle={\small\ttfamily},
  frame={tb},
  breaklines=true,
  columns=[l]{fullflexible},%
  numbers=left,%
  xrightmargin=0zw,%
  xleftmargin=3zw,%
  numberstyle={\scriptsize},%
  stepnumber=1,
  numbersep=1zw,%
  lineskip=-0.5ex%
}

\begin{document}
\title{測度論的確率論 2018 S1S2 \\ 
Homework 5}
\author{経済学研究科現代経済コース修士1年 / 池上 慧 (29186009) / sybaster.x@gmail.com}
\maketitle

\section{Ex 4.12}
\subsection{(a)}
$\left\{ A_n \right\} \in \mathcal{A}, n = 1,2, \cdots$を排反にとってくる。この時、
\begin{align*}
	\nu \left( \bigcup_{n = 1}^{\infty} A_n \right) &= \int_{\bigcup_{n = 1}^{\infty} A_n} f \mathrm{d}\mu = \int f \left( 1_{A_1} + 1_{A_2} + \cdots \right) \mathrm{d}\mu = \int \lim_{N \to \infty} f \sum_{n = 1}^N 1_{A_n} \mathrm{d}\mu 
\end{align*}
$f$が可測関数であり、指示関数も可測関数であることからCorollary 4.1より
\begin{align*}
	 \int \lim_{N \to \infty} f \sum_{n = 1}^N 1_{A_n} \mathrm{d}\mu = \lim_{N \to \infty} \sum_{n = 1}^N \int f 1_{A_n}\mathrm{d}\mu = \sum_{n = 1}^{\infty} \nu(A_n)
\end{align*}
であるので確かに$\nu$は測度となる。($\phi$については後で書く。)

\subsection{(b)}


\section{Ex 4.13}

\section{Ex 4.14}

\section{Ex 4.15}

\section{Ex 4.21}

\section{Ex 4.22}

\section{Ex 5.3}

\section{Ex 5.4}

\section{Ex 5.5}

\section{Ex 5.6}


\end{document}































