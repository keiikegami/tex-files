\documentclass{article}
\usepackage[margin = .7in]{geometry}
\usepackage[dvipdfmx]{graphicx}
\usepackage{listings}
\usepackage{amsmath}
\usepackage{amssymb}
\usepackage{amsfonts}
\usepackage{bm}
\usepackage{mathrsfs}
\lstset{%
  language={python},
  basicstyle={\small},%
  identifierstyle={\small},%
  commentstyle={\small\itshape},%
  keywordstyle={\small\bfseries},%
  ndkeywordstyle={\small},%
  stringstyle={\small\ttfamily},
  frame={tb},
  breaklines=true,
  columns=[l]{fullflexible},%
  numbers=left,%
  xrightmargin=0zw,%
  xleftmargin=3zw,%
  numberstyle={\scriptsize},%
  stepnumber=1,
  numbersep=1zw,%
  lineskip=-0.5ex%
}

\begin{document}
\title{測度論的確率論 2018 S1S2 \\ 
Homework 4}
\author{経済学研究科現代経済コース修士1年 / 池上 慧 (29186009) / sybaster.x@gmail.com}
\maketitle

\section{Ex4.2}
 $D = \left\{ x \in X \mid f(x) \neq 0 \right\} = \bigcup_{n = 1}^{\infty} \left\{ x\in X \mid |f(x)| > \frac{1}{n} \right\} = \bigcup_{n = 1}^{\infty} \left\{ x\in X \mid f(x) > \frac{1}{n} \right\}$
である。lemma3.5より$E_n \equiv \left\{ x\in X \mid f(x) > \frac{1}{n} \right\} $は可測集合の要素であるので、sigma fielsが可算個の和集合について閉じていることから$D$も可測集合に入る。したがって$\mu(D)$は定義されている。さらに、$E_n$が単調増大な集合なので、lemma1.3より
\begin{align*}
	\mu(D) = \mu \left(\bigcup_{n = 1}^{\infty} \left\{ x\in X \mid f(x) > \frac{1}{n} \right\}\right) = \lim_{n\to \infty} \mu(E_n)
\end{align*}
である。ここでチェビシェフの不等式から、任意の$n \geq 1$について
\begin{align*}
	\mu(E_n) \leq n \int_{E_n} f \mathrm{d}\mu
\end{align*}
を得る。さらに$f > 0$であることから、
\begin{align*}
	\int_{E_n} f \mathrm{d}\mu < \int f \mathrm{d}\mu  = 0
\end{align*}
これより、$0\leq \mu(E_n) \leq 0 \ \Rightarrow\ \mu(E_n) = 0$を得る。したがって、$\lim_{n\to \infty} \mu(E_n) = 0$であるので、$\mu(D) = 0$である。これは$f$が$0$でないような$X$の部分集合のの測度が$0$であることを意味するので、$f = 0\ \text{a.e.}$である。

\section{Ex4.3}

\section{Ex4.4}

\section{Ex4.8}

\section{Ex4.10}

\section{Ex4.11}

\section{Ex4.16}

\section{Ex4.17}


\end{document}






















