\documentclass{article}
\usepackage[margin = .7in]{geometry}
\usepackage[dvipdfmx]{graphicx}
\usepackage{listings}
\usepackage{amsmath}
\usepackage{amssymb}
\usepackage{amsfonts}
\usepackage{bm}
\usepackage{mathrsfs}
\lstset{%
  language={python},
  basicstyle={\small},%
  identifierstyle={\small},%
  commentstyle={\small\itshape},%
  keywordstyle={\small\bfseries},%
  ndkeywordstyle={\small},%
  stringstyle={\small\ttfamily},
  frame={tb},
  breaklines=true,
  columns=[l]{fullflexible},%
  numbers=left,%
  xrightmargin=0zw,%
  xleftmargin=3zw,%
  numberstyle={\scriptsize},%
  stepnumber=1,
  numbersep=1zw,%
  lineskip=-0.5ex%
}

\begin{document}
\title{測度論的確率論 2018 S1S2 \\ 
Homework 4}
\author{経済学研究科現代経済コース修士1年 / 池上 慧 (29186009) / sybaster.x@gmail.com}
\maketitle

\section{Ex4.2}
 $D = \left\{ x \in X \mid f(x) \neq 0 \right\} = \bigcup_{n = 1}^{\infty} \left\{ x\in X \mid |f(x)| > \frac{1}{n} \right\} = \bigcup_{n = 1}^{\infty} \left\{ x\in X \mid f(x) > \frac{1}{n} \right\}$
である。lemma3.5より$E_n \equiv \left\{ x\in X \mid f(x) > \frac{1}{n} \right\} $は可測集合の要素であるので、sigma fielsが可算個の和集合について閉じていることから$D$も可測集合に入る。したがって$\mu(D)$は定義されている。さらに、$E_n$が単調増大な集合なので、lemma1.3より
\begin{align*}
	\mu(D) = \mu \left(\bigcup_{n = 1}^{\infty} \left\{ x\in X \mid f(x) > \frac{1}{n} \right\}\right) = \lim_{n\to \infty} \mu(E_n)
\end{align*}
である。ここでチェビシェフの不等式から、任意の$n \geq 1$について
\begin{align*}
	\mu(E_n) \leq n \int_{E_n} f \mathrm{d}\mu
\end{align*}
を得る。さらに$f > 0$であることから、
\begin{align*}
	\int_{E_n} f \mathrm{d}\mu < \int f \mathrm{d}\mu  = 0
\end{align*}
これより、$0\leq \mu(E_n) \leq 0 \ \Rightarrow\ \mu(E_n) = 0$を得る。したがって、$\lim_{n\to \infty} \mu(E_n) = 0$であるので、$\mu(D) = 0$である。これは$f$が$0$でないような$X$の部分集合のの測度が$0$であることを意味するので、$f = 0\ \text{a.e.}$である。

\section{Ex4.3}
可積分であるので、$r$を正の実数、$\theta$を実数として、積分を以下のような極形式で表現できる。
\begin{align}
	\int f\mathrm{d}\mu = \int (Re f) \mathrm{d}\mu + \int (Im f) \mathrm{d}\mu = r e^{i\theta}
\end{align}
$e^{i\theta} = \cos \theta + i \sin\theta$であり、$r > 0$だから、
\begin{align*}
	\| \int f\mathrm{d}\mu \| = \| re^{i\theta} \| = r\| e^{i\theta} \| = r
\end{align*}
である。ここで$(1)$より以下を得る。
\begin{align*}
	r = e^{-i\theta} \int f\mathrm{d}\mu
\end{align*}
また、「可積分な関数$f$は、$c \in \mathbb{C}$に対して、$\int cf \mathrm{d}\mu = c \int f \mathrm{d}\mu$(主張$1$)」を所与とすれば、
\begin{align*}
	r = \int e^{-i\theta} f\mathrm{d}\mu
\end{align*}
である。$r\in \mathbb{R}$より$\int e^{-i\theta} f\mathrm{d}\mu$は実数である。よって、定義より以下を得る。
\begin{align*}
	\int e^{-i\theta} f\mathrm{d}\mu = Re\left( \int e^{-i\theta} f\mathrm{d}\mu \right) = \int Re\left( e^{-i\theta} f\right) \mathrm{d}\mu
\end{align*}
さらに、複素数の絶対値の定義より、
\begin{align*}
	\| e^{-i\theta} f \| = \sqrt{ Re\left( e^{-i\theta} f\right)^2 + Im\left( e^{-i\theta} f\right)^2 } \geq Re\left( e^{-i\theta} f\right)
\end{align*}
である。よってexample 4.2より、
\begin{align*}
	\int Re\left( e^{-i\theta} f\right) \mathrm{d}\mu \leq \int \| e^{-i\theta} f\| \mathrm{d}\mu
\end{align*}
であり、$\| e^{-i\theta} \| = 1$が恒等的に成立するので、
\begin{align*}
	\| \int f\mathrm{d}\mu \| = r = \int e^{-i\theta} f\mathrm{d}\mu = \int Re\left( e^{-i\theta} f\right) \mathrm{d}\mu \leq \int \| e^{-i\theta} f\| \mathrm{d}\mu = \int \| f \| \mathrm{d}\mu 
\end{align*}
となり題意は示された。

\subsection{主張$1$の証明}

\section{Ex4.4}
$E = \left\{ x \in X \mid f(x) \neq g(x) \right\}$の測度が$0$であることを示す。
\begin{align*}
	E = \bigcup_{q\in \mathbb{Q}} \left\{ \left\{ x\in X \mid f(x) > q \right\} \cap \left\{x \in X \mid q > g(x) \right\} \right\} \cup \bigcup_{q\in \mathbb{Q}} \left\{ \left\{ x\in X \mid f(x) < q \right\} \cap \left\{x \in X \mid q < g(x) \right\} \right\}
\end{align*}
である。前半を$E_1$、後半を$E_2$とする。lemma 3.5より$E_1, E_2 \in \mathcal{A}$である。仮定より、
\begin{align*}
	\int_{E_1} f \mathrm{d}\mu = \int_{E_1} g \mathrm{d}\mu
\end{align*}
であるはずだが、仮に$\mu(E_1) > 0$だとすると、$E_1$においては$f > g$なので、example 4.2より$\int_{E_1} f \mathrm{d}\mu > \int_{E_1} g \mathrm{d}\mu$となる。これは仮定に反するので、$\mu(E_1) = 0$である。同様に$\mu(E_2) = 0$であり、$\mu(E) = \mu(E_1) + \mu(E_2) = 0$より題意は示された。

\section{Ex4.8}
\subsection{a.e.に等しい関数を同一視することで$\left(L^{\infty}, \| \cdot \|_{\infty} \right)$がノルム空間となることの証明}
ノルムの非負値性、三角不等式、同次性の三つを確認すれば良い。
\subsubsection{非負値性}

ここでは$\forall f \in L^{\infty}\ \| f \|_{\infty} \geq 0$と$\| f \|_{\infty} = 0 \Leftrightarrow f=0\ \text{a.e.}$の二つを確認する。前半はノルムの定義より明らかである。後半は以下のように示される。

まず$\| f \|_{\infty} = 0 \Rightarrow f=0\ \text{a.e.}$を示す。左辺の意味するところは、
\begin{align*}
	\forall n \geq 1,\ \exists a\ \text{s.t.}\ \frac{1}{n} > a > 0,\ \mu(\left\{ x\in X \mid | f(x)| > a \right\}) = 0
\end{align*}
この時、任意の$n \geq 1$に対して$\mu(\left\{ x\in X \mid |f(x)| > \frac{1}{n} \right\}) \leq \mu(\left\{ x\in X \mid | f(x)| > a \right\}) = 0$とできることから、
\begin{align*}
	\mu(\left\{ x\in X \mid f(x) \neq 0 \right\}) = \lim_{n\to \infty} \mu(\left\{ x\in X \mid |f(x)| > \frac{1}{n} \right\}) = 0
\end{align*}
であることがわかる。これはすなわち$f=0\ \text{a.e.}$であることを意味する。ただし、上の式変形にはexercise 4.2と同じ式変形を用いている。

次に$f=0\ \text{a.e.} \Rightarrow \| f \|_{\infty} = 0$を確認する。仮定より任意に小さな非負実数$a$について$\mu(\left\{ x\in X \mid |f(x)| > a \right\}) = 0$であるので、$inf$の定義より$\| f \|_{\infty} = 0$である。以上で非負値性が示された。

\subsubsection{三角不等式}
$\forall f, g \in L^{\infty},\ \| f \|_{\infty} + \| g \|_{\infty} \geq \| f + g \|_{\infty}$を示す。通常の絶対値に対する三角不等式から、$|f(x) + g(x)| \leq |f(x)| + |g(x)|$である。従って、任意の非負実数$c$について、
\begin{align*}
	\left\{ x\in X \mid |f(x) + g(x)| > c \right\} \subset \left\{ x\in X \mid |f(x)| + |g(x)| > c \right\}  
\end{align*}
である。これより、
\begin{align*}
	\mu(\left\{ x\in X \mid |f(x) + g(x)| > c \right\}) \leq \mu(\left\{ x\in X \mid |f(x)| + |g(x)| > c \right\})
\end{align*}
である。なので、$\mu(\left\{ x\in X \mid |f(x)| + |g(x)| > c \right\}) = 0$ならば必ず$\mu(\left\{ x\in X \mid |f(x) + g(x)| > c \right\}) = 0$である。これより、
\begin{align*}
	\left\{ c \geq 0 \mid \mu(\left\{ x\in X \mid |f(x)| + |g(x)| > c \right\} ) = 0 \right\} \subset \left\{ c \geq 0\mid \mu(\left\{ x\in X \mid |f(x) + g(x)| > c \right\}) = 0 \right\}
\end{align*}
である。より広い範囲について$\inf$をとった時に$\inf$が元の集合に対するものよりも大きくなることはあり得ないので、
\begin{align}
	\inf \left\{ c \geq 0 \mid \mu(\left\{ x\in X \mid |f(x)| + |g(x)| > c \right\} = 0) \right\} \geq \inf \left\{ c \geq 0\mid \mu(\left\{ x\in X \mid |f(x) + g(x)| > c \right\}) = 0 \right\}
\end{align}
ここで、仮に
\begin{align*}
	&\inf \left\{ c \geq 0 \mid \mu(\left\{ x\in X \mid |f(x)| + |g(x)| > c \right\})=0 \right\} \\[8pt]
	> &\inf \left\{ c \geq 0 \mid \mu(\left\{ x\in X \mid |f(x)| > c \right\}) =0 \right\} + \inf \left\{ c \geq 0 \mid \mu(\left\{ x\in X \mid |g(x)| > c \right\}) =0\right\}
\end{align*}
であるとすると、左辺よりも小さく右辺よりも大きい正の実数$a$が必ず存在する。このような$a$に対して
\begin{align*}
	\mu(\left\{ x\in X \mid |f(x)| + |g(x)| > a \right\}) > 0
\end{align*}
である。$|f(x)| + |g(x)| > a$であるためには、一般性を失うことなく$|f(x)|$が$\inf \left\{ c \geq 0 \mid \mu(\left\{ x\in X \mid |f(x)| > c \right\}) =0 \right\}$を上回っている必要がある。すなわち、
\begin{align*}
	\mu(\left\{ x\in X \mid |f(x)| >  inf \left\{ c \geq 0 \mid \mu(\left\{ x\in X \mid |f(x)| > c \right\}) =0 \right\} \right\}) \geq \mu(\left\{ x\in X \mid |f(x)| + |g(x)| > a \right\})
\end{align*}
であり、定義より左辺は$0$である、これより
\begin{align*}
	0 \geq \mu(\left\{ x\in X \mid |f(x)| + |g(x)| > a \right\}) > 0
\end{align*}
であり矛盾をきたす。従って
\begin{align*}
	&\inf \left\{ c \geq 0 \mid \mu(\left\{ x\in X \mid |f(x)| + |g(x)| > c \right\})=0 \right\} \\[8pt]
	\leq &\inf \left\{ c \geq 0 \mid \mu(\left\{ x\in X \mid |f(x)| > c \right\}) =0 \right\} + \inf \left\{ c \geq 0 \mid \mu(\left\{ x\in X \mid |g(x)| > c \right\}) =0\right\}
\end{align*}
であることが示され、$(2)$と合わせて以下の三角不等式が成立することが確認できた。
\begin{align*}
	&\inf \left\{ c \geq 0\mid \mu(\left\{ x\in X \mid |f(x) + g(x)| > c \right\}) = 0 \right\}\\
	\leq &\inf \left\{ c \geq 0 \mid \mu(\left\{ x\in X \mid |f(x)| > c \right\}) =0 \right\} + \inf \left\{ c \geq 0 \mid \mu(\left\{ x\in X \mid |g(x)| > c \right\}) =0\right\}
\end{align*}

\subsubsection{同次性}
$\forall r \in \mathbb{R}$に対して$\| rf \|_{\infty} = |r| \| f\|_{\infty}$を示す。左辺を定義に従って書き下すと、
\begin{align*}
	\| rf \|_{\infty} &= \inf \left\{ a \geq 0 \mid \mu(\left\{ x\in X \mid |rf(x)| > a \right\}) = 0 \right\}\\
	&= \inf \left\{ a \geq 0 \mid \mu \left(\left\{ x\in X \mid |f(x)| > \frac{a}{|r|} \right\}\right) = 0 \right\}\\
	&= |r| \inf \left\{ a \geq 0 \mid \mu \left(\left\{ x\in X \mid |f(x)| > a \right\}\right) = 0 \right\}\\
	&= |r| \|f \|_{\infty}
\end{align*}
以上より同次性が示された。

\subsection{$\| f_n -f \|_{\infty} \to 0\ \Leftrightarrow\ \exists E\in \mathcal{A}\ \text{s.t.}\ \mu(E^c) = 0,\ \sup_{x\in E} |fn(x) - f(x)| \to 0$の証明}
まず$\| f_n -f \|_{\infty} \to 0\ \Rightarrow\ \exists E\in \mathcal{A}\ \text{s.t.}\ \mu(E^c) = 0,\ \sup_{x\in E} |fn(x) - f(x)| \to 0$を示す。$E$として以下の集合を考える。
\begin{align*}
	E \equiv \left\{ x\in X \mid \lim_{n \to \infty} f_n(x) = f(x) \right\}
\end{align*}
この$E$に対して、まず$\mu(E^c) = 0$を確認する。exercise 4.2と同様の式変形より、
\begin{align*}
	\mu(E^c) &= \mu\left( \bigcup_{m = 1}^{\infty} \left\{ x\in X \mid \lim_{n\to \infty} |f_n(x) - f(x)| > \frac{1}{m} \right\} \right)\\
	&= \lim_{m \to \infty} \mu \left( \left\{ x\in X \mid \lim_{n\to \infty} |f_n(x) - f(x)| > \frac{1}{m} \right\} \right)
\end{align*}
であるので、任意の$m$について$\mu \left( \left\{ x\in X \mid \lim_{n\to \infty} |f_n(x) - f(x)| > \frac{1}{m} \right\} \right) = 0$であることを確認すれば良い。仮定より、
\begin{align*}
	\forall m \geq 1,\ \exists N_m\ \text{s.t.}\ n \geq N_m \Rightarrow \mu\left( \left\{ x\in X \mid |f_n(x) - f(x)| > \frac{1}{m} \right\} \right)
\end{align*}
であるので、先の要件は確認された。

次に$\sup_{x\in E} |fn(x) - f(x)| \to 0$を確認する。まず、$E$が以下のように変形できることに注意する。
\begin{align*}
	E &= \left\{ x\in X \mid \lim_{n \to \infty} f_n(x) = f(x) \right\}\\
	&= \left\{ x\in X \mid \lim_{n \to \infty} |f_n(x)-f(x)| = 0 \right\}\\
	&= \bigcap_{m = 1}^{\infty} \left\{ x\in X \mid \lim_{n \to \infty} |f_n(x)-f(x)| < \frac{1}{m} \right\}\\
	&= \bigcap_{m = 1}^{\infty} \bigcup_{j = 1}^{\infty} \left\{ x\in X \mid |f_n(x)-f(x)| < \frac{1}{m}\ \forall n \geq j \right\}
\end{align*}
すなわち、$x\in E$の時、以下が必ず成立する。
\begin{align*}
	\forall m\ \exists j \ \text{s.t.}\ n \geq j \Rightarrow |fn(x) - f(x) | < \frac{1}{m}
\end{align*}
従って、$\sup$の定義より以下が成立する。
\begin{align*}
	\forall m\ \exists j \ \text{s.t.}\ n \geq j \Rightarrow \sup_{x \in E} |fn(x) - f(x) | \leq \frac{1}{m}
\end{align*}
これは収束の定義より$\sup_{x\in E} |fn(x) - f(x)| \to 0$を意味するので題意は示された。

逆向きの関係を示す。任意に$\epsilon > 0$を取ると、仮定より、
\begin{align*}
	\exists N\ \text{s.t.}\ n\geq N \Rightarrow \sup_{x\in E} |f_n(x) - f(x)| < \epsilon
\end{align*}
である。同じ$\epsilon,N$に対して、$a \geq \epsilon$のように実数$a$を用意すると、$n \geq N$である$n$について、$\sup$の定義より以下が成立する。
\begin{align*}
	\left\{ x\in X \mid |f_n(x) -f(x)| > a \right\} \subset E^c
\end{align*}
この時仮定より、$\mu(\left\{ x\in X \mid |f_n(x) -f(x)| > a \right\}) \leq \mu(E^c) = 0$なので$\mu(\left\{ x\in X \mid |f_n(x) -f(x)| > a \right\}) = 0$であることがわかる。以上の議論は以下のことを述べている。
\begin{align*}
	\forall \epsilon > 0,\ \exists N > 0\ \text{s.t.}\ \forall n \geq N,\ a \geq \epsilon \Rightarrow a \in \left\{ c \geq 0 \mid \mu\left( \left\{ x \in X \mid |f_n(x) - f(x)| > c \right\} \right) \right\}
\end{align*}
この最後の集合を$\star$と表記する。この時、
\begin{align*}
	\inf \star \leq \epsilon
\end{align*}
である。なぜなら、仮定より$\star$は空集合ではなく、$\inf \star > \epsilon$であったとすると$\inf \star > a > \epsilon$を満たす実数$a$が必ず存在し、その$a$が必ず$\star$に入ることに矛盾するからである。従って、以下が成立する。
\begin{align*}
	\forall \epsilon > 0,\ \exists N\ \text{s.t.}\ n\geq N \Rightarrow \inf \left\{ c \geq 0 \mid \mu\left( \left\{ x \in X \mid |f_n(x) - f(x)| > c \right\} \right) \right\} \leq \epsilon
\end{align*}
収束の定義よりこれは$\| f_n -f \|_{\infty} \to 0$を意味する。



\section{Ex4.10}
\subsection{(a)}
ヘルダーの不等式を使って証明する。任意に$f \in L^q$を取ってくるとき、
\begin{align*}
	\int |f|^p \mathrm{d}\mu = \int |f|^p 1 \mathrm{d}\mu \leq \left( \int \left(|f|^p \right)^{\frac{q}{p}} \mathrm{d}\mu \right)^{\frac{p}{q}} \left( \int \mathrm{d}\mu\right)^{1-\frac{p}{q}} =  \left( \int |f|^q \mathrm{d}\mu \right)^{\frac{p}{q}}\mu \left( X \right)^{1-\frac{p}{q}}
\end{align*}
とできる。仮定より$\mu(X) < \infty$であることを考慮すると、右辺は有限である。従って$f \in L^p$であることがわかる。

\subsection{(b)}
\begin{align*}
	g(x) = \begin{cases}
	x^{-\frac{1}{p}}\ &\text{when}\ x \in [1, \infty)\\
	0\ &\text{otherwise}
	\end{cases}
\end{align*}
とする。この時、
\begin{align*}
	\int |x^{-\frac{1}{p}}|^{p} \mathrm{d}\mu = \int_{[1,\infty)} \frac{1}{x} \mathrm{d}\mu
\end{align*}
である。2以上の任意の整数$n$について$g_n (x) = g1_{[1,n)}$とすると、$g$が積分区間において連続関数なので可測関数、$1_{[1,n)}$は可測関数であることからlemma 3.7より$g_n (x)$は常に可測関数であり、明らかに$g_n \uparrow g$であるので単調収束定理から
\begin{align*}
	\int_{[1,\infty)} \frac{1}{x} \mathrm{d}\mu = \int \lim_{n \to \infty} \frac{1}{x} 1_{[1, \infty)} \mathrm{d}\mu = \lim_{n \to \infty} \int_{\mathbb{R}} frac{1}{x} 1_{[1, \infty)} = \lim_{n \to \infty} {\rm log} n = \infty
\end{align*}
であるので先の積分は発散してしまう。そのため$g\notin L^p$である。

一方で、$q$に関しては、対応するリーマン積分を考えると$p-q < 0$より、
\begin{align*}
	\int_1^{\infty} x^{-\frac{q}{p}} \mathrm{d}x = -\left( \frac{p}{p-q} \right) = \frac{p}{q-p} < \infty
\end{align*}
である。リーマン積分が存在したのでtheorem 4.8よりルベーグ積分もこの値をとる。すなわち、
\begin{align*}
	\int_{\mathbb{R}} |x^{-\frac{1}{p}}|^q \mathrm{d}\mu = \int_1^{\infty} x^{-\frac{q}{p}} \mathrm{d}x < \infty
\end{align*}
であるので$g \in L^q$であることがわかる。

\section{Ex4.11}

\section{Ex4.16}

\section{Ex4.17}


\end{document}






















