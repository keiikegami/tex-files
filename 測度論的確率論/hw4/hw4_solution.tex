\documentclass{article}
\usepackage[margin = .7in]{geometry}
\usepackage[dvipdfmx]{graphicx}
\usepackage{listings}
\usepackage{amsmath}
\usepackage{amssymb}
\usepackage{amsfonts}
\usepackage{bm}
\usepackage{mathrsfs}
\lstset{%
  language={python},
  basicstyle={\small},%
  identifierstyle={\small},%
  commentstyle={\small\itshape},%
  keywordstyle={\small\bfseries},%
  ndkeywordstyle={\small},%
  stringstyle={\small\ttfamily},
  frame={tb},
  breaklines=true,
  columns=[l]{fullflexible},%
  numbers=left,%
  xrightmargin=0zw,%
  xleftmargin=3zw,%
  numberstyle={\scriptsize},%
  stepnumber=1,
  numbersep=1zw,%
  lineskip=-0.5ex%
}

\begin{document}
\title{測度論的確率論 2018 S1S2 \\ 
Homework 4}
\author{経済学研究科現代経済コース修士1年 / 池上 慧 (29186009) / sybaster.x@gmail.com}
\maketitle

\section{Ex4.2}
 $D = \left\{ x \in X \mid f(x) \neq 0 \right\} = \bigcup_{n = 1}^{\infty} \left\{ x\in X \mid |f(x)| > \frac{1}{n} \right\} = \bigcup_{n = 1}^{\infty} \left\{ x\in X \mid f(x) > \frac{1}{n} \right\}$
である。lemma3.5より$E_n \equiv \left\{ x\in X \mid f(x) > \frac{1}{n} \right\} $は可測集合の要素であるので、sigma fielsが可算個の和集合について閉じていることから$D$も可測集合に入る。したがって$\mu(D)$は定義されている。さらに、$E_n$が単調増大な集合なので、lemma1.3より
\begin{align*}
	\mu(D) = \mu \left(\bigcup_{n = 1}^{\infty} \left\{ x\in X \mid f(x) > \frac{1}{n} \right\}\right) = \lim_{n\to \infty} \mu(E_n)
\end{align*}
である。ここでチェビシェフの不等式から、任意の$n \geq 1$について
\begin{align*}
	\mu(E_n) \leq n \int_{E_n} f \mathrm{d}\mu
\end{align*}
を得る。さらに$f > 0$であることから、
\begin{align*}
	\int_{E_n} f \mathrm{d}\mu < \int f \mathrm{d}\mu  = 0
\end{align*}
これより、$0\leq \mu(E_n) \leq 0 \ \Rightarrow\ \mu(E_n) = 0$を得る。したがって、$\lim_{n\to \infty} \mu(E_n) = 0$であるので、$\mu(D) = 0$である。これは$f$が$0$でないような$X$の部分集合のの測度が$0$であることを意味するので、$f = 0\ \text{a.e.}$である。

\section{Ex4.3}
可積分であるので、$r$を正の実数、$\theta$を実数として、積分を以下のような極形式で表現できる。
\begin{align}
	\int f\mathrm{d}\mu = \int (Re f) \mathrm{d}\mu + \int (Im f) \mathrm{d}\mu = r e^{i\theta}
\end{align}
$e^{i\theta} = \cos \theta + i \sin\theta$であり、$r > 0$だから、
\begin{align*}
	\| \int f\mathrm{d}\mu \| = \| re^{i\theta} \| = r\| e^{i\theta} \| = r
\end{align*}
である。ここで$(1)$より以下を得る。
\begin{align*}
	r = e^{-i\theta} \int f\mathrm{d}\mu
\end{align*}
また、「可積分な関数$f$は、$c \in \mathbb{C}$に対して、$\int cf \mathrm{d}\mu = c \int f \mathrm{d}\mu$(主張$1$)」を所与とすれば、
\begin{align*}
	r = \int e^{-i\theta} f\mathrm{d}\mu
\end{align*}
である。$r\in \mathbb{R}$より$\int e^{-i\theta} f\mathrm{d}\mu$は実数である。よって、定義より以下を得る。
\begin{align*}
	\int e^{-i\theta} f\mathrm{d}\mu = Re\left( \int e^{-i\theta} f\mathrm{d}\mu \right) = \int Re\left( e^{-i\theta} f\right) \mathrm{d}\mu
\end{align*}
さらに、複素数の絶対値の定義より、
\begin{align*}
	\| e^{-i\theta} f \| = \sqrt{ Re\left( e^{-i\theta} f\right)^2 + Im\left( e^{-i\theta} f\right)^2 } \geq Re\left( e^{-i\theta} f\right)
\end{align*}
である。よってexample 4.2より、
\begin{align*}
	\int Re\left( e^{-i\theta} f\right) \mathrm{d}\mu \leq \int \| e^{-i\theta} f\| \mathrm{d}\mu
\end{align*}
であり、$\| e^{-i\theta} \| = 1$が恒等的に成立するので、
\begin{align*}
	\| \int f\mathrm{d}\mu \| = r = \int e^{-i\theta} f\mathrm{d}\mu = \int Re\left( e^{-i\theta} f\right) \mathrm{d}\mu \leq \int \| e^{-i\theta} f\| \mathrm{d}\mu = \int \| f \| \mathrm{d}\mu 
\end{align*}
となり題意は示された。

\subsection{主張$1$の証明}

\section{Ex4.4}
$E = \left\{ x \in X \mid f(x) \neq g(x) \right\}$の測度が$0$であることを示す。
\begin{align*}
	E = \bigcup_{q\in \mathbb{Q}} \left\{ \left\{ x\in X \mid f(x) > q \right\} \cap \left\{x \in X \mid q > g(x) \right\} \right\} \cup \bigcup_{q\in \mathbb{Q}} \left\{ \left\{ x\in X \mid f(x) < q \right\} \cap \left\{x \in X \mid q < g(x) \right\} \right\}
\end{align*}
である。前半を$E_1$、後半を$E_2$とする。lemma 3.5より$E_1, E_2 \in \mathcal{A}$である。仮定より、
\begin{align*}
	\int_{E_1} f \mathrm{d}\mu = \int_{E_1} g \mathrm{d}\mu
\end{align*}
であるはずだが、仮に$\mu(E_1) > 0$だとすると、$E_1$においては$f > g$なので、example 4.2より$\int_{E_1} f \mathrm{d}\mu > \int_{E_1} g \mathrm{d}\mu$となる。これは仮定に反するので、$\mu(E_1) = 0$である。同様に$\mu(E_2) = 0$であり、$\mu(E) = \mu(E_1) + \mu(E_2) = 0$より題意は示された。

\section{Ex4.8}
\subsection{a.e.に等しい関数を同一視することで$\left(L^{\infty}, \| \cdot \|_{\infty} \right)$がノルム空間となることを示す。}
以下ではノルムの非負値性、三角不等式、同次性の三つを確認する。
\subsubsection{非負値性}

ここでは$\forall f \in L^{\infty}\ \| f \|_{\infty} \geq 0$と$\| f \|_{\infty} = 0 \Leftrightarrow f=0\ \text{a.e.}$の二つを確認する。前半はノルムの定義より明らかである。後半は以下のように示される。

まず$\| f \|_{\infty} = 0 \Rightarrow f=0\ \text{a.e.}$を示す。左辺の意味するところは、
\begin{align*}
	\forall n \geq 1,\ \exists a\ \text{s.t.}\ \frac{1}{n} > a > 0,\ \mu(\left\{ x\in X \mid | f(x)| > a \right\}) = 0
\end{align*}
この時、任意の$n \geq 1$に対して$\mu(\left\{ x\in X \mid |f(x)| > \frac{1}{n} \right\}) \leq \mu(\left\{ x\in X \mid | f(x)| > a \right\}) = 0$とできることから、
\begin{align*}
	\mu(\left\{ x\in X \mid f(x) \neq 0 \right\}) = \lim_{n\to \infty} \mu(\left\{ x\in X \mid |f(x)| > \frac{1}{n} \right\}) = 0
\end{align*}
であることがわかる。これはすなわち$f=0\ \text{a.e.}$であることを意味する。ただし、上の式変形にはexercise 4.2と同じ式変形を用いている。

次に$f=0\ \text{a.e.} \Rightarrow \| f \|_{\infty} = 0$を確認する。仮定より任意に小さな非負実数$a$について$\mu(\left\{ x\in X \mid |f(x)| > a \right\}) = 0$であるので、$inf$の定義より$\| f \|_{\infty} = 0$である。以上で非負値性が示された。

\subsubsection{三角不等式}
$\forall f, g \in L^{\infty},\ \| f \|_{\infty} + \| g \|_{\infty} \geq \| f + g \|_{\infty}$を示す。通常の絶対値に対する三角不等式から、$|f(x) + g(x)| \leq |f(x)| + |g(x)|$である。従って、任意の非負実数$c$について、
\begin{align*}
	\left\{ x\in X \mid |f(x) + g(x)| > c \right\} \subset \left\{ x\in X \mid |f(x)| + |g(x)| > c \right\}  
\end{align*}
である。これより、
\begin{align*}
	\mu(\left\{ x\in X \mid |f(x) + g(x)| > c \right\}) \leq \mu(\left\{ x\in X \mid |f(x)| + |g(x)| > c \right\})
\end{align*}
である。なので、$\mu(\left\{ x\in X \mid |f(x)| + |g(x)| > c \right\}) = 0$ならば必ず$\mu(\left\{ x\in X \mid |f(x) + g(x)| > c \right\}) = 0$である。これより、
\begin{align*}
	\left\{ c \geq 0 \mid \mu(\left\{ x\in X \mid |f(x)| + |g(x)| > c \right\} ) = 0 \right\} \subset \left\{ c \geq 0\mid \mu(\left\{ x\in X \mid |f(x) + g(x)| > c \right\}) = 0 \right\}
\end{align*}
である。より広い範囲について$\inf$をとった時に$\inf$が元の集合に対するものよりも大きくなることはあり得ないので、
\begin{align}
	\inf \left\{ c \geq 0 \mid \mu(\left\{ x\in X \mid |f(x)| + |g(x)| > c \right\} = 0) \right\} \geq \inf \left\{ c \geq 0\mid \mu(\left\{ x\in X \mid |f(x) + g(x)| > c \right\}) = 0 \right\}
\end{align}
ここで、仮に
\begin{align*}
	&\inf \left\{ c \geq 0 \mid \mu(\left\{ x\in X \mid |f(x)| + |g(x)| > c \right\})=0 \right\} \\[8pt]
	> &\inf \left\{ c \geq 0 \mid \mu(\left\{ x\in X \mid |f(x)| > c \right\}) =0 \right\} + \inf \left\{ c \geq 0 \mid \mu(\left\{ x\in X \mid |g(x)| > c \right\}) =0\right\}
\end{align*}
であるとすると、左辺よりも小さく右辺よりも大きい正の実数$a$が必ず存在する。このような$a$に対して
\begin{align*}
	\mu(\left\{ x\in X \mid |f(x)| + |g(x)| > a \right\}) > 0
\end{align*}
である。$|f(x)| + |g(x)| > a$であるためには、一般性を失うことなく$|f(x)|$が$\inf \left\{ c \geq 0 \mid \mu(\left\{ x\in X \mid |f(x)| > c \right\}) =0 \right\}$を上回っている必要がある。すなわち、
\begin{align*}
	\mu(\left\{ x\in X \mid |f(x)| >  inf \left\{ c \geq 0 \mid \mu(\left\{ x\in X \mid |f(x)| > c \right\}) =0 \right\} \right\}) \geq \mu(\left\{ x\in X \mid |f(x)| + |g(x)| > a \right\})
\end{align*}
であり、定義より左辺は$0$である、これより
\begin{align*}
	0 \geq \mu(\left\{ x\in X \mid |f(x)| + |g(x)| > a \right\}) > 0
\end{align*}
であり矛盾をきたす。従って
\begin{align*}
	&\inf \left\{ c \geq 0 \mid \mu(\left\{ x\in X \mid |f(x)| + |g(x)| > c \right\})=0 \right\} \\[8pt]
	\leq &\inf \left\{ c \geq 0 \mid \mu(\left\{ x\in X \mid |f(x)| > c \right\}) =0 \right\} + \inf \left\{ c \geq 0 \mid \mu(\left\{ x\in X \mid |g(x)| > c \right\}) =0\right\}
\end{align*}
であることが示され、$(2)$と合わせて以下の三角不等式が成立することが確認できた。
\begin{align*}
	&\inf \left\{ c \geq 0\mid \mu(\left\{ x\in X \mid |f(x) + g(x)| > c \right\}) = 0 \right\}\\
	\leq &\inf \left\{ c \geq 0 \mid \mu(\left\{ x\in X \mid |f(x)| > c \right\}) =0 \right\} + \inf \left\{ c \geq 0 \mid \mu(\left\{ x\in X \mid |g(x)| > c \right\}) =0\right\}
\end{align*}


\section{Ex4.10}

\section{Ex4.11}

\section{Ex4.16}

\section{Ex4.17}


\end{document}






















