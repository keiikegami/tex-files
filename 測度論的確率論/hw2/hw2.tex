\documentclass{article}
\usepackage[margin = .7in]{geometry}
\usepackage[dvipdfmx]{graphicx}
\usepackage{listings}
\usepackage{amsmath}
\usepackage{amssymb}
\usepackage{amsfonts}
\usepackage{bm}
\usepackage{mathrsfs}
\lstset{%
  language={python},
  basicstyle={\small},%
  identifierstyle={\small},%
  commentstyle={\small\itshape},%
  keywordstyle={\small\bfseries},%
  ndkeywordstyle={\small},%
  stringstyle={\small\ttfamily},
  frame={tb},
  breaklines=true,
  columns=[l]{fullflexible},%
  numbers=left,%
  xrightmargin=0zw,%
  xleftmargin=3zw,%
  numberstyle={\scriptsize},%
  stepnumber=1,
  numbersep=1zw,%
  lineskip=-0.5ex%
}

\begin{document}
\title{測度論的確率論 2018 S1S2 \\ 
Homework 2}
\author{経済学研究科現代経済コース修士1年 / 池上 慧 (29186009) / sybaster.x@gmail.com}
\maketitle

\section{Ex 1.5}
 まず「$A_n \downarrow \phi,\ A_n \in \mathcal{A}\ \Rightarrow\ \lim_{n \to \infty} \mu \left( A_n \right) = 0$」$\Rightarrow$ 「$\mu$が$\sigma\left( \mathcal{A} \right)$上に拡張できる」を示す。Caratheodoryの拡張定理を用いるために、以下の二点が成立することを確認すれば良い。
\begin{enumerate}
	\item $A_n \in \mathcal{A}$が$n = 1,2,3,\dots$で排反に取られていて、$\bigcup_{n = 1}^{\infty} A_n \in \mathcal{A}$であるなら、$\mu \left( \bigcup_{n = 1}^{\infty} A_n \right) = \sum_{n = 1}^{\infty} \mu \left( A_n \right)$が成立する。
	\item $\mu(\phi) = 0$
\end{enumerate}

$1$番目を確認する。仮定されている有限加法性から以下が成立する。
\begin{align*}
	\mu \left(  \bigcup_{n = 1}^{\infty} A_n \right) = \mu \left( A_1 \cup  \bigcup_{n = 2}^{\infty} A_n \right) = \mu \left( A_1 \right) + \mu \left( \bigcup_{n = 2}^{\infty} A_n \right)
\end{align*}
ただし、$2$つ目の統合は、$\bigcup_{n = 1}^{\infty} A_n \in \mathcal{A}$かつ任意の$n$について$A_n \in \mathcal{A}$であることから、fieldの性質より$ \bigcup_{n = 2}^{\infty} A_n \in \mathcal{A}$となり、$\mu$の定義域に入るため成立することに注意する。この処理は繰り返し用いることができるので、$\mu \left( \bigcup_{n = 1}^{\infty} A_n \right) = \sum_{n = 1}^{\infty} \mu \left( A_n \right)$が成立することがわかる。

$2$番目を確認する。今、$B_n = A_1 \backslash A_n$とすると、仮定より$B_n \uparrow A_1 \backslash \phi$である。さらに$C_n = B_n \backslash B_{n+ 1}$とすると、$\left\{ B_n \right\}_{n = 1}^{\infty}$が単調な列であることから$\left\{ C_n \right\}_{n = 1}^{\infty}$は排反である。有限加法性より、

\begin{align*}
	\mu \left( B_n \right) = \mu \left( \bigcup_{m = 1}^n C_m \right) = \sum_{m = 1}^n \mu \left( C_m \right)
\end{align*}
である。$\mu\left( A_n \right) = \mu \left( A_1 \right) - \mu \left( A_1 \backslash A_n \right) = \mu \left( A_1 \right) - \mu \left( B_n \right)$なので、$\mu \left( A_n \right)$の極限を考えるには$\mu \left( B_n \right)$の極限を考えれば良い。上の式と$1$の性質より、

\begin{align*}
	\lim_{n \to \infty} \mu \left( B_n \right) = \sum_{m = 1}^{\infty} \mu \left( C_m \right) = \mu \left( \bigcup_{m = 1}^{\infty} C_m \right) = \mu \left( \bigcup_{n = 1}^{\infty} B_n \right) = \mu \left( A_1 \backslash \phi \right)
\end{align*}
である。以上より、

\begin{align*}
	\lim_{n \to \infty} \mu \left( A_n \right) = \mu \left( A_1 \right) - \lim_{n \to \infty} \mu\left( B_n \right) = \mu \left( A_1 \right) - \left( \mu \left( A_1 \right) - \mu \left( \phi \right) \right) = \mu \left( \phi \right)
\end{align*}
であり、仮定より$\mu \left( \phi \right) = 0$を得た。


 次に「$\mu$が$\sigma\left( \mathcal{A} \right)$上に拡張できる」$\Rightarrow$ 「$A_n \downarrow \phi,\ A_n \in \mathcal{A}\ \Rightarrow\ \lim_{n \to \infty} \mu \left( A_n \right) = 0$」を示す。レクチャーノートのlemma1.3を用いる。$\mu \left( A_1 \right) < \infty$であることを示せば、lemma1.3より$A_n \downarrow \phi,\ A_n \in \mathcal{A} \subset \sigma\left(\mathcal{A} \right)$のように取れば、測度の定義より空集合の測度は$0$なので、$\lim_{n \to \infty} \mu \left( A_n \right) = \mu \left( \phi \right) = 0$となり題意を示せる。従って$\mu \left( A_1 \right) < \infty$を確認すれば良い。拡張した測度においても$\mathcal{A}$上では元の関数$\mu$と同じ値をとり、元の$\mu$の値域は$\mathbb{R}_{+}$であり、$A_1 \in \mathcal{A}$である。これより$\mu \left( A_1 \right) < \infty$は明らかである。以上で必要十分条件であることが確認された。

\section{Ex 2.1}
 $m \in \mathbb{N}$に対して、$D_m = \left\{ x \in \mathbb{R}\mid | f(x) | < m\right\}$なる領域を定義する。$f(x)$の定義域は$\bigcup_{m = 1}^{\infty} D_m$でおおうことができる。これは可算個の和集合であるので、任意の$m$について$D_m$の中には高々可算個の不連続点しかないことを示せば、元の$f(x)$についても実数軸上で高々可算個の不連続点しか持ち得ないことになる。よって以下では任意に一つとった$m \in \mathbb{N}$について、$| f(x) | < m$であるとして話を進める。(なので$m$は省略する。)

 $n \in \mathbb{N}$に対して$A_n = \left\{ x \mid f(x_{+}) - f(x_{-}) > \frac{1}{n} \right\}$を考える。この時、いかなる大きさのジャンプを持つ$x$であれ$\bigcup_{n = 1}^{\infty} A_n$に入っている。なぜなら、任意の大きさ$\eta>0$のジャンプに対して、ある十分大きな$N$が存在して$\forall n \geq N\ \Rightarrow \eta > \frac{1}{n}$となるからである。これは可算個の和集合なので、任意の$n \in \mathbb{N}$で$A_n$が高々可算集合であることを確認すれば良い。

 今、関数$f(x)$は$-m < f(x) < m$で上限と下限を持つことを思い出す。ここで、$\frac{1}{n}$以上の大きさのジャンプが最大で何回できるかを考える。最大の回数が知りたいので、最小のジャンプ幅である$\frac{1}{n}$について考えればよく、$f(x)$が非減少関数であるので、下限から上限までジャンプだけで上がった時の場合が最大のジャンプ回数である。これは$\frac{2m}{\frac{1}{n}} = 2mn < \infty$である。すなわち有限回しかジャンプできないことがわかる。

 これより、各$m$について高々可算個しか不連続点が存在しないことが確認できたので、最初の議論より、一般に非減少関数$f(x)$は高々可算個しか不連続点を持たないことが示された。

\section{Ex 2.9}
 $x_n \in \mathbb{Q} \cap [0,1]\ , n = 1,2,3,\dots$を取ってくる。十分小さな$\epsilon > 0$に対して、$\left( x_n -\frac{\epsilon}{2^{n+1}}, x_n + \frac{\epsilon}{2^{n+1}} \right)$を$[0,1]$から除いた結果残った集合を考える。この集合を$A$と表記する。

まずこの集合が内点を持たない、つまり境界であることを確認する。任意の$x \in A$に対してどれだけ小さな$\eta > 0$を取っても、$\left( x-\eta, x + \eta\right)$内に$A$以外の要素が含まれてしまうことを確認すれば良い。$x \in \mathbb{R}$であり、$\mathbb{Q}$が$\mathbb{R}$で稠密であることから、$\left( x-\eta, x + \eta\right)$には必ず$\mathbb{Q} \cap [0,1]$の要素が含まれる。しかし、$A$の構成からして、$\mathbb{Q} \cap [0,1]$の要素は全て$A$に含まれていない。したがって、$A$は境界の定義を満たし、内点を持たない集合であることが確認された。

次に、$A$が実軸上のコンパクト集合であることを示す。ハイネボレルの被覆定理から、実軸上のコンパクト集合は有界閉集合と同値なので、$A$が有界であることと、$A$が閉集合であることの二つを確認する。前者については$A \subset [0,1]$であり、$[0,1]$が有界集合なので$A$も有界である。後者については、$A$の構成から考えると、$[0,1]\backslash A$は開集合の可算無限個の和集合となっている。開集合は和について閉じているので、$A^c$が開集合であることがわかる。補集合が開集合であることは$A$が閉集合であることと同値であり、後者についても示された。

最後に、$A$が正のルベーグ測度を持つように構成できることを確認する。$A \subset [0,1]$なので以下が成立する。
\begin{align*}
	\lambda \left( A \right) = \lambda \left( [0,1] \right) - \lambda\left( A^c \right)
\end{align*}
$\lambda \left( [0,1] \right) = 1$であり、$\lambda\left( A^c \right) \leq \sum_{n = 1}^{\infty} 2\frac{\epsilon}{2^{n+1}} = \epsilon$なので、$\lambda \left( A \right) \geq 1-\epsilon$となる。従って$0 < \epsilon < 1$とすれば、$\lambda \left( A \right) > 0$とすることができる。

以上より、ここで構成した$A$が内点を持たない実軸上のルベーグ測度が正のコンパクト集合であることが確認された。

\section{Ex 3.1}
\begin{enumerate}
	\item $f^{-1}\left( A^c \right) = \left( f^{-1}\left( A \right) \right)^c$
	
	\begin{align*}
		x \in \left( f^{-1}\left( A \right) \right)^c \Leftrightarrow x \not\in f^{-1}\left( A \right) \Leftrightarrow f(x) \not\in A \Leftrightarrow f(x) \in A^c \Leftrightarrow x \in f^{-1}\left( A^c \right)
	\end{align*}
	なので等号が成立する。
	\item $f^{-1}\left( \bigcup_{i\in I} A_i \right) = \bigcup_{i\in I} f^{-1} \left( A_i \right)$
	
	任意に$1$つの$x \in f^{-1}\left( \bigcup_{i\in I} A_i \right)$をとる。これは少なくとも$1$つの$i^* \in I$について$f(x) \in A_{i^*}$であることを意味する。これより、
	
	\begin{align*}
		x \in f^{-1}\left( A_{i^*}\right) \subset \bigcup_{i \in I} f^{-1}\left( A_i \right)
	\end{align*}
	であるので、$f^{-1}\left( \bigcup_{i\in I} A_i \right) \subset \bigcup_{i\in I} f^{-1} \left( A_i \right)$である。逆に、任意に$1$つの$x \in \bigcup_{i\in I} f^{-1} \left( A_i \right)$をとる。これは少なくともある$1$つの$i^* \in I$について$f(x) \in A_{i^*}$であることを意味する。以上より、
	\begin{align*}
		f(x) \in A_{i^*} \subset \bigcup_{i\in I} A_i\ \Rightarrow x \in f^{-1}\left( \bigcup_{i \in I} A_i \right)
	\end{align*}
	となり、逆向きの包含関係も成立する。
	
	\item $f^{-1}\left( \bigcap_{i \in I} A_i \right) = \bigcap_{i \in I} f^{-1}\left( A_i \right)$
	
	\begin{align*}
		x \in f^{-1}\left( \bigcap_{i \in I} A_i \right) \Leftrightarrow f(x) \in \bigcap_{i \in I} A_i \Leftrightarrow \forall i \in I\  f(x) \in A_i \Leftrightarrow \forall i \in I\ x \in f^{-1}\left( A_i \right) \Leftrightarrow x \in \bigcap_{i \in I} f^{-1}\left( A_i \right)
	\end{align*}
	
	なので等号が成立する。
\end{enumerate}

\section{Ex 3.2}
\subsection{(a)}
\begin{enumerate}
	\item $f\left( \bigcup_{i \in I} A_i \right) = \bigcup_{i\in I} f\left( A_i \right)$
	
	\begin{align*}
		f\left( \bigcup_{i \in I} A_i \right) = \left\{ f(x)\mid x \in \bigcup_{i \in I } A_i \right\} = \bigcup_{i\in I} \left\{ f(x)\mid x \in A_i \right\} = \bigcup_{i\in I} f\left( A_i \right)
	\end{align*}
	より等号が成立する。
	
	\item $f\left( \bigcap_{i \in I} A_i \right) \subset \bigcap_{i\in I} f\left( A_i \right)$
	
	任意に$1$つ$y \in f\left( \bigcap_{i \in I} A_i \right)$をとる。これは少なくとも一つ$x^* \in \bigcap_{i \in I} A_i$について$y = f(x^*)$となるものがあるということである。この時、$\forall i \in I\ x^* \in A_i$なので、$\forall i\in I\ y = f(x^*)\in f\left( A_i \right)\ \Rightarrow y \in \bigcap_{i \in I} f\left( A_i \right)$であるので、題意が示された。
\end{enumerate}

\subsection{(b)}
 $A_i = \left[ -\frac{1}{i}, \frac{1}{i} \right],\ i = 1,2,3,\dots$に対して、$x \in \mathbb{R}$上の関数を以下のように定める。
\begin{align*}
	f(x) = \begin{cases}
	1 & \text{if}\ x = 0 \\
	0 & \text{otherwise}
	\end{cases}
\end{align*}
この時、
\begin{align*}
	&\bigcap_{i\in I} A_i = \left\{ 0 \right\} \Rightarrow f\left( \bigcap_{i\in I} A_i \right) = \left\{ 1 \right\} \\
	&\bigcap_{i \in I} f\left( A_i \right) = \left\{ 0,1 \right\}
\end{align*}
である。確かに両者は異なり、前者が後者に含まれている。

\subsection{(c)}
 $A = \left\{ 0 \right\}$とし、実軸上の関数$f(x)$を$\forall x \in \mathbb{R}\ f(x) = 0$と定める。この時、
\begin{align*}
	&f\left( A^c \right) = \left\{ 0\right\} \neq \phi\\
	&\left( f\left( A \right) \right)^c = \mathbb{R}\backslash \left\{ 0 \right\} \neq \phi\\
	&f\left( A^c \right) \cap \left( f\left( A \right) \right)^c = \left\{ 0\right\} \cap \left( \mathbb{R}\backslash \left\{ 0 \right\}\right) = \phi
\end{align*}
であるので条件を満たすことが確認された。


\end{document}





























