\documentclass{article}
\usepackage[margin = .7in]{geometry}
\usepackage[dvipdfmx]{graphicx}
\usepackage{listings}
\usepackage{amsmath}
\usepackage{amssymb}
\usepackage{amsfonts}
\usepackage{bm}
\usepackage{mathrsfs}
\lstset{%
  language={python},
  basicstyle={\small},%
  identifierstyle={\small},%
  commentstyle={\small\itshape},%
  keywordstyle={\small\bfseries},%
  ndkeywordstyle={\small},%
  stringstyle={\small\ttfamily},
  frame={tb},
  breaklines=true,
  columns=[l]{fullflexible},%
  numbers=left,%
  xrightmargin=0zw,%
  xleftmargin=3zw,%
  numberstyle={\scriptsize},%
  stepnumber=1,
  numbersep=1zw,%
  lineskip=-0.5ex%
}

\begin{document}
\title{測度論的確率論 2018 S1S2 \\ 
Homework 2}
\author{経済学研究科現代経済コース修士1年 / 池上 慧 (2918009) / sybaster.x@gmail.com}
\maketitle

\section{Ex 1.5}
 まず「$A_n \downarrow \phi,\ A_n \in \mathcal{A}\ \Rightarrow\ \lim_{n \to \infty} \mu \left( A_n \right) = 0$」$\Rightarrow$ 「$\mu$が$\sigma\left( \mathcal{A} \right)$上に拡張できる」を示す。Caratheodoryの拡張定理を用いるために、以下の二点が成立することを確認すれば良い。
\begin{enumerate}
	\item $A_n \in \mathcal{A}$が$n = 1,2,3,\dots$で排反に取られていて、$\bigcup_{n = 1}^{\infty} A_n \in \mathcal{A}$であるなら、$\mu \left( \bigcup_{n = 1}^{\infty} A_n \right) = \sum_{n = 1}^{\infty} \mu \left( A_n \right)$が成立する。
	\item $\mu(\phi) = 0$
\end{enumerate}

$1$番目を確認する。仮定されている有限加法性から以下が成立する。
\begin{align*}
	\mu \left(  \bigcup_{n = 1}^{\infty} A_n \right) = \mu \left( A_1 \cup  \bigcup_{n = 2}^{\infty} A_n \right) = \mu \left( A_1 \right) + \mu \left( \bigcup_{n = 2}^{\infty} A_n \right)
\end{align*}
ただし、$2$つ目の統合は、$\bigcup_{n = 1}^{\infty} A_n \in \mathcal{A}$かつ任意の$n$について$A_n \in \mathcal{A}$であることから、fieldの性質より$ \bigcup_{n = 2}^{\infty} A_n \in \mathcal{A}$となり、$\mu$の定義域に入るため成立することに注意する。この処理は繰り返し用いることができるので、$\mu \left( \bigcup_{n = 1}^{\infty} A_n \right) = \sum_{n = 1}^{\infty} \mu \left( A_n \right)$が成立することがわかる。

$2$番目を確認する。今、$B_n = A_1 \backslash A_n$とすると、仮定より$B_n \uparrow A_1 \backslash \phi$である。さらに$C_n = B_n \backslash B_{n+ 1}$とすると、$\left\{ B_n \right\}_{n = 1}^{\infty}$が単調な列であることから$\left\{ C_n \right\}_{n = 1}^{\infty}$は排反である。有限加法性より、

\begin{align*}
	\mu \left( B_n \right) = \mu \left( \bigcup_{m = 1}^n C_m \right) = \sum_{m = 1}^n \mu \left( C_m \right)
\end{align*}
である。$\mu\left( A_n \right) = \mu \left( A_1 \right) - \mu \left( A_1 \backslash A_n \right) = \mu \left( A_1 \right) - \mu \left( B_n \right)$なので、$\mu \left( A_n \right)$の極限を考えるには$\mu \left( B_n \right)$の極限を考えれば良い。上の式と$1$の性質より、

\begin{align*}
	\lim_{n \to \infty} \mu \left( B_n \right) = \sum_{m = 1}^{\infty} \mu \left( C_m \right) = \mu \left( \bigcup_{m = 1}^{\infty} C_m \right) = \mu \left( \bigcup_{n = 1}^{\infty} B_n \right) = \mu \left( A_1 \backslash \phi \right)
\end{align*}
である。以上より、

\begin{align*}
	\lim_{n \to \infty} \mu \left( A_n \right) = \mu \left( A_1 \right) - \lim_{n \to \infty} \mu\left( B_n \right) = \mu \left( A_1 \right) - \left( \mu \left( A_1 \right) - \mu \left( \phi \right) \right) = \mu \left( \phi \right)
\end{align*}
であり、仮定より$\mu \left( \phi \right) = 0$を得た。


 次に「$\mu$が$\sigma\left( \mathcal{A} \right)$上に拡張できる」$\Rightarrow$ 「$A_n \downarrow \phi,\ A_n \in \mathcal{A}\ \Rightarrow\ \lim_{n \to \infty} \mu \left( A_n \right) = 0$」を示す。レクチャーノートのlemma1.3を用いる。$\mu \left( A_1 \right) < \infty$であることを示せば、lemma1.3より$A_n \downarrow \phi,\ A_n \in \mathcal{A} \subset \sigma\left(\mathcal{A} \right)$のように取れば、測度の定義より空集合の測度は$0$なので、$\lim_{n \to \infty} \mu \left( A_n \right) = \mu \left( \phi \right) = 0$となり題意を示せる。従って$\mu \left( A_1 \right) < \infty$を確認すれば良い。拡張した測度においても$\mathcal{A}$上では元の関数$\mu$と同じ値をとり、元の$\mu$の値域は$\mathbb{R}_{+}$であり、$A_1 \in \mathcal{A}$である。これより$\mu \left( A_1 \right) < \infty$は明らかである。以上で必要十分条件であることが確認された。

\section{Ex 2.1}

\section{Ex 2.9}

\section{Ex 3.1}

\section{Ex 3.2}

\end{document}





























