\documentclass{article}
\usepackage[margin = .7in]{geometry}
\usepackage[dvipdfmx]{graphicx}
\usepackage{listings}
\usepackage{amsmath}
\usepackage{amssymb}
\usepackage{amsfonts}
\usepackage{bm}
\usepackage{mathrsfs}
\lstset{%
  language={python},
  basicstyle={\small},%
  identifierstyle={\small},%
  commentstyle={\small\itshape},%
  keywordstyle={\small\bfseries},%
  ndkeywordstyle={\small},%
  stringstyle={\small\ttfamily},
  frame={tb},
  breaklines=true,
  columns=[l]{fullflexible},%
  numbers=left,%
  xrightmargin=0zw,%
  xleftmargin=3zw,%
  numberstyle={\scriptsize},%
  stepnumber=1,
  numbersep=1zw,%
  lineskip=-0.5ex%
}

\begin{document}
\title{測度論的確率論 2018 S1S2 \\ 
Homework 9}
\author{経済学研究科現代経済コース修士1年 / 池上 慧 (29186009) / sybaster.x@gmail.com}
\maketitle

\section{2.3.11}
まず$X_n \xrightarrow{p} 0\ \Leftrightarrow\ p_n \to 0$を示す。右側を仮定する。この時、任意に$\epsilon, \delta > 0$を取ると、
\begin{align*}
	\exists N\ \text{s.t.}\ n\geq N\ \Rightarrow\ P\left( |X_n| > \delta \right) = P(X_n = 1) = p_n < \epsilon
\end{align*}
であるので確率収束の定義より確かに$X_n \xrightarrow{p} 0$である。逆に左側を仮定すると、任意の$\epsilon, \delta > 0$に対して
\begin{align*}
	\exists N\ \text{s.t.}\ n\geq N\ \Rightarrow\ p_n = P(X_n = 1) = P(|X_n| > \delta) < \epsilon
\end{align*}
であるので収束の定義より確かに$p_n \to 0$である。

次に$X_n \to 0\ \text{a.s.}\ \Leftrightarrow\ \sum_n p_n < \infty$を示す。右側を仮定する。この時、$A_n = \left\{ |X_n| > \epsilon \right\}$とすると、
\begin{align*}
	\sum_n p_n = \sum_n P(A_n) < \infty
\end{align*}
なのでBorel-Cantelliより$P(A_n\ \text{i.o.}) = 0$であり、$P(A_n\ \text{i.o.}) = 0\ \Leftrightarrow\ X_n\ \to 0 \ \text{a.s.}$より確かに日代理側が成立する。次に逆向きを示す。$A_n$が独立な事象であることからTheorem 2.3.7より、
\begin{align*}
	\sum_n p_n = \sum P(A_n) = \infty\ \Rightarrow\ P(A_n\ \text{i.o.}) = 1
\end{align*}
である。この対偶は、
\begin{align*}
	P(A_n\ \text{i.o.}) < 1\ \Rightarrow\ \sum_n p_n < \infty
\end{align*}
であり、今仮定より$P(A_n\ \text{i.o.}) = 0$なので上の結果より確かに右側が成立する。

\section{2.3.13}

\section{2.3.14}

\section{2.5.9}

\section{2.5.10}

\section{2.5.11}

\end{document}



























