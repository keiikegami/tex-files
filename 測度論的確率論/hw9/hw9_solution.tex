\documentclass{article}
\usepackage[margin = .7in]{geometry}
\usepackage[dvipdfmx]{graphicx}
\usepackage{listings}
\usepackage{amsmath}
\usepackage{amssymb}
\usepackage{amsfonts}
\usepackage{bm}
\usepackage{mathrsfs}
\lstset{%
  language={python},
  basicstyle={\small},%
  identifierstyle={\small},%
  commentstyle={\small\itshape},%
  keywordstyle={\small\bfseries},%
  ndkeywordstyle={\small},%
  stringstyle={\small\ttfamily},
  frame={tb},
  breaklines=true,
  columns=[l]{fullflexible},%
  numbers=left,%
  xrightmargin=0zw,%
  xleftmargin=3zw,%
  numberstyle={\scriptsize},%
  stepnumber=1,
  numbersep=1zw,%
  lineskip=-0.5ex%
}

\begin{document}
\title{測度論的確率論 2018 S1S2 \\ 
Homework 9}
\author{経済学研究科現代経済コース修士1年 / 池上 慧 (29186009) / sybaster.x@gmail.com}
\maketitle

\section{2.3.11}
まず$X_n \xrightarrow{p} 0\ \Leftrightarrow\ p_n \to 0$を示す。右側を仮定する。この時、任意に$\epsilon, \delta > 0$を取ると、
\begin{align*}
	\exists N\ \text{s.t.}\ n\geq N\ \Rightarrow\ P\left( |X_n| > \delta \right) = P(X_n = 1) = p_n < \epsilon
\end{align*}
であるので確率収束の定義より確かに$X_n \xrightarrow{p} 0$である。逆に左側を仮定すると、任意の$\epsilon, \delta > 0$に対して
\begin{align*}
	\exists N\ \text{s.t.}\ n\geq N\ \Rightarrow\ p_n = P(X_n = 1) = P(|X_n| > \delta) < \epsilon
\end{align*}
であるので収束の定義より確かに$p_n \to 0$である。

次に$X_n \to 0\ \text{a.s.}\ \Leftrightarrow\ \sum_n p_n < \infty$を示す。右側を仮定する。この時、$A_n = \left\{ |X_n| > \epsilon \right\}$とすると、
\begin{align*}
	\sum_n p_n = \sum_n P(A_n) < \infty
\end{align*}
なのでBorel-Cantelliより$P(A_n\ \text{i.o.}) = 0$であり、$P(A_n\ \text{i.o.}) = 0\ \Leftrightarrow\ X_n\ \to 0 \ \text{a.s.}$より確かに日代理側が成立する。次に逆向きを示す。$A_n$が独立な事象であることからTheorem 2.3.7より、
\begin{align*}
	\sum_n p_n = \sum P(A_n) = \infty\ \Rightarrow\ P(A_n\ \text{i.o.}) = 1
\end{align*}
である。この対偶は、
\begin{align*}
	P(A_n\ \text{i.o.}) < 1\ \Rightarrow\ \sum_n p_n < \infty
\end{align*}
であり、今仮定より$P(A_n\ \text{i.o.}) = 0$なので上の結果より確かに右側が成立する。

\section{2.3.13}
何らかの正の無限大に発散する正実数列$\left\{ c_n \right\}$に対して、任意に$\epsilon > 0$を取ったときに$P\left( \frac{|X_n|}{c_n} > \epsilon\ \text{i.o.} \right) = 0$とできることを示す。Borel-Cantelliより、そのためには$\sum_n P\left( \frac{|X_n|}{c_n} > \epsilon \right) < \infty$とできることを示せば良い。すなわち足していく確率を級数が収束するように十分小さく抑えれば良い。
\begin{align}
	P\left( \frac{|X_n|}{c_n} > \epsilon \right) = P\left( \frac{|X_n|}{\epsilon} > c_n \right) \leq \frac{1}{2^n}
\end{align}
これを満たすような$c_n\to \infty$が存在すれば、$\sum_n \frac{1}{2^n} < \infty$なので題意は示される。以下ではこのような正実数列$\left\{ c_n \right\}\to \infty$が存在することを確認する。

まず各$n$について$(1)$を満たす$c_n$が存在することを確認する。$X_n$は$\mathbb{R}$ valued 確率変数であるので$\frac{|X_n|}{\epsilon} < \infty$である。従って、
\begin{align*}
	P\left( \frac{|X_n|}{\epsilon} > c \right)\to 0\ \text{as}\ c\to \infty
\end{align*}
である。この事実より、$(1)$を満たす$c_n$は各$n$について存在する。さらにこのような$c_n$について、
\begin{align*}
	P\left( \frac{|X_n|}{c_n} \leq \epsilon \right) \geq 1-\frac{1}{2^n}
\end{align*}
であるので、両辺$n\to \infty$とすると、
\begin{align*}
	\lim_{n\to \infty} P\left( \frac{|X_n|}{c_n} \leq \epsilon \right)= 1
\end{align*}
である。$c_n \to \infty$であるとき$\frac{1}{c_n} \to 0$であるので確かに上の式を満たす。以上より条件を満たす正の実数列$\left\{ c_n \right\}$が必ず存在することがわかった。

\section{2.3.14}
$\sup X_n < \infty\ \text{a.s.}\ \Leftrightarrow\ \sum_n P\left( X_n > A \right) < \infty\ \text{for some $A$}$を示す。右側を仮定すると、Borel-Cantelliとドモルガンの定理から
\begin{align*}
	\sum_n P\left( X_n > A \right) < \infty\ \Rightarrow\ P\left( X_n > A\ \text{i.o.} \right) = 0 = P\left( X_n \leq A\ \text{a.a.} \right) = 1
\end{align*}
である。ただしここで$a.a.$はDurrettにおける$\liminf A_n$に対応する記号であるとする。これは確かに左側が成立していることを意味している。逆向きは対偶を示す。すなわち、
\begin{align*}
	\forall A, \sum_n P\left( X_n > A \right) = \infty\ \Rightarrow\ P\left( \sup X_n < \infty \right) < 1
\end{align*}
を示す。任意に$A$を取って、$X_n$が独立な確率変数であることに注意すると、$X_n > A$という事象も独立であるのでTheorem 2.3.7より、
\begin{align*}
	\sum_n P\left( X_n > A \right) = \infty\ \Rightarrow\ P\left( X_n > A\ \text{i.o.} \right) = 1
\end{align*}
が成立する。これは$X_n$が確率$1$で任意の$A$をどこかの$n$で上回ることを意味しているので確かに$P\left( \sup X_n < \infty \right) = 0 < 1$が満たされる。以上で題意は示された。

\section{2.5.9}

\section{2.5.10}

\section{2.5.11}

\end{document}



























