\documentclass{article}
\usepackage[margin = .7in]{geometry}
\usepackage[dvipdfmx]{graphicx}
\usepackage{listings}
\usepackage{amsmath}
\usepackage{amssymb}
\usepackage{amsfonts}
\usepackage{bm}
\usepackage{mathrsfs}
\lstset{%
  language={python},
  basicstyle={\small},%
  identifierstyle={\small},%
  commentstyle={\small\itshape},%
  keywordstyle={\small\bfseries},%
  ndkeywordstyle={\small},%
  stringstyle={\small\ttfamily},
  frame={tb},
  breaklines=true,
  columns=[l]{fullflexible},%
  numbers=left,%
  xrightmargin=0zw,%
  xleftmargin=3zw,%
  numberstyle={\scriptsize},%
  stepnumber=1,
  numbersep=1zw,%
  lineskip=-0.5ex%
}

\begin{document}
\title{測度論的確率論 2018 S1S2 \\ 
Homework 6}
\author{経済学研究科現代経済コース修士1年 / 池上 慧 (29186009) / sybaster.x@gmail.com}
\maketitle

\section{Ex6.3}
\subsection{(a)}
$F$のサポートを以下で書くことにする。
\begin{align*}
	S_F = \left\{ x \in \mathbb{R} \mid \forall \epsilon > 0, \ F(x + \epsilon) - F(x - \epsilon) > 0 \right\}
\end{align*}
閉集合の定義より、$\mathbb{R} \setminus S_F$が開集合であることを示せばよい。任意に$x \in \mathbb{R} \setminus S_F$を取る時、$F$が非減少関数であることから、
\begin{align}
	\exists \epsilon > 0\ \text{s.t.}\ F(x + \epsilon) - F(x - \epsilon) = 0
\end{align}
である。ここで、必ず$\epsilon > \eta > 0$であるような$\eta$が存在し、そのような$\eta$について以下が成立する。
\begin{align*}
	\begin{cases}
	F(x + \epsilon) \geq F(x + \eta)\\
	F(x -\eta) \geq F(x -\epsilon)
	\end{cases}
\end{align*}
$(1)$より、$F(x + \eta) - F(x - \eta) = 0$であるので$(x-\eta, x+\eta) \subset \mathbb{R} \setminus S_F$である。よって定義より$\mathbb{R} \setminus S_F$は開集合であることが確認できたので題意は示された。

\subsection{(b)}
対偶を示す。すなわち以下を仮定して$F$が左連続でないことを示す。
\begin{align*}
	\exists x \in S_F\ \text{s.t.}\ \exists \epsilon > 0\ \text{s.t.}\ (x-\epsilon, x+\epsilon) \cap S_F = \left\{ x \right\}
\end{align*}
今、$x-\epsilon = x_0$として、左から$x$に収束する数列$\left\{ x_n \right\}_{n = 0}^{\infty}$を任意の$n$について$x_n < x$であるようにとる。この時、仮定よりこの数列の要素はどれもサポートに入らないので、
\begin{align*}
	\forall n,\ F(x_n) = F(x_{x+1})
\end{align*}
である。これより、$\lim_{n \to \infty} F(x_n) = f(x-\epsilon)$である。また右側から近づく数列についても同様に考えることができ、$f(x) = F(x + \epsilon)$である。一方で$F(\lim_{n \to \infty} x_n) = F(x)$である。仮に$F$が連続だとすると、
\begin{align*}
F(x - \epsilon) = F(x) = F(x + \epsilon)
\end{align*}
となり、これは$x$がサポートに入ることに矛盾する。従って題意は示された。

\section{Ex6.8}
任意に$p \in [0,1]$を取る。この時、$F^{-1}$の定義より以下は$F$が連続でなくても成立する。
\begin{align*}
	P\left( F(x) \leq p \right) = P\left( x \leq F^{-1}(p) \right) = F \left(F^{-1} (p)\right)
\end{align*}
$F$が連続でないとする。不連続点における左極限を$a$、右極限を$b$とすると、$p\in [a, b)$の時に、
\begin{align*}
	F \left(F^{-1} (p)\right) = b
\end{align*}
であるので、確かに$F$が連続でない時$F(X)$が一様分布に従わないことがわかる。一方で$F$が連続の時は、右極限と左翼弦が一致するため上のように$p$を取ることができない。従って、
\begin{align*}
	F \left(F^{-1} (p)\right) = p
\end{align*}
が成立する。よって題意は示された。

\section{Ex6.10}
\subsection{(a)}
測度の単調性と、アトムを持たないという仮定より以下が成立する。
\begin{align*}
	A_1 \subset A,\ A_1 \in \mathcal{F}\ \Rightarrow P(A) > P(A_1) > 0
\end{align*}
この論法で$A_1$に対しても同様に$A_2\ \text{s.t.}\ A_2 \subset A_1,\ A_2 \in \mathcal{F},\ P(A_1) > P(A_2) > 0$となる$A_2$が作れる。これを繰り返して減少列$A_1, A_2, A_3, \cdots$を作る。この時、
\begin{align}
	\lim_{n \to \infty} P\left(A_n \right) = 0
\end{align}
であることを示せば、任意の$\epsilon > 0$に対して、ある大きな$N$が存在して、$n \geq N \Rightarrow P(A_n) < \epsilon$であるので、そのような$n$のうちで一つの$A_n$を$B$とすれば題意は示されている。よって$(1)$を以下で示す。

減少列であることから、
\begin{align*}
	\lim_{n \to \infty} P(A_n) = P\left(\bigcap_n A_n \right)
\end{align*}
である。ここで、$\left\{ A_n \right\}$の構成の仕方から$\bigcap_n A_n \in \mathcal{F}$である。$P\left( \bigcap_n A_n \right) > 0$とすると、アトムを持たないという仮定より、
\begin{align*}
	\exists C \subset \bigcap_n A_n,\ C \in \mathcal{F} \ \Rightarrow\ P\left( \bigcap_n A_n \right) > p(C) > 0
\end{align*}
である。しかし、そのような$C$は明らかに$\left\{ A_n \right\}$のどこかに含まれるため、$P(C)\geq P\left( \bigcap_n A_n \right)$であり矛盾する。よって、$P\left( \bigcap_n A_n \right) = 0$であるので題意は示された。

\subsection{(b)}
任意の$n \in \mathbb{N}$について$B_n = \left\{ C \mid a - \frac{1}{n} < P(C) < a + \frac{1}{n}, C \subset A, C \in \mathcal{F}\right\}$が空集合でないことを示す。これを否定すると、
\begin{align*}
	\exists N > 0\ \text{s.t.}\ n \geq N\ \Rightarrow\ P(C) \leq a - \frac{1}{n}\ \text{or}\ a + \frac{1}{n} \leq P(C)\ \forall C \subset A, C \in \mathcal{F}
\end{align*}
である。$P(C) \leq a - \frac{1}{N}$の時、$P(A \setminus C) > \frac{1}{N} > 0$であるので$(a)$より、$D \subset A \setminus C$で任意に小さな$\epsilon > 0$に対して$0 < P(D) < \epsilon$であるような$D \in \mathcal{F}$が存在する。ここで$C$と$D$はその構成より排反なので、$P(C \cup D) = P(C) + P(D)$である。これは$a - \frac{1}{N} < P(C\cup D) < a + \frac{1}{N}$とできるような$D$が必ず存在することを意味する。$P(C) \geq a + \frac{1}{N}$のケースについても同様に考えることで$a - \frac{1}{N} < P(C\setminus D) < a + \frac{1}{N}$となるような$D$を適切にとってくることができる。
以上より、先の否定は成立しないので題意は示された。

\section{Ex6.12}
Lecture noteより任意の内点$x \in I$に対して任意の$y \in I$と任意の$a \in \left[ D_{-}\phi(x), D_{+}\phi(x) \right]$が取れて
\begin{align}
	\varphi(y) \geq \varphi(x) + a(y-x)
\end{align}
である。今、$S = \left\{ x_i \mid i \in \mathbb{N} \right\}$を$I$の可算稠密集合とする。これは$\mathbb{R}$が可分であることより必ず存在する。ここで$(3)$より、任意の$x\in I$に対して、
\begin{align}
	\varphi(x) \geq \varphi(x_i) + D_{+}\varphi(x_i) (x - x_i)
\end{align}
が任意の$i$について成り立つ。ここで、
\begin{align*}
	&a_i = D_{+}\varphi(x_i)\\
	&b_i = \varphi(x_i) - D_{+}\varphi(x_i) x_i
\end{align*}
とおいて、
\begin{align*}
	\psi(x) = \sup_i \left\{ a_i x + b_i \right\}
\end{align*}
と定義する。$(4)$より$\varphi(x) \geq \psi(x)$が成立することが確認できた。従って逆向きの不等号を証明すれば題意を示したことになる。

$(3)$より以下が成立する。
\begin{align*}
	\varphi(x_i) \geq \varphi(x) + D_{+}\varphi(x) (x_i - x)
\end{align*}
これより、
\begin{align*}
	\psi(x) &\geq \varphi(x_i) + D_{+}\varphi(x_i) (x - x_i)\\
	&\geq \varphi(x) - (x_i - x) (D_{+}\varphi(x_i) - D_{+}\varphi(x))
\end{align*}
が任意の$i$について成立する。ここで、「$D_{+} \varphi(x)$が右連続である(主張$1$)」を所与とすると、$x_i \downarrow x$の時に右辺第二項が$0$となるので、確かに$\psi(x) \geq \varphi(x)$が成立する。よって両方の不等号が示されたので、$\psi(x) =  \varphi(x)$が示された。

\subsection{主張$1$の証明}
$x < z < y$とする。
\begin{align*}
	\frac{\varphi(y) - \varphi(z)}{y-z}
\end{align*}
を考えると、lecture noteよりこれは$z$より大きな$y \in I$について非減少関数である。従って、そのような$y^{*}$について以下が成立する。
\begin{align*}
	\lim_{y\downarrow z} \frac{\varphi(y) -\varphi(z)}{y-z} \leq \frac{\varphi(y^{*})- \varphi(z)}{y^{*}-z}
\end{align*}
この両辺で$z\downarrow x$をとると、
\begin{align*}
	\lim_{z\downarrow x} D_{+}\varphi(z) \leq \lim_{z\downarrow x} \frac{\varphi(y^{*})- \varphi(z)}{y^{*}-z} = \frac{\varphi(y^{*})- \varphi(x)}{y^{*}-x}
\end{align*}
となる。ただし、等号は$\varphi(x)$が内部で連続であることより得られる。ここで、$y^{*} \downarrow x$を両辺でとると、左辺は関係ないので変化せず、
\begin{align*}
	\lim_{z\downarrow x} D_{+}\varphi(z) \leq \lim_{y^{*} \downarrow x} \frac{\varphi(y^{*})- \varphi(x)}{y^{*}-x} = D_{+}\varphi(x)
\end{align*}
を得る。

あとは逆向きの不等号も示せば題意が示されたことになる。つまり、$x < z$で常に$D_{+}\varphi(x) \leq D_{+}\varphi(z)$が成立することを示せば良い。ここで、$x < y < z < w$が内点としてとれる時、凸関数の性質から、
\begin{align*}
	\frac{\varphi(y) - \varphi(x)}{y-x} \leq \frac{\varphi(z) - \varphi(x)}{z-x} \leq \frac{\varphi(w) - \varphi(z)}{w-z}
\end{align*}
である。この両端に注目すると、確かに$D_{+}\varphi(x) \leq D_{+}\varphi(z)$であることがわかる。これより逆向きの不等号も示されたので、$D_{+}\varphi(x) = \lim_{z\downarrow x} D_{+}\varphi(z)$である。これは$D_{+}\varphi(z)$が右連続であることを示している。

\section{Ex6.14}
\begin{align*}
	E[X] = E\left[ X 1_{\left\{ X \leq \theta E[X] \right\}} \right] + E\left[ X 1_{\left\{ X > \theta E[X] \right\}} \right]
\end{align*}
両辺二乗して、第二項にヘルダーの不等式を用いて以下を得る。
\begin{align*}
	\left( E[X] \right)^2 \leq E\left[ X 1_{\left\{ X \leq \theta E[X] \right\}} \right]^2 + 2 E\left[ X 1_{\left\{ X \leq \theta E[X] \right\}} \right] E\left[ X 1_{\left\{ X > \theta E[X] \right\}} \right] + E[X^2] P\left( X > \theta E[X] \right)
\end{align*}
これを整理して以下を得る。
\begin{align*}
	P\left( X > \theta E[X] \right) \geq \frac{\left( E\left[ X 1_{\left\{ X \leq \theta E[X] \right\}} \right] - E[X] \right)^2}{E[X^2]}
\end{align*}
従って、題意を得るためには以下が成立していれば良い。
\begin{align*}
	\left( E\left[ X 1_{\left\{ X \leq \theta E[X] \right\}} \right] - E[X] \right)^2 \geq \left( E[X] - \theta E[X] \right)^2
\end{align*}
上から打ち切って期待値をとっているため、$E\left[ X 1_{\left\{ X \leq \theta E[X] \right\}} \right] < E[X]$であるので、左辺の中身は負である。一方で、$\theta \in [0,1]$であるので右辺の中身は正である。これより、以下を示せばよい。
\begin{align*}
	E[X] - E\left[ X 1_{\left\{ X \leq \theta E[X] \right\}} \right]  \geq E[X] - \theta E[X] 
\end{align*}
ここで、$\theta E[X]$よりも小さい値についてしか期待値を取れないので、$E\left[ X 1_{\left\{ X \leq \theta E[X] \right\}} \right] \leq \theta E[X]$であることから上は確かに成立する。よって題意は示された。

\section{Ex6.15}
\subsection{(a)}
$p > q > 0$とする。この時、$f(x) = -x^{\frac{q}{p}}$は凸関数である。従ってJensenの不等式より、
\begin{align*}
	&E\left[ -\left( |X|^p \right)^{\frac{q}{p}} \right] \geq - \left( E\left[ |X|^p \right]\right)^{\frac{q}{p}}\\[8pt]
	\Leftrightarrow&\ E[\left( |X|^p \right)^{\frac{q}{p}}] \leq \left( E[|X|^p] \right)^{\frac{q}{p}}\\[8pt]
	\Leftrightarrow&\ \left( E[|X|^q]\right)^{\frac{1}{q}} \leq \left( E[|X|^p]\right)^{\frac{1}{p}}
\end{align*}

\subsection{(b)}
$\frac{r-q}{r-p} + \frac{q-p}{r-p} = 1$であるので、ヘルダーの不等式を用いて以下を得る。
\begin{align*}
	\left( E[|X|^p] \right)^{\frac{r-q}{r-p}} \left( E[|X|^r] \right)^{\frac{q-p}{r-p}} = \left( E \left[\left( |X|^{p\frac{r-q}{r-p}}\right)^{\frac{r-p}{r-q}} \right] \right)^{\frac{r-q}{r-p}} \left( E \left[\left( |X|^{r\frac{q-p}{r-p}}\right)^{\frac{r-p}{q-p}} \right] \right)^{\frac{q-p}{r-p}} \geq E\left[ |X|^{\frac{p(r-q) + r(q-p)}{r-p}} \right] = E[|X|^q]
\end{align*}
よって題意は示された。

\end{document}






























