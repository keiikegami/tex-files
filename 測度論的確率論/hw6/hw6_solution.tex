\documentclass{article}
\usepackage[margin = .7in]{geometry}
\usepackage[dvipdfmx]{graphicx}
\usepackage{listings}
\usepackage{amsmath}
\usepackage{amssymb}
\usepackage{amsfonts}
\usepackage{bm}
\usepackage{mathrsfs}
\lstset{%
  language={python},
  basicstyle={\small},%
  identifierstyle={\small},%
  commentstyle={\small\itshape},%
  keywordstyle={\small\bfseries},%
  ndkeywordstyle={\small},%
  stringstyle={\small\ttfamily},
  frame={tb},
  breaklines=true,
  columns=[l]{fullflexible},%
  numbers=left,%
  xrightmargin=0zw,%
  xleftmargin=3zw,%
  numberstyle={\scriptsize},%
  stepnumber=1,
  numbersep=1zw,%
  lineskip=-0.5ex%
}

\begin{document}
\title{測度論的確率論 2018 S1S2 \\ 
Homework 6}
\author{経済学研究科現代経済コース修士1年 / 池上 慧 (29186009) / sybaster.x@gmail.com}
\maketitle

\section{Ex6.3}


\section{Ex6.8}

\section{Ex6.10}

\section{Ex6.12}

\section{Ex6.14}
\begin{align*}
	E[X] = E\left[ X 1_{\left\{ X \leq \theta E[X] \right\}} \right] + E\left[ X 1_{\left\{ X > \theta E[X] \right\}} \right]
\end{align*}
両辺二乗して、第二項にヘルダーの不等式を用いて以下を得る。
\begin{align*}
	\left( E[X] \right)^2 \leq E\left[ X 1_{\left\{ X \leq \theta E[X] \right\}} \right]^2 + 2 E\left[ X 1_{\left\{ X \leq \theta E[X] \right\}} \right] E\left[ X 1_{\left\{ X > \theta E[X] \right\}} \right] + E[X^2] P\left( X > \theta E[X] \right)
\end{align*}
これを整理して以下を得る。
\begin{align*}
	P\left( X > \theta E[X] \right) \geq \frac{\left( E\left[ X 1_{\left\{ X \leq \theta E[X] \right\}} \right] - E[X] \right)^2}{E[X^2]}
\end{align*}
従って、題意を得るためには以下が成立していれば良い。
\begin{align*}
	\left( E\left[ X 1_{\left\{ X \leq \theta E[X] \right\}} \right] - E[X] \right)^2 \geq \left( E[X] - \theta E[X] \right)^2
\end{align*}
上から打ち切って期待値を撮っているため、$E\left[ X 1_{\left\{ X \leq \theta E[X] \right\}} \right] < E[X$であるので、左辺の中身は負である。一方で、$\theta \in [0,1]$であるので右辺の中身は正である。これより、以下を示せばよい。
\begin{align*}
	E[X] - E\left[ X 1_{\left\{ X \leq \theta E[X] \right\}} \right]  \geq E[X] - \theta E[X] 
\end{align*}
ここで、$\theta E[X]$よりも小さい値についてしか期待値を取れないので、$E\left[ X 1_{\left\{ X \leq \theta E[X] \right\}} \right] \leq \theta E[X]$であることから上は確かに成立する。よって題意は示された。

\section{Ex6.15}
\subsection{(a)}
$p > q > 0$とする。この時、$f(x) = -x^{\frac{q}{p}}$は凸関数である。従ってJensenの不等式より、
\begin{align*}
	&E\left[ -\left( |X|^p \right)^{\frac{q}{p}} \right] \geq - \left( E\left[ |X|^p \right]\right)^{\frac{q}{p}}\\[8pt]
	\Leftrightarrow&\ E[\left( |X|^p \right)^{\frac{q}{p}}] \leq \left( E[|X|^p] \right)^{\frac{q}{p}}\\[8pt]
	\Leftrightarrow&\ \left( E[|X|^q]\right)^{\frac{1}{q}} \leq \left( E[|X|^p]\right)^{\frac{1}{p}}
\end{align*}

\subsection{(b)}
$\frac{r-q}{r-p} + \frac{q-p}{r-p} = 1$であるので、ヘルダーの不等式を用いて以下を得る。
\begin{align*}
	\left( E[|X|^p] \right)^{\frac{r-q}{r-p}} \left( E[|X|^r] \right)^{\frac{q-p}{r-p}} = \left( E \left[\left( |X|^{p\frac{r-q}{r-p}}\right)^{\frac{r-p}{r-q}} \right] \right)^{\frac{r-q}{r-p}} \left( E \left[\left( |X|^{r\frac{q-p}{r-p}}\right)^{\frac{r-p}{q-p}} \right] \right)^{\frac{q-p}{r-p}} \geq E\left[ |X|^{\frac{p(r-q) + r(q-p)}{r-p}} \right] = E[|X|^q]
\end{align*}
よって題意は示された。

\end{document}






























