\documentclass{article}
\usepackage[margin = .7in]{geometry}
\usepackage[dvipdfmx]{graphicx}
\usepackage{listings}
\usepackage{amsmath}
\usepackage{amssymb}
\usepackage{amsfonts}
\usepackage{bm}
\usepackage{mathrsfs}
\lstset{%
  language={python},
  basicstyle={\small},%
  identifierstyle={\small},%
  commentstyle={\small\itshape},%
  keywordstyle={\small\bfseries},%
  ndkeywordstyle={\small},%
  stringstyle={\small\ttfamily},
  frame={tb},
  breaklines=true,
  columns=[l]{fullflexible},%
  numbers=left,%
  xrightmargin=0zw,%
  xleftmargin=3zw,%
  numberstyle={\scriptsize},%
  stepnumber=1,
  numbersep=1zw,%
  lineskip=-0.5ex%
}

\begin{document}
\title{測度論的確率論 2018 S1S2 \\ 
Homework 10}
\author{経済学研究科現代経済コース修士1年 / 池上 慧 (29186009) / sybaster.x@gmail.com}
\maketitle

\section{3.2.1}
\begin{align*}
	f_n(x) = \begin{cases} 0 & \text{when $\left[ 2^n x \right]$ is even}\\[8pt]
	2 & \text{when $\left[ 2^n x \right]$ is odd}
	\end{cases}
\end{align*}
なる関数列を考える。ただし$\left[ x \right]$は$x$を超えない最大の整数を表す記号とする。この時、任意の$n$について$f_n(x)$は$0,2$のいずれかしか取り得ないので、明らかに$n \to \infty$で$f_n(x) \neq 1 ,\forall x \in \mathbb{R}$である。しかし分布関数を考えると、任意の$x \in \mathbb{R}$を考えると、
\begin{align*}
	F_n(x) = \int_0^x f_n(y) \mathrm{d}y = \begin{cases} \frac{\left[ 2^n x \right] - 1}{2} \times 2 \times \frac{1}{2^n} + 2\left( x - \frac{\left[ 2^n x \right]}{2^n} \right) & \text{when $\left[ 2^n x \right]$ is odd} \\[8pt]
	\frac{\left[ 2^n x \right] }{2} \times 2 \times \frac{1}{2^n}  & \text{when $\left[ 2^n x \right]$ is even}
	\end{cases}
\end{align*}
である。前者は整理すると$2x - \frac{1 + \left[ 2^n x \right]}{2^n}$であり、$2^n x - 1 \leq \left[ 2^n x \right] \leq 2^n x$なので、
\begin{align*}
	x - \frac{1}{2^n}\leq 2x - \frac{1 + \left[ 2^n x \right]}{2^n} \leq x
\end{align*}
であり、はさみうちの原理から$n \to \infty$で$x$に収束する。また後者についても同様の議論から、
\begin{align*}
	x-\frac{1}{2^n} \leq \frac{\left[ 2^n x \right]}{2^n} \leq x
\end{align*}
となるのでこちらも$x$に収束する。従って、$x \in \mathbb{R}$に各点で分布関数$F_n$が一様分布の分布関数に収束するので一様分布に分布収束していることがわかる。

\section{3.2.6}
\subsection{距離の公理}
まず距離の公理を満たすことを確認する。

\subsubsection{対称性}
$y = x - \epsilon, z = x + \epsilon$と置くと、
\begin{align*}
	\left\{ \epsilon \mid F(x-\epsilon) - \epsilon \leq G(x) \leq F(x+\epsilon) + \epsilon, \forall x \right\} &= \left\{ \epsilon \mid F(y) \leq G(y + \epsilon) + \epsilon\ \text{and}\  G(z-\epsilon) - \epsilon \leq F(z), \forall y , z\right\} \\
	&= \left\{ \epsilon \mid G(x - \epsilon) - \epsilon \leq F(x) \leq G(x + \epsilon) + \epsilon, \forall x \right\}
\end{align*}
であるので$\inf$を取る対象となる集合が変わらないため$\rho\left( F, G \right) = \rho\left( G, F \right)$である。

\subsubsection{非負性}
$\rho\left( F, G \right) < 0$とすると実数の連続性より$\rho\left( F, G \right) < \alpha < 0$なる実数$\alpha$が必ず存在し、この時以下が満たされる。
\begin{align*}
	\forall x, F\left( x - \alpha \right) - \alpha \leq G(x) \leq F\left( x + \alpha \right) + \alpha
\end{align*}
しかし、$x - \alpha > x > x + \alpha$であるので、分布関数の定義より$F(x-\alpha) \geq F(x + \alpha)$であり、さらに$F(x-\alpha) -\alpha \geq F(x + \alpha) + \alpha$である。これと先の不等式より、$G(x) = F(x-\alpha) -\alpha = F(x + \alpha) + \alpha$が任意の$x$について成立していることになる。しかしこの時、$x\to \infty$で$G(x) \to 1+\alpha,1 -\alpha$であり、$1$には収束しないので分布関数の定義を満たさない。従って、$\rho\left( F, G \right) \geq 0$である。

\subsubsection{$\rho\left( F, G \right)  =  0 \Leftrightarrow F = G$}
$(\Leftarrow part)$ 
\begin{align*}
	\rho\left( F, G \right) = \inf\left\{ \epsilon \mid F(x-\epsilon) - \epsilon \leq F(x) \leq F(x+\epsilon) + \epsilon, \forall x \right\}
\end{align*}
任意の$\epsilon \geq 0$について$F(x-\epsilon) \leq F(x), F(x + \epsilon) \geq F(x)$であるので、$\inf$を取る集合は任意の$\epsilon \geq 0$を含む。非負であることは先に示したので、この時確かに$\rho\left( F, G \right)  =  0$である。

\noindent $(\Rightarrow part)$
\begin{align*}
	F(x - \epsilon) - \epsilon \leq G(x) \leq F(x + \epsilon) + \epsilon
\end{align*}
に対して$\epsilon \to 0$とすると、仮定より$F(x_{-}) \leq G(x) \leq F(x_{+})$が成立する。これより、もし$x$が$F$の連続点であれば、$G(x) = F(x)$である。また、もし$x$が連続点でない場合出会っても、不連続点が高々可算個しかないことから上から不連続点$x$に近づく連続点の列$\left\{ x_m \right\}$を考えることができ、分布関数の右連続性と、連続点においては$F.G$が等しいという先の結果より、
\begin{align*}
	G(x) = G(x_{+}) = \lim_{x_m \downarrow x} G(x_m) = \lim_{x_m \downarrow x} F(x_m) = F(x_{+}) = F(x)
\end{align*}
となり、不連続点においても$F = G$となることが確認できる。

\subsubsection{三角不等式}
三つの分布関数$F,G,H$について、
\begin{align*}
	\rho\left( F, G \right) + \rho\left( G, H \right) \geq \rho\left( F, H \right)
\end{align*}
が成立することを示す。まず、$\rho\left( F, G \right) = \epsilon_1, \rho\left( G, H \right) = \epsilon_2, \epsilon = \epsilon_1 + \epsilon_2$と置く。この時任意の$x$について、
\begin{align*}
	F\left( x - \epsilon \right) - \epsilon = F\left(x - \epsilon_1 - \epsilon_2 \right) - \epsilon_1 - \epsilon_2 \leq G\left( x - \epsilon_2 \right) - \epsilon_2 \leq H(x)
\end{align*}
が成立する。ただし一つ目の不等号は$\epsilon_1$の定義より二つめの不等号は$\epsilon_2$の定義より得られる。同様にして、
\begin{align*}
	F\left( x + \epsilon \right) + \epsilon = F\left(x + \epsilon_1 + \epsilon_2 \right) +\epsilon_1 + \epsilon_2 \geq G\left( x + \epsilon_2 \right) + \epsilon_2 \geq H(x)
\end{align*}
も得られる。以上より、
\begin{align*}
	F\left( x - \epsilon \right) - \epsilon \leq H(x) \leq F\left( x + \epsilon \right) + \epsilon
\end{align*}
が任意の$x$について満たされるので、$\rho\left( F, H \right)$は$\epsilon$以下の値となることがわかる。以上より題意の不等式が示された。

\subsection{分布収束との同値性}
次に$\rho\left( F_n, F \right) \to 0\ \Leftrightarrow\ F_n \xrightarrow{w} F$を示す。

\noindent $(\Rightarrow part)$

仮定より任意の$\epsilon > 0$と任意の$x$に対して十分大きな$n$をとれば、
\begin{align*}
	F(x-\epsilon) - \epsilon \leq F_n(x) \leq F(x+\epsilon) + \epsilon
\end{align*}
である。左側の不等式に注目して両辺$\liminf$を取ることを考えると、$\epsilon \to 0$とできることより、$F(x_{-}) \leq \liminf_n F_n(x)$を得る。また同様に右の不等式に注目すると$\limsup_n F_n(x) \leq F(x_{+})$を得る。$\liminf_n F_n(x) \leq \limsup_n F_n(x)$は定義より常に成立するので、
\begin{align*}
	F(x_{-}) \leq \liminf_n F_n(x) \leq \limsup_n F_n(x) \leq F(x_{+})
\end{align*}
が成立する。従ってもし$x$が$F$の連続点であれば、$F(x) = F(x_{-}) = F(x_{+})$であるので、はさみうちの原理より、$F(x) = \lim_n F_n(x)$である。仮に$x$が$F$の不連続点であるならば、不連続点は高々可算個であることより$x$に上から近く連続点の列$\left\{ x_m \right\}$が存在して、分布関数が右連続であることより、
\begin{align*}
	F(x) = \lim_m F(x_m) = \lim_m \lim_n F_n(x_m) = \lim_n F_n(x)
\end{align*}
である。ただし極限の交換は任意の$n$について$F_n$も分布関数であることより右極限が存在することと、連続点であれば$F$に各点で収束することから保証される(この議論は後の問題でも必要となるが、同様の議論であるため以下では省略する)。これより不連続点においても収束するので$F_n(x) \to F(x), \forall x$であり、分布収束することが示された。

\noindent $(\Leftarrow part)$

仮定より、
\begin{align*}
	\forall \epsilon > 0, \exists N\ \text{s.t.}\ n\geq N \Rightarrow\ \left| F_n(x) - F(x) \right| < \epsilon\ \forall x
\end{align*}
である。分布関数は非減少なので、$F_n(x-\epsilon) \leq F_n(x), F_n(x) \leq F_n(x+\epsilon)$であることから、
\begin{align*}
	\forall \epsilon > 0, \exists N\ \text{s.t.}\ n\geq N \Rightarrow\ F_n(x-\epsilon) - \epsilon < F(x) < F_n(x+\epsilon) + \epsilon\ \forall x
\end{align*}
である。$\rho(F_n, F)$の定義よりこれは$\rho(F_n, F) \to 0$を意味する。


\section{3.2.9}
「分布関数$F$が連続であるならば$F$は一様連続である(主張$1$)」を所与とする。この時、
\begin{align*}
	\forall \epsilon > 0, \exists \delta > 0, \text{s.t.}\ |x-y| \leq \delta\ \Rightarrow\ \left| F(x) - F(y) \right| \leq \epsilon
\end{align*}
である。$F_n \Rightarrow F$であるのでEx 3.2.6.より$\rho\left( F_n, F \right) \to 0$である。すなわち一般性を失うことなく、$\epsilon \geq \delta \geq 0$に対して、
\begin{align}
	F\left( x-\delta \right) -\delta \leq F_n(x) \leq F(x + \delta) + \delta
\end{align}
である。ここで、$|x - (x-\delta)| \leq \delta$であるので一様連続性から$\left| F(x) - F(x-\delta) \right| \leq \epsilon$である。すなわち、$F(x) - \epsilon \leq F(x - \delta) \leq F(x) + \epsilon$である。同様の議論を$x, x+\delta$に対しても適用することで先の$(1)$と合わせて以下を得る。
\begin{align*}
	F(x) - \delta - \epsilon \leq F_n(x) \leq F(x) + \delta + \epsilon
\end{align*}
$\epsilon \geq \delta$であるので、十分大きな$n$において$F(x) - 2\epsilon \leq F_n(x) \leq F(x)+ 2\epsilon$が任意の$\epsilon > 0$に対して任意の$x$で成立することがわかった。すなわち、
\begin{align*}
	\sup_{x\in \mathbb{R}} \left| F_n(x) - F(x) \right| \leq 2\epsilon
\end{align*}
が十分大きな$n$について成立するので題意は示された。

\subsection{主張$1$の証明}
任意に$m \in \mathbb{N}$を取る。これに対して$(0,1)$を$\left( \frac{j-1}{m}, \frac{j}{m} \right), j = 1,\cdots, m$で$m$個に分割することを考える。$F$が連続で非減少関数であるので、開集合の連続関数による逆像は開集合となることから、$F^{-1}\left( \left( \frac{j-1}{m}, \frac{j}{m} \right) \right)$は各$j$について開区間である。この$m$個の区間のルベーグ測度で最小のものを$\delta$とかく。この時、
\begin{align*}
	|x-y| < \delta\ \Rightarrow\ \left| F(x) - F(y) \right| < \frac{2}{m}
\end{align*}
が成立する。なぜなら$x,y$を上のようにとる時、$[x, y]$に最小区間以外の区間が丸々入ることはあり得ないからである。従って、任意の$\epsilon > 0$に対して、$\frac{2}{m} < \epsilon\ \Rightarrow\ \frac{2}{\epsilon} < m$を満たすように$m$をとりその時の最小区間の長さを取ってくれば良いので一様連続性を満たすことがわかる。

\section{3.2.11}
$X_n\ \xrightarrow{w}\ X_{\infty}\ \Rightarrow\ P\left( X_n = m \right) \to P\left( X_{\infty} = m \right)\ \forall m$を示す。$X_n$の分布関数を$F_n$でかくと、$a \in (m-1, m), b \in (m, m+1)$として$P\left( X_n = m \right) = F_n(b) - F_n(a)$である。$X_{\infty}$の分布関数を$F$とかくとすると、仮定より以下を得る。
\begin{align*}
	P\left( X_n = m \right) = F_n(b) - F_n(a) \to F(b) - F(a)
\end{align*}
ここで$b\downarrow m, a\uparrow m$とすることを考えると、
\begin{align*}
	F(b) - F(a) \to F(m_{+}) - F(m_{-}) = P\left( X_{\infty} = m\right)
\end{align*}
従って、$P\left( X_n = m \right) \to P\left( X_{\infty} = m \right)\ \forall m$を得る。

次に$X_n\ \xrightarrow{w}\ X_{\infty}\ \Leftarrow\ P\left( X_n = m \right) \to P\left( X_{\infty} = m \right)\ \forall m$を示す。$G$を$\mathbb{R}$の開集合とする。この時、
\begin{align*}
	\liminf_{n\to \infty} P\left( X_n \in G \right) &= \liminf_{n\to \infty} \sum_{m \in G \cap \mathbb{Z}} P\left( X_n = m \right)\\
	&= \liminf_{n\to \infty} E\left[ 1\left( X_n \in G\cap \mathbb{Z} \right) \right]\\
	&\geq E\left[ \liminf_{n\to \infty} 1\left( X_n \in G\cap \mathbb{Z} \right) \right]\\
	&= \sum_{m \in G \cap \mathbb{Z}} \liminf_{n\to \infty} P\left( X_n = m \right)\\
	&= \sum_{m \in G \cap \mathbb{Z}} P\left( X_{\infty} = m \right)\\
	&= P\left( X_{\infty} \in G\cap \mathbb{Z} \right)
\end{align*}
を得る。ただし、不等号はFatouの補題より得られ、4番目の等号は仮定より得られる。従って、Th 3.2.11(2)より確かに$X_n \xrightarrow{w} X_{\infty}$である。

\section{3.2.12}
まず確率収束するなら分布収束することを示す。まず、
\begin{align*}
	X_n > x\ \text{and}\ |X - X_n| \leq \epsilon\ \Rightarrow\ x-\epsilon < X
\end{align*}
が成立する。これの対偶より$x - \epsilon \geq X\ \Rightarrow\ X_n \leq X\ \text{or}\ |X- X_n| > \epsilon$が成立する。すなわち、
\begin{align*}
	P\left( X \leq x-\epsilon \right) \leq P\left( X_n \leq x \right) + P\left(  |X- X_n| > \epsilon \right)
\end{align*}
である。これより$F(x-\epsilon) - P\left(  |X- X_n| > \epsilon \right) \leq F_n(x)$であり、$X_n \xrightarrow{p}X$であることより左辺第二項は$0$に収束することから、$F(x-\epsilon) \leq \liminf_{n} F_n(x)$である。同じように以下が成立することを使う。
\begin{align*}
	X > x+ \epsilon\ \text{and}\ |X-X_n| \leq \epsilon\ \Rightarrow\ x < X_n
\end{align*}
この対偶は$X_n\leq x\ \Rightarrow\ X\leq x + \epsilon\ \text{or}\ |X-X_n| > \epsilon$であるので、以下を得る。
\begin{align*}
	P\left(X_n\leq x \right) \leq P\left( X\leq x + \epsilon \right) + P\left( |X-X_n| > \epsilon \right)
\end{align*}
仮定より右辺第2項は$0$に収束するので$\limsup_n F_n(x) \leq F(x+\epsilon)$を得る。以上を合わせると、任意の$\epsilon > 0$に対して、
\begin{align*}
	F(x-\epsilon) \leq \liminf_{n} F_n(x) \leq \limsup_n F_n(x) \leq F(x+\epsilon)
\end{align*}
が成立している。$x$が$F$の連続点であるならば、$\epsilon \to 0$とすることを考えて$F(x) = \lim_n F_n(x)$である。また、$x$が$F$の連続点でない場合に関しても、不連続点は高々可算個なので$x$に上から近く連続点の列$\left\{ x_m \right\}$を取ることができる。これに対して、$F, F_n$が右連続であることと、連続点においては収束が示されていることより、
\begin{align*}
	F(x) = \lim_m F(x_m) = \lim_m \lim_n F_n(x_m) = \lim_n F_n(x)
\end{align*}
が成立する。従って各$x$について$\lim_n F_n(x) = F(x)$となっていることが示されたので分布収束することが確かめられた。

次に、定数に分布収束するのであれば上記の逆も成立することを確かめる。定数に分布収束するということは、ある定数$c$について、
\begin{align*}
	F_n(x) \to 1\left( c \leq x \right)\ \text{as}\ n\to \infty
\end{align*}
を意味する。この時、任意の$\epsilon > 0$について
\begin{align*}
	P\left( |X_n - c| > \epsilon \right) = 1 - P\left( |X_n - c| \leq \epsilon \right) = 1- \left( F_n(c+\epsilon) - F_n(c-\epsilon) \right)
\end{align*}
であり、両辺$n\to \infty$での極限を考えると、
\begin{align*}
	\lim_{n\to \infty} P\left( |X_n - c| > \epsilon \right) = 1-\lim_{n\to \infty} F_n(c+\epsilon) + \lim_{n\to \infty} F_n(c-\epsilon) = 1-1+0 = 0
\end{align*}
であるので、確かに$X_n \xrightarrow{p} c$であることが確認された。

\section{3.2.13}
任意の$z \in \mathbb{R}$と任意の$\epsilon > 0$に対して成立する以下二つの関係を使う。
\begin{align*}
	\begin{cases}
	X_n > z-c\ \text{and}\ |Y_n - c| \leq \epsilon\ \Rightarrow\ z-\epsilon < X_n + Y_n\\
	X_n + Y_n > z\ \text{and}\ |Y_n - c| \leq \epsilon\ \Rightarrow\ z-c-\epsilon < X_n
	\end{cases}
\end{align*}
これらの対偶より以下が成立する。
\begin{align*}
	\begin{cases}
	P\left( z-\epsilon \geq X_n + Y_n \right) \leq P\left( X_n \leq z-c \right) + P\left( |Y_n - c| > \epsilon \right)\\[8pt]
	P\left( z-c-\epsilon \geq X_n \right) \leq P\left( X_n + Y_n \leq z \right) + P\left( |Y_n - c| > \epsilon \right)
	\end{cases}
\end{align*}
ここで$X_n + Y_n = Z_n$と書き、その分布関数を$F_{Z_n}$と表記し、$X_n$の分布関数を$F_{X_n}$と表記することにすると、以下を得る。
\begin{align*}
	\begin{cases}
	F_{Z_n}\left( z-\epsilon \right) \leq F_{X_n}\left( z-c \right) + P\left( |Y_n - c| > \epsilon \right)\\[8pt]
	F_{X_n}(z-c-\epsilon) - P\left( |Y_n - c| > \epsilon \right) \leq F_{Z_n}(z)
	\end{cases}
\end{align*}
1つ目の式に対しては両辺$\limsup$を、2つ目の式に対しては両辺$\liminf$を取ることを考える。仮定よりどちらの式においても$P\left( |Y_n - c| > \epsilon \right) \to 0$であり、$X_n \xrightarrow{w} X$である。$X$の分布関数を$F_X$で書くことにすると、1つ目の式において$z$を$z+\epsilon$に置き換えると、
\begin{align*}
	\begin{cases}
	\limsup_n F_{Z_n}(z) \leq F_X(z-c+\epsilon)\\
	F_X(z-c-\epsilon) \leq \liminf_n F_{Z_n}(z)
	\end{cases}
\end{align*}
従って以下が成立する。
\begin{align*}
	F_X(z-c-\epsilon) \leq \liminf_n F_{Z_n}(z) \leq \limsup_n F_{Z_n}(z) \leq F_X(z-c+\epsilon)
\end{align*}
これより、$z-c$が$F_X$の連続点であるならば$\epsilon \to 0$とすることで、
\begin{align*}
	F_X(z-c) = \lim_n F_{Z_n}(z)
\end{align*}
である。また、$z-c$が不連続点であっても、不連続点は高々可算個であることより、$z-c$に上から近く連続点の列$\left\{ y_m \right\}$が存在して、分布関数が右連続であることより、
\begin{align*}
	F_X(z-c) = \lim_m F_X(y_m) = \lim_m \lim_n F_{Z_n}(y_m + c) = \lim_n F_{Z_n}(z)
\end{align*}
であるので不連続点においても収束している。従って$X_n + Y_n \xrightarrow{w} X + c$である。

\section{3.2.14}
任意の$z \in \mathbb{R}$と任意の$\epsilon > 0$に対して成立する以下の関係を使う。
\begin{align*}
	\begin{cases}
	X_n > \frac{z}{c}\ \text{and}\ |Y_n - c| \leq \epsilon\ \Rightarrow\ X_n Y_n > z\frac{c-\epsilon}{c}\\[8pt]
	X_n Y_n > cz\ \text{and}\ |Y_n - c| \leq \epsilon\ \Rightarrow\ X_n > \frac{c}{c+\epsilon}z
	\end{cases}
\end{align*}
これの対偶が真であることより以下が成立する。
\begin{align*}
	\begin{cases}
	P\left( X_n Y_n \leq z\frac{c-\epsilon}{c} \right) \leq P\left( X_n \leq \frac{z}{c} \right) + P\left(  |Y_n - c| > \epsilon \right)\\[8pt]
	P\left( X_n \leq \frac{c}{c+\epsilon}z \right) - P\left( |Y_n - c| > \epsilon \right)  \leq P\left( X_n Y_n \leq cz \right)
	\end{cases}
\end{align*}
1つ目の式において両辺$\limsup$をとり、2つ目の式において$\liminf$を取ることを考える。仮定から$P\left( |Y_n - c| > \epsilon \right) \to 0$であることと、$X_n$の分布関数$F_{X_n}$が$X$の分布関数$F_X$に各点収束することより、
\begin{align*}
	\begin{cases}
	\limsup_n P\left( X_n Y_n \leq z\frac{c-\epsilon}{c} \right) \leq F_X\left( \frac{z}{c} \right)\\[8pt]
	F_X\left( \frac{c}{c+\epsilon}z \right) \leq \liminf_n P\left( X_n Y_n \leq cz \right)
	\end{cases}
\end{align*}
ここで、上の式において$z$を$\frac{c^2}{c-\epsilon}z$で置き換えても成立するので以下を得る。
\begin{align*}
	F_X\left( \frac{c}{c+\epsilon} z \right) \leq \liminf_n P\left( X_n Y_n \leq cz \right) \leq \limsup_n P\left( X_n Y_n \leq cz \right) \leq F_X\left( \frac{c}{c-\epsilon} z \right)
\end{align*}
従って$z$が$F_X$の連続点であるなら$\epsilon \to 0$で$F_X(z) = \lim_n P\left( X_n Y_n \leq cz \right)$である。また、$z$が$F_X$の不連続点であっても、不連続点は高々可算個であることより、$z$に上から近く連続点の列$\left\{ y_m \right\}$が存在して、分布関数が右連続であることより、
\begin{align*}
	F_X(z) = \lim_m F_X(y_m) = \lim_m \lim_n P\left( X_n Y_n \leq cy_m \right) = \lim_n P\left( X_n Y_n \leq cz \right)
\end{align*}
であるので、不連続点においても$F_X(z)$に収束する。従って$X_n Y_n \xrightarrow{w} cX$であることが示された。

\end{document}
























