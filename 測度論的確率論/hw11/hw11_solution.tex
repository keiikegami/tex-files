\documentclass{article}
\usepackage[margin = .7in]{geometry}
\usepackage[dvipdfmx]{graphicx}
\usepackage{listings}
\usepackage{amsmath}
\usepackage{amssymb}
\usepackage{amsfonts}
\usepackage{bm}
\usepackage{mathrsfs}
\lstset{%
  language={python},
  basicstyle={\small},%
  identifierstyle={\small},%
  commentstyle={\small\itshape},%
  keywordstyle={\small\bfseries},%
  ndkeywordstyle={\small},%
  stringstyle={\small\ttfamily},
  frame={tb},
  breaklines=true,
  columns=[l]{fullflexible},%
  numbers=left,%
  xrightmargin=0zw,%
  xleftmargin=3zw,%
  numberstyle={\scriptsize},%
  stepnumber=1,
  numbersep=1zw,%
  lineskip=-0.5ex%
}

\begin{document}
\title{測度論的確率論 2018 S1S2 \\ 
Homework 11}
\author{経済学研究科現代経済コース修士1年 / 池上 慧 (29186009) / sybaster.x@gmail.com}
\maketitle


\section{Ex11.6}
確率変数列$\left\{ X_n : n \in \mathbb{N}\right\}$に対して$\left\{ \mathcal{L}(X_n) : n \in \mathbb{N} \right\}$が一様にタイトであることを仮定する。この時、任意に$\epsilon > 0$と$c_n \to 0$を取ると、
\begin{align*}
	P\left( \left| c_n X_n \right| \geq \epsilon \right) = P\left( \left| X_n \right| \geq \frac{\epsilon}{|c_n|} \right)
\end{align*}
であるので、任意の$\delta > 0$に対してある$N \geq 0$が存在して以下を満たす。
\begin{align*}
	n\geq N\ \Rightarrow\ P\left( \left| c_n X_n \right| \geq \epsilon \right) \leq P\left( \left| X_n \right| \geq \frac{\epsilon}{\delta} \right) = 1-\mathcal{L}(X_n)\left( \frac{\epsilon}{\delta} \right) + \mathcal{L}(X_n)\left(- \frac{\epsilon}{\delta} \right)
\end{align*}
つまり、
\begin{align*}
	\forall \epsilon,M > 0, \exists N\ \text{s.t.}\ n\geq N\ \Rightarrow\ P\left( \left| c_n X_n \right| \geq \epsilon \right) \leq 1-\mathcal{L}(X_n)\left( M \right) + \mathcal{L}(X_n)\left(-M \right)
\end{align*}
である。一葉タイト性の仮定より、任意の$n$について上式の右辺を任意に小さい$\eta$で上から抑えることのできる$M$が必ず存在するので、
\begin{align*}
	\forall \epsilon, \eta > 0,\ \exists N\ \text{s.t.}\ n\geq N\ \Rightarrow\ P\left( \left| c_n X_n \right| \geq \epsilon \right) \leq \eta
\end{align*}
が成立し、これは確率収束の定義から$c_nX_n\xrightarrow{p}0$を意味する。

\noindent
次に逆向きを示す。すなわち任意の$c_n \to 0$に対して$c_n X_n \xrightarrow{p} 0$が成立することを仮定する。これは以下を意味する。
\begin{align*}
	\forall \epsilon , \delta > 0,\ \exists N\ \text{s.t.}\ n\geq N\ \Rightarrow\ 1-\mathcal{L}(X_n)\left( \frac{\epsilon}{\left| c_n \right|} \right) + \mathcal{L}(X_n)\left( -\frac{\epsilon}{\left| c_n \right|} \right) \leq \delta
\end{align*}

\section{Ex12.2}

\section{Ex12.3}

\section{Ex12.4}

\section{Ex12.6}

\end{document}





















