\documentclass{article}
\usepackage[margin = .7in]{geometry}
\usepackage[dvipdfmx]{graphicx}
\usepackage{listings}
\usepackage{amsmath}
\usepackage{amssymb}
\usepackage{amsfonts}
\usepackage{bm}
\usepackage{mathrsfs}
\lstset{%
  language={python},
  basicstyle={\small},%
  identifierstyle={\small},%
  commentstyle={\small\itshape},%
  keywordstyle={\small\bfseries},%
  ndkeywordstyle={\small},%
  stringstyle={\small\ttfamily},
  frame={tb},
  breaklines=true,
  columns=[l]{fullflexible},%
  numbers=left,%
  xrightmargin=0zw,%
  xleftmargin=3zw,%
  numberstyle={\scriptsize},%
  stepnumber=1,
  numbersep=1zw,%
  lineskip=-0.5ex%
}

\begin{document}
\title{測度論的確率論 2018 S1S2 \\ 
Homework 7}
\author{経済学研究科現代経済コース修士1年 / 池上 慧 (29186009) / sybaster.x@gmail.com}
\maketitle

\section{Theorem 7.2}
まず$P(\epsilon_i = 0) = P(\epsilon_i = 1) = \frac{1}{2}$を示す。$\epsilon_i \in \left\{ 0,1\right\}$なので、$1$となる確率のみ求めれば良い。$i = 1$の時は、
\begin{align*}
	P(\epsilon_1 = 1) = P\left(\omega \mid \omega \in \left[\frac{1}{2}, 1\right)\right) = \frac{1}{2}
\end{align*}
より確かに成り立つ。$i > 1$の時は、
\begin{align*}
	P(\epsilon_i = 1) = 2^{i-1} \times \frac{1}{2^i} = \frac{1}{2}
\end{align*}
ここで$2^{i-1}$は$\left\{ \epsilon_j \right\}_1^{i-1}$が取りうる場合の数であり、$\frac{1}{2^i}$は実現した$\left\{ \epsilon_j \right\}_1^{i-1}$に対して$\left\{ \epsilon_j \right\}_i^{\infty}$を使って表現できる$\omega$の集合の測度、すなわち$P\left(\omega \mid \omega \in \left[0, \frac{1}{2^i}\right) \right)$である。よって確かに任意の$i$について$P(\epsilon_i = 0) = P(\epsilon_i = 1) = \frac{1}{2}$が確かめられた。

次に独立性を示す。$D_i \in \left\{ 0,1\right\}$として、任意の$\left\{ i_1, i_2, \cdots, i_m \right\} \subset \mathbb{N}$に対して以下が成立することを示す。
\begin{align*}
	P\left( \epsilon_{i_1} = D_1, \cdots, \epsilon_{i_m} = D_m \right) = P\left( \epsilon_{i_1} = D_1\right) \times \cdots \times P\left( \epsilon_{i_m} = D_,\right)
\end{align*}
先の結果より、右辺は$\frac{1}{2^m}$なので、左辺もこの値を取ることを示せばよい。先と同じように考えて以下をえる。
\begin{align*}
	P\left( \epsilon_{i_1} = D_1, \cdots, \epsilon_{i_m} = D_m \right) = 2^{i_m - m} \times \frac{1}{2^{i_m}} = \frac{1}{2^m}
\end{align*}
ここで$2^{i_m - m}$は少数第$i_m$位までに自由に動かせる桁を組み合わせた時の場合の数であり、$\frac{1}{2^{i_m}}$は各組み合わせに対して動かせる幅の測度である。よって題意は示された。


\section{Ex 7.1}
レクチャーノートより$\left\{ X_n \mid n \in \mathbb{N}\right\}$が独立であることは以下と同値である。
\begin{align*}
	\forall \left\{ i_1, i_2, \cdots, i_m \right\} \subset \mathbb{N},\ L\left( X_{i_1}, \cdots, X_{i_m} \right) = L\left( X_{i_1} \right)\times \cdots \times L\left( X_{i_m} \right)
\end{align*}
従って$\left\{ X_n \mid n \in \mathbb{N}\right\}$が独立の時、任意の$n \in \mathbb{N}$に対して、
\begin{align*}
	L\left( X_1, \cdots, X_n \right) \times L(X_{n+1}) &= L\left( X_{1} \right)\times \cdots \times L\left( X_{n} \right) \times L\left( X_{n+1} \right)\\
	&= L\left( X_1, \cdots, X_n, X_{n+1} \right)\\ &= L\left( \left(X_1, \cdots, X_n\right), X_{n+1} \right)
\end{align*}
なので確かに$\left(X_1, \cdots, X_n\right)$と$X_{n+1}$は独立である。また逆に、任意の$n$について$\left(X_1, \cdots, X_n\right)$と$X_{n+1}$が独立の時、
\begin{align*}
	L\left( \left(X_1, \cdots, X_n\right), X_{n+1} \right) &= L\left( X_1, \cdots, X_n \right) \times L(X_{n+1})\\
	&= L\left( \left(X_1, \cdots, X_{n-1}\right), X_n \right) \times L(X_{n+1})\\
	&= L\left( \left(X_1, \cdots, X_{n-1}\right)\right) \times L(X_{n}) times L(X_{n+1})\\
	& \vdots\\
	&= L\left( X_1 \right)\times \cdots \times L\left( X_{n+1} \right)
\end{align*}
が成立する。任意に$\left\{ i_1, i_2, \cdots, i_m \right\} \subset \mathbb{N}$をとった時、上記の議論より$\left( X_1, \cdots, X_{i_m} \right)$が独立なので、そのサブセットによって構成される$\left( X_{i_1}, \cdots, X_{i_m} \right)$も独立となる。

\section{Ex 7.3}
$X$と同じ確率変数をもう一つ用意し$Y$とする。この時分布を$\mu(x),\mu(y)$と書いて以下の積分を考える。
\begin{align*}
	\int_{\mathbb{R}}\int_{\mathbb{R}} \left( f(x) - f(y) \right)  \left( g(x) - g(y)\right) \mathrm{d}\mu(x) \mathrm{d}\mu(y)
\end{align*}
$f,g$はどちらも非減少関数であるので、$x \geq y$の時は$f(x) - f(y), g(x) - g(y) \geq 0$であり、$x < y$の時は$f(x) - f(y), g(x) - g(y) \leq 0$である。これより被積分関数は常に非負であることがわかる。従って上記の積分は非負である。$x,y$は同一視できるので、積分の線型性より以下を得る。
\begin{align*}
	\int_{\mathbb{R}}\int_{\mathbb{R}} \left( f(x) - f(y) \right)  \left( g(x) - g(y)\right) \mathrm{d}\mu(x) \mathrm{d}\mu(y) = 2 E \left[ f(x)g(x) \right] - 2E\left[ f(x) \right] E\left[ g(y) \right]
\end{align*}
これより、
\begin{align*}
	2 E \left[ f(x)g(x) \right] - 2E\left[ f(x) \right] E\left[ g(x) \right] \geq 0\ \Leftrightarrow\ E \left[ f(x)g(x) \right] \geq E\left[ f(x) \right] E\left[ g(x) \right]
\end{align*}

また、$g$が非増加関数の時、$-g$が非減少関数であり、上記が適用できる。
\begin{align*}
	E \left[ -f(x)g(x) \right] \geq E\left[ f(x) \right] E\left[ -g(x) \right] \ \Leftrightarrow\ E \left[ f(x)g(x) \right] \leq E\left[ f(x) \right] E\left[ g(x) \right]
\end{align*}
よって題意は示された。

\section{Ex 7.8}
まず連続性を示す。$z, z^{\prime} \in \mathbb{R}$をとる。積分の線型性とEx 4.3より以下を得る。
\begin{align*}
	\left| f\star g(z) - f\star g(z^{\prime}) \right| \leq \int \left| f(z-y) - f(z^{\prime} -y) \right| \left| g(y) \right| \mathrm{d}y
\end{align*}
ここで関数に対して以下を定義する。ただし$\mu$は関数の定義域に対して定義された測度である。
\begin{align*}
	\| g \|_{\infty}  = \inf \left\{ C \mid \mu\left( \left\{ \left| g \right| > C \right\} \right) = 0 \right\}
\end{align*}
これを用いて、
\begin{align*}
	\int \left| f(z-y) - f(z^{\prime} -y) \right| \left| g(y) \right| \mathrm{d}y \leq \| g \|_{\infty} \int \left| f(z-y) - f(z^{\prime} -y) \right| \mathrm{d}y
\end{align*}
である。仮定より$\| g \|_{\infty} < \infty$であるので、$\left| z-z^{\prime} \right| < \delta$ならば$\int \left| f(z-y) - f(z^{\prime} -y) \right| \mathrm{d}y$が任意に小さい$\epsilon > 0$で抑えられるような、$\delta$が存在することを示せば定義より連続性が示されたことになる。以下ではこれを示す。

$z^{\prime} - y = w$で変数変換することで以下を得る。
\begin{align*}
	\int \left| f(z-y) - f(z^{\prime} -y) \right| \mathrm{d}y
\end{align*}



\section{Durrett 2.1.10}
積分の線型性より以下を得る。
\begin{align*}
	P(X+Y = n) &= E\left[ 1(x + y = n) \right] = E\left[ \sum_m 1(x = m) 1(y = n-m) \right]\\ &= \sum_m E\left[ 1(x = m) 1(y = n-m) \right] = \sum_m E\left[ 1(x = m) \right] E\left[1(y = n-m) \right] = \sum_m P(X = m) p(Y = n-m)
\end{align*}

\section{Durrett 2.1.11}
前問より以下が成立する。
\begin{align*}
	P(X+Y = n) &= \sum_m P(X = m) P(Y = n-m) = \sum_m \frac{e^{-\lambda} \lambda^m}{m!} \frac{e^{-\mu} \mu^{(n-m)}}{(n-m)!}\\[8pt]
	&=\frac{e^{-(\lambda + \mu) (\lambda + \mu)^n}}{n!} \sum_m \frac{n!}{(\lambda + \mu)^n} \frac{\lambda^m \mu^{(n-m)}}{m! (n-m)!}\\[8pt]
	&= \frac{e^{-(\lambda + \mu) (\lambda + \mu)^n}}{n!} \sum_m \binom nm \left( \frac{\lambda}{\lambda + \mu} \right)^m \left( 1 - \frac{\lambda}{\lambda + \mu} \right)^{(n-m)}\\[8pt]
	&= \frac{e^{-(\lambda + \mu) (\lambda + \mu)^n}}{n!} 
\end{align*}
よって題意は示された。

\end{document}

















