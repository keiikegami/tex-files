\documentclass{jsarticle}
\usepackage[margin = .7in]{geometry}
\usepackage[dvipdfmx]{graphicx}
\usepackage{listings}
\usepackage{amsmath}
\usepackage{bm}
\usepackage{ascmac}
\lstset{%
  language={python},
  basicstyle={\small},%
  identifierstyle={\small},%
  commentstyle={\small\itshape},%
  keywordstyle={\small\bfseries},%
  ndkeywordstyle={\small},%
  stringstyle={\small\ttfamily},
  frame={tb},
  breaklines=true,
  columns=[l]{fullflexible},%
  numbers=left,%
  xrightmargin=0zw,%
  xleftmargin=3zw,%
  numberstyle={\scriptsize},%
  stepnumber=1,
  numbersep=1zw,%
  lineskip=-0.5ex%
}

\begin{document}
\title{卒論テーマ候補 :ゆびすま}
\author{池上 慧}
\maketitle

\section{「ゆびすま」とは}
「ゆびすま」とは2人以上で行われるゲームである。ここでは2人で行われるケースを想定する。プレイヤーは毎回「攻め」と「守り」の役目を交互に行う。プレイヤーは毎回好きな本数の親指を上げる。「攻め」のプレイヤーは今回上がる親指の本数を予想し、掛け声とともにその予想した数をコールしながら、自分でも好きな本数だけ親指を上げる。「守り」のプレイヤーは掛け声の後に親指を好きな本数だけあげる。「攻め」がコールした数と実際にあげられた親指の総数が等しかったなら「攻め」の勝ちであり、そうでなければ「引き分け」である。引き分けたら役割を交代してどちらかが勝つまで続けるものとする。(本来であれば勝てば腕を一本減らすことができ、先に二回勝利した方の勝ちというルールであるが、ここでは最初の一回の勝利が起こるまでの、2本vs2本の状況のみを想定する。)

\section{研究の意図するところ}
ゆびすまで攻め手の戦略を決める肝は、「自分のコールと自分で揚げる親指の数の差」であることは明らかである。ここで「$a$とコールしながら$b$本の親指を上げる」という行動を$(a,b)$と表記することにすると、$\left\{ (0,0), (1,1), (2,2)\right\}$は自分のコールと自分のゆびの数が等しい組(set1)で、$\left\{ (1,0), (2,1)\right\}$を自分のコールが自分のゆびの数よりも1つ多い組(set2)、$\left\{ (2,0)\right\}$を自分のコールが自分のゆびの数よりも2つ多い組(set3)の3つに戦略を分類することができる。

これらはそれぞれ同じ組の中では勝利確率は等しい。以下で計算するように、相手の戦略を$\left\{ 0,1,2\right\}$、すなわちゆびを上げる本数だと捉えると、この全てをサポートに持つ混合戦略ナッシュ均衡がこのゲームの唯一のナッシュ均衡として存在し、その時以上のセットには全て等確率、すなわち$\frac{1}{3}$ずつの確率が振られることになる。この時、各セットに含まれる戦略の数は異なっていることに注意すると、個別の戦略に振られる確率はset3の中身、set2の中身、set1の中身の順番で大きくなるはずである。

しかし、この結果は直感に反するものである。$\left\{ (2,0)\right\}$を選ぶということは自分は1本も上げないで、相手が2本とも上げることを想定しているということであるが、これが実現する可能性は低そうに思えてしまう。実際、何も説明せずに友人とこのゲームを繰り返し行った時、いきなり$\left\{ (2,0)\right\}$が選ばれる回数は少なかった。

むしろ選ばれがちなのはset2、すなわち$\left\{ (1,0), (2,1)\right\}$の2つであった。本研究ではゆびすまにおいてset2が戦略として選ばれやすいことの理由として、「攻め手が知らず知らずのうちに守り手の左右の手を別々の意思決定主体が操作するものだと想定してゲームをプレイしている」という仮説を検証する。

以下で計算するように、守り手の腕を別々の主体と想定した場合の混合戦略ナッシュ均衡は現実的なものが3つ存在し、そのどれもでset2に高い確率が付与されている。逆にset1やset3には確率が付与されない均衡も存在し、それらの戦略を極端な手として嫌ってしまう傾向と合致していると言える。

\section{ゆびすまのナッシュ均衡}
計算は別紙。

$(q, r, s)$でcase1において守り手が$\left\{ 0,1,2\right\}$をそれぞれ取る確率を、$p$でcase2において守り手の左右の腕が取る混合戦略、すなわち親指を上げる確率を、$(x, y, z)$で攻め手が取る混合戦略、すなわちset1, set2, set3をとる確率を表すとする。この時、それぞれのケースでいかが混合戦略ナッシュ均衡となる。

\begin{itembox}[l]{case1 : 相手が統一された意思決定主体だと想定}
    \begin{align}
    	\begin{cases}
		(q, r, s) = (\frac{1}{3}, \frac{1}{3}, \frac{1}{3})\\
		(x, y, z) = (\frac{1}{3}, \frac{1}{3}, \frac{1}{3})
	\end{cases}
    \end{align}
\end{itembox}

\begin{itembox}[l]{case2 : 相手の左右の手を別々の意思決定主体だと想定}
\begin{align}
    	&\begin{cases}
		p = \frac{1}{3}\\
		(x, y, z) = (\frac{1}{3}, \frac{2}{3}, 0)
	\end{cases}\\[10pt]
	&\begin{cases}
		p = \frac{1}{2}\\
		(x, y, z) = (0, 1, 0)
	\end{cases}\\[10pt]
	&\begin{cases}
		p = \frac{2}{3}\\
		(x, y, z) = (0, \frac{2}{3}, \frac{1}{3})
	\end{cases}
\end{align}
\end{itembox}

\section{データへのフィットの比べ方}
ゆびすまを実際に対戦させてデータを集めることを想定する。そのデータに対して先の二つのモデルの説明度合いを比較してどちらのモデルが採択されるかを調べたい。単純に考えられるのは以下のように尤度を比較すること。

case1に対しては通常の意味で対数尤度が計算できる。case2に対しては以下のようにEMアルゴリズムを用いてその対数尤度を計算する。
$i = 1, \dots, n$を$n$回のプレイのindexとする。$Y_i$を攻め手が$i$回目のプレイで行った戦略を示す確率変数であるとする。すなわちそのサポートは$\left\{ (0,0), (1,1), (2,2), (1,0), (2,1), (2,0)\right\}$の6つである。$j = 1, 2, 3$で均衡のindexとして、それぞれ先の均衡(2), (3), (4)を示すとする。この時、$f_j(y_i)$は均衡$j$がプレイ$i$で実現している時に、攻め手が出す戦略の確率分布関数である。先に定義された戦略のset内では均等に確率が分配されるとすると、それぞれ以下のようになる。
\begin{align}
	f_1(y_i) = \begin{cases}
			\frac{1}{9} & y_i \in \text{set1}\\[8pt]
			\frac{1}{3} & y_i \in \text{set2}\\[8pt]
			0 & y_i \in \text{set3}
		\end{cases}\qquad
	f_2(y_i) = \begin{cases}
			0 & y_i \in \text{set1}\\[8pt]
			\frac{1}{2} & y_i \in \text{set2}\\[8pt]
			0 & y_i \in \text{set3}
		\end{cases}\qquad
	f_3(y_i) = \begin{cases}
			0 & y_i \in \text{set1}\\[8pt]
			\frac{1}{3} & y_i \in \text{set2}\\[8pt]
			\frac{1}{3} & y_i \in \text{set3}
		\end{cases}
\end{align}

ここで均衡$j$が選ばれる確率を$\psi_j$で書くとすると、$Y_i$は混合分布である$\sum_j \psi_j f_j(y_i)$に従うとできる。missing variableとして各プレイがどの均衡から生じたものなのかを示した確率変数$u_i$を考えると、完全なデータが与えられた時の尤度は、$I(\cdot)$を指示関数とすると以下のように書ける。
\begin{align}
	L(Y, u | \psi) = \Pi_i\ \Pi_j\ (\psi_j\cdot f_j(y_i))^{I(u_i = j)}
\end{align}

Q関数の定義は以下である。
\begin{align*}
	Q(\psi | \psi^{(0)}) = E_{\psi^{(0)}}\left[ {\rm log}\ L(Y, u | \psi)\ |\ Y \right]
\end{align*}

Q関数はパラメータについて最大化されるので、パラメータに関連する部分だけ残せば最大化のための一階条件を問題なく得られる。従って以下では$\psi$に関係のない項は省略してイコールで記す。ただし、$\psi^{0}$はEMアルゴリズムないで一回前のiterationで得られたパラメータの値であり、期待値はmissing variableについて取られているとする。
\begin{align}
	E_{\psi^{(0)}}\left[ {\rm log}\ L(Y, u | \psi)\ |\ Y \right] &= \sum_i \sum_j E_{\psi^{(0)}}\left[ I(u_i = j) {\rm log}\ \psi_j\ |\ Y \right]\nonumber \\
	&= \sum_j {\rm log}\ \psi_j\ \sum_i E_{\psi^{(0)}}\left[ I(u_i = j)\ |\ Y \right]
\end{align}

これをパラメータ$\psi_1, \psi_2, \psi_3$について最大化することを考える。一階条件より
\begin{align}
	\psi_j = \frac{1}{n} \sum_i E_{\psi^{(0)}}\left[ I(u_i = j)\ |\ Y \right]
\end{align}
である。

ここで、ベイズの公式より
\begin{align*}
	E_{\psi^{(0)}}\left[ I(u_i = j)\ |\ Y \right] = P_{\psi^{(0)}}( u_i = j\ |\ Y ) = \frac{p_{\psi^{(0)}} (y_i\ |\ u_i = j) \cdot p_{\psi^{(0)}} (u_i = j)}{p_{\psi^{(0)}} (y_i)} = \frac{f_j(y_i) \cdot \psi_j^{(0)}}{\sum_j \psi_j^{(0)} f_j(y_i)}
\end{align*}
を得るので、先の結果と合わせることでパラメータの更新式として以下を得る。
\begin{align}
	\forall j \quad \psi_j^{(1)} = \frac{1}{n} \sum_i \frac{f_j(y_i) \cdot \psi_j^{(0)}}{\sum_j \psi_j^{(0)} f_j(y_i)}
\end{align}

これによって収束させたパラメータについて先のQ関数の値を計算し、それを持ってcase2の対数尤度とする。case1の対数尤度とcase2のQ関数の値を比較してモデルのフィットを比べる。(これは理論的に正当か?モデルの検定にはどのようなものがあるのか?)


\end{document}




















