\documentclass{jsarticle}
\usepackage[margin = .7in]{geometry}
\usepackage[dvipdfmx]{graphicx}
\usepackage{listings}
\usepackage{amsmath}
\usepackage{amsfonts}
\usepackage{bm}
\lstset{%
  language={python},
  basicstyle={\small},%
  identifierstyle={\small},%
  commentstyle={\small\itshape},%
  keywordstyle={\small\bfseries},%
  ndkeywordstyle={\small},%
  stringstyle={\small\ttfamily},
  frame={tb},
  breaklines=true,
  columns=[l]{fullflexible},%
  numbers=left,%
  xrightmargin=0zw,%
  xleftmargin=3zw,%
  numberstyle={\scriptsize},%
  stepnumber=1,
  numbersep=1zw,%
  lineskip=-0.5ex%
}

\begin{document}
\title{卒論テーマ候補 :ゆびすま}
\author{池上 慧}
\maketitle

\section{「ゆびすま」とは}
「ゆびすま」とは2人以上で行われるゲームである。ここでは2人で行われるケースを想定する。プレイヤーは毎回「攻め」と「守り」の役目を交互に行う。プレイヤーは毎回好きな本数の親指を上げる。「攻め」のプレイヤーは今回上がる親指の本数を予想し、掛け声とともにその予想した数をコールしながら、自分でも好きな本数だけ親指を上げる。「守り」のプレイヤーは掛け声の後に親指を好きな本数だけあげる。「攻め」がコールした数と実際にあげられた親指の総数が等しかったなら「攻め」の勝ちであり、そうでなければ「引き分け」である。引き分けたら役割を交代してどちらかが勝つまで続けるものとする。(本来であれば勝てば腕を一本減らすことができ、先に二回勝利した方の勝ちというルールであるが、ここでは最初の一回の勝利が起こるまでの、2本vs2本の状況のみを想定する。)

\section{研究の意図するところ}
ゆびすまで攻め手の戦略を決める肝は、「自分のコールと自分で揚げる親指の数の差」であることは明らかである。ここで「$a$とコールしながら$b$本の親指を上げる」という行動を$(a,b)$と表記することにすると、$\left\{ (0,0), (1,1), (2,2)\right\}$は自分のコールと自分のゆびの数が等しい組(set1)で、$\left\{ (1,0), (2,1)\right\}$を自分のコールが自分のゆびの数よりも1つ多い組(set2)、$\left\{ (2,0)\right\}$を自分のコールが自分のゆびの数よりも2つ多い組(set3)の3つに戦略を分類することができる。

これらはそれぞれ同じ組の中では勝利確率は等しい。以下で計算するように、相手の戦略を$\left\{ 0,1,2\right\}$、すなわちゆびを上げる本数だと捉えると、この全てをサポートに持つ混合戦略ナッシュ均衡がこのゲームの唯一のナッシュ均衡として存在し、その時以上のセットには全て等確率、すなわち$\frac{1}{3}$ずつの確率が振られることになる。この時、各セットに含まれる戦略の数は異なっていることに注意すると、個別の戦略に振られる確率はset3の中身、set2の中身、set1の中身の順番で大きくなるはずである。

しかし、この結果は直感に反するものである。$\left\{ (2,0)\right\}$を選ぶということは自分は1本も上げないで、相手が2本とも上げることを想定しているということであるが、これが実現する可能性は低そうに思えてしまう。実際、何も説明せずに友人とこのゲームを繰り返し行った時、いきなり$\left\{ (2,0)\right\}$が選ばれる回数は少なかった。

むしろ選ばれがちなのはset2、すなわち$\left\{ (1,0), (2,1)\right\}$の2つであった。本研究ではゆびすまにおいてset2が戦略として選ばれやすいことの理由として、「攻め手が知らず知らずのうちに守り手の左右の手を別々の意思決定主体が操作するものだと想定してゲームをプレイしている」という仮説を検証する。

ゆびすまを

\section{ゆびすまのナッシュ均衡}

\section{データへのフィットの比べ方}
通常の意味での尤度の最大値vsEMアルゴリズムでの尤度の最大値をやる。後者の方が大きかったらケースの方がいいと言っていい?

\end{document}




















