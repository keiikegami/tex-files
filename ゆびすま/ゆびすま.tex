\documentclass{jsarticle}
\usepackage[margin = .7in]{geometry}
\usepackage[dvipdfmx]{graphicx}
\usepackage{listings}
\usepackage{amsmath}
\usepackage{amsfonts}
\usepackage{bm}
\lstset{%
  language={python},
  basicstyle={\small},%
  identifierstyle={\small},%
  commentstyle={\small\itshape},%
  keywordstyle={\small\bfseries},%
  ndkeywordstyle={\small},%
  stringstyle={\small\ttfamily},
  frame={tb},
  breaklines=true,
  columns=[l]{fullflexible},%
  numbers=left,%
  xrightmargin=0zw,%
  xleftmargin=3zw,%
  numberstyle={\scriptsize},%
  stepnumber=1,
  numbersep=1zw,%
  lineskip=-0.5ex%
}

\begin{document}
\title{卒論テーマ候補 :ゆびすま}
\author{池上 慧}
\maketitle

\section{「ゆびすま」とは}
「ゆびすま」とは2人以上で行われるゲームである。ここでは2人で行われるケースを想定する。プレイヤーは毎回「攻め」と「守り」の役目を交互に行う。プレイヤーは毎回好きな本数の親指を上げる。「攻め」のプレイヤーは今回上がる親指の本数を予想し、掛け声とともにその予想した数をコールしながら、自分でも好きな本数だけ親指を上げる。「守り」のプレイヤーは掛け声の後に親指を好きな本数だけあげる。「攻め」がコールした数と実際にあげられた親指の総数が等しかったなら「攻め」の勝ちであり、そうでなければ「引き分け」である。引き分けたら役割を交代してどちらかが勝つまで続けるものとする。(本来であれば勝てば腕を一本減らすことができ、先に二回勝利した方の勝ちというルールであるが、ここでは最初の一回の勝利が起こるまでの、2本vs2本の状況のみを想定する。)

\section{研究の意図するところ}

\section{ゆびすまのナッシュ均衡}

\section{データへのフィットの比べ方}
通常の意味での尤度の最大値vsEMアルゴリズムでの尤度の最大値をやる。後者の方が大きかったらケースの方がいいと言っていい?

\end{document}




















