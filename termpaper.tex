\documentclass{article}
\usepackage[margin = .7in]{geometry}
\usepackage{multirow,array}

\begin{document}
\title{ECON106/2214\ Games and Decisions\\ 2016 Term Paper}
\author{Kei Ikegami}
\maketitle

\section{Introduction}
\par
In this paper I analyze singles party in Japan by using Game theory. Most of those parties are held by men group and women group composed of same number of persons in order to make couples. Because one-to-one matching can be allowed in making couples, it gives rise to highly strategic situations among the participants, which include tactics between the same sex and the other sex.
\par
In the following paragraph, I show you the peculiar background of this game in Japan. After that I formalize this situation in game theoretic mode. And in the last paragraph I summarize the results and clarify the way of further studies.

\section{Background}
\par
In most cases, singles parties are carried out by 4 or 5 men and women. They meet up to make couples. The main function of such parties is to make the time to talk with the members of the other group. After 2 or 3 hours, some participants can get the partner and other ones can not. 
\par
In the typical type of singles party in japan, the participants talk over a square table, with all men sitting on one side and all women sitting on the other side. Then they inevitably have more time to talk with the member sitting in front of them than the other members. The aim of this meeting is, however, finding the partner. And all people think the more choice is better. Thus "Sekigae" is done there. "Sekigae" is the Japanese term which means changing seats. By doing "Sekigae", they get the equal time to talk with all the members of the other sex group.

\section{Analysis}
	
	\begin{table}[h]
	\begin{center}
                \setlength{\extrarowheight}{2pt}
                \begin{tabular}{*{16}{c|}}
                  \multicolumn{2}{c}{} & \multicolumn{1}{c}{} & \multicolumn{2}{c}{Women}\\\cline{3-6}
                  \multicolumn{1}{c}{} &  & $[A, A]$  & $[A, B]$ & $[B, A]$ & $[B.B]$\\\cline{2-6}
                  \multirow{4}*{Men}  & $(a,a)$ & $(x,y,x,y)$ & $(x,y,x,y)$ & $(y,x,x,y)$ & $(y,x,x,y)$\\\cline{2-6}
                  & $(a,b)$ & $(x,y,x,y)$ & $(x,x,x,x)$ & $(y,y,y,y)$ & $(y,x,y,x)$\\\cline{2-6}
                  & $(b,a)$ & $(x,y,y,x)$ & $(y,y,y,y)$ & $(x,x,x,x)$ & $(y,x,x,y)$\\\cline{2-6}
                  & $(b,b)$ & $(x,y,y,x)$ & $(y,x,y,x)$ & $(x,y,y,x)$ & $(y,x,y,x)$\\\cline{2-6}
                \end{tabular}
        \end{center}
        \caption{no difference vs no difference}
  	\end{table}
	
	\begin{table}[h]
		\begin{center}
                \setlength{\extrarowheight}{2pt}
                \begin{tabular}{*{16}{c|}}
                  \multicolumn{2}{c}{} & \multicolumn{1}{c}{} & \multicolumn{2}{c}{Women}\\\cline{3-6}
                  \multicolumn{1}{c}{} &  & $[H, H]$  & $[H, L]$ & $[H, L]$ & $[L.L]$\\\cline{2-6}
                  \multirow{4}*{Men}  & $(a,a)$ & $(x,y,x_h,y)$ & $(x,y,x_h,y)$ & $(y,x,x_l,y)$ & $(y,x,x_l,y)$\\\cline{2-6}
                  & $(a,b)$ & $(x,y,x_h,y)$ & $(x,x,x_h,x_l)$ & $(y,y,y,y)$ & $(y,x,y,x_l)$\\\cline{2-6}
                  & $(b,a)$ & $(x,y,y,x_h)$ & $(y,y,y,y)$ & $(x,x,x_l,x_h)$ & $(y,x,x_l,y)$\\\cline{2-6}
                  & $(b,b)$ & $(x,y,y,x_h)$ & $(y,x,y,x_l)$ & $(x,y,y,x_h)$ & $(y,x,y,x_l)$\\\cline{2-6}
                \end{tabular}
                \end{center}
                \caption{difference vs no difference}
          \end{table}

	\begin{table}[h]
		\begin{center}
		\begingroup
                \renewcommand{\arraystretch}{1.5}
                \setlength{\extrarowheight}{2pt}
                \begin{tabular}{*{4}{c|}}
                  \multicolumn{2}{c}{} & \multicolumn{2}{c}{b}\\\cline{3-4}
                  \multicolumn{1}{c}{} &  & $H$  & $L$ \\\cline{2-4}
                  \multirow{2}*{a}  & $H$ & $(\frac{x_h+y}{2}, \frac{x_h+y}{2})$ & $(\frac{x_h+y}{2}, \frac{x_l+y}{2})$ \\\cline{2-4}
                  & $L$ & $(\frac{x_l+y}{2}, \frac{x_h+y}{2})$ & $(\frac{x_l+y}{2}, \frac{x_l+y}{2})$ \\\cline{2-4}
                \end{tabular}
                \endgroup
                \end{center}
                \caption{women's payoff matrix with change seats}
  	\end{table}
	
	\begin{table}[h]
                \begin{center}
                \begingroup
                \renewcommand{\arraystretch}{1.8}
                \begin{tabular}{*{4}{c|}}
                  \multicolumn{2}{c}{} & \multicolumn{2}{c}{b}\\\cline{3-4}
                  \multicolumn{1}{c}{} &  & $H$  & $L$ \\\cline{2-4}
                  \multirow{2}*{a}  & $H$ & $((x_h - y)\theta + \frac{x_h+y}{2}, (y - x_h)\theta + \frac{x_h+y}{2})$ & $((x_h - y)\theta+\frac{x_h+y}{2}, (x_l - y)\theta + \frac{x_l+y}{2})$ \\\cline{2-4}
                  & $L$ & $((y-x_l)\theta + \frac{x_l+y}{2}, (y-x_h)\theta + \frac{x_h+y}{2})$ & $((y-x_l)\theta + \frac{x_l+y}{2}, (x_l - y)\theta + \frac{x_l+y}{2})$ \\\cline{2-4}
                \end{tabular}
                \endgroup
                \end{center}
                \caption{women's payoff matrix without change seats}
  	\end{table}
	
	\begin{table}[h]
	\begin{center}
                \setlength{\extrarowheight}{2pt}
                \begin{tabular}{*{16}{c|}}
                  \multicolumn{2}{c}{} & \multicolumn{1}{c}{} & \multicolumn{2}{c}{Women}\\\cline{3-6}
                  \multicolumn{1}{c}{} &  & $[H, H]$  & $[H, L]$ & $[H, L]$ & $[L.L]$\\\cline{2-6}
                  \multirow{4}*{Men}  & $(h,h)$ & $(x_h,y,x_h,y)$ & $(x_h,y,x_h,y)$ & $(y,x_h,x_l,y)$ & $(y,x_h,x_l,y)$\\\cline{2-6}
                  & $(h,l)$ & $(x_h,y,x_h,y)$ & $(x_h,x_l,x_h,x_l)$ & $(y,y,y,y)$ & $(y,x_l,y,x_l)$\\\cline{2-6}
                  & $(l,h)$ & $(x_l,y,y,x_h)$ & $(y,y,y,y)$ & $(x_l,x_h,x_l,x_h)$ & $(y,x_h,x_l,y)$\\\cline{2-6}
                  & $(l,l)$ & $(x_l,y,y,x_h)$ & $(y,x_l,y,x_l)$ & $(x_l,y,y,x_h)$ & $(y,x_l,y,x_l)$\\\cline{2-6}
                \end{tabular}
        \end{center}
        \caption{difference vs difference}
  	\end{table}




\section{Conclusion}

\end{document}



















