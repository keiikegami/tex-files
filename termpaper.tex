\documentclass{article}
\usepackage[margin = .7in]{geometry}
\usepackage{multirow,array}
\usepackage[dvipdfmx]{graphicx}
\usepackage{listings}
\usepackage{amsmath}
\usepackage{bm}

\begin{document}
\title{ECON106/2214\ Games and Decisions\\ 2016 Term Paper}
\author{Kei Ikegami}
\maketitle

\section{Introduction}
\par
In this paper I analyze singles party in Japan by using Game theory. Most of those parties are held by men group and women group composed of same number of persons in order to make couples. Because one-to-one matching can be allowed in making couples, it gives rise to highly strategic situations among the participants, which include tactics between the same sex and the other sex.
\par
In the following paragraph, I show you the peculiar background of this game in Japan. After that I formalize this situation in game theoretic mode. And in the last paragraph I summarize the results and clarify the way of further studies.

\section{Background}
\par
In most cases, singles parties are carried out by 4 or 5 men and women. They meet up to make couples. The main function of such parties is to make the time to talk with the members of the other group. After 2 or 3 hours, some participants can get the partner and other ones can not. 
\par
In the typical type of singles party in japan, the participants talk over a square table, with all men sitting on one side and all women sitting on the other side. Then they inevitably have more time to talk with the member sitting in front of them than the other members. The aim of this meeting is, however, finding the partner. And all people think the more choice is better. Thus "Sekigae" is done there. "Sekigae" is the Japanese term which means changing seats. By doing "Sekigae", they get the equal time to talk with all the members of the other sex group.
\par
It is sure, however, that "Sekigae" deprives the participant of the time to make bigger the probability of getting a couple with each one candidate, because he or she could make a better impression on the future partner if he or she had the more time to talk with the person.
\par
This trade off arises the strategic situation on the happy party. Then I analyze the condition which they should have "Sekigae" by using game theory. Furthermore I study the effect of quality gap in the same sex group. For simplification, in this paper, I consider only "2 men vs 2 women" situation, but the essence of the result will not change in more persons case. 
\par
The most important point of this paper is the definition of the success of this party. I define the success in this case by "all participants can get the partner" rather than "the expected number of members getting the partner". This is because almost all participants actually want their one-night stand rather than their companion for life through the party, and so the most essential purpose in the meeting is that all members have so good an experience that they are looking forward to the next party. According to this purpose, the success is "all participants can get the partner".
\section{Analysis}
	\subsection{No quality gap case}
	\par
	First I formalize both group have no quality gap in each group.
	\par
	Let A and B be two men, and a and b be two women. Then each man have two choices, i.e. a or b. And each woman also have two choices, i.e. A or B. In order to summarize the payoff, I write $[\alpha, \beta]$ as indicating "a chooses $\alpha$ and b chooses $\beta$ ", and $(\alpha, \beta)$ as indicating "A chooses $\alpha$ and B chooses $\beta$ ". The payoff matrix can be expressed as Table1 by using this notation, where $x$ is the payoff of getting the partner and $y$ is the payoff of no matching.
	\begin{table}[h]
	\begin{center}
                \setlength{\extrarowheight}{2pt}
                \begin{tabular}{*{16}{c|}}
                  \multicolumn{2}{c}{} & \multicolumn{1}{c}{} & \multicolumn{2}{c}{Women}\\\cline{3-6}
                  \multicolumn{1}{c}{} &  & $[A, A]$  & $[A, B]$ & $[B, A]$ & $[B.B]$\\\cline{2-6}
                  \multirow{4}*{Men}  & $(a,a)$ & $(x,y,x,y)$ & $(x,y,x,y)$ & $(y,x,x,y)$ & $(y,x,x,y)$\\\cline{2-6}
                  & $(a,b)$ & $(x,y,x,y)$ & $(x,x,x,x)$ & $(y,y,y,y)$ & $(y,x,y,x)$\\\cline{2-6}
                  & $(b,a)$ & $(x,y,y,x)$ & $(y,y,y,y)$ & $(x,x,x,x)$ & $(y,x,x,y)$\\\cline{2-6}
                  & $(b,b)$ & $(x,y,y,x)$ & $(y,x,y,x)$ & $(x,y,y,x)$ & $(y,x,y,x)$\\\cline{2-6}
                \end{tabular}
        \end{center}
        \caption{no difference vs no difference}
  	\end{table}
	\par
	Given the communication skills of members are same, with "Sekigae", all members have the same amount of information about the other sex members. Then, letting $P_\alpha(\beta)$ be the probability of $\alpha$ chooses $\beta$, $P_A(a)= P_B(a) = P_a(A) = P_b(A) = \frac{1}{2}$. Now I assume that the decision of each person is independent. In this case, the probability of achieving $\left( (a, b), [A, B] \right)$ or $\left( (b, a), [B, A] \right)$ is $2 \left(\frac{1}{2}\right)^4 = \frac{1}{8}$.
	\par
	Next I consider without "Sekigae" case. I can set the talk pair as "A - a" and "B - b" without loss of generality and let $\theta$ be the probability gain by longer time talking, i.e. $P_A(a)= P_B(b) = P_a(A) = P_b(B) = \frac{1}{2} + \theta$, where $-\frac{1}{2} \leq\theta \leq \frac{1}{2}$. Because the decision of each person is independent, $P((a,b)) = P([A, B]) = \left( \frac{1}{2} + \theta \right) \left( \frac{1}{2} + \theta\right)$ and $P((b, a)) = P([B, A]) =  \left( \frac{1}{2} - \theta \right) \left( \frac{1}{2} - \theta\right)$. This leads to that the probability of achieving $\left( (a, b), [A, B] \right)$ or $\left( (b, a), [B, A] \right)$ is $ \left( \frac{1}{2} + \theta \right)^4 + \left( \frac{1}{2} - \theta\right)^4 = 2\theta^4 + 3\theta^2 + \frac{1}{8}$.
	By the above discussion, the probability of the success is bigger when "Sekigae" is not carried out since $\theta^2$ and $\theta^4 \geq 0$. Furthermore this says that $\theta$ can be negative, or the talk can be used as the tool of giving their own negative impression. And note that I assume that $\theta$ is equal among all members.

	\subsection{One side quality gap case}
	I assume that the quality gap can be viewed by the other group. And the gap exists in the men group without loss of generality. Then H (high quality) and L (low quality) are the two men. By using the same notation of the previous setting, I can express the payoffs in the figure 2, where $x_h$ is the payoff of matching with the high quality man and $x_l$ is one of that with the low quality ($x_h > x_l$).
	
	\begin{table}[h]
		\begin{center}
                \setlength{\extrarowheight}{2pt}
                \begin{tabular}{*{16}{c|}}
                  \multicolumn{2}{c}{} & \multicolumn{1}{c}{} & \multicolumn{2}{c}{Women}\\\cline{3-6}
                  \multicolumn{1}{c}{} &  & $[H, H]$  & $[H, L]$ & $[H, L]$ & $[L.L]$\\\cline{2-6}
                  \multirow{4}*{Men}  & $(a,a)$ & $(x,y,x_h,y)$ & $(x,y,x_h,y)$ & $(y,x,x_l,y)$ & $(y,x,x_l,y)$\\\cline{2-6}
                  & $(a,b)$ & $(x,y,x_h,y)$ & $(x,x,x_h,x_l)$ & $(y,y,y,y)$ & $(y,x,y,x_l)$\\\cline{2-6}
                  & $(b,a)$ & $(x,y,y,x_h)$ & $(y,y,y,y)$ & $(x,x,x_l,x_h)$ & $(y,x,x_l,y)$\\\cline{2-6}
                  & $(b,b)$ & $(x,y,y,x_h)$ & $(y,x,y,x_l)$ & $(x,y,y,x_h)$ & $(y,x,y,x_l)$\\\cline{2-6}
                \end{tabular}
                \end{center}
                \caption{difference vs no difference}
          \end{table}
          
          \par
          With "Sekigae", by the same logic of the previous setting, men put the same probability on each choice, i.e. $P_H(a) = P_L(a) = \frac{1}{2}$. Then all rows in the figure 2 happen with a probability of quarter 1. In this situation the payoff matrix of the women side can be expressed as figure 3 due to the calculation of the expected value from figure 2.

	\begin{table}[h]
		\begin{center}
		\begingroup
                \renewcommand{\arraystretch}{1.5}
                \setlength{\extrarowheight}{2pt}
                \begin{tabular}{*{4}{c|}}
                  \multicolumn{2}{c}{} & \multicolumn{2}{c}{b}\\\cline{3-4}
                  \multicolumn{1}{c}{} &  & $H$  & $L$ \\\cline{2-4}
                  \multirow{2}*{a}  & $H$ & $(\frac{x_h+y}{2}, \frac{x_h+y}{2})$ & $(\frac{x_h+y}{2}, \frac{x_l+y}{2})$ \\\cline{2-4}
                  & $L$ & $(\frac{x_l+y}{2}, \frac{x_h+y}{2})$ & $(\frac{x_l+y}{2}, \frac{x_l+y}{2})$ \\\cline{2-4}
                \end{tabular}
                \endgroup
                \end{center}
                \caption{women's payoff matrix with Sekigae}
  	\end{table}
	
	\par
	Since $x_h > x_l$, H is a strict dominance strategy for both a and b. This means that $[H, H]$ is realized in figure 2. Then two matching can never be achieved in this case.
	\par
	Next I consider no "Sekigae" case. Let the talk pair be "H - a" and "L - b" without loss of generality and $\theta$ be the probability gain by the longer talks ($-\frac{1}{2} \leq \theta \leq \frac{1}{2}$). Given that $P_H(a) = P_L(b) = \frac{1}{2} + \theta$, women side have the payoff matrix expressed in figure 4.
		
	\begin{table}[h]
                \begin{center}
                \begingroup
                \renewcommand{\arraystretch}{1.8}
                \begin{tabular}{*{4}{c|}}
                  \multicolumn{2}{c}{} & \multicolumn{2}{c}{b}\\\cline{3-4}
                  \multicolumn{1}{c}{} &  & $H$  & $L$ \\\cline{2-4}
                  \multirow{2}*{a}  & $H$ & $((x_h - y)\theta + \frac{x_h+y}{2}, (y - x_h)\theta + \frac{x_h+y}{2})$ & $((x_h - y)\theta+\frac{x_h+y}{2}, (x_l - y)\theta + \frac{x_l+y}{2})$ \\\cline{2-4}
                  & $L$ & $((y-x_l)\theta + \frac{x_l+y}{2}, (y-x_h)\theta + \frac{x_h+y}{2})$ & $((y-x_l)\theta + \frac{x_l+y}{2}, (x_l - y)\theta + \frac{x_l+y}{2})$ \\\cline{2-4}
                \end{tabular}
                \endgroup
                \end{center}
                \caption{women's payoff matrix without Sekigae}
  	\end{table}
	
	\par
	For example, I calculate the payoffs of the case in which both a and b chooses H.
	\begin{align*}
	\text{a's payoff} = (\frac{1}{2} + \theta)(\frac{1}{2} - \theta)x_h + (\frac{1}{2} + \theta)(\frac{1}{2} + \theta)x_h + (\frac{1}{2} - \theta)(\frac{1}{2} - \theta)y + (\frac{1}{2} - \theta)(\frac{1}{2} + \theta)y = (x_h - y)\theta + \frac{h_h + y}{2} \\[10pt]
	\text{b's payoff} = (\frac{1}{2} + \theta)(\frac{1}{2} - \theta)y + (\frac{1}{2} + \theta)(\frac{1}{2} + \theta)y + (\frac{1}{2} - \theta)(\frac{1}{2} - \theta)x_h + (\frac{1}{2} - \theta)(\frac{1}{2} + \theta)x_h = (y - x_h)\theta + \frac{h_h + y}{2}
	\end{align*}
	the rest of all components in figure 4 can be calculated by this logic.
	\par
	For simplicity let $\theta > 0$. And I assume that $x_h > x_l > y$. This is the key assumption. Now H strictly dominates L for a by the assumption. Thus if b's payoff of choosing L is bigger than one of H, $[H, L]$ can be achieved. The condition is as follows.
	\begin{align*}
	&\qquad (x_l - y)\theta + \frac{x_l + y}{2} - \left\{ (y - x_h)\theta + \frac{x_h + y}{2} \right\} > 0\\
	&\Leftrightarrow (x_h + x_l - 2y)\theta + \frac{x_l - x_h}{2} > 0\\[8pt]
	&\Leftrightarrow \theta > \frac{x_h - x_l}{2} \frac{1}{x_h + x_l - 2y}
	\end{align*}
	The last line is due to the $x_h > x_l > y$. And this assumption allows $\frac{x_h - x_l}{2} \frac{1}{x_h + x_l - 2y}$ to be less than $\frac{1}{2}$.
	\par
	If this condition of $\theta$ can be achieved, the probability of matching all members is the probability of "H chooses a and L chooses b". Now I get the below.
	\begin{align*}
	P((a, b)) = (\frac{1}{2} + \theta)^2
	> \left( \frac{1}{2} + \frac{x_h - x_l}{2} \frac{1}{x_h + x_l - 2y} \right)^2
	= \left( \frac{x_h - y}{x_h + x_l -2y} \right)
	= \left( \frac{1}{1 + \frac{x_l - y}{x_h -y}}\right)
	\end{align*}
	The above shows that the lower bound of the probability of success is the decreasing function of $\frac{x_l - y}{x_h - y}$. And this is the increasing function of $x_h$. Then in this one side gap setting "Sekigae" should not be carried out. And the bigger gap among one group exists, the more likely all members get their own partners.
	\par
	Let's compare the result with the no gap situation.
	\begin{align*}
	\left( 2\theta^4 + 3\theta^2 + \frac{1}{8} \right) - \left( \theta + \frac{1}{2} \right)^2 < 0\\
	\Leftrightarrow -0.10339 \dots < \theta < \frac{1}{2}
	\end{align*}
	This means that one side gap setting gives the bigger probability of success than no gap setting if $\frac{1}{2} > \theta > \frac{x_h - x_l}{2} \frac{1}{x_h + x_l - 2y}$ and "Sekigae" is not carried out.
	
	\subsection{Two side quality gap case}
	In this case, all the players know which of the other group members is high quality or low quality and know that among their own group. This is a highly strategic situation. The payoffs are summarized in figure 5, where $x_h$ is the payoff for matching with a high quality person and $x_l$ is one with a low quality person ($x_h > x_l$). I assume that $x_l > y$. This is the same assumption in the previous setting and plays a big roll. Note that $[\alpha, \beta]$ means "high quality woman chooses $\alpha$ quality man and low quality woman chooses $\beta$ quality man". The same notation is used for $(\alpha, \beta)$.
	
	\begin{table}[h]
	\begin{center}
                \setlength{\extrarowheight}{2pt}
                \begin{tabular}{*{16}{c|}}
                  \multicolumn{2}{c}{} & \multicolumn{1}{c}{} & \multicolumn{2}{c}{Women}\\\cline{3-6}
                  \multicolumn{1}{c}{} &  & $[H, H]$  & $[H, L]$ & $[H, L]$ & $[L.L]$\\\cline{2-6}
                  \multirow{4}*{Men}  & $(h,h)$ & $(x_h,y,x_h,y)$ & $(x_h,y,x_h,y)$ & $(y,x_h,x_l,y)$ & $(y,x_h,x_l,y)$\\\cline{2-6}
                  & $(h,l)$ & $(x_h,y,x_h,y)$ & $(x_h,x_l,x_h,x_l)$ & $(y,y,y,y)$ & $(y,x_l,y,x_l)$\\\cline{2-6}
                  & $(l,h)$ & $(x_l,y,y,x_h)$ & $(y,y,y,y)$ & $(x_l,x_h,x_l,x_h)$ & $(y,x_h,x_l,y)$\\\cline{2-6}
                  & $(l,l)$ & $(x_l,y,y,x_h)$ & $(y,x_l,y,x_l)$ & $(x_l,y,y,x_h)$ & $(y,x_l,y,x_l)$\\\cline{2-6}
                \end{tabular}
        \end{center}
        \caption{difference vs difference}
  	\end{table}

	\par
	First I consider "Sekigae" case and assume that all people in this model think "High quality persons match high quality one". I introduce this assumption as an idea accepted in common in order to study the success. This assumption make the payoff matrix very simple as in table 6.
	
	\begin{table}[h]
	\begin{center}
                \setlength{\extrarowheight}{2pt}
                \begin{tabular}{*{4}{c|}}
                  \multicolumn{2}{c}{} & \multicolumn{2}{c}{l}\\\cline{3-4}
                  \multicolumn{1}{c}{} &  & $H$  & $L$ \\\cline{2-4}
                  \multirow{2}*{L}  & $h$ & $(y,y)$ & $(y,y)$\\\cline{2-4}
                  & $l$ & $(y,y)$ & $(x_l, x_l)$ \\\cline{2-4}
                \end{tabular}
        \end{center}
        \caption{payoff matrix for the low quality persons}
  	\end{table}
	
	\par
	Let $x_h > x_l > y$.
	This matrix is the left upper part of table 5. And in this situation choosing the low quality dominates the other choice for both of them. Then $((h, l), [H, L])$ is always achieved as a whole. This means the probability of two matching is 1. Furthermore I get the same result under another assumption, which is "Low quality persons can not match with High quality persons, so should choose Low quality persons". In this situation The payoff matrix can be simplified like the previous one and choosing high quality dominates choosing low quality for the two high quality participants. Then $((h, l), [H, L])$ is always realized.
	\par
	In no "Sekigae" case, I write the payoff gain from the longer time of talking as $\tau$, but it ends up to the same result under the common idea assumptions provided in the previous setting however $\tau$ is large. Thus next I relieve the assumption and consider the general case equilibriums.
	\par
	I derive the mixed strategy Nash equilibrium in "Sekigae" case. Let $P_\alpha(\beta)$ be the probability that $\alpha$ chooses $\beta$. And denote $P_H(h) = a, P_L(h) = b, P_h(H) = c, P_l(H) = d$. Then the expected payoff of $H$ when he chooses h is as follows.
	\begin{align*}
	k
	\end{align*}
	




\section{Conclusion}

\end{document}



















